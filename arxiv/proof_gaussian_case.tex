\section{Missing Proof of the Different Covariates Setting}\label{sec_maintools}

%\subsection{Generalization Bounds}\label{app_proof_main_thm}

\begin{corollary}[Two tasks with different covariates]\label{thm_main_informal}
	%For the setting of two tasks, let $\delta > 0$ be a fixed error margin, $\rho_2 > 1$ and $\rho_1 \gtrsim \delta^{-2}\cdot \lambda_{\min}(M)^{-4} \norm{\Sigma_1} \max(\norm{\beta_1}^2, \norm{\beta_2}^2)$.
	For the setting of two tasks, let $C$ be a fixed constant.
	Let $B, W_1, W_2$ be any local minimum of equation \eqref{eq_mtl}.
 	%There exist two deterministic functions $\Delta_{\bias}$ and $\Delta_{\vari}$ that only depend on scaled model distance $\beta_1 - \frac{W_1}{W_2} \beta_2$, sample sizes $n_1 = \rho_1 \cdot p, n_2 = \rho_2 \cdot p$, and covariate shift matrix $M$ such that
	We have that
	\begin{align*}
		\bigabs{L(\hat{\beta}_2^{\MTL}) - \Pi_{\vari} - (\beta_1 - \frac{W_1}{W_2}\beta_2)^{\top}\Pi_{\bias} (\beta_1 - \frac{W_1}{W_2}\beta_2)} \le \frac{ C\cdot \max(\norm{\beta_1}, \norm{\beta_2}) \cdot \sqrt{\norm{\Sigma_1}} } {\lambda_{\min}^2(M)) } \cdot \frac{1}{\sqrt{\rho_1}}
	\end{align*}
%	\begin{enumerate}
%		\item[a)] \textbf{Positive transfer:} If $\Delta_{\bias} < \Delta_{\vari} -  \frac{\delta}{\sqrt{\rho_1}} $, then w.h.p. over the randomness of $X_1, X_2, \varepsilon_1, \varepsilon_2$, we have
%			\[ \te(\hat{\beta}_2^{\MTL}) < \te(\hat{\beta}_2^{\STL}).  \]
%		\item[b)] \textbf{Negative transfer:} If $\Delta_{\bias} > \Delta_{\vari} + \frac{\delta}{\sqrt{\rho_1}}$, then w.h.p. over the randomness of $X_1, X_2, \varepsilon_1, \varepsilon_2$, we have
%			\[ \te(\hat{\beta}_2^{\MTL}) > \te(\hat{\beta}_2^{\STL}). \]
%	\end{enumerate}
\end{corollary}
In words, the above result shows that a deterministic function $\Delta_{\bias} - \Delta_{\vari}$ determines whether the prediction loss of the empirical multi-task learning estimator is lower than that of single-task learning, up to an error that scales as $\delta / \sqrt{\rho_1}$.
We make several remarks about Corollary \ref{thm_main_informal}.
First, as the amount of data from the source task increases, we get more accurate predictions  since the error scales down.
This applies to many practical settings where collecting labeled data for the target task is expensive and auxillary (source) task data is easier to obtain.
%Theorem \ref{thm_main_informal} applies to settings where large amounts of source task data are available but the target sample size is small.
%For such settings, we obtain a sharp transition from positive transfer to negative transfer determined by $\Delta_{\bias} - \Delta_{\vari}$.
%determined by the covariate shift matrix and the model shift.
%The bounds get tighter and tighter as $\rho_1$ increases.
Second, the deterministic function $\Delta_{\bias} - \Delta_{\vari}$ depends on the three task properties that we care about and the precise form can be found in Section \ref{sec_proof_general}.
Finally, later on in Section \ref{sec_special}, we will study how varying each task property affects the performance of multi-task learning in depth.

Lemma \ref{lem_cov_shift} derives the asymptotic limit of the variance equation \eqref{eq_var_2task}.
To see this, we rescale $X_1$ with $W_1 / W_2$ and set the matrix $\Sigma$ as $\sigma^2 \Sigma_2$.
%allows us to get a tight bound on equation \eqref{eq_te_var}, that only depends on \textit{sample size}, \textit{covariate shift} and the scalar $\hat{v}$.
As a remark, the concentration error $o(\|A\|)$ on the right hand side of equation \eqref{lem_cov_shift_eq} reduces provided with stronger moment assumptions on $X_1$ and $X_2$. % p^{-1/2+\epsilon}
Recall from Section \ref{sec_prelim} that the feature vectors are generated by $\Sigma_i^{1/2} z$, where $z\in\real^p$ consists of i.i.d. entries with mean zero and unit variance.
Provided that for every entry of $z$, its $k$-th moment exists, then we can obtain a concentration error at most $\bigo{\norm{A} p^{-1/2 + 2/k + o(1)}}$.
\todo{We remark that one can probably derive the same asymptotic result using free probability theory (see e.g. \cite{nica2006lectures}), but our results \eqref{lem_cov_shift_eq} and \eqref{lem_cov_derv_eq} also give an almost sharp error bound $\bigo{ p^{-1/2+\epsilon}}$.}
%We will use the matrix fractional notation $\frac{X}{Y}$ for two PSD matrices that share the same eigenspace.
%That is, suppose that the SVD of $X, Y$ is $X = U D_1 U^{\top}$ and $Y = U D_2 U^{\top}$.
%We denote $\frac{X}{Y} = U D_{1} D_2^{-1} U^{\top}$.
%Using the notation, we introduce an important quantity that describes the limit of the bias equation as $p$ goes to infinity.
%Our next result shows that the bias equation \eqref{eq_bias_2task} is described by the above limiting bias asymptotically.
%\begin{lemma}[Bias asymptotics]\label{lem_cov_derivative}
%In the setting of Theorem \ref{thm_main_informal}, let $w \in \R^p$ be any vector that is independent of $X_1$ and $X_2$.
%With high probability, we have that
%\begin{equation}\label{lem_cov_derv_eq}
%\begin{split}
%\bignorm{\Sigma_2^{1/2} \bigbrace{\frac{X_1^{\top}X_1 + X_2^{\top}X_2}{n_1 + n_2}}^{-1} \Sigma_1^{1/2} w}^2
%= w^{\top} \Pi_\bias w + o(\|w\|^2),
%\end{split}
%\end{equation}
%where $\Pi_\bias$ was defined in \eqref{def Pibias} with $M\equiv M(1): = \Sigma_1^{1/2}\Sigma_2^{-1/2}$.
%\end{lemma}
To apply the above result to equation \eqref{eq_bias_2task}, we first approximate $X_1^{\top}X_1$ by $\Sigma_1$, and then use Lemma \ref{lem_cov_derivative} with $w = \Sigma_1 (\beta_1 - \frac{W_1}{W_2} \beta_2)$.
Note that we can rescale $X_1$ by $W_1 / W_2$ and $\Sigma_1$ by $(W_1 / W_2)^2$, and the rest of the steps follows.
As a remark, the approximation of $X_1^{\top}X_1$ by $\Sigma_1$ incurs a generalization error that scales down with $\rho_1$, which is the $\delta / \sqrt{\rho_1}$ term in Theorem \ref{thm_main_informal} more precisely. The formal proof of Theorem \ref{thm_main_informal} using Lemma \ref{lem_cov_shift} and Lemma \ref{lem_cov_derivative} will be presented in Section \ref{app_proof_main_thm}.
Finally, while the limiting bias $\Pi$ provides a precise connection with general covariate shift matrices $M$, it can be difficult to interpret.
In Section \ref{sec_special}, we show that $\Pi$ can be significantly simplified in an isotropic covariance setting.
%While the general form of these functions can be complex (as are previous generalization bounds for MTL), they admit interpretable forms for simplified settings.

%\textbf{Proof overview.}\todo{}
%Theorem \ref{lem_cov_shift_informal} extends a well-known result for the single-task setting when $X_1, \rho_1, a_1$ are all equal to zero \cite{S07}.
%Applying Theorem \ref{lem_cov_shift_informal} to \eqref{eq_te_var}, we can calculate the amount of reduced variance compared to STL, which is given asymptotically by $\Delta_{\vari}$.
%For the bias term in equation \eqref{eq_te_model_shift}, we apply the approximate isometry property to $X_1^{\top}X_1$, which is close to $n_1^2\Sigma_1$. This results in the error term $\delta$, which scales as $(1 + 1/\sqrt{\rho_1})^4-1$.
%Then, we apply a similar identity to Theorem \ref{lem_cov_shift_informal} for bounding the bias term, noting that the derivative of $R(z)$ with respect to $z$ can be approximated by $R_\infty'(z)$.
%This estimates the negative effect given by $\Delta_{\bias}$. %, which will be used to estimate the first term on the right hand side of \eqref{eq_te_model_shift}.
%During this process, we will get the $\Delta_{\bias}$ term up to an error $\delta$ depending on $\rho_1$.
%The proof of Theorem \ref{thm_main_informal} is presented in Appendix \ref{app_proof_main_thm} and the proof of Lemma \ref{lem_cov_shift_informal} is in Appendix \ref{sec_maintools}.
%
%
%
%The formal statement is stated in Theorem \ref{thm_many_tasks} and its proof can be found in Appendix \ref{app_proof_many_tasks}.
%The technical crux of our approach is to derive the asymptotic limit of $\te(\hat{\beta}_t^{\MTL})$ in the high-dimensional setting, when $p$ approaches infinity.
%We derive a precise limit of $\bigtr{(X_1^{\top}X_1 + X_2^{\top}X_2)^{-1}\Sigma_2}$, which is a deterministic function that only depends on $\Sigma_1, \Sigma_2$ and $n_1/p, n_2/p$ (see Lemma \ref{lem_cov_shift} in Appendix \ref{app_proof_main} for the result).
%Based on the result, we show how to determine positive versus negative transfer as follows.
%, where $\lambda_{\min}(M)$ is the smallest singular value of $M_1$


%\begin{lemma}[Variance bound]\label{lem_cov_shift_informal}
%	In the setting of two tasks,
%	let $n_1 = \rho_1 \cdot p$ and $n_2 = \rho_2 \cdot$ be the sample size of the two tasks.
%	Let $\lambda_1, \dots, \lambda_p$ be the singular values of the covariate shift matrix $\Sigma_1^{1/2}\Sigma_2^{-1/2}$ in decreasing order.
%	%let $n_1 = \rho_1 \cdot p$ and $n_2 = \rho_2 \cdot p$ denote the sample sizes of each task.
%	%Let $\Sigma_1$ and $\Sigma_2$ denote the covariance matrix of each task.
%	With high probability, the variance of the multi-task estimator $\hat{\beta}_t^{\MTL}$ equals
%	%let $M = \Sigma_1^{1/2}\Sigma_2^{-1/2}$ and $\lambda_1, \lambda_2, \dots, \lambda_p$ be the singular values of $M^{\top}M$ in descending order.
%%	For any constant $\e>0$, w.h.p. over the randomness of $X_1, X_2$, we have that
%	{\small\begin{align*}%\label{eq_introX1X2}
%		%\bigtr{(X_1^{\top}X_1 + X_2^{\top}X_2)^{-1}\Sigma_2} =
%		\frac{\sigma^2}{n_1+n_2}\cdot \bigtr{ (\hat{v}^2 a_1 \Sigma_2^{-1/2}\Sigma_1\Sigma_2^{-1/2} + a_2\id)^{-1}} +\bigo{{p^{-1/2+o(1)}}},
%	\end{align*}}%
%	where $a_1, a_2$ are solutions of the following equations:
%	{\small\begin{align*}
%		a_1 + a_2 = 1- \frac{1}{\rho_1 + \rho_2},\quad a_1 + \frac1{\rho_1 + \rho_2}\cdot \frac{1}{p}\sum_{i=1}^p \frac{\hat{v}^2\lambda_i^2 a_1}{\hat{v}^2\lambda_i^2 a_1 + a_2} = \frac{\rho_1}{\rho_1 + \rho_2}.
%	\end{align*}}
%%are both fixed values that roughly scales with the sample sizes $\rho_1, \rho_2$, and satisfy $a_1 + a_2 = 1 - (\rho_1 + \rho_2)^{-1}$ plus another deterministic equation.
%\end{lemma}


%which deals with the inverse of the sum of two random matrices, which
%any is can be viewed as a special case of Theorem \ref{thm_model_shift}.

%\begin{lemma}[Variance asymptotics]\label{lem_cov_shift}
	%Let $X_i\in\real^{n_i\times p}$ be a random matrix that contains i.i.d. row vectors with mean $0$ and variance $\Sigma_i\in\real^{p\times p}$, for $i = 1, 2$.
	%Suppose $X_1=Z_1\Sigma_1^{1/2}\in \R^{n_1\times p}$ and $X_2=Z_2\Sigma_2^{1/2}\in \R^{n_2\times p}$ satisfy Assumption \ref{assm_secA1} with $\rho_1:=n_1/p>1$ and $\rho_2:=n_2/p>1$ being fixed constants.
	%Denote by $M = \Sigma_1^{1/2}\Sigma_2^{-1/2}$ and
%	Let $\Sigma \in \real^{p \times p}$ be any fixed and deterministic matrix.
%	In the setting of Theorem \ref{thm_main_informal},
%	with probability $1-\oo(1)$ over the randomness of $X_1$ and $X_2$, we have that %for any constant $\e>0$,
	%When $n_1 = c_1 p$ and $n_2 = c_2 p$, we have that with high probability over the randomness of $X_1$ and $X_2$, the following equation holds
%	\begin{align}\label{lem_cov_shift_eq}
%		\bigtr{(X_1^{\top}X_1 + X_2^{\top}X_2)^{-1} \Sigma} = \bigtr{ (a_1 \Sigma_1 + a_2\Sigma_2)^{-1} \Sigma} + o(\norm{\Sigma}), %\bigo{\|\| p^{-1/2+\epsilon}}
%	\end{align}
%where $(a_1,a_2)$ is the solution to equation \eqref{eq_a12extra} with $\lambda_i$ being the singular values of $M\equiv M(1): = \Sigma_1^{1/2}\Sigma_2^{-1/2}$.
%\end{lemma}


In this subsection, we provide the proof of Corollary \ref{thm_main_informal} using Theorem \ref{thm main RMT}.
%Lemma  \ref{lem_cov_shift} and \ref{lem_cov_derivative}.

Our key insight is a bias-variance decomposition of the expected prediction loss of $\hat{\beta}_2^{\MTL} = \hat{B} W_2$.
Using the local optimality condition of $B$, we obtain that
\begin{align*}
	 \hat{B}(W_1, W_2) &= (W_1^2 X_1^{\top}X_1 + W_2^2 X_2^{\top}X_2)^{-1} (W_1 X_1^{\top}Y_1 + W_2 X_2^{\top}Y_2)\\
	&= \frac{1}{W_2} \left( \frac{W_1^2}{W_2^2}  X_1^{\top}X_1 + X_2^{\top}X_2\right)^{-1} \left(\frac{W_1}{W_2} X_1^{\top}Y_1 + X_2^{\top}Y_2\right) \\
	&= \frac{1}{W_2}\left[\beta_2 + \left(\frac{W_1^2}{W_2^2} X_1^{\top}X_1 + X_2^{\top}X_2\right)^{-1}\bigbrace{X_1^{\top}X_1\left(\frac{W_1}{W_2}\beta_1 - \frac{W_1^2}{W_2^2} \beta_2\right) + \left(\frac{W_1}{W_2} X_1^{\top}\varepsilon_1 + X_2^{\top}\varepsilon_2\right)}\right].
\end{align*}
Recall that $\hat{\beta}_i^{\MTL} = \hat{B} W_i$. Therefore, we have
\begin{align}
	\exarg{\epsilon_1, \epsilon_2}{L(\hat{\beta}_2^{\MTL}) \mid X_1, X_2}
	=&~ \frac{W_1^2}{W_2^2} \bignorm{\Sigma_2^{1/2}(\frac{W_1^2}{W_2^2} X_1^{\top}X_1 + X_2^{\top}X_2)^{-1} X_1^{\top}X_1 (\beta_1 - \frac{W_1}{W_2} \beta_2)}^2 \label{eq_bias_2task} \\
			&+~  \sigma^2\cdot \bigtr{\Sigma_2(\frac{W_1^2}{W_2^2} X_1^{\top}X_1 + X_2^{\top}X_2)^{-1} }. \label{eq_var_2task}
\end{align}
Equation \eqref{eq_bias_2task} is the bias of $\hat{\beta}_t^{\MTL}$ and
equation \eqref{eq_var_2task} is the variance of $\hat{\beta}_t^{\MTL}$.
%minus the variance of $\hat{\beta}_t^{\STL}$, which is always negative.
Comparing the above to single-task learning, that is,
\begin{align}
	\exarg{\epsilon_2}{L(\hat{\beta}_2^{\STL}) \mid X_2} = \sigma^2 \cdot \bigtr{\Sigma_2 (X_2^{\top} X_2)^{-1}}, \label{eq_var_stl}
\end{align}
we observe that while the bias of $\hat{\beta}_2^{\MTL}$ is always larger than that of $\hat{\beta}_2^{\STL}$, which is zero, the variance of $\hat{\beta}_2^{\MTL}$ is always lower than that of $\hat{\beta}_2^{\STL}$.\footnote{To see why this is true, we apply the Woodbury matrix identity over equation \eqref{eq_var_2task} and use the fact that for the product of two PSD matrices, its trace is always nonnegative.}
In other words, training both tasks together helps predict the target task by reducing variance while incurring a bias.
Therefore, whether multi-task learning outperforms single-task learning is determined by the bias-variance decomposition!

 \todo{derive the test loss} Expectation was given in \eqref{eq_bias_2task} and \eqref{eq_var_2task}.

The proof of Lemma \ref{thm_model_shift} is based on Lemma \ref{lem_cov_shift}, Lemma \ref{lem_cov_derivative}, and the following bound on the singular values of $Z^{(1)}$: for any constant $\e>0$, we have
\begin{align}
\al_-(\rho_1) - \OO(p^{-1/2+e})  \preceq {n_1^{-1}}{(Z^{(1)})^{\top} Z^{(1)}}  \preceq   \al_+(\rho_1) + \OO(p^{-1/2+e}) \quad \text{w.h.p.}  \label{eq_isometric}
\end{align}
%with overwhelming probability.
In fact, $n_1^{-1}(Z^{(1)})^{\top}Z^{(1)}$ is a standard sample covariance matrix, and it is well-known that its eigenvalues are all inside the support of the Marchenko-Pastur law $[\al_-(\rho_1)-\oo(1) ,\al_+(\rho_1)+\oo(1)]$ with probability $1-\oo(1)$ \cite{No_outside}. The estimate \eqref{eq_isometric} provides tight bounds on the concentration errors, and it will be formally proved in Lemma \ref{SxxSyy} below.


%First, we provide the steps for the bias-variance decomposition.
%Using the local optimality condition of $B$, we obtain that
%\begin{align*}
%	 \wh{B}(W_1, W_2) &= (W_1^2 (X^{(1)})^{\top}X^{(1)} + W_2^2 (X^{(2)})^{\top}X^{(2)})^{-1} (W_1 (X^{(1)})^{\top}Y_1 + W_2 (X^{(2)})^{\top}Y_2)\\
%	&= \frac{1}{W_2} \left( \frac{W_1^2}{W_2^2}  (X^{(1)})^{\top}X^{(1)} + (X^{(2)})^{\top}X^{(2)}\right)^{-1} \left(\frac{W_1}{W_2} (X^{(1)})^{\top}Y_1 + (X^{(2)})^{\top}Y_2\right) \\
%	&= \frac{1}{W_2}\left[\beta_2 + \left(\frac{W_1^2}{W_2^2} (X^{(1)})^{\top}X^{(1)} + (X^{(2)})^{\top}X^{(2)}\right)^{-1}\bigbrace{(X^{(1)})^{\top}X^{(1)}\left(\frac{W_1}{W_2}\beta_1 - \frac{W_1^2}{W_2^2} \beta_2\right) + \left(\frac{W_1}{W_2} (X^{(1)})^{\top}\varepsilon_1 + (X^{(2)})^{\top}\varepsilon_2\right)}\right].
%\end{align*}
%Recall that $\wh{\beta}_2^{\MTL} = \wh{B}(W_1, W_2) W_2$, and one can verify that equation \eqref{eq_bias_2task} and \eqref{eq_var_2task} hold.

During the proof we shall also give explicit expressions for $\Delta_{\bias}$ and $\Delta_{\vari}$, which might be of interest to some readers.
With $a_i$, $i=1,2,3,4$, given in Lemma \ref{lem_cov_shift} and Lemma \ref{lem_cov_derivative},
we define the following matrix
\be\label{defnpihat}\Pi \equiv \Pi(\wh v)= \frac{\rho_1^2}{(\rho_1 + \rho_2)^2}\cdot \wh v^2{M} \frac{(1 +   a_3)\id + \wh v^2 a_4 {M}^{\top} {M}}{( \wh v^2 a_1 {M}^{\top} {M}+ a_2 )^2} {M}^{\top}.\ee
%which is defined in a way such that {\it in certain sense} it is the asymptotic limit of the random matrix
%$$\wh v\Sigma_1^{1/2} (\wh{v}^2 (X^{(1)})^{\top}X^{(1)} + (X^{(2)})^{\top}X^{(2)})^{-1} \Sigma_2 (\wh{v}^2 (X^{(1)})^{\top}X^{(1)} + (X^{(2)})^{\top}X^{(2)})^{-1}\Sigma_1^{1/2}.$$
Moreover, we introduce two factors that will appear often in our statements and discussions:
$$\al_-(\rho_1):=\left(1- \rho_1^{-1/2}\right)^2,\quad \al_+(\rho_1):=\left(1 + \rho_1^{-1/2}\right)^2.$$
In fact, $\al_-(\rho_1)$ and $\al_+(\rho_1)$ correspond to the largest and smallest singular values of $Z^{(1)}/\sqrt{n_1}$, respectively, as given by the famous Mar{\v c}enko-Pastur law \cite{MP}. In particular, as $\rho_1$ increases, both $\al_-$ and $\al_+$ will converge to 1 and $Z^{(1)}/\sqrt{n_1}$ will be more close to an isometry. Finally, we introduce the error term
\be\label{eq_deltaextra}
 \delta(\wh v):=\frac{\al_+^2(\rho_1) - 1 }{\al_-^{2}(\rho_1)\wh v^2\lambda_{\min}^2(M)} \cdot  \norm{\Sigma_1^{1/2}(\beta_1 - \wh{v}\beta_2)}^2.\ee
%where $\lambda_{\min}(\wh M)$ is the smallest singular value of $\wh M$.
Note that this factor converges to 0 as $\rho_1$ increases. It is not hard to see that Theorem \ref{thm_main_informal} is an immediate consequence of the following lemma.

%$$\delta:=\left[\frac{n_1 \lambda_1}{(\sqrt{n_1}-\sqrt{p})^2\lambda_p  +  (\sqrt{n_2}-\sqrt{p})^2}\right]^2\cdot \norm{\Sigma_1^{1/2}(\beta_1 - \wh{w}\beta_2)}^2.$$
%{\cor may be able to get a better bound, but the statement will be long}





\begin{lemma}%[Theorem \ref{thm_main_informal} restated]
\label{thm_model_shift}
%For $i=1,2$, let $Y_i = X_i\beta_i + \varepsilon_i$ be two independent data models, where $X_i$, $\beta_i$ and $\varepsilon_i$ are also independent of each other. Suppose that $X_i=Z_i\Sigma_i^{1/2}\in \R^{n_i\times p}$ satisfy Assumption \ref{assm_secA1} with $\rho_i:=n_i/p>1$ being fixed constants, and $\e_i\in \R^{n_i}$ are random vectors with i.i.d. entries with mean zero, variance $\sigma^2$ and all moments as in \eqref{assmAhigh}.
Consider two data models $Y_i = X_i\beta_i + \varepsilon_i$, $i=1,2$, that satisfy Assumption \ref{assm_secA2} with $\sigma_1^2=\sigma_2^2=\sigma^2$. Then with overwhelming probability, we have
	\begin{align}
	 	\te(\wh{\beta}_{t}^{\MTL}) \le \te(\wh{\beta}_t^{\STL}) \quad \text{ when: } \ \ &\Delta_{\vari} - \Delta_{\bias} \ge   \delta(\wh v), \label{upper}\\
		\te(\wh{\beta}_t^{\MTL}) \ge \te(\wh{\beta}_t^{\STL}) \quad \text{ when: } \ \ &\Delta_{\vari} - \Delta_{\bias} \le - \delta(\wh v), \label{lower}
	\end{align}
	where
	\begin{align} %\bigtr{{\Sigma_2^{-1}}}
		\Delta_{\vari} &\define {\sigma^2}\bigbrace{\frac{1}{\rho_2 - 1} -  \frac{1}{\rho_1 + \rho_2}\cdot \frac1p \bigtr{( \wh v^2 a_1 M^{\top}M +  a_2\id)^{-1}} } \label{Deltavarv} \\
		\Delta_{\bias} &\define (\beta_1 - \wh{v}\beta_2)^{\top} \Sigma_1^{1/2} \Pi (\wh v)\Sigma_1^{1/2} (\beta_1 - \wh{v}\beta_2). \label{Deltabetav}
	\end{align}
\end{lemma}



%\begin{remark}
The main error $\delta$ in Lemma \ref{thm_model_shift} comes from approximating $n_1^{-1}(Z^{(1)})^{\top}Z^{(1)}$ by $\id$ using estimate \eqref{eq_isometric}; see the estimate \eqref{bounddelta-} below. In order to improve this estimate and obtain an exact asymptotic result, one needs to study the singular value distribution of the following random matrix:
$$((X^{(1)})^{\top}X^{(1)})^{-1}(X^{(2)})^{\top}X^{(2)} +  \wh{v}^2 .$$
In fact, the eigenvalues of $\cal X:=((X^{(1)})^{\top}X^{(1)})^{-1}(X^{(2)})^{\top}X^{(2)}$ have been studied in the name of Fisher matrices; see e.g. \cite{Fmatrix}. However, since $\cal X$ is non-symmetric, it is known that the singular values of $\cal X$ are different from its eigenvalues. To the best of our knowledge, the asymptotic singular value behavior of $\cal X$ is still unknown in random matrix literature, and the study of the singular values of $\cal X +\wh v^2$ will be even harder. We leave this problem to future study.
%\end{remark}





\begin{proof}[Proof of Lemma \ref{thm_model_shift}]
%\noindent
%To prove Theorem \ref{thm_cov_shift}, we study the spectrum of the random matrix model:
%$$Q= \Sigma_1^{1/2}  (Z^{(1)})^{\top} Z^{(1)} \Sigma_1^{1/2}  + \Sigma_2^{1/2}  (Z^{(2)})^{\top} Z^{(2)} \Sigma_2^{1/2} ,$$
%where $\Sigma_{1,2}$ are $p\times p$ deterministic covariance matrices, and $X^{(1)}=(x_{ij})_{1\le i \le n_1, 1\le j \le p}$ and $X^{(2)}=(x_{ij})_{n_1+1\le i \le n_1+n_2, 1\le j \le p}$ are $n_1\times p$ and $n_2 \times p$ random matrices, respectively, where the entries $x_{ij}$, $1 \leq i \leq n_1+n_2\equiv n$, $1 \leq j \leq p$, are real independent random variables satisfying
%\begin{equation}\label{eq_12moment} %\label{assm1}
%\mathbb{E} z_{ij} =0, \ \quad \ \mathbb{E} \vert z_{ij} \vert^2  = 1.
%\end{equation}
%\todo{A proof outline; including the following key lemma.}
By equations \eqref{eq_te_model_shift} and \eqref{eq_te_var}, we can write
\begin{align*}
L(\wh{\beta}_t^{\STL}) - L(\wh{\beta}_t^{\MTL})=\delta_{\vari}(\wh v) - \delta_{\bias}(\wh v),
\end{align*}
where we denote
\begin{align*}
\delta_{\vari}(\wh v)&=\sigma^2 \left(  \bigtr{((X^{(2)})^{\top}X^{(2)})^{-1}\Sigma_2} -  \bigtr{( \wh{v}^2 (X^{(1)})^{\top}X^{(1)} + (X^{(2)})^{\top}X^{(2)})^{-1} \Sigma_2}\right),
\end{align*}
and
\begin{align}\label{revise_deltabias}
 \delta_{\bias}(\wh v)&= \wh{v}^2 \bignorm{\Sigma_2^{1/2}(\wh{v}^2 (X^{(1)})^{\top}X^{(1)} + (X^{(2)})^{\top}X^{(2)})^{-1} (X^{(1)})^{\top}X^{(1)} (\beta_1 - \wh{v} \beta_2)}^2.
\end{align}
We introduce the notation $\wh M \equiv \wh M(v)= v\Sigma_1^{1/2}\Sigma_2^{-1/2}$ for $v\in \R$. Then the proof is divided into the following four steps.
\begin{itemize}
\item[(i)] We first consider $ \wh M(v)$ for a fixed $v\in \R$. Then we use Lemma \ref{lem_minv} and Lemma \ref{lem_cov_shift} to calculate the variance reduction $\delta_{\vari}(v)$, which will lead to the $\Delta_{\vari}$ term.
%$$\sigma^2 \cdot \bigtr{((X^{(2)})^{\top}X^{(2)})^{-1}},\quad \sigma^2 \cdot \bigtr{({v}^2 (X^{(1)})^{\top}X^{(1)} + (X^{(2)})^{\top}X^{(2)})^{-1} \Sigma_2},$$
%and the difference between them

\item[(ii)] Using the approximate isometry property of $Z^{(1)}$ in equation \eqref{eq_isometric}, we will approximate the bias term $ \delta_{\bias}(v)$ by
%$${v}^2 \bignorm{\Sigma_2^{1/2}({v}^2 (X^{(1)})^{\top}X^{(1)} + (X^{(2)})^{\top}X^{(2)})^{-1} (X^{(1)})^{\top}X^{(1)} (\beta_1 - {v} \beta_2)}^2$$
%through
\be\label{deltabetapf}
\wt\delta_{\bias}(v):={v}^2 n_1^2\bignorm{\Sigma_2^{1/2}({v}^2 (X^{(1)})^{\top}X^{(1)} + (X^{(2)})^{\top}X^{(2)})^{-1} \Sigma_1 (\beta_1 - {v} \beta_2)}^2.\ee

\item[(iii)] We use Lemma \ref{lem_cov_derivative} to calculate \eqref{deltabetapf}, which will lead to the $\Delta_{\bias}$ term.

\item[(iv)] Finally we use a standard $\e$-net argument to extend the above results to $\wh M(\wh v)$ for a possibly random $\wh v$ which depends on $Y_1$ and $Y_2$.
\end{itemize}


\paragraph{Step I: Variance reduction.} Consider $\wh M(v)$ for a fixed $v\in \R$. Using Lemma \ref{lem_cov_shift}, we can obtain that for any constant $\e>0$,
$$  \sigma^2 \cdot \bigtr{((X^{(2)})^{\top}X^{(2)})^{-1}\Sigma_2} = \frac{\sigma^2}{\rho_2-1}\left( 1+ \OO(p^{-1/2+e})\right),$$
and
$$ \sigma^2 \cdot \bigtr{( {v}^2 (X^{(1)})^{\top}X^{(1)} + (X^{(2)})^{\top}X^{(2)})^{-1} \Sigma_2} =   \frac {\sigma^2} {\rho_1 + \rho_2}\cdot \frac1p \bigtr{( a_1(v) \wh M^{\top}\wh M + a_2(v)\id)^{-1}}\left( 1+ \OO(p^{-1/2+e})\right) ,$$
with overwhelming probability, where $a_1(v)$ and $a_2(v)$ satify equation \eqref{eq_a12extra} with $\wh v$ replaced by $v$. The subtraction of these two terms gives
\be\label{deltavaral-} \delta_{\vari}(v)=\Delta_{\vari}(v) +\OO( \sigma^2 p^{-1/2+e}) \quad \text{w.h.p.},
\ee
where $\Delta_{\vari}(v)$ is defined as in equation \eqref{Deltavarv} but with $\wh v$ replaced by $v$.



\paragraph{Step II: Bounding the bias term.}
Next we use equation \eqref{eq_isometric} to approximate $\delta_{\bias}(v)$ with $\wt\delta_{\bias}(v)$. %in \eqref{deltabetapf}.
%relate the first term in equation \eqref{eq_te_model_shift} to $\Delta_{\bias}$.
\begin{claim}\label{prop_model_shift}
	In the setting of Lemma \ref{thm_model_shift},
	we denote $K(v) := (v^2(X^{(1)})^{\top}X^{(1)} + (X^{(2)})^{\top}X^{(2)})^{-1}$, and
	\begin{align*}
		%\delta_1 &= v^2 \bignorm{\Sigma_2^{1/2} K (X^{(1)})^{\top}X^{(1)}(\beta_1 - v\beta_2)}^2, \\
		%\delta_2 &= n_1^2\cdot v^2 \bignorm{\Sigma_2^{1/2}K\Sigma_1(\beta_1 - v\beta_2)}, \\
		\delta_{err}(v) := n_1^2 v^2 \bignorm{\Sigma_1^{1/2} K(v) \Sigma_2 K(v) \Sigma_1^{1/2}} \cdot \bignorm{\Sigma_1^{1/2} (\beta_1 - v\beta_2)}^2.
	\end{align*}
	Then we have w.h.p.
	\begin{align*}
		 \left| \delta_{\bias}(v)-\wt\delta_{\bias}(v)\right|
		\le  \left( \al_+^2(\rho_1)-1 + \OO(p^{-1/2+\e})\right)\delta_{err}(v).
	\end{align*}
%	We have that
%	\begin{align*}
%		-2n_1^2\bigbrace{{2\sqrt{\frac{p}{n_1}}} + {\frac{p}{n_1}}} \delta_3
%		\le  \delta_1 - \delta_2
%		\le n_1^2\bigbrace{2\sqrt{\frac{p}{n_1}} + \frac{p}{n_1}}\bigbrace{2 + 2\sqrt{\frac{p}{n_1}} + \frac{p}{n_1}}\delta_3.
%	\end{align*}
%	For the special case when $\Sigma_1 = \id$ and $\beta_1 - \beta_2$ is i.i.d. with mean $0$ and variance $d^2$, we further have
%	\begin{align*}
%		\bigbrace{1 - \sqrt{\frac{p}{n_1}}}^4 \Delta_{\bias}
%		\le \bignorm{\Sigma_2^{1/2} ((X^{(1)})^{\top}X^{(1)} + (X^{(2)})^{\top}X^{(2)})^{-1}(X^{(1)})^{\top}X^{(1)}(\beta_1 - \beta_2)}^2.
%	\end{align*}
\end{claim}

\begin{proof}
	%The proof follows by applying equation \eqref{eq_isometric}.
	%Recall that $(X^{(1)})^{\top}X^{(1)} = \Sigma_1^{1/2}(Z^{(1)})^{\top}Z^{(1)}\Sigma_1^{1/2}$.
We can calculate that
%	Let $\alpha = \bignorm{\Sigma_2^{1/2} K \Sigma_1 (\beta_1 - \wh{w}\beta_2)}^2$.
	%We have
	\begin{align}
%		& \bignorm{\Sigma_2^{1/2}((X^{(1)})^{\top}X^{(1)} + (X^{(2)})^{\top}X^{(2)})^{-1}(X^{(1)})^{\top}X^{(1)}(\beta_1 - \wh{w}\beta_2)}^2 \nonumber \\
		 \delta_{\bias}(v)-\wt\delta_{\bias}(v)&= {2v^2n_1}(\beta_1 - v\beta_2)^{\top}\Sigma_1^{1/2} \cE\left(\Sigma_1^{1/2}K \Sigma_2 K \Sigma_1^{1/2}\right) \Sigma_1^{1/2} (\beta_1 - v\beta_2) \nonumber
		\\
		&+ v^2\bignorm{\Sigma_2^{1/2} K \Sigma_1^{1/2}\cE \Sigma_1^{1/2}(\beta_1 - v\beta_2)}^2. \label{eq_lem_model_shift_1}
%		\le& n_1\bigbrace{{n_1^2}{} + \frac{2n_1}p(p + 2\sqrt{{n_1}p}) + (p + 2\sqrt{{n_1}p})^2} \alpha = n_1^2\bigbrace{1 + \sqrt{\frac{p}{n_1}}}^4 \alpha. \nonumber
	\end{align}
	where we denote $\cE: = (Z^{(1)})^{\top}Z^{(1)} - {n_1}\id$.
  Using equation \eqref{eq_isometric}, we can bound
	$$\|\cal E\|\le \left( \al_+(\rho_1)-1 + \OO(p^{-1/2+\e})\right)n_1, \quad \text{w.h.p.}$$
	Thus we can estimate that
	\begin{align*}
	| \delta_{\bias}(v)-\wt\delta_{\bias}(v)|&\le v^2 \left( 2n_1  \|\cal E\| +  \|\cal E\|^2 \right) \bignorm{\Sigma_1^{1/2} K \Sigma_2 K \Sigma_1^{1/2}} \bignorm{\Sigma_1^{1/2} (\beta_1 - v\beta_2)}^2 \\
	&=  v^2 \left[\left( n_1 + \|\cal E\|\right)^2 -n_1^2 \right] \bignorm{\Sigma_1^{1/2} K \Sigma_2 K \Sigma_1^{1/2}} \bignorm{\Sigma_1^{1/2} (\beta_1 - v\beta_2)}^2 \\
	& \le v^2 n_1^2 \left[ \al_+^2(\rho_1) + \OO(p^{-1/2+\e}) -1\right]\bignorm{\Sigma_1^{1/2} K \Sigma_2 K \Sigma_1^{1/2}} \bignorm{\Sigma_1^{1/2} (\beta_1 - v\beta_2)}^2,
	\end{align*}
	which concludes the proof by the definition of $\delta_{err}$.
%	we can bound the second term on the RHS of equation \eqref{eq_lem_model_shift_1} as
%	\begin{align*}
%		& \bigabs{(\beta_1 -  \beta_2)^{\top} \Sigma_1^{1/2} \cE \Sigma_1^{1/2} K \Sigma_2 K \Sigma_1 (\beta_1 - v\beta_2)}\le n_1  \|\cal E\| \cdot \bignorm{\Sigma_1^{1/2} K \Sigma_2 K \Sigma_1^{1/2}} \bignorm{\Sigma_1^{1/2} (\beta_1 - v\beta_2)}^2 \\
%		= & \bigabs{\bigtr{\cE \Sigma_1^{1/2}K\Sigma_2 K \Sigma_1(\beta_1 - \wh{w}\beta_2)(\beta_1 - \wh{w}\beta_2)^{\top} \Sigma_1^{1/2}}} \\
%		\le & \norm{\cE} \cdot \bignormNuclear{\Sigma_1^{1/2} K \Sigma_2 K \Sigma_1 (\beta_1 - \wh{w}\beta_2) (\beta_1 - \wh{w}\beta_2)^{\top} \Sigma_1^{1/2}} \\
%		\le & n_1 \bigbrace{2\sqrt{\frac{p}{n_1}} + \frac{p}{n_1}} \cdot \bignormNuclear{\Sigma_1^{1/2} K \Sigma_2 K \Sigma_1 (\beta_1 - \wh{w}\beta_2)(\beta_1 - \wh{w}\beta_2)^{\top} \Sigma_1^{1/2}} \tag{by equation \eqref{eq_isometric}} \\
%		\le   & n_1 \bigbrace{2\sqrt{\frac{p}{n_1}} + \frac{p}{n_1}} \bignorm{\Sigma_1^{1/2}K \Sigma_2 K \Sigma_1^{1/2}} \cdot \bignorm{\Sigma_1^{1/2}(\beta_1 - \wh{w}\beta_2)}^2 \tag{since the matrix inside is rank 1}
%	\end{align*}
%	The third term in equation \eqref{eq_lem_model_shift_1} can be bounded with
%	\begin{align*}
%		\bignorm{\Sigma_2^{1/2}K\Sigma_1^{1/2}\cE\Sigma_1^{1/2}(\beta_1 - v\beta_2)}^2
%		\le n_1^2 \bigbrace{2\sqrt{\frac{p}{n_1}} + \frac{p}{n_1}}^2 \bignorm{\Sigma_1^{1/2}K\Sigma_{2}K\Sigma_1^{1/2}} \cdot \bignorm{\Sigma_1^{1/2}(\beta_1 -  \beta_2)}^2.
%	\end{align*}
%	Combined together we have shown the right direction for $\delta_1 - \delta_2$.
%	For the left direction, we simply note that the third term in equation \eqref{eq_lem_model_shift_1} is positive.
%	And the second term is bigger than $-2n_1^2(2\sqrt{\frac{p}{n_1}} + \frac{p}{n_1}) \alpha$ using equation \eqref{eq_isometric}.
\end{proof}
Note by equation \eqref{eq_isometric}, we also obtain that with overwhelming probability,
\begin{align*}
&v^2 n_1^2 \Sigma_1^{1/2} K \Sigma_2 K \Sigma_1^{1/2} =n_1^2 \wh M (\wh M^\top (Z^{(1)})^{\top} Z^{(1)} \wh M + (Z^{(2)})^\top Z^{(2)})^{-2}\wh M^\top \\
&\preceq  n_1^2 \wh M \left[n_1 \al_-(\rho_1)\wh M^\top \wh M + n_2 \al_-(\rho_2) + \OO(p^{1/2+\e})\right]^{-2}\wh M^\top \\
&\preceq  \left[ \al_-^2(\rho_1) \wh M\wh M^\top + 2\frac{\rho_2}{\rho_1} \al_-(\rho_1)\al_-(\rho_2) + \left(\frac{\rho_2}{\rho_1}\right)^2 \al_-^2(\rho_2) (\wh M \wh M^\top )^{-1}\right]^{-1}+  \OO(p^{-1/2+\e}) \\
&\preceq [\al_-^2(\rho_1) \lambda_{\min}^2(\wh M)]^{-1}\cdot (1 - c)
\end{align*}
for a small enough constant $c>0$. Together with Claim \ref{prop_model_shift}, we get that with overwhelming probability,
\be\label{bounddelta-}
\left| \delta_{\bias}(v)-\wt\delta_{\bias}(v)\right|
		\le (1-c) \delta(v)
\ee
for some small constant $c>0$, where recall that $\delta(v)$ was defined in equation \eqref{eq_deltaextra}.


\paragraph{Step III: The limit of $\wt\delta_{\bias}(v)$.}
Using Lemma \ref{lem_cov_derivative}, we obtain that with overwhelming probability,
\begin{align*}
\wt\delta_{\bias}(v) &=\frac{\rho_1^2}{(\rho_1 + \rho_2)^2}\cdot v^2 (\beta_1-v\beta_2)^\top\Sigma_1 \Sigma_2^{-1/2}  \frac{(1 +  a_3(v))\id +v^2 a_4(v) {M}^{\top}{M}}{( v^2 a_1(v) {M}^{\top}{M}+a_2(v) )^2} \Sigma_2^{-1/2} \Sigma_1(\beta_1-v\beta_2) +\OO(p^{-1/2+\e}) \\
&= (\beta_1 - {v}\beta_2)^{\top} \Sigma_1^{1/2} \Pi(v) \Sigma_1^{1/2} (\beta_1 - {v}\beta_2) +\OO(p^{-1/2+\e}) =: \Delta_{\bias}(v) +\OO(p^{-1/2+\e}),
\end{align*}
where $a_3(v)$ and $a_4(v)$ satify equation \eqref{eq_a34extra} with $\wh v$ replaced by $v$, and $\Pi$ was defined in equation \eqref{defnpihat}. Together equations \eqref{deltavaral-} and \eqref{bounddelta-}, we obtain that w.h.p.,
\be\label{dicho_varbeta}
\begin{cases}\delta_{\vari}(v)>\delta_{\bias}(v), & \text{ if } \ \ \Delta_{\vari}(v) - \Delta_{\bias}(v) \ge   \delta(v),\\
\delta_{\vari}(v)<\delta_{\bias}(v),  & \text{ if }  \ \ \Delta_{\vari}(v) - \Delta_{\bias}(v) \le -  \delta(v).\end{cases}
\ee



\paragraph{Step IV: An $\e$-net argument.} Finally, it remains to extend the above result \eqref{dicho_varbeta} to $v=\wh v$, which is random and depends on $X^{(1)}$ and $X^{(2)}$. We first show that for any fixed constant $C_0>0$, there exists a overwhelming probability event $\Xi$ on which equation \eqref{dicho_varbeta}
%\eqref{lem_cov_shift_eq} and \eqref{lem_cov_derv_eq}
holds uniformly for all $v\in [-C_0, C_0]$. In fact, we first consider $v$ that belongs to a discrete set
$$V:=\{v_k = kp^{-1}: -(C_0p +1)\le k \le C_0p +1\}.$$
Then using the arguments for the first three steps and a simple union bound, we get that equation
\eqref{dicho_varbeta} holds simultaneously for all $v\in V$ with overwhelming probability. On the other hand, by equation \eqref{eq_isometric} the event
$$\Xi_1:=\left\{ \al_-(\rho_1)/2 \preceq  \frac{(Z^{(1)})^{\top} Z^{(1)}}{n_1}  \preceq   2\al_+(\rho_1) ,\  \al_-(\rho_2)/2 \preceq  \frac{(Z^{(2)})^\top Z^{(2)}}{n_2}  \preceq   2\al_+(\rho_2)\right\}$$
holds with overwhelming probability. Now it is easy to check that on $\Xi_1$, the following estimates holds simultaneously for all $v_k \le v\le v_{k+1}$:
\begin{align*}
& |\delta_{\vari}(v) -\delta_{\vari}(v_k)|\lesssim p^{-1}\delta_{\vari}(v_k),\ \ |\delta_{\bias}(v) -\delta_{\bias}(v_k)|\lesssim p^{-1}\delta_{\bias}(v_k), \ \   |\delta(v)-\delta(v_k)|\lesssim p^{-1}\delta(v_k),\\
& |\Delta_{\bias}(v) -\Delta_{\bias}(v_k)|\lesssim p^{-1}\Delta_{\bias}(v_k),\ \ |\Delta_{\vari}(v) -\Delta_{\vari}(v_k)|\lesssim p^{-1}\Delta_{\vari}(v_k).
\end{align*}
Then a simple application of triangle inequality gives that the event
$$\Xi_2=\{\eqref{dicho_varbeta} \text{ holds simultaneously for all }-C_0\le v \le C_0\}$$
holds with overwhelming probability. On the other hand, on $\Xi_1$ it is easy to check that for any small constant $\e>0$, there exists a large enough constant $C_0>0$ depending on $\e$ such that
\begin{align*}
& |\delta_{\vari}(v) -\delta_{\vari}(C_0)|\le \e\delta_{\vari}(C_0),\quad |\delta_{\bias}(v) -\delta_{\bias}(C_0)|\le \e\delta_{\bias}(C_0), \quad  |\delta(v)-\delta(C_0)|\le \e\delta(C_0),\\
& |\Delta_{\bias}(v) -\Delta_{\bias}(C_0)|\le \e\Delta_{\bias}(C_0),\quad |\Delta_{\vari}(v) -\Delta_{\vari}(C_0)|\le \e\Delta_{\vari}(C_0),
\end{align*}
for all $v\ge C_0$. Similar estimates hold for $v\le -C_0$ if we replace $C_0$ with $-C_0$ in the above estimates. Together with the estimate at $\pm C_0$, we get that equation \eqref{dicho_varbeta} holds simultaneously for all $v\in \R$ on the overwhelming probability event $\Xi_1\cap \Xi_2$. This concludes the proof of Lemma \ref{thm_model_shift}, since $\wh v$ must be one of the real values.
\end{proof}



\paragraph{Notations.}
Define the index sets
$$\cal I_0:=\llbracket 1,p\rrbracket, \quad  \cal I_1:=\llbracket p+1,p+n_1\rrbracket, \quad \cal I_2:=\llbracket p+n_1+1,p+n_1+n_2\rrbracket ,\quad \cal I:=\cal I_0\cup \cal I_1\cup \cal I_2  .$$
 We will consistently use the latin letters $i,j\in\sI_{0}$ and greek letters $\mu,\nu\in\sI_{1}\cup \sI_{2}$. Correspondingly, the indices of the matrices $Z^{(1)}$ and $Z^{(2)}$ are labelled as
 \be\label{labelZ}
 Z^{(1)}= (Z^{(1)}_{\mu i}:i\in \mathcal I_0, \mu \in \mathcal I_1), \quad Z^{(2)}= (Z^{(2)}_{\nu i}:i\in \mathcal I_0, \nu \in \mathcal I_2).\ee

Moreover, we also define the following averaged resolvents, which are the (weighted) partial traces of $G$:
\be\label{defm}
\begin{split}
m(z) :=\frac1p\sum_{i\in \cal I_0} G_{ii}(z) ,\quad & m_0(z):=\frac1p\sum_{i\in \cal I_0} \lambda_i^2 G_{ii}(z),\\
 m_1(z):= \frac{1}{n_1}\sum_{\mu \in \cal I_1}G_{\mu\mu}(z) ,\quad & m_2(z):= \frac{1}{n_2}\sum_{\nu\in \cal I_2}G_{\nu\nu}(z) .
\end{split}
\ee
We will show that these averaged resolvents satisfy some deterministic self-consistent equations asymptotically, which will be the core part of the proof. We refer the reader to the discussions below \eqref{0self_Gii} and \eqref{0self_Gmu1}.

\subsection{Warmup: The Gaussian Case}
Now we give a heuristic derivation of the matrix limit when the entries of $Z^{(1)}$ and $Z^{(2)}$ are i.i.d. Gaussian. In this case,
%However, notice that if the entries of $(Z^{(1)})\equiv (Z^{(1)})^{\text{Gauss}}$ and $(Z^{(2)})\equiv (Z^{(2)})^{\text{Gauss}}$ are i.i.d. Gaussian, then
by the rotational invariance of the multivariate Gaussian distribution we have
\be\label{eq in Gauss} Z^{(1)} U\Lambda \stackrel{d}{=} Z^{(1)} \Lambda, \quad Z^{(2)} V \stackrel{d}{=} Z^{(2)},\ee
where ``$\stackrel{d}{=}$" means ``equal in distribution". Hence it suffices to consider the following resolvent
 \begin{equation} \label{resolv Gauss1}
   G(z)= \left( {\begin{array}{*{20}c}
   { -z\id_{p} } & n^{-1/2}\Lambda (Z^{(1)})^\top & n^{-1/2} (Z^{(2)})^\top  \\
   {n^{-1/2} Z^{(1)} \Lambda  } & {-\id_{n_1}} & 0 \\
   {n^{-1/2} Z^{(2)}} & 0 & {-\id_{n_2}}
   \end{array}} \right)^{-1}.
 \end{equation}

\medskip
\noindent{\bf Schur complements.}
%Now we discuss about how to obtain the matrix limit in equation \eqref{defn_piw}.
%\HZ{Divide into several parts like sec 7 to improve readability.}
The core quantities of the derivation are the following resolvent minors, which are defined by removing certain rows and columns of the matrix $H$.
\begin{definition}[Resolvent minors]\label{defn_Minor}
 For any $ (p+n)\times (p+n)$ matrix $\cal A$ and $a\in \mathcal I$, we define the minor $\cal A^{(a)}:=(\cal A_{a_1 a_2}:a_1, a_2 \in \mathcal I\setminus \{a\})$ as the $ (p+n-1)\times (p+n-1)$ matrix obtained by removing the $a$-th row and column in $\cal A$. Note that we keep the names of indices when defining $\cal A^{(a)}$, i.e. $(\cal A^{(a)})_{a_1a_2}= \cal A_{a_1 a_2}$ for $a_1,a_2\ne a$. Correspondingly, we define the resolvent minor for $G$ in equation \eqref{resolv Gauss1} as %(recall equation \eqref{green2})
\begin{align*}
G^{(\mathfrak c)}:&=\left[ \left( {\begin{array}{*{20}c}
   { -z\id_{p} } & n^{-1/2}\Lambda (Z^{(1)})^\top & n^{-1/2} (Z^{(2)})^\top  \\
   {n^{-1/2} Z^{(1)} \Lambda  } & {-\id_{n_1}} & 0 \\
   {n^{-1/2} Z^{(2)}} & 0 & {-\id_{n_2}}
   \end{array}} \right)^{(a)}\right]^{-1} ,
%= \left( {\begin{array}{*{20}c}
%   { \mathcal G^{(\mathbb T)}} & \mathcal G^{(\mathbb T)} W^{(\mathbb T)}  \\
%   {\left(W^{(\mathbb T)}\right)^\top\mathcal G^{(\mathbb T)}} & { \mathcal G_R^{(\mathbb T)} }  \\
%\end{array}} \right)  ,
\end{align*}
and define the partial traces $m^{(a)}$, $m_0^{(a)}$, $m_1^{(a)}$ and $m_2^{(a)}$ by replacing $G$ with $G^{(a)}$ in equation \eqref{defm}. We adopt the convention that for the resolvent minor $G^{(a)}$ defined as above, $G^{(a)}_{a_1a_2} = 0$ if $a_1 =a$ or $a_2=a$.
\end{definition}
%Note that the resolvent minor $G^{(\mathfrak c)}$ is defined such that it is independent of the entries in the $\mathfrak c$-th row and column of $H$. One will see a crucial use of this fact in the heuristic proof below.
 Using Schur complement formulas in equation (\ref{resolvent2}), we have that for $i \in \cal I_0$, %$\mu \in \cal I_1$ and $\nu\in \cal I_2$,
%\HZ{This part needs more explanation - cannot understand.}
\begin{align}
\frac{1}{{G_{ii} }}&=  - z - \frac{\lambda_i^2}{n} \sum_{\mu,\nu\in \mathcal I_1} Z^{(1)}_{\mu i}Z^{(1)}_{\nu i}G^{\left( i \right)}_{\mu\nu} - \frac{1}{n} \sum_{\mu,\nu\in \mathcal I_2} Z^{(2)}_{\mu i}Z^{(2)}_{\nu i}G^{\left( i \right)}_{\mu\nu} -2 \frac{\lambda_i}{n} \sum_{\mu\in \cal I_1,\nu\in \mathcal I_2} Z^{(1)}_{\mu i}Z^{(2)}_{\nu i}G^{\left( i \right)}_{\mu\nu} , \label{0self_Gii}
\end{align}
and for $\mu \in \cal I_1$ and $\nu\in \cal I_2$,
\begin{align}
\frac{1}{{G_{\mu\mu} }}&=  - 1 - \frac{1}{n} \sum_{i,j\in \mathcal I_0}\lambda_i \lambda_j Z^{(1)}_{\mu i}Z^{(1)}_{\mu j} G^{\left(\mu\right)}_{ij}, \quad \frac{1}{{G_{\nu\nu} }}=  - 1 - \frac{1}{n} \sum_{i,j\in \mathcal I_0}  Z^{(2)}_{\nu i}Z^{(2)}_{\nu j}  G^{\left(\nu\right)}_{ij},
\label{0self_Gmu1}
\end{align}
where we recall the notations in equation \eqref{labelZ}. To see why equation \eqref{0self_Gii} holds, we have that by Schur complement formula,
\begin{align*}
\frac{1}{G_{ii}}= -z - H_i G^{(i)}H_{i}^\top, \quad H_i = \left( \mathbf 0_{p-1}, (n^{-1/2}\lambda_i (Z^{(1)})^\top_{i\mu }:\mu \in \cal I_1),(n^{-1/2} (Z^{(2)})^\top_{i\nu }:\nu \in \cal I_2)\right),
\end{align*}
where $H_i$ is actually the $i$-th row of $H$ with the $(i,i)$-th entry removed. Expanding the above expression, we obtain equation \eqref{0self_Gii}. The two expressions in equation \eqref{0self_Gmu1} are easier to obtain using Schur complement formula.

\medskip
\noindent{\bf Concentration estimates.} Now for the right-hand side of equation \eqref{0self_Gii}, notice that the resolvent minor $G^{(i)}$ is defined such that it is independent of the entries $Z^{(1)}_{\mu i}$ and $Z^{(2)}_{\nu i}$. Hence by the concentration inequalities in Lemma \ref{largedeviation}, we have that the  right-hand side of equation \eqref{0self_Gii} concentrates around the partial expectation over the entries $\{Z^{(1)}_{\mu i}: \mu \in \cal I_1 \}\cup \{Z^{(2)}_{\nu i}: \nu \in \cal I_2\}$, i.e., with overwhelming probability,
\begin{align*}
\frac{1}{{G_{ii} }}&=  - z - \frac{\lambda_i^2}{n} \sum_{\mu \in \mathcal I_1}  G^{\left( i \right)}_{\mu\mu} - \frac{1}{n} \sum_{\mu\in \mathcal I_2} G^{\left( i \right)}_{\mu\mu} +\oo(1)= - z - \lambda_i^2 \frac{\rho_1}{\rho_1+\rho_2} m_1^{(i)}(z)-  \frac{\rho_2}{\rho_1+\rho_2} m_2^{(i)}(z)+\oo(1),
\end{align*}
where we used the definition of $m_1^{(i)}$ and $m_2^{(i)}$ in equation \eqref{defm} with $G$ replaced by $G^{(i)}$. Intuitively, since we have removed only one column and one row out of the $(p+n)$ columns and rows in $H$, $m_1^{(i)}$ and $m_2^{(i)}$ should be close to the original $m_1$ and $m_2$. Hence we obtain from the above equation that
\begin{align}\label{1self_Gii}
 G_{ii}  = -\left( z +\lambda_i^2 \frac{\rho_1}{\rho_1+\rho_2} m_1 +  \frac{\rho_2}{\rho_1+\rho_2}m_2+\oo(1)\right)^{-1}.
\end{align}
Similarly, we can obtain from equation \eqref{0self_Gmu1} that for $\mu \in \cal I_1$ and $\nu\in \cal I_2$,
\be\label{1self_Gmu} G_{\mu \mu }=-\left(1+\frac{p}{n} m_0 + \oo(1)\right)^{-1},\quad G_{\nu\nu}=-\left(1+\frac{p}{n} m+\oo(1)\right)^{-1},\ee
with overwhelming probability. The rigorous derivation of the above concentration estimates will be given in the proof of Lemma \ref{lemm_selfcons_weak}; see equations \eqref{self_Gii}-\eqref{erri}.

\medskip
\noindent\textbf{Deriving the self-consistent equations.} Now taking average of equation \eqref{1self_Gmu}, we obtain that
\be\label{2self_Gmu} m_1= \frac{1}{n_1}\sum_{\mu \in \cal I_1}G_{\mu\mu}=-\left(1+\frac{p}{n} m_0 + \oo(1)\right)^{-1},\quad m_2=\frac{1}{n_2}\sum_{\nu \in \cal I_2}G_{\nu\nu}=-\left(1+\frac{p}{n} m+\oo(1)\right)^{-1},
\ee
with overwhelming probability. As a byproduct, comparing equations \eqref{1self_Gmu} and \eqref{2self_Gmu}, we obtain that for $\mu \in \cal I_1$ and $\nu\in \cal I_2$,
\be\label{2.5self_Gmu} G_{\mu \mu }=m_1 +\oo(1),\quad G_{\nu\nu}=m_2+\oo(1),\ee
with overwhelming probability. Together with the definition of $m$ and $m_0$ in equation \eqref{defm}, the two equations in equation \eqref{2self_Gmu} give that
\be\label{3self_Gmu}  \frac1{m_1}= -1- \frac{1}{n} \sum_{i=1}^p \lambda_i^2 G_{ii}+ \oo(1),\quad \frac1{m_2}=-1-\frac{1}{n} \sum_{i=1}^p G_{ii}  + \oo(1),\ee
with overwhelming probability. Plugging equation \eqref{1self_Gii} into equation \eqref{3self_Gmu}, we obtain that
\be\label{approximate m1m2}
\begin{split}
& \frac1{m_1}= -1+ \frac{1}{n} \sum_{i=1}^p \frac{\lambda_i^2 }{ z +\lambda_i^2 \frac{\rho_1}{\rho_1+\rho_2} m_1(z) +  \frac{\rho_2}{\rho_1+\rho_2}m_2(z)+\oo(1)}+ \oo(1),\\
& \frac1{m_2}=-1+\frac{1}{n} \sum_{i=1}^p \frac{1}{ z +\lambda_i^2 \frac{\rho_1}{\rho_1+\rho_2} m_1(z) +  \frac{\rho_2}{\rho_1+\rho_2}m_2(z)+\oo(1)}  + \oo(1),
\end{split}
\ee
with overwhelming probability, which give a system of approximate self-consistent equations for $(m_1,m_2)$.
A rigorous derivation of these two equations will be given in the proof of Lemma \ref{lemm_selfcons_weak}. Compare \eqref{approximate m1m2} to the deterministic self-consistent equations in equation \eqref{selfomega_a}, one can observe that we should have
\be\label{approx m12 add}
(m_1,m_2) =\left(-\frac{\rho_1+\rho_2}{\rho_1}a_{1}(z),-\frac{\rho_1+\rho_2}{\rho_2}a_{2}(z)\right)+\oo(1) \quad \text{with overwhelming probability. }
\ee
To justify this identity rigorously, we need to know that the self-consistent equations are stable, that is, a small perturbation of the equations leads to a small perturbation of the solution. This will be given by Lemma \ref{lem_stabw}.


\medskip
\noindent\textbf{Deriving the matrix limit.}  Inserting the approximate identity \eqref{approx m12 add} into equations \eqref{1self_Gii} and \eqref{2.5self_Gmu}, we get that for  $i \in \cal I_0$, $\mu \in \cal I_1$ and $\nu\in \cal I_2$,
$$G_{ii}(z)=(-z +\lambda_i^2 a_{1}(z) + a_2(z)+\oo(1)^{-1},\quad G_{\mu\mu}=-\frac{\rho_1+\rho_2}{\rho_1}a_{1}(z)+\oo(1),\quad G_{\nu\nu}=-\frac{\rho_1+\rho_2}{\rho_2}a_{2}(z)+\oo(1),$$
with overwhelming probability. These explain the diagonal entries of $\Gi$ in equation \eqref{defn_piw}. For the off-diagonal entries, they are close to zero due to concentration. For example, for $i\ne j\in \cal I_1$, by Schur complement formula in equation (\ref{resolvent3}), we have
$$G_{ij}=-G_{ii}\Big({\lambda_i}{n^{-1/2}}\sum_{\mu \in \cal I_1} Z^{(1)}_{\mu i} G^{(i)}_{\mu j} + {n^{-1/2}}\sum_{\mu \in \cal I_2} Z^{(2)}_{\mu i} G^{(i)}_{\mu j} \Big).$$
Using Lemma \ref{largedeviation}, we can show that $n^{-1/2}\sum_{\mu \in \cal I_1} Z^{(1)}_{\mu i} G^{(i)}_{\mu j}$ and $n^{-1/2}\sum_{\mu \in \cal I_2} Z^{(2)}_{\mu i} G^{(i)}_{\mu j}$ are both close to zero. The other off-diagonal entries can be bounded in the same way. The bound on the off-diagonal entries will be proved rigorously in Lemma \ref{Z_lemma}.


\medskip
\noindent{\bf Deriving the bias asymptotics.} Finally, we describe how to derive the bias asymptotics. We can write the left-hand side of equation \eqref{lem_cov_derv_eq} using the derivative of $\cal G$ with respect to $z$ at $z=0$. More precisely, using equation \eqref{eigen2extra} we can obtain that
\begin{align}
&n^2\bignorm{\Sigma_2^{1/2} \bigbrace{ (X^{(1)})^{\top}X^{(1)} + (X^{(2)})^{\top}X^{(2)} }^{-1} \Sigma_1^{1/2} w}^2 \nonumber\\
&=n^2 w^\top \Sigma_1^{1/2}\Sigma_2^{-1/2}V\left(   \Lambda U^\top (Z^{(1)})^\top Z^{(1)} U\Lambda  + V^\top (Z^{(2)})^\top Z^{(2)}V\right)^{-2}V^\top\Sigma_2^{-1/2}\Sigma_1^{1/2}w \nonumber\\
&=  w^\top \Sigma_1^{1/2}\Sigma_2^{-1/2}V\cal G'(0)V^\top\Sigma_2^{-1/2}\Sigma_1^{1/2}w,\label{calculate G'}
\end{align}
where we used equation \eqref{rewrite X as R} in the second step. Since the matrix limit of $\cal G(z)$ is given by equation \eqref{matrix limit}, it is natural to guess that the matrix limit of $\cal G'(0)$ is given by
\be\label{cal G'0}\cal G'(0) \approx \left.\frac{\dd}{\dd z}\right|_{z=0}(-z\id_p+a_{1}(z)\Lambda^2 + a_{2}(z)\id_p)^{-1} = \frac{\id_p- a_1'(0)\Lambda^2 - a_2'(0)\id_p}{(a_{1}(0)\Lambda^2 + a_{2}(0)\id_p)^2}.\ee
If we let $a_3:=-a_1'(0)$ and $a_4:=-a_2'(0)$, then taking implicit differentiation of equation \eqref{selfomega_a} we can check that $(a_3,a_4)$ satisfies equation \eqref{eq_a34extra}. Then inserting \eqref{cal G'0} into \eqref{calculate G'}, we obtain that
\begin{align}
& n^2\bignorm{\Sigma_2^{1/2} \bigbrace{ (X^{(1)})^{\top}(X^{(1)}) + (X^{(2)})^{\top}(X^{(2)}) }^{-1} \Sigma_1^{1/2} w}^2 \nonumber\\
&\approx  w^\top \Sigma_1^{1/2}\Sigma_2^{-1/2}V\frac{a_3\Lambda^2 +(1+ a_4)\id_p}{(a_{1}\Lambda^2 + a_{2}\id_p)^2}V^\top \Sigma_2^{-1/2}\Sigma_1^{1/2}w= w^{\top} \Pi_\bias w, \label{calculatePibias}
\end{align}
where in the last step we used $M = \Sigma_1^{1/2}\Sigma_2^{-1/2}$ and $V \Lambda^2 V^\top=M^\top M$. This concludes equation \eqref{lem_cov_derv_eq}.
Note that in order to have the approximate identity for $\cal G'(0)$ in equation \eqref{cal G'0}, we not only need to know the asymptotics of $\cal G(0)$, but also need to know the asymptotics of $\cal G(z)$ for general $z$ around $z=0$. This is the main reason why we need to take a general $z$ in the definition of resolvents. The rigorous proof of equation \eqref{lem_cov_derv_eq} will be given in Section \ref{sec pf RMTlemma}, where we justify the approximate identity in equation \eqref{cal G'0}.
%In the above definition, we have taken the argument of $\cal G$ to be a general complex number, because we will need to use $\cal G'(0)$ in the proof of Lemma \ref{lem_cov_derivative}, which requires a good estimate of $\cal G(z)$ for $z$ around the origin.
%For the variance asymptotic limit, we study the resolvent
%	\[ R(z):= \bigbrace{\Sigma_2^{-1/2}( (X^{(1)})^{\top}(X^{(1)}) + (X^{(2)})^{\top}(X^{(2)}))\Sigma_2^{-1/2} - z \id}^{-1}, \text{ for any } z\in \C \text{ around } z=0. \]
%Using the techniques from \citet{Anisotropic} and \citet{yang2019spiked}, we find the asymptotic limit of $R(z)$ for any $z$ as $p$ goes to infinity, denoted by $R_\infty(z)$, with an almost optimal convergence rate of $p$.
%In particular, when $z=0$, the asymptotic limit of equation \eqref{lem_cov_shift_eq} is given by
%	\[ \tr[\Sigma_2^{-1/2} \Sigma \Sigma_2^{-1/2}R_\infty(0)]. \]
%
%For the bias asymptotic limit, we show in \todo{where?} that
%$$\bignorm{\Sigma_2^{1/2} ((X^{(1)})^{\top}(X^{(1)}) + (X^{(2)})^{\top}(X^{(2)})^{-1} w}^2= w^\top \Sigma_2^{-1/2}R'(0)\Sigma_2^{-1/2} w.$$
%Hence its limit can be calculated through $R_\infty'(z)$, which gives the expression in \eqref{lem_cov_derv_eq}.
%We leave the full proof of Lemma \ref{lem_cov_shift} and Lemma \ref{lem_cov_derivative} to Appendix \ref{sec_maintools}.
%Combining the above two results, we provide the proof of Theorem \ref{thm_main_informal} in Section \ref{app_proof_main_thm}.


The above arguments are the core of the main proof. To have a rigorous proof, we need to estimate each error carefully, and extend the Gaussian case to the more general case where the entries of $Z^{(1)}$ and $Z^{(2)}$ only satisfy certain moment assumptions. These will make the real argument rather tedious, but the methods we used are standard in the random matrix literature \cite{erdos2017dynamical,Anisotropic}. For the full rigorous proof, we refer the reader to Section \ref{appendix RMT}.
