\subsection{Warmup: Proof Overview for the Gaussian Case}
Now we give a heuristic derivation of the matrix limit when the entries of $Z^{(1)}$ and $Z^{(2)}$ are i.i.d. Gaussian. In this case,
%However, notice that if the entries of $(Z^{(1)})\equiv (Z^{(1)})^{\text{Gauss}}$ and $(Z^{(2)})\equiv (Z^{(2)})^{\text{Gauss}}$ are i.i.d. Gaussian, then
by the rotational invariance of the multivariate Gaussian distribution we have
\be\label{eq in Gauss} Z^{(1)} U\Lambda \stackrel{d}{=} Z^{(1)} \Lambda, \quad Z^{(2)} V \stackrel{d}{=} Z^{(2)},\ee
where ``$\stackrel{d}{=}$" means ``equal in distribution". Hence it suffices to consider the following resolvent
 \begin{equation} \label{resolv Gauss1}
   G(z)= \left( {\begin{array}{*{20}c}
   { -z\id_{p} } & n^{-1/2}\Lambda (Z^{(1)})^\top & n^{-1/2} (Z^{(2)})^\top  \\
   {n^{-1/2} Z^{(1)} \Lambda  } & {-\id_{n_1}} & 0 \\
   {n^{-1/2} Z^{(2)}} & 0 & {-\id_{n_2}}
   \end{array}} \right)^{-1}.
 \end{equation}

\medskip
\noindent{\bf Schur complements.}
%Now we discuss about how to obtain the matrix limit in equation \eqref{defn_piw}.
%\HZ{Divide into several parts like sec 7 to improve readability.}
The core quantities of the derivation are the following resolvent minors, which are defined by removing certain rows and columns of the matrix $H$.
\begin{definition}[Resolvent minors]\label{defn_Minor}
 For any $ (p+n)\times (p+n)$ matrix $\cal A$ and $a\in \mathcal I$, we define the minor $\cal A^{(a)}:=(\cal A_{a_1 a_2}:a_1, a_2 \in \mathcal I\setminus \{a\})$ as the $ (p+n-1)\times (p+n-1)$ matrix obtained by removing the $a$-th row and column in $\cal A$. Note that we keep the names of indices when defining $\cal A^{(a)}$, i.e. $(\cal A^{(a)})_{a_1a_2}= \cal A_{a_1 a_2}$ for $a_1,a_2\ne a$. Correspondingly, we define the resolvent minor for $G$ in equation \eqref{resolv Gauss1} as %(recall equation \eqref{green2})
\begin{align*}
G^{(\mathfrak c)}:&=\left[ \left( {\begin{array}{*{20}c}
   { -z\id_{p} } & n^{-1/2}\Lambda (Z^{(1)})^\top & n^{-1/2} (Z^{(2)})^\top  \\
   {n^{-1/2} Z^{(1)} \Lambda  } & {-\id_{n_1}} & 0 \\
   {n^{-1/2} Z^{(2)}} & 0 & {-\id_{n_2}}
   \end{array}} \right)^{(a)}\right]^{-1} ,
%= \left( {\begin{array}{*{20}c}
%   { \mathcal G^{(\mathbb T)}} & \mathcal G^{(\mathbb T)} W^{(\mathbb T)}  \\
%   {\left(W^{(\mathbb T)}\right)^\top\mathcal G^{(\mathbb T)}} & { \mathcal G_R^{(\mathbb T)} }  \\
%\end{array}} \right)  ,
\end{align*}
and define the partial traces $m^{(a)}$, $m_0^{(a)}$, $m_1^{(a)}$ and $m_2^{(a)}$ by replacing $G$ with $G^{(a)}$ in equation \eqref{defm}. We adopt the convention that for the resolvent minor $G^{(a)}$ defined as above, $G^{(a)}_{a_1a_2} = 0$ if $a_1 =a$ or $a_2=a$.
\end{definition}
%Note that the resolvent minor $G^{(\mathfrak c)}$ is defined such that it is independent of the entries in the $\mathfrak c$-th row and column of $H$. One will see a crucial use of this fact in the heuristic proof below.
 Using Schur complement formulas in equation (\ref{resolvent2}), we have that for $i \in \cal I_0$, %$\mu \in \cal I_1$ and $\nu\in \cal I_2$,
%\HZ{This part needs more explanation - cannot understand.}
\begin{align}
\frac{1}{{G_{ii} }}&=  - z - \frac{\lambda_i^2}{n} \sum_{\mu,\nu\in \mathcal I_1} Z^{(1)}_{\mu i}Z^{(1)}_{\nu i}G^{\left( i \right)}_{\mu\nu} - \frac{1}{n} \sum_{\mu,\nu\in \mathcal I_2} Z^{(2)}_{\mu i}Z^{(2)}_{\nu i}G^{\left( i \right)}_{\mu\nu} -2 \frac{\lambda_i}{n} \sum_{\mu\in \cal I_1,\nu\in \mathcal I_2} Z^{(1)}_{\mu i}Z^{(2)}_{\nu i}G^{\left( i \right)}_{\mu\nu} , \label{0self_Gii}
\end{align}
and for $\mu \in \cal I_1$ and $\nu\in \cal I_2$,
\begin{align}
\frac{1}{{G_{\mu\mu} }}&=  - 1 - \frac{1}{n} \sum_{i,j\in \mathcal I_0}\lambda_i \lambda_j Z^{(1)}_{\mu i}Z^{(1)}_{\mu j} G^{\left(\mu\right)}_{ij}, \quad \frac{1}{{G_{\nu\nu} }}=  - 1 - \frac{1}{n} \sum_{i,j\in \mathcal I_0}  Z^{(2)}_{\nu i}Z^{(2)}_{\nu j}  G^{\left(\nu\right)}_{ij},
\label{0self_Gmu1}
\end{align}
where we recall the notations in equation \eqref{labelZ}. To see why equation \eqref{0self_Gii} holds, we have that by Schur complement formula,
\begin{align*}
\frac{1}{G_{ii}}= -z - H_i G^{(i)}H_{i}^\top, \quad H_i = \left( \mathbf 0_{p-1}, (n^{-1/2}\lambda_i (Z^{(1)})^\top_{i\mu }:\mu \in \cal I_1),(n^{-1/2} (Z^{(2)})^\top_{i\nu }:\nu \in \cal I_2)\right),
\end{align*}
where $H_i$ is actually the $i$-th row of $H$ with the $(i,i)$-th entry removed. Expanding the above expression, we obtain equation \eqref{0self_Gii}. The two expressions in equation \eqref{0self_Gmu1} are easier to obtain using Schur complement formula.

\medskip
\noindent{\bf Concentration estimates.} Now for the right-hand side of equation \eqref{0self_Gii}, notice that the resolvent minor $G^{(i)}$ is defined such that it is independent of the entries $Z^{(1)}_{\mu i}$ and $Z^{(2)}_{\nu i}$. Hence by the concentration inequalities in Lemma \ref{largedeviation}, we have that the  right-hand side of equation \eqref{0self_Gii} concentrates around the partial expectation over the entries $\{Z^{(1)}_{\mu i}: \mu \in \cal I_1 \}\cup \{Z^{(2)}_{\nu i}: \nu \in \cal I_2\}$, i.e., with overwhelming probability,
\begin{align*}
\frac{1}{{G_{ii} }}&=  - z - \frac{\lambda_i^2}{n} \sum_{\mu \in \mathcal I_1}  G^{\left( i \right)}_{\mu\mu} - \frac{1}{n} \sum_{\mu\in \mathcal I_2} G^{\left( i \right)}_{\mu\mu} +\oo(1)= - z - \lambda_i^2 \frac{\rho_1}{\rho_1+\rho_2} m_1^{(i)}(z)-  \frac{\rho_2}{\rho_1+\rho_2} m_2^{(i)}(z)+\oo(1),
\end{align*}
where we used the definition of $m_1^{(i)}$ and $m_2^{(i)}$ in equation \eqref{defm} with $G$ replaced by $G^{(i)}$. Intuitively, since we have removed only one column and one row out of the $(p+n)$ columns and rows in $H$, $m_1^{(i)}$ and $m_2^{(i)}$ should be close to the original $m_1$ and $m_2$. Hence we obtain from the above equation that
\begin{align}\label{1self_Gii}
 G_{ii}  = -\left( z +\lambda_i^2 \frac{\rho_1}{\rho_1+\rho_2} m_1 +  \frac{\rho_2}{\rho_1+\rho_2}m_2+\oo(1)\right)^{-1}.
\end{align}
Similarly, we can obtain from equation \eqref{0self_Gmu1} that for $\mu \in \cal I_1$ and $\nu\in \cal I_2$,
\be\label{1self_Gmu} G_{\mu \mu }=-\left(1+\frac{p}{n} m_0 + \oo(1)\right)^{-1},\quad G_{\nu\nu}=-\left(1+\frac{p}{n} m+\oo(1)\right)^{-1},\ee
with overwhelming probability. The rigorous derivation of the above concentration estimates will be given in the proof of Lemma \ref{lemm_selfcons_weak}; see equations \eqref{self_Gii}-\eqref{erri}.

\medskip
\noindent\textbf{Deriving the self-consistent equations.} Now taking average of equation \eqref{1self_Gmu}, we obtain that
\be\label{2self_Gmu} m_1= \frac{1}{n_1}\sum_{\mu \in \cal I_1}G_{\mu\mu}=-\left(1+\frac{p}{n} m_0 + \oo(1)\right)^{-1},\quad m_2=\frac{1}{n_2}\sum_{\nu \in \cal I_2}G_{\nu\nu}=-\left(1+\frac{p}{n} m+\oo(1)\right)^{-1},
\ee
with overwhelming probability. As a byproduct, comparing equations \eqref{1self_Gmu} and \eqref{2self_Gmu}, we obtain that for $\mu \in \cal I_1$ and $\nu\in \cal I_2$,
\be\label{2.5self_Gmu} G_{\mu \mu }=m_1 +\oo(1),\quad G_{\nu\nu}=m_2+\oo(1),\ee
with overwhelming probability. Together with the definition of $m$ and $m_0$ in equation \eqref{defm}, the two equations in equation \eqref{2self_Gmu} give that
\be\label{3self_Gmu}  \frac1{m_1}= -1- \frac{1}{n} \sum_{i=1}^p \lambda_i^2 G_{ii}+ \oo(1),\quad \frac1{m_2}=-1-\frac{1}{n} \sum_{i=1}^p G_{ii}  + \oo(1),\ee
with overwhelming probability. Plugging equation \eqref{1self_Gii} into equation \eqref{3self_Gmu}, we obtain that
\be\label{approximate m1m2}
\begin{split}
& \frac1{m_1}= -1+ \frac{1}{n} \sum_{i=1}^p \frac{\lambda_i^2 }{ z +\lambda_i^2 \frac{\rho_1}{\rho_1+\rho_2} m_1(z) +  \frac{\rho_2}{\rho_1+\rho_2}m_2(z)+\oo(1)}+ \oo(1),\\
& \frac1{m_2}=-1+\frac{1}{n} \sum_{i=1}^p \frac{1}{ z +\lambda_i^2 \frac{\rho_1}{\rho_1+\rho_2} m_1(z) +  \frac{\rho_2}{\rho_1+\rho_2}m_2(z)+\oo(1)}  + \oo(1),
\end{split}
\ee
with overwhelming probability, which give a system of approximate self-consistent equations for $(m_1,m_2)$.
A rigorous derivation of these two equations will be given in the proof of Lemma \ref{lemm_selfcons_weak}. Compare \eqref{approximate m1m2} to the deterministic self-consistent equations in equation \eqref{selfomega_a}, one can observe that we should have
\be\label{approx m12 add}
(m_1,m_2) =\left(-\frac{\rho_1+\rho_2}{\rho_1}a_{1}(z),-\frac{\rho_1+\rho_2}{\rho_2}a_{2}(z)\right)+\oo(1) \quad \text{with overwhelming probability. }
\ee
To justify this identity rigorously, we need to know that the self-consistent equations are stable, that is, a small perturbation of the equations leads to a small perturbation of the solution. This will be given by Lemma \ref{lem_stabw}.


\medskip
\noindent\textbf{Deriving the matrix limit.}  Inserting the approximate identity \eqref{approx m12 add} into equations \eqref{1self_Gii} and \eqref{2.5self_Gmu}, we get that for  $i \in \cal I_0$, $\mu \in \cal I_1$ and $\nu\in \cal I_2$,
$$G_{ii}(z)=(-z +\lambda_i^2 a_{1}(z) + a_2(z)+\oo(1)^{-1},\quad G_{\mu\mu}=-\frac{\rho_1+\rho_2}{\rho_1}a_{1}(z)+\oo(1),\quad G_{\nu\nu}=-\frac{\rho_1+\rho_2}{\rho_2}a_{2}(z)+\oo(1),$$
with overwhelming probability. These explain the diagonal entries of $\Gi$ in equation \eqref{defn_piw}. For the off-diagonal entries, they are close to zero due to concentration. For example, for $i\ne j\in \cal I_1$, by Schur complement formula in equation (\ref{resolvent3}), we have
$$G_{ij}=-G_{ii}\Big({\lambda_i}{n^{-1/2}}\sum_{\mu \in \cal I_1} Z^{(1)}_{\mu i} G^{(i)}_{\mu j} + {n^{-1/2}}\sum_{\mu \in \cal I_2} Z^{(2)}_{\mu i} G^{(i)}_{\mu j} \Big).$$
Using Lemma \ref{largedeviation}, we can show that $n^{-1/2}\sum_{\mu \in \cal I_1} Z^{(1)}_{\mu i} G^{(i)}_{\mu j}$ and $n^{-1/2}\sum_{\mu \in \cal I_2} Z^{(2)}_{\mu i} G^{(i)}_{\mu j}$ are both close to zero. The other off-diagonal entries can be bounded in the same way. The bound on the off-diagonal entries will be proved rigorously in Lemma \ref{Z_lemma}.


\medskip
\noindent{\bf Deriving the bias asymptotics.} Finally, we describe how to derive the bias asymptotics. We can write the left-hand side of equation \eqref{lem_cov_derv_eq} using the derivative of $\cal G$ with respect to $z$ at $z=0$. More precisely, using equation \eqref{eigen2extra} we can obtain that
\begin{align}
&n^2\bignorm{\Sigma_2^{1/2} \bigbrace{ (X^{(1)})^{\top}X^{(1)} + (X^{(2)})^{\top}X^{(2)} }^{-1} \Sigma_1^{1/2} w}^2 \nonumber\\
&=n^2 w^\top \Sigma_1^{1/2}\Sigma_2^{-1/2}V\left(   \Lambda U^\top (Z^{(1)})^\top Z^{(1)} U\Lambda  + V^\top (Z^{(2)})^\top Z^{(2)}V\right)^{-2}V^\top\Sigma_2^{-1/2}\Sigma_1^{1/2}w \nonumber\\
&=  w^\top \Sigma_1^{1/2}\Sigma_2^{-1/2}V\cal G'(0)V^\top\Sigma_2^{-1/2}\Sigma_1^{1/2}w,\label{calculate G'}
\end{align}
where we used equation \eqref{rewrite X as R} in the second step. Since the matrix limit of $\cal G(z)$ is given by equation \eqref{matrix limit}, it is natural to guess that the matrix limit of $\cal G'(0)$ is given by
\be\label{cal G'0}\cal G'(0) \approx \left.\frac{\dd}{\dd z}\right|_{z=0}(-z\id_p+a_{1}(z)\Lambda^2 + a_{2}(z)\id_p)^{-1} = \frac{\id_p- a_1'(0)\Lambda^2 - a_2'(0)\id_p}{(a_{1}(0)\Lambda^2 + a_{2}(0)\id_p)^2}.\ee
If we let $a_3:=-a_1'(0)$ and $a_4:=-a_2'(0)$, then taking implicit differentiation of equation \eqref{selfomega_a} we can check that $(a_3,a_4)$ satisfies equation \eqref{eq_a34extra}. Then inserting \eqref{cal G'0} into \eqref{calculate G'}, we obtain that
\begin{align}
& n^2\bignorm{\Sigma_2^{1/2} \bigbrace{ (X^{(1)})^{\top}(X^{(1)}) + (X^{(2)})^{\top}(X^{(2)}) }^{-1} \Sigma_1^{1/2} w}^2 \nonumber\\
&\approx  w^\top \Sigma_1^{1/2}\Sigma_2^{-1/2}V\frac{a_3\Lambda^2 +(1+ a_4)\id_p}{(a_{1}\Lambda^2 + a_{2}\id_p)^2}V^\top \Sigma_2^{-1/2}\Sigma_1^{1/2}w= w^{\top} \Pi_\bias w, \label{calculatePibias}
\end{align}
where in the last step we used $M = \Sigma_1^{1/2}\Sigma_2^{-1/2}$ and $V \Lambda^2 V^\top=M^\top M$. This concludes equation \eqref{lem_cov_derv_eq}.
Note that in order to have the approximate identity for $\cal G'(0)$ in equation \eqref{cal G'0}, we not only need to know the asymptotics of $\cal G(0)$, but also need to know the asymptotics of $\cal G(z)$ for general $z$ around $z=0$. This is the main reason why we need to take a general $z$ in the definition of resolvents. The rigorous proof of equation \eqref{lem_cov_derv_eq} will be given in Section \ref{sec pf RMTlemma}, where we justify the approximate identity in equation \eqref{cal G'0}.
%In the above definition, we have taken the argument of $\cal G$ to be a general complex number, because we will need to use $\cal G'(0)$ in the proof of Lemma \ref{lem_cov_derivative}, which requires a good estimate of $\cal G(z)$ for $z$ around the origin.
%For the variance asymptotic limit, we study the resolvent
%	\[ R(z):= \bigbrace{\Sigma_2^{-1/2}( (X^{(1)})^{\top}(X^{(1)}) + (X^{(2)})^{\top}(X^{(2)}))\Sigma_2^{-1/2} - z \id}^{-1}, \text{ for any } z\in \C \text{ around } z=0. \]
%Using the techniques from \citet{Anisotropic} and \citet{yang2019spiked}, we find the asymptotic limit of $R(z)$ for any $z$ as $p$ goes to infinity, denoted by $R_\infty(z)$, with an almost optimal convergence rate of $p$.
%In particular, when $z=0$, the asymptotic limit of equation \eqref{lem_cov_shift_eq} is given by
%	\[ \tr[\Sigma_2^{-1/2} \Sigma \Sigma_2^{-1/2}R_\infty(0)]. \]
%
%For the bias asymptotic limit, we show in \todo{where?} that
%$$\bignorm{\Sigma_2^{1/2} ((X^{(1)})^{\top}(X^{(1)}) + (X^{(2)})^{\top}(X^{(2)})^{-1} w}^2= w^\top \Sigma_2^{-1/2}R'(0)\Sigma_2^{-1/2} w.$$
%Hence its limit can be calculated through $R_\infty'(z)$, which gives the expression in \eqref{lem_cov_derv_eq}.
%We leave the full proof of Lemma \ref{lem_cov_shift} and Lemma \ref{lem_cov_derivative} to Appendix \ref{sec_maintools}.
%Combining the above two results, we provide the proof of Theorem \ref{thm_main_informal} in Section \ref{app_proof_main_thm}.


The above arguments are the core of the main proof. To have a rigorous proof, we need to estimate each error carefully, and extend the Gaussian case to the more general case where the entries of $Z^{(1)}$ and $Z^{(2)}$ only satisfy certain moment assumptions. These will make the real argument rather tedious, but the methods we used are standard in the random matrix literature \cite{erdos2017dynamical,Anisotropic}. For the full rigorous proof, we refer the reader to Section \ref{appendix RMT}.
