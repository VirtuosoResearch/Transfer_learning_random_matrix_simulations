%We shall refer to random matrices of the form $X^\top X$ as sample covariance matrices following the standard notations in high-dimensional statistics. The second lemma extends Lemma \ref{lem_minv} for a single sample covariance matrix to the sum of two independent sample covariance matrices. It is the main random matrix theoretical input of this paper.
%%which deals with the inverse of the sum of two random matrices, which
%%any is can be viewed as a special case of Theorem \ref{thm_model_shift}.
%
%\begin{lemma}[Variance bound: Lemma \ref{lem_cov_shift_informal} restated]\label{lem_cov_shift}
%	%Let $X_i\in\real^{n_i\times p}$ be a random matrix that contains i.i.d. row vectors with mean $0$ and variance $\Sigma_i\in\real^{p\times p}$, for $i = 1, 2$.
%	Suppose $X_1=Z_1\Sigma_1^{1/2}\in \R^{n_1\times p}$ and $X_2=Z_2\Sigma_2^{1/2}\in \R^{n_2\times p}$ satisfy Assumption \ref{assm_secA1} with $\rho_1:=n_1/p>1$ and $\rho_2:=n_2/p>1$ being fixed constants.
%	Denote by $M = \Sigma_1^{1/2}\Sigma_2^{-1/2}$ and let $\lambda_1, \lambda_2, \dots, \lambda_p$ be the singular values of $M$ in descending order. Let $A$ be any $p\times p$ matrix that is independent of $X_1$ and $X_2$. We have that for any constant $\e>0$,
%	%When $n_1 = c_1 p$ and $n_2 = c_2 p$, we have that with high probability over the randomness of $X_1$ and $X_2$, the following equation holds
%	\begin{align}\label{lem_cov_shift_eq}
%		\bigtr{(X_1^{\top}X_1 + X_2^{\top}X_2)^{-1}A} = \frac{1}{\rho_1+\rho_2}\frac1p\bigtr{ (a_1 \Sigma_1 + a_2\Sigma_2)^{-1} A} +\bigo{\|A\| p^{-1/2+\epsilon}}
%	\end{align}
%with high probability, where $(a_1, a_2)$ is the solution to the following deterministic equations:
%	\begin{align}
%		a_1 + a_2 = 1- \frac{1}{\rho_1 + \rho_2},\quad a_1 + \frac1{\rho_1 + \rho_2}\cdot \frac{1}{p}\sum_{i=1}^p \frac{\lambda_i^2 a_1}{\lambda_i^2 a_1 + a_2} = \frac{\rho_1}{\rho_1 + \rho_2}. \label{eq_a12extra}
%	\end{align}
%\end{lemma}
%
%Finally, the last lemma describes the asymptotic limit of $(X_1^{\top}X_1 + X_2^{\top}X_2)^{-1}\Sigma_2(X_1^{\top}X_1 + X_2^{\top}X_2)^{-1}$, which will be needed when we estimate the first term on the right-hand side of \eqref{eq_te_mtl_2task}.


%We will give the proof of this lemma in Section \ref{sec_maintools}.






%\subsection{Proof for Two Tasks with General Covariates}\label{app_proof_main}

%To illustrate the idea, we observe that by using Lemma \ref{lem_minv}, we have that
%\[ \te(\hat{\beta}_t^{\STL}) = \frac{\sigma^2}{n_2 - p}\bigtr{\Sigma_2^{-1}}. \]
%We shall also derive the limit of $\te(\hat{\beta}_t^{\MTL})$.
%\todo{write a brief technical overview}

%We shall consider the case where the entries of $\e_1$ and $\e_2$ have the same variance $\sigma_1^2=\sigma_2^2=\sigma^2$.
%Moreover, we shall denote $\beta_1$ and $\beta_2$ by $\beta_1$ and $\beta_2$, standing for ``source task" and ``target task", respectively.
%First, we introduce several quantities that will be used in our statement, and they are also related to the quantities in Lemma \ref{lem_cov_shift} and Lemma \ref{lem_cov_derivative}. Given the optimal ratio $\hat v$,
%%let $\hat{M} = \hat{v} \Sigma_1^{1/2}\Sigma_2^{-1/2}$ denote the weighted covariate shift matrix, and ${\hat\lambda}_1\ge {\hat\lambda}_2 \ge \dots \ge {\hat\lambda}_p$ be the eigenvalues of $\hat{M}^{\top}\hat{M}$. Define
%let $(\hat a_1, \hat a_2)$ be the solution to the following system of deterministic equations,
%	\be
%		 \hat a_1 +  \hat a_2 = 1- \frac{1}{\rho_1 + \rho_2},\quad  \hat a_1 + \frac1{\rho_1 + \rho_2}\cdot \frac1p\sum_{i=1}^p \frac{ \hat v^2 \lambda_i^2 \hat a_1}{ \hat v^2 \lambda_i^2\hat a_1 +  \hat a_2} = \frac{\rho_1}{\rho_1 + \rho_2}.\label{eq_a2} \\
%		 \ee
%		 After obtaining $(\hat a_1,\hat a_2)$, we can solve the following linear equations to get $(\hat a_3,\hat a_4)$:
%\begin{gather}
%		\left(\rho_2 \hat a_2^{-2}- \hat b_0\right)\cdot \hat  a_3 - \hat b_1 \cdot \hat a_4
%		=\hat b_0, \quad \left(\rho_1\hat a_1^{-2} - \hat b_2  \right)\cdot \hat a_4 - \hat b_1 \cdot \hat a_3 =\hat b_1 .\label{eq_a3}
%%		\left(\frac{n_1}{\hat a_1^2} -  \sum_{i=1}^p \frac{\hat \lambda_i^4   }{  (\hat a_2 + \hat \lambda_i^2\hat a_1)^2  }\right)\hat a_4 -\left(\sum_{i=1}^p \frac{\hat \lambda_i^2  }{  (\hat a_2 + \hat \lambda_i^2\hat a_1)^2  }\right)\hat a_3
%%		= \sum_{i=1}^p \frac{\hat \lambda_i^2 }{  (\hat a_2 + \hat \lambda_i^2\hat a_1)^2  }. \label{eq_a4}
%	\end{gather}
%where we denoted
%$$\hat b_k:= \frac1{p}\sum_{i=1}^p \frac{\hat v^{2k} \lambda_i^{2k}}{ (\hat a_2 +\hat v^2 \lambda_i^2\hat a_1)^2  },\quad k=0,1,2.$$


%For our purpose, a validation set that is much larger than the number of output layer parameters $r\cdot t$ suffices (e.g. $\rho_i \cdot p^{0.5}$).
%The size of the training set is then $\rho_i (p - p^{0.99})$.
%The advantage of tuning the output layers on the validation set is to reduce the effect of noise from $\hat{B}$.
%For more details, we refer the reader to Appendix \ref{app_proof_sec3}.

%The single-task estimator $\hat{\beta}_t^{\STL}$ is given by $(X_t^{\top}X_t)^{-1}X_t^{\top}Y_t$.
%For an estimator $\hat{\beta}\in\real^p$, we define the out-of-sample prediction loss as

%which can be further decomposed as the bias plus the variance of $\hat{\beta}$.
%In order to relate the btradeoff to properties of the data, we need tight concentration bounds for the bias and variance.
%We consider the high-dimensional regime where $n_i$ is a fixed constant $\rho_i > 1$  multiple of $p$ for every $1\le i\le t$, and $p$ is sufficiently large.
%We focus on a setting where $\rho_t$ is small compared to $\set{\rho_i}_{i=1}^{t-1}$.
%This setting captures the need to add more labeled data to reduce the prediction loss of the target task.

%However, this result only applies to a single task.
%Therefore, our goal is to extend this result to multiple tasks.

%\textbf{Notations.}
%When there is no ambiguity, we drop the subscript $t$ from $\te_t(\hat{\beta}_t^{\MTL})$ and write $\te(\hat{\beta}_t^{\MTL})$ for simplicity.
%We refer to the first task as the source task when there are only two tasks.
%We call $M = \Sigma_1^{1/2}\Sigma_2^{-1/2}$ the covariate shift matrix.


%\subsection{Analyzing the Tradeoff via Random Matrix Theory}


\iffalse
First, we give the basic assumption for our main objects---the random matrices $X_i$, $i=1,2$.

\begin{assumption}[Moment assumptions]%\label{assm_secA1}
We will consider $n\times p$ random matrices of the form $X=Z\Sigma^{1/2}$, where $\Sigma$  is a $p\times p$ deterministic positive definite symmetric matrix, and $Z=(z_{ij})$ is an $n\times p$ random matrix with real i.i.d. entries with mean zero and variance one. Note that the rows of $X$ are i.i.d. centered random vectors with covariance matrix $\Sigma$. For simplicity, we assume that all the moments of $z_{ij}$ exists, that is, for any fixed $k\in \N$, there exists a constant $C_k>0$ such that
\begin{equation}\label{assmAhigh}
\mathbb{E} |z_{ij}|^k \le C_k ,\quad 1\le i \le n, \ \ 1\le j \le p.
\end{equation}
 We assume that $n=\rho p$ for some fixed constant $\rho>1$. Without loss of generality, after a rescaling we can assume that the norm of $\Sigma$ is bounded by a constant $C>0$. Moreover, we assume that $\Sigma$ is well-conditioned: $\kappa(\Sigma)\le C$, where $\kappa(\cdot)$ denotes the condition number.
\end{assumption}
Here we have assumed \eqref{assmAhigh} solely for simplicity of representation. If the entries of $Z$ only have finite $a$-th moment for some $a>4$, then all the results below still hold except that we need to replace $\OO(p^{-\frac12+\e})$ with $\OO( p^{-\frac12+\frac2a +\epsilon})$ in some error bounds.
We will not get deeper into this issue in this section, but refer the reader to Corollary \ref{main_cor} in Section \ref{sec locallaw1}.
\fi

%As an example, for the setting of two tasks, we can decompose $L(\hat{\beta}_t^{\MTL}) - L(\hat{\beta}_t^{\STL})$ into a bias term and a variance term as follows .
%Recall that $\hat{\beta}_t^{\MTL}$ is defined as $BW_t$ after solving equation \eqref{eq_mtl}.
%We can  the test error of $\hat{\beta}_{t}^{\MTL}$ on the target task into two parts as follows.
%{\small\begin{align}
%	\te(\hat{\beta}_t^{\MTL}) - L(\hat{\beta}_t^{\STL}) =& ~ \hat{v}^2 \bignorm{\Sigma_2^{1/2} (\hat{v}^2 X_1^{\top}X_1 + X_2^{\top}X_2)^{-1}X_1^{\top}X_1 (\beta_1 - \hat{v}\beta_2)}^2 \label{eq_te_model_shift} \\
%	+&~ \sigma^2 \bigbrace{\bigtr{(\hat{v}^2 X_1^{\top}X_1 + X_2^{\top}X_2)^{-1}\Sigma_2} - \bigtr{(X_2^{\top}X_2)^{-1}\Sigma_2}}. \label{eq_te_var}
%\end{align}}%
%In the above, $\hat{v} = W_1 / W_2$ where $W_1, W_2$ are obtained from solving equation \eqref{eq_mtl_eval} (recalling that $W_1, W_2$ are scalars for two tasks).
%The role of $\hat{v}$ is to scale the shared subspace $B$ to fit each task.

%Hence, the bias term introduces a negative effect that depends on the \textit{similarity} between $\beta_1$ and $\beta_2$.
%Intuitively, the more \textit{samples} we have, the smaller the variance is.
%Meanwhile, \textit{covariate shift} also affects how small the variance can be.

%Then we make the following assumptions on the data models.
%\begin{assumption}[Linear regression model]\label{assm_secA2}
%For some fixed $t\in \N$, let $Y_i = X_i\beta_i + \varepsilon_i$, $1\le i \le t$, be independent data models, where $X_i$, $\beta_i$ and $\varepsilon_i$ are also independent of each other. Suppose that $X_i=Z_i\Sigma_i^{1/2}\in \R^{n_i\times p}$ satisfy Assumption \ref{assm_secA1} with $\rho_i:=n_i/p>1$ being fixed constants.
%$\e_i\in \R^{n_i}$ are random vectors with i.i.d. entries with mean zero, variance $\sigma_i^2$ and all moments as in \eqref{assmAhigh}.
%\end{assumption}


%As a remark, since the spectral norm of $U_r$ is less than $1$, we have that $\norm{U_r(i)} < 1$ for all $1 \le i \le t$. Compared to Theorem \ref{thm_main_informal}, we can get a simple expression for the two functions $\Delta_{vari}$ and $\Delta_{\bias}$. The proof of Theorem \ref{thm_many_tasks} can be found in Appendix \ref{app_proof_many_tasks}.
%In this section we consider the setting with $k$ many that have the same covariates.
%Since every task has the same number of data points as well as the same covariance, the only differences between different tasks are their models $\set{\beta_i}_{i=1}^k$.
%For this setting, we derive solutions for the multi-task training and the transfer learning setting that match our insights qualitatively from Section \ref{sec_denoise}.
%Using Lemma \ref{lem_minv} and some concentration bounds, we can complete the proof of Theorem \ref{thm_many_tasks}.

\iffalse
Next, we derive a closed-form solution of the multi-task learning estimator for the case of two tasks.
From \cite{WZR20}, we know that we need to explicitly restrict the output dimension $r$ of $B$ so that there is transfer between the two tasks.
Hence for the case of two tasks, we consider the setting where $r=1$.
For simplicity of notations, we shall denote $(X_i^{tr},Y_i^{tr})$ and $(X_i^{val},Y_i^{val})$ as $(X_i,Y_i)$ and  $(\wt X_i,\wt Y_i)$, respectively. Then equation \eqref{eq_mtl} simplifies to
\begin{align}\label{eq_mtl_2task}
	f(B; w_1, w_2) = \bignorm{X_1 B w_1 - Y_1}^2 + \bignorm{X_2 B w_2 - Y_2}^2,
\end{align}
where $B\in\real^p$ and $w_1, w_2$ are both real numbers. To solve the above problem, suppose that $w_1, w_2$ are fixed, by local optimality, we find the optimal $B$ as
\begin{align}
	& \hat{B}(w_1, w_2) = (w_1^2 X_1^{\top}X_1 + w_2^2 X_2^{\top}X_2)^{-1} (w_1 X_1^{\top}Y_1 + w_2 X_2^{\top}Y_2) \label{hatB}\\
	&= \frac{1}{w_2} \left( \frac{w_1^2}{w_2^2}  X_1^{\top}X_1 + X_2^{\top}X_2\right)^{-1} \left(\frac{w_1}{w_2} X_1^{\top}Y_1 + X_2^{\top}Y_2\right) \nonumber\\
	&= \frac{1}{w_2}\left[\beta_2 + \left(\frac{w_1^2}{w_2^2} X_1^{\top}X_1 + X_2^{\top}X_2\right)^{-1}\bigbrace{X_1^{\top}X_1\left(\frac{w_1}{w_2}\beta_1 - \frac{w_1^2}{w_2^2} \beta_2\right) + \left(\frac{w_1}{w_2} X_1^{\top}\varepsilon_1 + X_2^{\top}\varepsilon_2\right)}\right]. \nonumber
\end{align}
As a remark, when $w_1 = w_2 = 1$, we obtain linear regression.
If $\beta_1$ is a scaling of $\beta_2$, then  $w_1, w_2$ can be scaled accordingly to fix both tasks more accurately than linear regression.

%For the discussions below, we assume that the entries of $\e_1$ and $\e_2$ all have the same variance $\sigma^2$. This holds for most parts of our discussion, except in Proposition \ref{prop_var_transition}. We will derive different expressions for the validation loss and the test error

Next we consider $N_i$ independent samples of the training set $\{(\wt x_k^{(i)},\wt y_k^{(i)}): 1\le k \le N_i\}$ from task-$i$, $i=1,2$. With these sample, we form the random matrices $\wt X_i \in \R^{N_i\times p}$ and $\wt Y_i\in \R^{N_i}$, $i=1,2,$ whose row vectors are given by $\wt x_k^{(i)}$ and $\wt y_k^{(i)}$. We assume that $N_1$ and $N_2$ satisfy $N_1/N_2=n_1/n_2$ and $N_i \ge n_i^{1-\e_0}$ for some constant $\e_0>0$. Then we write the validation loss in \eqref{eq_mtl_eval} as
\begin{align}\label{eq_mtl_2tasktilde}
	g(w_1,w_2) = \bignorm{\wt X_1 \hat B w_1 - \wt Y_1}^2 + \bignorm{\wt X_2 \hat B w_2 - \wt Y_2}^2.
\end{align}
Inserting \eqref{hatB} into \eqref{eq_mtl_2tasktilde}, one can see that the optimal solution of $g$ only depends on the ratio $v:=w_1/w_2$.
Hence we overload the notation by writing $g(v)$ in the following discussion.
The expectation of $g(v)$ can be written as follows.
\begin{align}
		\val(v) \define& \exarg{\varepsilon_1,\e_2} {\sum_{i=1}^2 \left\|\Sigma_i^{1/2}( \hat B w_i - \beta_i) \right\|^2} \nonumber\\
	=&  N_1 \cdot \bignorm{\Sigma_1^{1/2}\left(v^2 X_1^{\top}X_1 + X_2^{\top}X_2\right)^{-1}X_2^{\top}X_2\left (\beta_1 - v\beta_2\right)}^2 \nonumber \\
	&+ N_2 \cdot v^2\bignorm{\Sigma_2^{1/2}\left(v^2 X_1^{\top}X_1 + X_2^{\top}X_2\right)^{-1}X_1^{\top}X_1\left(\beta_1 - v\beta_2\right)}^2 \nonumber \\
		&+ N_1   \cdot v^2 \bigtr{\Sigma_1\left(v^2 X_1^{\top}X_1 + X_2^{\top}X_2\right)^{-2} \left(\sigma_1^2 \cdot v^2X_1^{\top}X_1 + \sigma_2^2 \cdot X_2^{\top}X_2\right)} \nonumber \\
		&+ N_2  \cdot \bigtr{\Sigma_2\left(v^2 X_1^{\top}X_1 + X_2^{\top}X_2\right)^{-2} \left(\sigma_1^2 \cdot v^2  X_1^{\top}X_1 + \sigma_2^2  \cdot X_2^{\top}X_2\right)}. \label{revise_eq_val_mtl}
\end{align}

{\color{red}\begin{align*}
			& f(W_1, W_2) = \bignorm{X_1 \hat B w_1 - Y_1}^2 + \bignorm{X_2 \hat B w_2 - Y_2}^2\\
			& =\bignorm{X_1\left( v^2 X_1^{\top}X_1 + X_2^{\top}X_2\right)^{-1} \left(v^2 X_1^{\top}Y_1 + vX_2^{\top}Y_2\right) - Y_1}^2 \\
			&+ \bignorm{X_2 \left( v^2 X_1^{\top}X_1 + X_2^{\top}X_2\right)^{-1} \left(vX_1^{\top}Y_1 + X_2^{\top}Y_2\right) - Y_2}^2 \\
			& =\bignorm{X_1\left( v^2 X_1^{\top}X_1 + X_2^{\top}X_2\right)^{-1} \left(v^2 X_1^{\top}\e_1 + vX_2^{\top}\e_2\right) - \e_1 + X_1\left( v^2 X_1^{\top}X_1 + X_2^{\top}X_2\right)^{-1}  X_2^{\top}X_2(v\beta_2-\beta_1) }^2 \\
			&+ \bignorm{X_2 \left( v^2 X_1^{\top}X_1 + X_2^{\top}X_2\right)^{-1} \left(vX_1^{\top}\e_1 + X_2^{\top}\e_2\right) - \e_2 + vX_2 \left( v^2 X_1^{\top}X_1 + X_2^{\top}X_2\right)^{-1}  X_1^{\top}X_1(\beta_1-v\beta_2) }^2 \\
			&=\val(v)\cdot \left( 1+\OO(p^{-1/2\e})\right) \quad \text{w.h.p.},
		\end{align*}
		where
		\begin{align*}
		\val(v)&=\bignorm{X_1\left( v^2 X_1^{\top}X_1 + X_2^{\top}X_2\right)^{-1}  X_2^{\top}X_2(v\beta_2-\beta_1) }^2 \\
			&+ v^2\bignorm{X_2 \left( v^2 X_1^{\top}X_1 + X_2^{\top}X_2\right)^{-1}  X_1^{\top}X_1(\beta_1-v\beta_2) }^2 \\
			&+\sigma^2 \tr\left(v^2 X_1\left( v^2 X_1^{\top}X_1 + X_2^{\top}X_2\right)^{-1}   X_1^{\top} -\id\right)^2 \\
			&+\sigma^2 \tr\left(v^2 X_1\left( v^2 X_1^{\top}X_1 + X_2^{\top}X_2\right)^{-1}X_2^{\top}X_2 \left( v^2 X_1^{\top}X_1 + X_2^{\top}X_2\right)^{-1} X_1^\top\right)\\
			&+\sigma^2 \tr\left(X_2 \left( v^2 X_1^{\top}X_1 + X_2^{\top}X_2\right)^{-1} X_2^{\top} -\id\right)^2 \\
			&+\sigma^2 \tr\left(v^2 X_2\left( v^2 X_1^{\top}X_1 + X_2^{\top}X_2\right)^{-1}X_1^{\top}X_1 \left( v^2 X_1^{\top}X_1 + X_2^{\top}X_2\right)^{-1} X_2^\top\right)\\
			&=\bignorm{X_1\left( v^2 X_1^{\top}X_1 + X_2^{\top}X_2\right)^{-1}  X_2^{\top}X_2(v\beta_2-\beta_1) }^2 \\
			&+ v^2\bignorm{X_2 \left( v^2 X_1^{\top}X_1 + X_2^{\top}X_2\right)^{-1}  X_1^{\top}X_1(\beta_1-v\beta_2) }^2 +(n_1+n_2-p)\sigma^2.
		\end{align*}
	}





Hence to minimize $g(v)$, it suffices to minimize $\val(v)$ over $v$.
Let $\hat v=\hat{w_1}/\hat{w_2}$ be the global minimizer of $g(v)$.
Now we can define the multi-task learning estimator for the target task as
	\[ \hat{\beta}_2^{\MTL} = \hat{w}_{2}\hat{B}(\hat{w}_1, \hat{w}_2) .\]
%	where $t=2$ since we are considering the two task case, and it also stands for the ``target task".
%The intuition for deriving $\hat{\beta}_2^{\MTL}$ is akin to performing multi-task training in practice.
%Let $\hat{v} = \hat{w_1} / \hat{w_2}$ for the simplicity of notation.
The prediction loss of using $\hat{\beta}_2^{\MTL}$ for the target task is
\begin{align}
	\te(\hat{\beta}_2^{\MTL}) =&~ \hat{v}^2 \bignorm{\Sigma_2^{1/2}(\hat{v}^2 X_1^{\top}X_1 + X_2^{\top}X_2)^{-1} X_1^{\top}X_1 (\beta_1 - \hat{v} \beta_2)}^2 \nonumber \\
			&+~  \bigtr{\Sigma_2(\hat{v}^2 X_1^{\top}X_1 + X_2^{\top}X_2)^{-2}\left(\sigma_1^2 \cdot \hat v^2  X_1^{\top}X_1 + \sigma_2^2  \cdot X_2^{\top}X_2\right) }, \label{eq_te_mtl_2task}
\end{align}
which only depends on $\hat v$, the sample covariance matrices, and $\beta_1,\beta_2$.
\fi


%\section{Main Results}\label{sec_general}

%\begin{proposition}[Bias-variance tradeoff]
%	Variance always reduces and bias always increases.
%\end{proposition}

%In this section, we compare the prediction loss of the multi-task learning estimator to that of the single-task learning estimator.
%First, we consider the two-task case.
%We provide precise asymptotics of the bias and variance of the multi-task learning estimator.
%We apply recent developments from the random matrix theory literature to characterize the bias and variance.
%The results scale with key properties of task data such as sample size and covariance shift, and allow us to study the performance of multi-task learning by varying these properties (Section \ref{sec_special}).
%Second, we consider the multiple-task case where all tasks have the same features but different labels.
%We extend the bias-variance decomposition of the two-task case to this setting and show qualitatively similar results.
%For the single-task learning estimator, there are well-known results that relate its prediction loss to sample size and noise variance.

%We focus on the high-dimensional linear regression setting \cite{}

%A well-known result in the high-dimensional linear regression setting states that $\tr[(X_2^{\top}X_2)^{-1}\Sigma_2]$ is concentrated around $1 / (\rho_2 - 1)$ (e.g. Chapter 6 of \cite{S07}), which scales with the sample size of the target task.
%Our main technical contribution is to extend this result to two tasks.
%We show how the variance of the multi-task estimator scales with sample size and covariate shift in the following result.

