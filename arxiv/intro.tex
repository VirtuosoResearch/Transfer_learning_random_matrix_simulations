\begin{abstract}
	Hard parameter sharing is a widely used approach to learn jointly from multiple tasks in many applications, such as text classification. The idea is to use a shared feature space for all tasks, while each task also has a separate layer for making the prediction.
%	It becomes increasingly important to understand when they work well on typical data. For example, the performance of hard parameter sharing depends on dataset properties such as their sample sizes. Yet, rigorously formulating such intuition can be challenging.
	This paper studies a fundamental question to better understand this approach: How does hard parameter sharing work given multiple linear regression tasks?
	We consider the high-dimensional setting, where the sample size and feature dimension become increasingly large in a fixed ratio. We present tight generalization bounds for two canonical cases: (i) multiple tasks with the same feature covariates; (ii) two tasks with arbitrarily different sample size ratios and covariance matrices. We also demonstrate that these estimates for the empirical loss are incredibly accurate in moderate dimensions. Finally, our results provide new explanations for several intriguing phenomena. For example, increasing one task's sample size helps another task initially by reducing variance, but hurts eventually due to increasing bias. This suggests progressively adding data for optimizing hard parameter sharing, and we validate its efficiency in text classification tasks.
\end{abstract}

\section{Introduction}\label{sec introduction}

\iffalse
%Multi-task learning is an inductive learning mechanism to improve generalization performance using related task data.
%Many state-of-the-art results in computer vision and natural language processing are obtained using multi-task learning.
Multi-task learning is a powerful approach to improve performance for many tasks in computer vision, natural language processing, and other areas \cite{C97,ZY17,R17}.
%In multi-task learning, having related task data is fundamental to its performance.
%Multi-task learning is particularly powerful when there is limited labeled data for a task to be solved, meanwhile more labeled data from different but related tasks is available.
%By combining multiple information sources, it is possible to share all the information in the same model.
In many settings, multiple source tasks are available to help with predicting a particular target task.
\todo{clarify setting is different from traditional MTL}
%For example, many applications in , and many other areas have been achieved by learning from multiple tasks together.
The performance of multi-task learning depends on the relationship between the source and target tasks \cite{C97}.
%	We define that multi-task learning provides \textit{positive transfer} if it outperforms single-task learning, or \textit{negative transfer} otherwise.
When the sources are relatively different from the target, multi-task learning (MTL) has often been observed to perform worse than single-task learning (STL) \cite{AP16,BS17}, which is referred to as \textit{negative transfer} \cite{PY09}.
While many empirical approaches have been proposed to mitigate negative transfer \cite{ZY17}, a precise understanding of when negative transfer occurs remains elusive in the literature \cite{R17}.
%This phenomenon, known as \textit{negative transfer}, is fundamental to the understanding of multi-task learning.

%Inspired by the theory, we propose an incremental training schedule to improve multi-task training.
%We consider a setting where the target task has limited labeled data and show
%On the other hand, unless the structures across task data are well-understood, applying multi-task learning on several different datasets often result in suboptimal models (or negative transfer in more technical terms).

Understanding negative transfer requires developing generalization bounds that scale tightly with properties of each task data, such as its sample size.
This presents a technical challenge in the multi-task setting because of the difference among task features, even for two tasks.
ithout a tight lower bound for multi-task learning, comparing its performance to single-task learning results in vacuous bounds.
\todo{add more technical motivation (or maybe later)}
From a practical standpoint, developing a better understanding of multi-task learning in terms of properties of task data can provide guidance for downstream applications \cite{RH19}.
%For example,
%On the other hand, uneven sample sizes (or dominating tasks) have been empirically observed to cause negative transfer \cite{YKGLHF20}.
%The benefit of learning multi-task representations has also been studied for certain half-spaces \cite{} and sparse regression \cite{}.
%When all tasks are sufficiently similar, adding more labeled data improves the generalization performance for predicting a particular task \cite{WZR20}.

%\textbf{Setup and Main Results.}
In this work, we study the bias and variance of multi-task learning in the high-dimensional linear regression setting \cite{HMRT19,BLLT20}.
Our key observation is that three properties of task data, including \textit{task similarity}, \textit{sample ratio}, and \textit{covariate shift}, can affect whether multi-task learning outperforms single-task learning (which we refer to as \textit{positive transfer}).
As an example, we vary each property in Figure \ref{fig_model_shift_phasetrans} for two linear regression tasks and measure the improvement of multi-task learning over single-task learning for a particular task.
We observe that the effect of transfer can be either positive or negative as we vary each property.
These phenomena cannot be explained using previous techniques \cite{WZR20}.
The high-dimensional linear regression setting allows us to measure the three properties precisely.
Here we define each property for the case of two tasks, while our definition applies to general settings.
We refer to the first task as the source task and the second as the target task.
\squishlist
	\item \textbf{Task similarity:} Assume that both tasks follow a linear model with parameters $\beta_1, \beta_2\in\real^p$, respectively.
	We measure the distance between them by $\norm{\beta_1 - \beta_2}$.
	\item \textbf{Sample ratio:} Let $n_1 = \rho_1 \cdot p, n_2 = \rho_2 \cdot p$ be the sample size of each task, where $\rho_1, \rho_2>1$ are both fixed values that do not grow with $p$.
	We measure the source/target sample ratio by $\rho_1 / \rho_2$.
%	Importantly, $\rho_2$ can be a small constant (say $2$) to capture the need for more labeled data.
	\item \textbf{Covariate shift:} Assume that the task features are random vectors with positive semidefinite covariance matrices $\Sigma_1\in\real^{p\times p}$ and $\Sigma_2\in\real^{p\times p}$, respectively.
	%$x = \Sigma_i^{1/2}z$, where $z\in\real^p$ consists of i.i.d. entries with mean zero and unit variance, and is a positive semidefinite matrix.
	We measure covariate shift with matrix $\Sigma_1^{1/2}\Sigma_2^{-1/2}$.
\squishend
\fi


Hard parameter sharing (HPS) is a widely used approach to learn from multiple tasks and goes back to the seminal work of \citet{C97}.
It is generally applied by sharing the feature layers between all tasks while keeping an output layer for every task.
Recent work has revived interest in this classical approach through many applications \cite{MTDNN19,ZSSGM18}.
Hard parameter sharing offers two critical advantages:
(i) it reduces model parameters since all tasks use the same feature space;
(ii) it reduces the need for labeled data from each task while augmenting the entire training dataset.

Hard parameter sharing is often thought of as a mechanism for inducing inductive bias.
Since all tasks share the same feature space, by restricting the space's, HPS acts as a regularizer and reduces overfitting \cite{KD12,WZR20}.
Another source of inductive bias comes from the tasks and depends on datasets' properties such as sample sizes and task covariances \cite{WZR20}.
Previous generalization theories have considered when all tasks' sample sizes are equal \cite{B00,MPR16}.
Using Rademacher complexity or VC-based techniques, the generalization error scales down as all tasks' sample sizes increase \cite{AZ05,M06}.
%For, the generalization error scales down as the sample sizes of all tasks increase, when applied to the multi-task setting \cite{B00,AZ05,M06,MPR16,WZR20}.


This paper presents a new generalization theory to understand the hard parameter sharing approach by studying its performance in the high-dimensional linear regression setting.
Linear regression is arguably one of the most fundamental problems in statistics and machine learning.
We are interested in the \textit{high-dimensional} setting, where each dataset's sample size and the feature dimension grow linearly at a fixed ratio.
This is motivated by many multi-task learning applications, where the amount of labeled data from each dataset is usually insufficient for learning a single task.
For example, this is the case if a dataset's sample size is only a small constant factor of the feature dimension \cite{socher2013recursive}.
The high-dimensional setting is challenging but is crucial for understanding how datasets' sample sizes impact generalization performance.
