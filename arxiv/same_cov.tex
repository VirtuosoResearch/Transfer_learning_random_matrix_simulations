\section{Bias-Variance Tradeoff in the Same Covariates Setting}


We begin by considering the case where all tasks have the same sample size and feature covariates, that is, $n_i = n$ and $X^{(i)} = X\in\real^{n\times p}$ for all $i = 1, \dots, t$.
We provide a sharp generalization error bound of hard parameter sharing estimators.
We show that the hard parameter sharing estimator finds a ``low-rank approximation of all tasks'', in the sense formalized below.
%This setting is prevalent in applications of multi-task learning to image classification, where there are multiple prediction labels/tasks for every image \cite{chexnet17,EA20}.
%We consider an arbitrary local minimum $B, W_1, \dots, W_2$ of the optimization objective.
%We extend the bias-variance decomposition from the two-task case to the multiple-task case.
%We observe that the expected prediction loss of $\hat{\beta}_t^{\MTL}$ conditional on $X$ consists of a bias and a variance equation as follows
%\begin{align}
%	\exarg{\varepsilon_1, \dots, \varepsilon_t}{L(\hat{\beta}_t^{\MTL}) \mid X}
%	=& \bignorm{\Sigma^{1/2} \bigbrace{B^{\star} \cW^{\top} (\cW \cW^{\top})^{-1} W_t - \beta_t}}^2 \label{eq_bias_multiple} \\
%	&+ \sigma^2 \cdot (W_t^{\top} (\cW \cW^{\top})^{-1} W_t) \cdot \bigtr{\Sigma (X^{\top} X)^{-1}} \label{eq_var_multiple}
%\end{align}
%One can see that equation \eqref{eq_bias_multiple} is the bias of the multi-task learning estimator and equation \eqref{eq_var_multiple} is its variance.
%Compared to the prediction loss of single-task learning (cf. equation \eqref{eq_var_stl}), we observe that the variance equation \eqref{eq_var_multiple} is always smaller because $W_t^{\top} (\cW \cW^{\top})^{-1} W_t \le 1$.
%On the other hand, the bias equation \eqref{eq_bias_multiple} is always larger because of the difference between the task models.
%We show the generalization error of hard parameter sharing estimators.
%Before stating the result, we define the following notations.
Let $B^\star := [{\beta}_1,{\beta}_2,\dots,{\beta}_{t}] \in \real^{p\times t}$ denote the ground truth model parameters.
Let $\Sigma$ denote the (same) population covariance matrix of all tasks.
Let $A^{\star} {A^{\star}}^{\top}$ denote the best rank-$r$ subspace approximation of ${B^{\star}}^\top\Sigma B^{\star}$, that is,
\[ A^{\star} \define \argmin_{U\in\real^{t\times r} : U^{\top} U = \id_{r\times r}} \inner{U U^{\top}} {{B^{\star}}^{\top} \Sigma B^{\star}}. \]
For $i = 1,\dots, t$, let $a_i^{\star} \in\real^r$ denote the $i$-th column of $A^{\star}{A^{\star}}^{\top}$.
Our main result in this section is stated as follows.

\begin{theorem}[Multiple tasks with the same covariates]\label{thm_many_tasks}
%Suppose $X=Z\Sigma^{1/2}\in \R^{n\times p}$ satisfy Assumption \ref{assm_secA1} with $\rho:=n/p>1$ being some fixed constant. Consider data models  $Y_i = X\beta_i + \varepsilon_i$, $i=1,2,\cdots, t$, where $\e_i\in \R^{n}$ are random vectors with i.i.d. entries with mean zero, variance $\sigma^2$ and all moments as in \eqref{assmAhigh}. Moreover, assume that $X$, $\beta_i$ and $\e_i$ are all independent of each other.
	%Let $n = c \cdot p$.
	%Let $X\in\real^{n\times p}$ and $Y_i = X\beta_i + \varepsilon_i$, for $i = 1,\dots,k$.
%	Consider $t$ data models $Y_i = X\beta_i + \varepsilon_i$, $i=1,2,\cdots, t$, where $X$ has covariance matrix $\Sigma$, and the entries of $\e_i$ are i.i.d. with mean zero and variance $\sigma^2$.gT
	%that satisfy Assumption \ref{assm_secA2} in the appendix.
	Consider the multiple-task case where all tasks have the same covariates $X\in\real^{n\times p}$ and population covariance matrix $\Sigma\in\real^{p\times p}$.
	Suppose that $X$ satisfies Assumption \ref{assume_rm}.
	Recall that $c$ is any fixed value within $(0, \frac{\varphi-4}{2\varphi})$.
	% Let $\delta$ be any fixed value such that $\delta \le \oo \left( \|B^\star\|^2 + \sigma^2\right)$.
	Let $L(B^{\star}a_i^{\star}) = \norm{\Sigma^{1/2} (B^{\star} a_i^{\star}- \beta^{(i)})}^2$.
	Let $C_1 = \frac{\rho\cdot \normFro{\Sigma^{1/2} B^{\star}}^2}{(\sqrt{\rho} - 1)^2\lambda_{\min}^2(\Sigma^{1/2} B^{\star}) + \sigma^2}$ and $C_2 =  C_1\cdot \norm{\Sigma^{1/2} B^{\star}}$, both of which do not grow with $p$.
	Then, for any task $i = 1, 2, \dots, t$, the prediction loss of the hard parameter sharing estimator $\hat{\beta}_i^{\MTL}$ satisfies that
	\begin{align}
		\bigabs{L(\hat{\beta}_i^{\MTL}) - L(B^{\star}a_i^{\star}) - \sigma^2  \cdot \frac{p \cdot \norm{a_i^{\star}}^2}{n - p}} \lesssim p^{-c}\bigbrace{C_1 L(B^{\star}a_i^{\star}) + C_1 \sigma^2 \frac{p \cdot{\norm{a_i^{\star}}^2}}{n-p} + C_2}. \label{eq_thm_many}
	\end{align}
	%\begin{itemize}
	%	\item \textbf{Positive transfer:} If $\left(1 - \norm{v_t}^2 \right)\frac{\sigma^2}{\rho - 1} -  > \delta$, then w.h.p over the randomness of $X, \varepsilon_1, \dots, \varepsilon_t$, we have that
	%	 \[ \te(\hat{\beta}_t^{\MTL}) < \te(\hat{\beta}_t^{\STL}). \]
	%	\item \textbf{Negative transfer:} If $\left(1 - \norm{v_t}^2\right)\frac{\sigma^2}{\rho - 1} - \norm{\Sigma^{1/2}(B^{\star} U_r v_t - \beta_t)}^2 < -\delta$, then w.h.p. over the randomness of $X, \varepsilon_1, \dots, \varepsilon_t$, we have that
	%	\[ \te(\hat{\beta}_t^{\MTL}) > \te(\hat{\beta}_t^{\STL}). \]
	%\end{itemize}
\end{theorem}
	Theorem \ref{thm_many_tasks} provides a sharp generalization error bound that is asymptotically tight as $p$ goes to infinity.
	The limiting loss of hard parameter sharing consists of two parts, a bias term that measures the error compared to the ground truth, and a variance term that scales with noise variance and sample size.
	Our result implies that the variance of hard parameter sharing always reduces compared to single-task learning, whose variance is equal to $\frac{\sigma^2 \cdot p} {n - p}$ by Lemma \ref{lem_minv}.
	This is because $\norm{a_i^{\star}}^2$ is at most one since the spectral norm of $U_r$, which is a projection matrix, is at most one.

	%First, we provide the steps for showing the bias-variance decomposition in equation \eqref{eq_bias_multiple} and \eqref{eq_var_multiple}.
	\paragraph{Proof overview.} The key idea for proving Theorem \ref{thm_many_tasks} is a bias-variance decomposition of the prediction loss of hard parameter sharing.
	We obtain the decomposition as follows.
	Since all tasks have the same covariates, the optimization objective \eqref{eq_mtl} becomes
	\begin{align}
		f(A, B) = \sum_{j=1}^t \bignorm{X B A_j - Y^{(j)}}^2, \label{eq_mtl_same_cov}
	\end{align}
	where we recall that $B \in \real^{p \times r}$ and $A_1, A_2, \dots, A_t \in \R^r$. % for $1 < r < t$ by our assumption.
	Without loss of generality, we assume that $AA^{\top}$ is invertible.
	Otherwise, we can add a small amount of random perturbation to $A$ (e.g. an inverse exponential in $p$ amount of isotropic Gaussian noise), and the result still holds.
	Using the local optimality condition over $B$, that is, $\frac{\partial f}{\partial B} = 0$, we obtain $\hat{B}$ as a function of the output layers as follows
	\begin{align}
		\hat{B} \define (X^{\top}X)^{-1} X^{\top} \bigbrace{\sum_{j=1}^t Y^{(j)} A_j^{\top}} (A  A^{\top})^{-1}. \label{eq_Bhat} %\\
		%&= (B^\star A ^{\top}) (A A^{\top})^{-1} + (X^{\top}X)^{-1}X^{\top}   \bigbrace{\sum_{j=1}^t \varepsilon_i A_i^{\top}} (A  A^{\top})^{-1}.
	\end{align}
	In the above, we use the fact that $AA^{\top}$ is invertible by our discussion above and $X^{\top}X$ is invertible by Fact \ref{lem_minv}.
	By plugging in $\hat{B}$ back into equation \eqref{eq_mtl_same_cov}, we obtain an objective that only depends on the output layer as follows:
	\begin{align}
		g(A) \define \sum_{j=1}^t \bignorm{X (X^{\top}X)^{-1}X^{\top} \bigbrace{\sum_{k=1}^t Y^{(k)} A_{k}^{\top}} (AA^{\top})^{-1} A_j - Y^{(j)}}^2. \label{eq_mtl_output_layer}
	\end{align}

	\begin{claim}\label{lem_exp_opt}
		In the setting of Theorem \ref{thm_many_tasks}, we have that
		\begin{align}
			\exarg{\set{\varepsilon^{(j)}}_{j=1}^t, X}{g(A)} = n \bignorm{\Sigma^{1/2} B^{\star} \bigbrace{A^{\top} (AA^{\top})^{-1} A - \id_{t\times t}}}^2 + \sigma^2 (n\cdot t - p \cdot r). \label{eq_gA}
		\end{align}
		As a result, the minimum of $\ex{g(A)}$, denoted by $A^{\star}$, is the best rank-$r$ approximation of ${B^{\star}}^{\top}\Sigma B^{\star}$.
	\end{claim}

	One can see that the expected prediction loss admits a bias-variance decomposition in equation \eqref{eq_gA}.
	The first part measures the bias of hard parameter sharing compared to single-task learning for all tasks.
	The second part measures the variance of hard parameter sharing and scales with $\sigma^2$.
	In order to minimize the expected prediction loss, it suffices to minimize the bias equation since the variance equation does not depend on $A$.
	%We denote $Q := \cal W^{\top} (\cal W\cal W^{\top})^{-1} \cal W \in\real^{t\times t}$, whose $(i,j)$-th entry is given by $W_i^{\top} (\cal W\cal W^{\top})^{-1} W_j$.
	%Let $B^{\star} = [\beta_1, \beta_2, \dots, \beta_k] \in\real^{p \times k}$ denote the true model parameters.
	%Now we can write $\delta_{\bias}(\cal W)$ succinctly as
	%\begin{align*}
	%	\delta_{\bias}(Q) \equiv \delta_{\bias}(\cal W) = \bignormFro{\Sigma^{1/2}B^{\star}  %\bigbrace{Q -\id}}^2 .
	%\end{align*}
	For the bias equation, we observe that the minimizer is equal to the best rank-$r$ (subspace) approximation to ${B^{\star}}^{\top} \Sigma B^{\star}$, which is denoted by $U_{r} U_r^{\top}$.
	Recall that we obtain the hard parameter sharing estimator $\hat{\beta}_i^{\MTL} = \hat{B} \hat{A}_i$ from a global minimum solution $\hat{A}, \hat{B}$ of the training objective $f(A, B)$.
	Since $\hat{A}, \hat{B}$ is a global minimum of the training objective, we know that $\hat{B}$ is equal to $\hat{B}(\hat{A})$ given by equation \eqref{eq_Bhat}.
	Therefore, $\hat{A}$ is a global minimum of $g(A)$ in equation \eqref{eq_mtl_output_layer}.
	Our next claim shows that the subspace spanned by the rows of $\hat{A}$ is close to that of $A^{\star}$.
	\begin{claim}\label{claim_opt_dist}
		Let $U_{\hat{A}} U_{\hat{A}}^{\top} \in\real^{t\times t}$ denote the subspace $\hat{A}^{\top} (\hat{A}\hat{A}^{\top})^{-1} \hat{A}$.
		In the setting of Theorem \ref{thm_many_tasks}, we have that
		\[ \bignormFro{U_{\hat{A}} U_{\hat{A}}^{\top} - A^{\star} {A^{\star}}^{\top}}
				\lesssim p^{-c} C_1. \]
	\end{claim}
	The above claim shows that as $p$ goes to infinity, the distance between the column subspace of $U_{\hat{A}}$ and ${A^{\star}}$ goes to zero.
	Finally, the following claim shows that the prediction loss of the hard parameter sharing estimator and the optimal solution of Claim \ref{lem_exp_opt} is small.
	\begin{claim}\label{claim_pred_err}
		In the setting of Theorem \ref{thm_many_tasks},
		let $\hat{a}_i = \hat{A}^{\top} (\hat{A}\hat{A}^{\top})^{-1} \hat{A}_i$.
		We have that the prediction loss of $\hat{\beta}_i^{\MTL} = \hat{B} \hat{A}_i$ satisfies that
		\begin{align*}
			\bigabs{L(\hat{\beta}_i^{\MTL}) - L(B^{\star} \hat{a}_i) - \norm{\hat{a}_i}^2 \cdot \bigtr{\Sigma (X^{\top}X)^{-1}}}
			\le 2\cdot p^{-1/2 + \e} \cdot \bigbrace{L(B^{\star} \hat{a}_i) + \norm{\hat{a}_i}^2 \cdot \bigtr{\Sigma (X^{\top}X)^{-1}}}.
		\end{align*}
	\end{claim}
	The proof of Claim \ref{lem_exp_opt}, Claim \ref{claim_opt_dist}, and Claim \ref{claim_pred_err} can be found in Appendix \ref{app_proof_error_same_cov}.
	Provided with these results, we are ready to prove Theorem \ref{thm_many_tasks}.
	\begin{proof}[Proof of Theorem \ref{thm_many_tasks}]
		Using Claim \ref{claim_pred_err}, we get that the prediction loss of $\hat{\beta}_i^{\MTL}$ is within an $\OO(p^{-1/2+\e})$ fraction from $L(B^{\star}\hat{a}_i)$ and $\norm{\hat{a}_i}^2 \bigtr{\Sigma(X^{\top}X)^{-1}}$.
		For the latter, we use Claim \ref{claim_opt_dist} to upper bound the difference between $\norm{\hat{a}_i}^2$ and $\norm{A^{\star}}^2$, and Fact \ref{lem_minv} to get a precise estimate of $\bigtr{\Sigma (X^{\top}X)^{-1}}$.
		For $L(B^{\star}\hat{a}_i)$, we again use Claim \ref{claim_opt_dist} to upper bound the distance between $\hat{a}_i$ and $A^{\star}$.
		Combined together, we have shown that equation \eqref{eq_thm_many} holds and the proof is complete.
	\end{proof}

