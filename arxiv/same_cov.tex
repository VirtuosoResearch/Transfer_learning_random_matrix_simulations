\section{Bias-Variance Tradeoff in the Same Covariates Setting}


We begin by considering the case where all tasks have the same sample size and feature covariates, that is, $n_i = n$ and $X^{(i)} = X\in\real^{n\times p}$ for all $i = 1, \dots, t$.
We provide a sharp generalization error bound of hard parameter sharing estimators.
We show that the hard parameter sharing estimator finds a ``low-rank approximation of all tasks'', in the sense formalized below.
%This setting is prevalent in applications of multi-task learning to image classification, where there are multiple prediction labels/tasks for every image \cite{chexnet17,EA20}.
%We consider an arbitrary local minimum $B, W_1, \dots, W_2$ of the optimization objective.
%We extend the bias-variance decomposition from the two-task case to the multiple-task case.
%We observe that the expected prediction loss of $\hat{\beta}_t^{\MTL}$ conditional on $X$ consists of a bias and a variance equation as follows
%\begin{align}
%	\exarg{\varepsilon_1, \dots, \varepsilon_t}{L(\hat{\beta}_t^{\MTL}) \mid X}
%	=& \bignorm{\Sigma^{1/2} \bigbrace{B^{\star} \cW^{\top} (\cW \cW^{\top})^{-1} W_t - \beta_t}}^2 \label{eq_bias_multiple} \\
%	&+ \sigma^2 \cdot (W_t^{\top} (\cW \cW^{\top})^{-1} W_t) \cdot \bigtr{\Sigma (X^{\top} X)^{-1}} \label{eq_var_multiple}
%\end{align}
%One can see that equation \eqref{eq_bias_multiple} is the bias of the multi-task learning estimator and equation \eqref{eq_var_multiple} is its variance.
%Compared to the prediction loss of single-task learning (cf. equation \eqref{eq_var_stl}), we observe that the variance equation \eqref{eq_var_multiple} is always smaller because $W_t^{\top} (\cW \cW^{\top})^{-1} W_t \le 1$.
%On the other hand, the bias equation \eqref{eq_bias_multiple} is always larger because of the difference between the task models.
%We show the generalization error of hard parameter sharing estimators.
%Before stating the result, we define the following notations.
Let $B^\star := [{\beta}_1,{\beta}_2,\dots,{\beta}_{t}] \in \real^{p\times t}$ denote the ground truth model parameters.
Let $\Sigma$ denote the (same) population covariance matrix of all tasks.
Let $A^{\star} {A^{\star}}^{\top}$ denote the best rank-$r$ subspace approximation of ${B^{\star}}^\top\Sigma B^{\star}$, that is,
\[ A^{\star} \define \argmin_{U\in\real^{t\times r} : U^{\top} U = \id_{r\times r}} \inner{U U^{\top}} {{B^{\star}}^{\top} \Sigma B^{\star}}. \]
For $i = 1,\dots, t$, let $a_i^{\star} \in\real^r$ denote the $i$-th column of $A^{\star}{A^{\star}}^{\top}$.
Our main result in this section is stated as follows.

\begin{theorem}[Multiple tasks with the same covariates]\label{thm_many_tasks}
%Suppose $X=Z\Sigma^{1/2}\in \R^{n\times p}$ satisfy Assumption \ref{assm_secA1} with $\rho:=n/p>1$ being some fixed constant. Consider data models  $Y_i = X\beta_i + \varepsilon_i$, $i=1,2,\cdots, t$, where $\e_i\in \R^{n}$ are random vectors with i.i.d. entries with mean zero, variance $\sigma^2$ and all moments as in \eqref{assmAhigh}. Moreover, assume that $X$, $\beta_i$ and $\e_i$ are all independent of each other.
	%Let $n = c \cdot p$.
	%Let $X\in\real^{n\times p}$ and $Y_i = X\beta_i + \varepsilon_i$, for $i = 1,\dots,k$.
%	Consider $t$ data models $Y_i = X\beta_i + \varepsilon_i$, $i=1,2,\cdots, t$, where $X$ has covariance matrix $\Sigma$, and the entries of $\e_i$ are i.i.d. with mean zero and variance $\sigma^2$.gT
	%that satisfy Assumption \ref{assm_secA2} in the appendix.
	Consider the multiple-task case where all tasks have the same covariates $X\in\real^{n\times p}$ and population covariance matrix $\Sigma\in\real^{p\times p}$.
	Suppose that $X$ satisfies Assumption \ref{assume_rm}.
	Recall that $c$ is any fixed value within $(0, \frac{\varphi-4}{2\varphi})$.
	% Let $\delta$ be any fixed value such that $\delta \le \oo \left( \|B^\star\|^2 + \sigma^2\right)$.
	Let $L(B^{\star}a_i^{\star}) = \norm{\Sigma^{1/2} (B^{\star} a_i^{\star}- \beta^{(i)})}^2$.
	Let $C_1 = \frac{\rho\cdot \normFro{\Sigma^{1/2} B^{\star}}^2}{(\sqrt{\rho} - 1)^2\lambda_{\min}^2(\Sigma^{1/2} B^{\star}) + \sigma^2}$ and $C_2 =  C_1\cdot \norm{\Sigma^{1/2} B^{\star}}$, both of which do not grow with $p$.
	Then, for any task $i = 1, 2, \dots, t$, the prediction loss of the hard parameter sharing estimator $\hat{\beta}_i^{\MTL}$ satisfies that
	\begin{align}
		\bigabs{L(\hat{\beta}_i^{\MTL}) - L(B^{\star}a_i^{\star}) - \sigma^2  \cdot \frac{p \cdot \norm{a_i^{\star}}^2}{n - p}} \lesssim p^{-c}\bigbrace{C_1 L(B^{\star}a_i^{\star}) + C_1 \sigma^2 \frac{p \cdot{\norm{a_i^{\star}}^2}}{n-p} + C_2}. \label{eq_thm_many}
	\end{align}
	%\begin{itemize}
	%	\item \textbf{Positive transfer:} If $\left(1 - \norm{v_t}^2 \right)\frac{\sigma^2}{\rho - 1} -  > \delta$, then w.h.p over the randomness of $X, \varepsilon_1, \dots, \varepsilon_t$, we have that
	%	 \[ \te(\hat{\beta}_t^{\MTL}) < \te(\hat{\beta}_t^{\STL}). \]
	%	\item \textbf{Negative transfer:} If $\left(1 - \norm{v_t}^2\right)\frac{\sigma^2}{\rho - 1} - \norm{\Sigma^{1/2}(B^{\star} U_r v_t - \beta_t)}^2 < -\delta$, then w.h.p. over the randomness of $X, \varepsilon_1, \dots, \varepsilon_t$, we have that
	%	\[ \te(\hat{\beta}_t^{\MTL}) > \te(\hat{\beta}_t^{\STL}). \]
	%\end{itemize}
\end{theorem}
	Theorem \ref{thm_many_tasks} provides a sharp generalization error bound that is asymptotically tight as $p$ goes to infinity.
	The limiting loss of hard parameter sharing consists of two parts, a bias term that measures the error compared to the ground truth, and a variance term that scales with noise variance and sample size.
	Our result implies that the variance of hard parameter sharing always reduces compared to single-task learning, whose variance is equal to $\frac{\sigma^2 \cdot p} {n - p}$ by Lemma \ref{lem_minv}.
	This is because $\norm{a_i^{\star}}^2$ is at most one since the spectral norm of $U_r$, which is a projection matrix, is at most one.

	%First, we provide the steps for showing the bias-variance decomposition in equation \eqref{eq_bias_multiple} and \eqref{eq_var_multiple}.
	The key idea for proving Theorem \ref{thm_many_tasks} is a bias-variance decomposition of the prediction loss of hard parameter sharing.
	We obtain the decomposition as follows.
	Since all tasks have the same covariates, the optimization objective \eqref{eq_mtl} becomes
	\begin{align}
		f(A, B) = \sum_{j=1}^t \bignorm{X B A_j - Y^{(j)}}^2, \label{eq_mtl_same_cov}
	\end{align}
	where we recall that $B \in \real^{p \times r}$ and $A_1, A_2, \dots, A_t \in \R^r$. % for $1 < r < t$ by our assumption.
	Without loss of generality, we assume that $AA^{\top}$ is invertible.
	Otherwise, we can add a small amount of random perturbation to $A$ (e.g. an inverse exponential in $p$ amount of isotropic Gaussian noise), and the result still holds.
	Using the local optimality condition over $B$, that is, $\frac{\partial f}{\partial B} = 0$, we obtain $\hat{B}$ as a function of the output layers as follows
	\begin{align}
		\hat{B} \define (X^{\top}X)^{-1} X^{\top} \bigbrace{\sum_{j=1}^t Y^{(j)} A_j^{\top}} (A  A^{\top})^{-1}. \label{eq_Bhat} %\\
		%&= (B^\star A ^{\top}) (A A^{\top})^{-1} + (X^{\top}X)^{-1}X^{\top}   \bigbrace{\sum_{j=1}^t \varepsilon_i A_i^{\top}} (A  A^{\top})^{-1}.
	\end{align}
	In the above, we use the fact that $AA^{\top}$ is invertible by our discussion above and $X^{\top}X$ is invertible by Fact \ref{lem_minv}.
	By plugging in $\hat{B}$ back into equation \eqref{eq_mtl_same_cov}, we obtain an objective that only depends on the output layer as follows:
	\begin{align}
		g(A) \define \sum_{j=1}^t \bignorm{X (X^{\top}X)^{-1}X^{\top} \bigbrace{\sum_{k=1}^t Y^{(k)} A_{k}^{\top}} (AA^{\top})^{-1} A_j - Y^{(j)}}^2. \label{eq_mtl_output_layer}
	\end{align}

	\begin{claim}\label{lem_exp_opt}
		In the setting of Theorem \ref{thm_many_tasks}, we have that
		\begin{align}
			\exarg{\set{\varepsilon^{(j)}}_{j=1}^t, X}{g(A)} = n \bignorm{\Sigma^{1/2} B^{\star} \bigbrace{A^{\top} (AA^{\top})^{-1} A - \id_{t\times t}}}^2 + \sigma^2 (n\cdot t - p \cdot r). \label{eq_gA}
		\end{align}
		As a result, the minimum of $\ex{g(A)}$, denoted by $A^{\star}$, is the best rank-$r$ approximation of ${B^{\star}}^{\top}\Sigma B^{\star}$.
	\end{claim}

	One can see that the expected prediction loss admits a bias-variance decomposition in equation \eqref{eq_gA}.
	The first part measures the bias of hard parameter sharing compared to single-task learning for all tasks.
	The second part measures the variance of hard parameter sharing and scales with $\sigma^2$.
	In order to minimize the expected prediction loss, it suffices to minimize the bias equation since the variance equation does not depend on $A$.
	%We denote $Q := \cal W^{\top} (\cal W\cal W^{\top})^{-1} \cal W \in\real^{t\times t}$, whose $(i,j)$-th entry is given by $W_i^{\top} (\cal W\cal W^{\top})^{-1} W_j$.
	%Let $B^{\star} = [\beta_1, \beta_2, \dots, \beta_k] \in\real^{p \times k}$ denote the true model parameters.
	%Now we can write $\delta_{\bias}(\cal W)$ succinctly as
	%\begin{align*}
	%	\delta_{\bias}(Q) \equiv \delta_{\bias}(\cal W) = \bignormFro{\Sigma^{1/2}B^{\star}  %\bigbrace{Q -\id}}^2 .
	%\end{align*}
	For the bias equation, we observe that the minimizer is equal to the best rank-$r$ (subspace) approximation to ${B^{\star}}^{\top} \Sigma B^{\star}$, which is denoted by $U_{r} U_r^{\top}$.
	Recall that we obtain the hard parameter sharing estimator $\hat{\beta}_i^{\MTL} = \hat{B} \hat{A}_i$ from a global minimum solution $\hat{A}, \hat{B}$ of the training objective $f(A, B)$.
	Since $\hat{A}, \hat{B}$ is a global minimum of the training objective, we know that $\hat{B}$ is equal to $\hat{B}(\hat{A})$ given by equation \eqref{eq_Bhat}.
	Therefore, $\hat{A}$ is a global minimum of $g(A)$ in equation \eqref{eq_mtl_output_layer}.
	Our next claim shows that the subspace spanned by the rows of $\hat{A}$ is close to that of $A^{\star}$.
	\begin{claim}\label{claim_opt_dist}
		Let $U_{\hat{A}} U_{\hat{A}}^{\top} \in\real^{t\times t}$ denote the subspace $\hat{A}^{\top} (\hat{A}\hat{A}^{\top})^{-1} \hat{A}$.
		In the setting of Theorem \ref{thm_many_tasks}, we have that
		\[ \bignormFro{U_{\hat{A}} U_{\hat{A}}^{\top} - A^{\star} {A^{\star}}^{\top}}
				\lesssim p^{-c} C_1. \]
	\end{claim}
	The above claim shows that as $p$ goes to infinity, the distance between the column subspace of $U_{\hat{A}}$ and ${A^{\star}}$ goes to zero.
	Finally, the following claim shows that the prediction loss of the hard parameter sharing estimator and the optimal solution of Claim \ref{lem_exp_opt} is small.
	\begin{claim}\label{claim_pred_err}
		In the setting of Theorem \ref{thm_many_tasks},
		let $\hat{a}_i = \hat{A}^{\top} (\hat{A}\hat{A}^{\top})^{-1} \hat{A}_i$.
		We have that the prediction loss of $\hat{\beta}_i^{\MTL} = \hat{B} \hat{A}_i$ satisfies that
		\begin{align*}
			\bigabs{L(\hat{\beta}_i^{\MTL}) - L(B^{\star} \hat{a}_i) - \norm{\hat{a}_i}^2 \cdot \bigtr{\Sigma (X^{\top}X)^{-1}}}
			\le 2\cdot p^{-1/2 + \e} \cdot \bigbrace{L(B^{\star} \hat{a}_i) + \norm{\hat{a}_i}^2 \cdot \bigtr{\Sigma (X^{\top}X)^{-1}}}.
		\end{align*}
	\end{claim}

	Provided with Claim \ref{claim_opt_dist} and Claim \ref{claim_pred_err}, we are ready to prove Theorem \ref{thm_many_tasks}.
	\begin{proof}[Proof of Theorem \ref{thm_many_tasks}]
		Using Claim \ref{claim_pred_err}, we get that the prediction loss of $\hat{\beta}_i^{\MTL}$ is within an $\OO(p^{-1/2+\e})$ fraction from $L(B^{\star}\hat{a}_i)$ and $\norm{\hat{a}_i^2} \bigtr{\Sigma(X^{\top}X)^{-1}}$.
		For the latter, we use Claim \ref{claim_opt_dist} to upper bound the difference between $\norm{\hat{a}_i}^2$ and $\norm{A^{\star}}^2$, and Fact \ref{lem_minv} to get a precise estimate of $\bigtr{\Sigma (X^{\top}X)^{-1}}$.
		For $L(B^{\star}\hat{a}_i)$, we again use Claim \ref{claim_opt_dist} to upper bound the distance between $\hat{a}_i$ and $A^{\star}$.
		Combined together, we have shown that equation \eqref{eq_thm_many} holds and the proof is complete.
	\end{proof}
	%A proof of equation \eqref{eq_gA}, which is based on elementary calculations, can be found in Appendix \ref{app_proof_error_same_cov}.
	%Provided with the bias-variance decomposition of the expected prediction loss, we are ready to show the generalization error of the hard parameter sharing estimator.

	%We state a concentration error bound between $g(\cal W)$ and its expectation.

	%\begin{lemma}\label{lem_error_same_cov}
	%	In the setting of Theorem \ref{thm_many_tasks}, we have that
	%	\[ g(\cW) = \ex{g(\cW)} \cdot (1 + o(1)) + \OO(\sigma^2 p^{1/2 + c}). \]
	%\end{lemma}
	%The proof of Lemma \ref{lem_error_same_cov} is via standard concentration bounds.
	%The details can be found in Appendix \ref{app_proof_error_same_cov}.
	%Based on the above result, we are ready to prove Theorem \ref{thm_many_tasks}.
	%We use the notation $U_X U_X^{\top} = X (X^{\top} X)^{-1} X^{\top} \in\real^{n \times n}$ to denote the projection matrix to $X$, where $U_X\in\real^{n \times p}$.




	%In order to derive the minimizer, our proof involves two steps.
	%First, we consider the expectation of $g(\cW)$ over $\varepsilon_1, \varepsilon_2, \dots, \varepsilon_t$, and $X$.
	%We show that the minimizer of $\ex{g(\cW)}$ has a simple closed form solution similar to principal component analysis.
	%Second, we show that the concentration error between $g(\cW)$ and $\ex{g(\cW)}$ is small provided with $n$ samples.
	%Intuitively, this is because $\cW$ only has $r \times t$ parameters, which is much smaller than $n$ by our assumption.
	%Hence we use standard concentration bounds and $\epsilon$-net arguments to show that the concentration error is small for $g(\cW)$.

	%Now we switch $\hat{B}$ back into equation \eqref{eq_mtl_same_cov} to
	%Then as in \eqref{approxvalid}, we pick $N$ independent samples of the training set for each task with $N\ge n^{1-\e_0}$, and use the concentration result, Lemma \ref{largedeviation}, to get the validation loss as



%	\[ \abs{Z_1} \le \sum_{j=1}^t \sigma \cdot \norm{X B^{\star} \cW^{\top} (\cW\cW^{\top})^{-1} W_j - X \beta_j} \le \sigma \cdot \sqrt{t \cdot Z_0}, \]
%	where the second part follows by using Cauchy-Shwartz inequality.

	%We provide tights bounds on the concentration error from the randomness of $\varepsilon_1, \dots, \varepsilon_t$, and $X$, respectively.
	%\begin{align}
	%	g(\cW) &= \exarg{\varepsilon_1, \dots, \varepsilon_t}{g(\cW) \mid X} \cdot (1 \pm \OO(p^{-1/2 + c})), \text{ and} \label{approxvalid_1} \\
	%	\exarg{\varepsilon_1, \dots, \varepsilon_t}{g(\cW) \mid X} &= \ex{g(\cW)} \cdot(1 + o(1)) \label{approxvalid_2}
	%\end{align}
	%Together, they imply that $g(\cW) = \ex{g(\cW)} \cdot (1 + o(1))$. Therefore, next we focus on proving equation \eqref{approxvalid_1} and \eqref{approxvalid_2}.

	%For equation \eqref{approxvalid_1}, we observe that the summands of $g(\cW)$ w.r.t. $\varepsilon_1, \dots, \varepsilon_t$ belong to either of the following three types:
	%\begin{enumerate}
	%	\item[(i)] $\varepsilon_i^{\top} A \varepsilon_i$, for any $i = 1,\dots, t$ and a fixed $A\in\real^{n\times n}$ that is independent of $\varepsilon_i$;
	%	\item[(ii)] $\varepsilon_i^{\top} A \varepsilon_j$, for any $i \neq j$ and a fixed $A$ that is independent of both $\varepsilon_i$ and $\varepsilon_j$;
	%	\item[(iii)] $\varepsilon_i^{\top} A$, for any $i = 1,\dots, t$ and a fixed $A\in\real^n$ that is independent of $\varepsilon_i$.
	%\end{enumerate}
	%For all three types, using Lemma \ref{largedeviation} in Appendix \ref{sec_maintools} and the fact that all moments of $\varepsilon_i$ exist, we conclude that
	%\begin{enumerate}
	%	\item[(i)] $\varepsilon_i^{\top} A \varepsilon_i = \sigma^2 (1 + \OO(p^{-1/2 + c})) \normFro{A}^2$;
	%	\item[(ii)] $\abs{}$
	%	\item[(iii)]
	%\end{enumerate}

	%We use the fact that our random vectors have i.i.d. entries.
%Before doing that, we first need to fix the setting for the following discussions, because we want to keep track of the error rate carefully instead of obtaining an asymptotic result only.
	%Recall that $Y_i = X_i\beta_i + \varepsilon_i$ and $\wt Y_i = \wt X_i\beta_i + %\wt\varepsilon_i$, $i=1,2$, all satisfy Assumption \ref{assm_secA2}. Then we rewrite %\eqref{eq_mtl_2tasktilde} as
%$$	g( v) = \sum_{i=1}^2\left\| \wt X_i\wt\beta_i  - \wt \e_i\right\|^2 , \quad \wt\beta:=\hat %B w_i-\beta_i.$$
	%Since $ \wt X_i\wt\beta$ and $ \wt \e_i$ are independent random vectors with i.i.d. centered entries, we can use the concentration result,  to get that for any constant $\e>0$,
	%\begin{align*}
		%\left|\left\| \wt X_i\wt\beta_i  - \wt \e_i\right\|^2 -  \exarg{\wt X_i,\wt{\e}_i} {\left\| \wt X_i\wt\beta_i  - \wt \e_i\right\|^2} \right| & =\left|\left\| \wt X_i\wt\beta_i  %- \wt \e_i\right\|^2 - N_i (\wt\beta_i^\top \Sigma_i \wt\beta_i + \sigma_i^2) \right| \\
%&\le N_i^{1/2+\e} (\wt\beta_i^\top \Sigma_i \wt\beta_i + \sigma_i^2),
	%$\end{align*}
	%with high probability. Thus we obtain that
	%$$g(v)= \left[\sum_{i=1}^2 N_i\left\|\Sigma_i^{1/2}( \hat B w_i - \beta_i) \right\|^2 + (N_1\sigma^2_1+N_2\sigma^2_2)\right]\cdot \left( 1+\OO(p^{-(1-\e_0)/2+\e})\right),$$
%where we also used $N_i\ge p^{-1+\e_0}$. Inserting \eqref{hatB} into the above expression and using
	% again the concentration result, Lemma \ref{largedeviation}, we get that
	%$$ \sum_{i=1}^2 N_i\left\|\Sigma_i^{1/2}( \hat B w_1 - \beta_i) \right\|^2 = \val(v)\cdot \left( 1+\OO(p^{-1/2+\e})\right)$$
%with high probability.
%-----old-------
%Suppose that the entries of $\e_1$ and $\e_2$ have variance $\sigma^2$.  Using a validation set that is sub-sampled from the original training dataset, we get a validation loss as follows
%\begin{align}
%		&\val(\hat{B}; w_1, w_2):= \exarg{\varepsilon_1,\e_2} \sum_{i=1}^2 \left\|\Sigma_i^{1/2}( \hat B w_1 - \beta_i) \right\|^2 \\
%	&=  n_1 \cdot \bignorm{\Sigma_1^{1/2}\left(\frac{w_1^2}{w_2^2} X_1^{\top}X_1 + X_2^{\top}X_2\right)^{-1}X_2^{\top}X_2\left (\beta_s - \frac{w_1}{w_2}\beta_t\right)}^2 \nonumber \\
%		&+ n_1 \sigma^2 \cdot \frac{w_1^2}{w_2^2} \bigtr{\left(\frac{w_1^2}{w_2^2}  X_1^{\top}X_1 + X_2^{\top}X_2\right)^{-1}\Sigma_1} \nonumber \\
%		&+ n_2 \cdot \frac{w_1^2}{w_2^2}\bignorm{\Sigma_2^{1/2}\left(\frac{w_1^2}{w_2^2} X_1^{\top}X_1 + X_2^{\top}X_2\right)^{-1}X_1^{\top}X_1\left(\beta_s - \frac{w_1}{w_2}\beta_t\right)}^2 \nonumber \\
%		&+ n_2 \sigma^2 \cdot \bigtr{\left(\frac{w_1^2}{w_2^2} X_1^{\top}X_1 + X_2^{\top}X_2\right)^{-1}\Sigma_2}. \label{eq_val_mtl}
%\end{align}
%\nc
%------------------
%Thus we conclude the proof.




%From the above we can obtain three conceptual insights that are consistent with Section \ref{sec_denoise} and \ref{sec_insight}.
%\begin{itemize}
%	\item The de-noising effect of multi-task learning.
%	\item Multi-task training vs single-task training can be either positive or negative.
%	\item Transfer learning is better than the other two. And the improvement over multi-task training increases as the model distances become larger.
%\end{itemize}






