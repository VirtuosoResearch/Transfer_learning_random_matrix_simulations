\subsection{The Random-Effects Model}\label{app_iso_cov}

\subsubsection{Proof for Sample Imbalance}  
We now give the proof of Proposition \ref{cor_MTL_loss}. Similar to the proof of Theorem \ref{thm_many_tasks}, our proof is divided into the following steps: 
\begin{itemize}
\item simplifying the optimization objective using concentration estimates;

\item finding the minimizer of the optimization objective;

\item deriving a bias-variance decomposition of the prediction loss $L(\hat \beta_2^{\MTL})$; 

\item calculating the bias term and the variance term using Theorem \ref{thm_main_RMT}. 
\end{itemize}

\iffalse
%Our key insight is a bias-variance decomposition of the expected prediction loss of $\hat{\beta}_2^{\MTL} = \hat{B} W_2$.
%Using the local optimality condition of $B$, we obtain that
%\begin{align*}
%	 \hat{B}(W_1, W_2) &= (W_1^2 X_1^{\top}X_1 + W_2^2 X_2^{\top}X_2)^{-1} (W_1 X_1^{\top}Y_1 + W_2 X_2^{\top}Y_2)\\
%	&= \frac{1}{W_2} \left( \frac{W_1^2}{W_2^2}  X_1^{\top}X_1 + X_2^{\top}X_2\right)^{-1} \left(\frac{W_1}{W_2} X_1^{\top}Y_1 + X_2^{\top}Y_2\right) \\
%	&= \frac{1}{W_2}\left[\beta_2 + \left(\frac{W_1^2}{W_2^2} X_1^{\top}X_1 + X_2^{\top}X_2\right)^{-1}\bigbrace{X_1^{\top}X_1\left(\frac{W_1}{W_2}\beta_1 - \frac{W_1^2}{W_2^2} \beta_2\right) + \left(\frac{W_1}{W_2} X_1^{\top}\varepsilon_1 + X_2^{\top}\varepsilon_2\right)}\right].
%\end{align*}
Recall that $\hat{\beta}_i^{\MTL} = \hat{B} W_i$. Therefore, we have
\begin{align}
	\exarg{\epsilon_1, \epsilon_2}{L(\hat{\beta}_2^{\MTL}) \mid X_1, X_2}
	=&~ \frac{W_1^2}{W_2^2} \bignorm{\Sigma_2^{1/2}(\frac{W_1^2}{W_2^2} X_1^{\top}X_1 + X_2^{\top}X_2)^{-1} X_1^{\top}X_1 (\beta_1 - \frac{W_1}{W_2} \beta_2)}^2 \label{eq_bias_2task} \\
			&+~  \sigma^2\cdot \bigtr{\Sigma_2(\frac{W_1^2}{W_2^2} X_1^{\top}X_1 + X_2^{\top}X_2)^{-1} }. \label{eq_var_2task}
\end{align}
equation \eqref{eq_bias_2task} is the bias of $\hat{\beta}_t^{\MTL}$ and
equation \eqref{eq_var_2task} is the variance of $\hat{\beta}_t^{\MTL}$.
%minus the variance of $\hat{\beta}_t^{\STL}$, which is always negative.
Comparing the above to single-task learning, that is,
\begin{align}
	\exarg{\epsilon_2}{L(\hat{\beta}_2^{\STL}) \mid X_2} = \sigma^2 \cdot \bigtr{\Sigma_2 (X_2^{\top} X_2)^{-1}}, \label{eq_var_stl}
\end{align}
we observe that while the bias of $\hat{\beta}_2^{\MTL}$ is always larger than that of $\hat{\beta}_2^{\STL}$, which is zero, the variance of $\hat{\beta}_2^{\MTL}$ is always lower than that of $\hat{\beta}_2^{\STL}$.\footnote{To see why this is true, we apply the Woodbury matrix identity over equation \eqref{eq_var_2task} and use the fact that for the product of two PSD matrices, its trace is always nonnegative.}
In other words, training both tasks together helps predict the target task by reducing variance while incurring a bias.
Therefore, whether multi-task learning outperforms single-task learning is determined by the bias-variance decomposition!

% \todo{derive the test loss} Expectation was given in equation \eqref{eq_bias_2task} and equation \eqref{eq_var_2task}.
 \fi

\paragraph{Simplifying the optimization objective.} 
Under the two-task setting of Corollary \ref{cor_MTL_loss}, the optimization objective \eqref{eq_mtl} becomes
	\begin{align}
		f(A_1,A_2, B) =   \bignorm{X^{(1)} B A_1 - Y^{(1)}}^2+ \bignorm{X^{(2)} B A_2 - Y^{(2)}}^2, \label{eq_mtl_2task_cov}
	\end{align}
	where $B \in \real^{p}$ and $A_1, A_2\in \R$ because $r=1$ in the two-task case. % for $1 < r < t$ by our assumption.
	Without loss of generality, we assume that both $A_1$ and $A_2$ are nonzero, since otherwise we can add a small perturbation $\delta$ to them and later remove this perturbation by taking $\delta\to 0$ in the final result. Again using the local optimality condition ${\partial f}/{\partial B} = 0$, for any $A_1$ and $A_2$ we obtain the optimal $\hat{B}$ as follows
	\begin{align}
		\hat{B} \define  (A_1^2 (X^{(1)})^{\top}X^{(1)} + A_2^2 (X^{(2)})^{\top}X^{(2)})^{-1} (A_1 (X^{(1))})^{\top}Y^{(1)} + A_2 (X^{(2)})^{\top}Y^{(2)}). \label{eq_Bhat_2task} %\\
		%&= (B^\star A ^{\top}) (A A^{\top})^{-1} + (X^{\top}X)^{-1}X^{\top}   \bigbrace{\sum_{j=1}^t \varepsilon_i A_i^{\top}} (A  A^{\top})^{-1}.
	\end{align}
 Now inserting $\hat B$ back into equation \eqref{eq_mtl_2task_cov}, we obtain an objective that only depends on $A:=A_1/A_2$ as follows 
 \begin{align}
		 g(A) \define & \left\| X^{(1)} \Sigma(A)^{-1} (X^{(2)})^\top X^{(2)} (A\beta_2-\beta_1)+ \left(A^2 X^{(1)}\Sigma(A)^{-1} (X^{(1)})^\top-\id_{n_1}\right)\epsilon^{(1)} + A X^{(1)}\Sigma(A)^{-1} (X^{(2)})^\top \epsilon^{(2)} \right\|^2 \nonumber\\
		 + & \left\| X^{(2)} \Sigma(A)^{-1} (X^{(1)})^\top X^{(1)} (A\beta_1-A^2\beta_2)+ \left(X^{(2)}\Sigma(A)^{-1} (X^{(2)})^\top-\id_{n_2}\right)\epsilon^{(2)} + A X^{(2)}\Sigma(A)^{-1} (X^{(1)})^\top \epsilon^{(1)} \right\|^2, 
		\label{eq_mtl_A12}
	\end{align}
	where we denote
	$$ \Sigma(A):= A^2 (X^{(1)})^\top X^{(1)}  + (X^{(2)})^\top X^{(2)} .$$
We can calculate the partial expectation of $g(A)$ over the randomness in $\epsilon_1$ and $\epsilon_2$:
\begin{align*}	
&\exarg{\epsilon^{(1)}, \epsilon^{(2)}}{g(A) \mid X_1, X_2 ,\beta_1,\beta_2} \\
&= \left\| X^{(1)} \Sigma(A)^{-1} (X^{(2)})^\top X^{(2)} (A\beta_2-\beta_1)\right\|^2 +\left\| X^{(2)} \Sigma(A)^{-1} (X^{(1)})^\top X^{(1)} (A\beta_1-A^2\beta_2)\right\|^2  \\
& + \exarg{\epsilon^{(1)}}{(\epsilon^{(1)})^\top \left(A^2 X^{(1)}\Sigma(A)^{-1} (X^{(1)})^\top-\id_{n_1}\right)^2\epsilon^{(1)}} +\exarg{ \epsilon^{(2)}}{(\epsilon^{(2)})^\top \left(X^{(2)}\Sigma(A)^{-1} (X^{(2)})^\top-\id_{n_2}\right)^2\epsilon^{(2)}}   \\
&+  A^2 \exarg{ \epsilon^{(2)}}{ (\epsilon^{(2)} )^\top X^{(2)} \Sigma(A)^{-1}  (X^{(1)})^\top X^{(1)}\Sigma(A)^{-1} (X^{(2)})^\top \epsilon^{(2)}} \\
& +A^2 \exarg{ \epsilon^{(1)}} { (\epsilon^{(1)})^\top X^{(1)} \Sigma(A)^{-1}  (X^{(2)})^\top X^{(2)}\Sigma(A)^{-1} (X^{(1)})^\top \epsilon^{(1)}  }\\
		&= \left\| X^{(1)} \Sigma(A)^{-1} (X^{(2)})^\top X^{(2)} (A\beta_2-\beta_1)\right\|^2 +\left\| X^{(2)} \Sigma(A)^{-1} (X^{(1)})^\top X^{(1)} (A\beta_1-A^2\beta_2)\right\|^2  \\
		&+ \sigma^2\tr\left[ \left(A^2 X^{(1)}\Sigma(A)^{-1} (X^{(1)})^\top-\id_{n_1}\right)^2\right] +\sigma^2\tr\left[ \left(X^{(2)}\Sigma(A)^{-1} (X^{(2)})^\top-\id_{n_2}\right)^2 \right]\\
&+ 2 A^2 \sigma^2 \tr\left[ \Sigma(A)^{-1}  (X^{(1)})^\top X^{(1)}\Sigma(A)^{-1} (X^{(2)})^\top X^{(2)} \right]  \nonumber\\
		& = \left\| X^{(1)} \Sigma(A)^{-1} (X^{(2)})^\top X^{(2)} (\beta_1-A\beta_2)\right\|^2 +A^2\left\| X^{(2)} \Sigma(A)^{-1} (X^{(1)})^\top X^{(1)} (\beta_1-A\beta_2)\right\|^2 +\sigma^2(n_1+n_2-p).
\end{align*}
For the random-effects model, we can further take the partial expectation over $\beta_1$ and $\beta_2$: 
\begin{align}	
&\exarg{\epsilon^{(1)}, \epsilon^{(2)} ,\beta_1,\beta_2}{g(A) \mid X_1, X_2} \nonumber\\
&=[(A-1)^2\kappa^2 + (A^2+1)d^2/2] \cdot p^{-1}\tr\left[  (X^{(2)})^\top X^{(2)} \Sigma(A)^{-1}(X^{(1)})^\top X^{(1)} \Sigma(A)^{-1} (X^{(2)})^\top X^{(2)}\right] \nonumber\\
&+ A^2[(A-1)^2\kappa^2 + (A^2+1)d^2/2] \cdot p^{-1}\tr\left[  (X^{(1)})^\top X^{(1)} \Sigma(A)^{-1}(X^{(2)})^\top X^{(2)} \Sigma(A)^{-1} (X^{(1)})^\top X^{(1)}\right] +\sigma^2(n_1+n_2-p) \nonumber\\
&=[(A-1)^2\kappa^2 + (A^2+1)d^2/2] \cdot p^{-1} \tr\left[ (X^{(1)})^\top X^{(1)} \Sigma(A)^{-1}(X^{(2)})^\top X^{(2)}\right]+\sigma^2(n_1+n_2-p), \label{EgA}
\end{align}
where in the first step we used that the entries of $\beta_1-A\beta_2\in \R^p$ are i.i.d. Gaussian random variables with mean zero and variance $p^{-1}[(A-1)^2\kappa^2 + (A^2+1)d^2/2]$, and in the second step we used  %Using $(X^{(2)})^\top X^{(2)}=\hat\Sigma- A^2 (X^{(1)})^\top X^{(1)}$, we can further simplify that
\begin{align*}
 & (X^{(2)})^\top X^{(2)} \Sigma(A)^{-1}(X^{(1)})^\top X^{(1)} \Sigma(A)^{-1} (X^{(2)})^\top X^{(2)} + A^2   (X^{(1)})^\top X^{(1)} \Sigma(A)^{-1}(X^{(2)})^\top X^{(2)} \Sigma(A)^{-1} (X^{(1)})^\top X^{(1)}  \\
 &=  (X^{(2)})^\top X^{(2)} \Sigma(A)^{-1}(X^{(1)})^\top X^{(1)} \Sigma(A)^{-1} (X^{(2)})^\top X^{(2)} + A^2  (X^{(1)})^\top X^{(1)} \Sigma(A)^{-1}(X^{(1)})^\top X^{(1)} \Sigma(A)^{-1}(X^{(2)})^\top X^{(2)}  \\
 & =   (X^{(1)})^\top X^{(1)} \Sigma(A)^{-1}(X^{(2)})^\top X^{(2)} .
\end{align*}
Here in the first step of the above derivation, we used that 
 $$(X^{(1)})^\top X^{(1)} \Sigma(A)^{-1}(X^{(2)})^\top X^{(2)} =\left( A^2 [(X^{(2)})^\top X^{(2)}]^{-1} + [(X^{(1)})^\top X^{(1)}]^{-1} \right)^{-1} = (X^{(2)})^\top X^{(2)} \Sigma(A)^{-1}(X^{(1)})^\top X^{(1)}.$$
 
Using the concentration estimates in Lemma \ref{largedeviation}, we can control the difference between $g(A)$ and its partial expectation. %$\exarg{\epsilon^{(1)}, \epsilon^{(2)}}{g(A) \mid X_1, X_2 ,\beta_1,\beta_2}$.
 \begin{claim}\label{claim_largedev1}
 In the setting of Proposition \ref{cor_MTL_loss}, we have that with high probability,
$$\left| g(A) - \exarg{\epsilon^{(1)}, \epsilon^{(2)} ,\beta_1,\beta_2}{g(A) \mid X_1, X_2 }\right| \le p^{1/2+e_1} \left(\sigma^2 + \kappa^2+d^2 \right) $$
for any small constant $e_1>0$.
 \end{claim}
 \begin{proof}
By Fact \ref{lem_minv} (iii), we have that with high probability,
 \be\label{op_X12}
\|X^{(1)}\|\le \sqrt{(\sqrt{n_1} + \sqrt{p})^2 + n_1 \cdot p^{-c_{\varphi}}}\lesssim \sqrt{p}, \quad \|X^{(2)}\|\le \sqrt{(\sqrt{n_2} + \sqrt{p})^2 + n_2 \cdot p^{-c_{\varphi}}}\lesssim \sqrt{p}, 
 \ee
 and
 \be\label{op_Sig1}
\| \Sigma(A)^{-1}\|\le \frac1{A^2((\sqrt{n_1} - \sqrt{p})^2 - n_1 \cdot p^{-c_{\varphi}})+((\sqrt{n_2} - \sqrt{p})^2 - n_2 \cdot p^{-c_{\varphi}}) }\lesssim \frac{1}{(A^2+1)p},
 \ee
 where we also used that $1+\tau\le \rho_1\le \tau^{-1}$ and $1+\tau\le \rho_2\le \tau^{-1}$ for a small constant $\tau>0$. 
 
 Now we expand the first term on the right-hand side of \eqref{eq_mtl_A12}: 
 \begin{align*}
 & \left\| X^{(1)} \Sigma(A)^{-1} (X^{(2)})^\top X^{(2)} (A\beta_2-\beta_1)+ \left(A^2 X^{(1)}\Sigma(A)^{-1} (X^{(1)})^\top-\id_{n_1}\right)\epsilon^{(1)} + A X^{(1)}\Sigma(A)^{-1} (X^{(2)})^\top \epsilon^{(2)} \right\|^2 \nonumber\\
 &=h_1(A) + h_2(A) +h_3(A) + 2h_4(A) + 2h_5(A) + 2h_6(A) ,
 \end{align*}
 where
 \begin{align*}
&h_1(A):=  (\beta_1-A\beta_2)^\top (X^{(2)})^\top X^{(2)}  \Sigma(A)^{-1}  (X^{(1)})^\top X^{(1)} \Sigma(A)^{-1} (X^{(2)})^\top X^{(2)} (\beta_1-A\beta_2)  ,\\
& h_2(A) := {(\epsilon^{(1)})^\top \left(A^2 X^{(1)}\Sigma(A)^{-1} (X^{(1)})^\top-\id_{n_1}\right)^2\epsilon^{(1)}}  ,\\
& h_3(A):=  A^2 { (\epsilon^{(2)} )^\top X^{(2)} \Sigma(A)^{-1}  (X^{(1)})^\top X^{(1)}\Sigma(A)^{-1} (X^{(2)})^\top \epsilon^{(2)}} , \\
& h_4(A):=  (\epsilon^{(1)})^\top\left(A^2 X^{(1)}\Sigma(A)^{-1} (X^{(1)})^\top-\id_{n_1}\right) X^{(1)} \Sigma(A)^{-1} (X^{(2)})^\top X^{(2)} (A\beta_2-\beta_1), \\
&h_5(A):=A (\epsilon^{(2)})^\top X^{(2)}\Sigma(A)^{-1} (X^{(1)})^\top X^{(1)} \Sigma(A)^{-1} (X^{(2)})^\top X^{(2)} (A\beta_2-\beta_1),\\
&h_6(A):=A (\epsilon^{(2)})^\top X^{(2)}\Sigma(A)^{-1} (X^{(1)})^\top\left(A^2 X^{(1)}\Sigma(A)^{-1} (X^{(1)})^\top-\id_{n_1}\right)\epsilon^{(1)}.
\end{align*}
Next we estimate each term using Lemma \ref{largedeviation} in Appendix \ref{app_tool}.

For $h_1(A)$, using Lemma \ref{largedeviation} in Appendix \ref{app_tool} and the fact that the entries of $\beta_1-A\beta_2\in \R^p$ are i.i.d. Gaussian random variables with mean zero and variance $p^{-1}[(A-1)^2\kappa^2 + (A^2+1)d^2/2]$, we obtain the following estimate with high probability for any small constant $\e_1>0$:
\begin{align}
&\left|h_1(A) - \exarg{\beta_1,\beta_2}{h_1(A) \mid X_1, X_2}\right|\nonumber \\
&\le p^{\e_1}\cdot p^{-1}[(A-1)^2\kappa^2 + (A^2+1)d^2/2]\cdot \left\| (X^{(2)})^\top X^{(2)}  \Sigma(A)^{-1}  (X^{(1)})^\top X^{(1)} \Sigma(A)^{-1} (X^{(2)})^\top X^{(2)} \right\|_F \nonumber\\
&\le p^{e_1}\cdot p^{-1}[(A-1)^2\kappa^2 + (A^2+1)d^2/2]\cdot p^{1/2} \left\| (X^{(2)})^\top X^{(2)}  \Sigma(A)^{-1}  (X^{(1)})^\top X^{(1)} \Sigma(A)^{-1} (X^{(2)})^\top X^{(2)} \right\| \nonumber \\
&\lesssim p^{1/2+e_1}\cdot \frac{(A-1)^2\kappa^2 + (A^2+1)d^2/2}{(A^2+1)^2}\lesssim p^{1/2+e_1}(\kappa^2 + d^2).\label{eq_hA1}
\end{align}
 Here in the third step we used \eqref{op_X12} and \eqref{op_Sig1} to bound the operator norm:
\begin{align*}
&\left\| (X^{(2)})^\top X^{(2)}  \Sigma(A)^{-1}  (X^{(1)})^\top X^{(1)} \Sigma(A)^{-1} (X^{(2)})^\top X^{(2)} \right\| \\
&\le \left\| (X^{(2)})^\top X^{(2)}\right\|^2 \cdot \left\| (X^{(1)})^\top X^{(1)}\right\|\cdot \left\| \Sigma(A)^{-1} \right\|^2 \lesssim \frac{p}{(A^2+1)^2}.
\end{align*}  
Similarly, using Lemma \ref{largedeviation}, the fact that the entries of $\epsilon^{(1)},\epsilon^{(2)}\in \R^p$ are i.i.d. Gaussian random variables with mean zero and variances $\sigma^2$, and bounds \eqref{op_X12}-\eqref{op_Sig1}, we can obtain that with high probability,
\be\label{eq_hA23}\left|h_2(A) - \exarg{\epsilon^{(1)}}{h_2(A) \mid X_1, X_2}\right| \lesssim p^{1/2+\e_1}\sigma^2, \quad \left|h_3(A) - \exarg{\epsilon^{(2)}}{h_3(A) \mid X_1, X_2}\right| \lesssim p^{1/2+\e_1}\sigma^2.\ee
For $h_4(A)$, using Lemma \ref{largedeviation} in Appendix \ref{app_tool}, we obtain the following estimate with high probability for any small constant $\e_1>0$:
\begin{align}
\left|h_4(A)\right| &\le p^{e_1}\cdot \sigma \sqrt{p^{-1}[(A-1)^2\kappa^2 + (A^2+1)d^2/2]}\cdot \left\| \left(A^2 X^{(1)}\Sigma(A)^{-1} (X^{(1)})^\top-\id_{n_1}\right) X^{(1)} \Sigma(A)^{-1} (X^{(2)})^\top X^{(2)} \right\|_F\nonumber \\
&\le p^{e_1}\cdot \sigma \sqrt{p^{-1}[(A-1)^2\kappa^2 + (A^2+1)d^2/2]} \cdot p^{1/2} \left\| \left(A^2 X^{(1)}\Sigma(A)^{-1} (X^{(1)})^\top-\id_{n_1}\right) X^{(1)} \Sigma(A)^{-1} (X^{(2)})^\top X^{(2)} \right\|  \nonumber\\
&\lesssim p^{e_1}\cdot \sigma \sqrt{ (A-1)^2\kappa^2 + (A^2+1)d^2/2} \cdot \left\|  \id_{n_1}-A^2 X^{(1)}\Sigma(A)^{-1} (X^{(1)})^\top\right\|\cdot  \left\|X^{(1)}\right\| \left\|\Sigma(A)^{-1}\right\|\left\| (X^{(2)})^\top X^{(2)} \right\|  \nonumber\\
&\lesssim p^{1/2+e_1}\frac{\sigma \sqrt{ (A-1)^2\kappa^2 + (A^2+1)d^2/2}}{A^2+1} \lesssim p^{1/2+e_1}(\sigma^2 + \kappa^2 + d^2),\label{eq_hA4}
\end{align}
where in the fourth step we used $\left\|  \id_{n_1}-A^2 X^{(1)}\Sigma(A)^{-1} (X^{(1)})^\top\right\|\le 1$ and \eqref{op_X12}-\eqref{op_Sig1}, and in the last step we used AM-GM inequality. With same argument, we can show that 
\be\label{eq_hA56} \left|h_5(A)\right| \le p^{1/2+e_1}(\sigma^2 + \kappa^2 + d^2), \quad \left|h_6(A)\right| \le p^{1/2+e_1}\sigma^2,\ee
with high probability for any small constant $e_1>0$. %Combining \eqref{eq_hA1}, \eqref{eq_hA23}, \eqref{eq_hA4} and \eqref{eq_hA56}, we conclude that

Finally, we can obtain similar estimates for the second term on the right-hand side of \eqref{eq_mtl_A12} as in  \eqref{eq_hA1}, \eqref{eq_hA23}, \eqref{eq_hA4} and \eqref{eq_hA56}, which conclude the proof of Claim \ref{claim_largedev1}.
 \end{proof}


 \paragraph{Minimizer of the optimization objective.} We show that for the isotropic model, the minimizer of $g(A)$, denoted by $\hat A$, is very close to 1 under the condition \eqref{choiceofpara0}. This will be used to calculate the predication loss $\te(\hat{\beta}_2^{\MTL})$ loss.
%Our main observation is that the optimum of $\val(v)$ is approximately $1$.This allows us to simplify the bias and variance bounds significantly.
%Furthermore, under certain choice of parameters, we can simplify  
%More precisely, we have the following claim.
%Later on, we will combine it with Lemma \ref{prop_model_shift_tight} together to analyze the bias-variance tradeoff.

\begin{claim}\label{lem_hat_v}
%Suppose the assumptions of Lemma \ref{prop_model_shift_tight} hold. Assume that $ \kappa^2 \sim pd^2 \sim \sigma^2$ are of the same order.
 In the setting of Proposition \ref{cor_MTL_loss}, we have that with high probability,
%There exists a constant $C>0$ such that
	\be\label{hatw_add1}|\hat A -1|\le  \frac{d^2}{\kappa^2} + p^{-1/4+e_2}. 
	\ee
	for any small constant $e_2>0$.
	%is $\hat{w} = 1 \pm \bigo{\frac 1 {n_1+n_2}}$ \todo{(figure out the constants)}
\end{claim}

\begin{proof}
Corresponding to equation \eqref{EgA}, we define the function
\begin{align*}
h(A):= [(A-1)^2\kappa^2 + (A^2+1)d^2/2] \cdot p^{-1} \tr\left[ (X^{(1)})^\top X^{(1)} \Sigma(A)^{-1}(X^{(2)})^\top X^{(2)}\right].
\end{align*}
Using Fact \ref{lem_minv} (iii), we can obtain that with high probability,
\begin{align} 
&\left( A^2 \left[(\sqrt{n_2}-\sqrt{p})^{2}-n_2p^{-c_\varphi}\right]^{-1} +\left[(\sqrt{n_1}-\sqrt{p})^{2}-n_1p^{-c_\varphi}\right]^{-1} \right)^{-1} \nonumber\\
&\preceq (X^{(1)})^\top X^{(1)} \Sigma(A)^{-1}(X^{(2)})^\top X^{(2)} =\left( A^2 [(X^{(2)})^\top X^{(2)}]^{-1} + [(X^{(1)})^\top X^{(1)}]^{-1} \right)^{-1} \nonumber\\
&\preceq \left( A^2 \left[(\sqrt{n_2}+\sqrt{p})^{2}+n_2p^{-c_\varphi}\right]^{-1} +\left[(\sqrt{n_1}+\sqrt{p})^{2}+n_1p^{-c_\varphi}\right]^{-1} \right)^{-1}. \label{op_uplow}
\end{align}
Then we get that
\be\label{eq_h1}h(1)= d^2 \cdot p^{-1} \tr\left[ (X^{(1)})^\top X^{(1)} \Sigma(1)^{-1}(X^{(2)})^\top X^{(2)}\right]\lesssim pd^2\left(\rho_1^{-1} + \rho_2^{-1}\right)^{-1}\lesssim pd^2.\ee
where we used that $1+\tau\le \rho_1\le \tau^{-1}$ and $1+\tau\le \rho_2\le \tau^{-1}$. On the other hand, if $|A-1|\ge \delta$ for a small constant $\delta>0$, then we have 
\begin{align}\label{eq_hA}
h(A)\gtrsim (1+A^2)\kappa^2 \cdot p\left(\rho_1^{-1} + A^2 \rho_2^{-1}\right)^{-1} \gtrsim p\kappa^2 ,
\end{align}
where in the first step we used that $|A-1|^2\gtrsim 1+A^2$ for $|A-1|\ge \delta$, $1+\tau\le \rho_1\le \tau^{-1}$ and $1+\tau\le \rho_2\le \tau^{-1}$, and in the second step we used equation \eqref{choiceofpara0}. Hence using Claim \ref{claim_largedev1} and equations \eqref{eq_h1}-\eqref{eq_hA}, we obtain that with high probability,
\begin{align}
g(A)&\ge  h(A) +\sigma^2(n_1+n_2-p) -p^{-1/2+e_1}\cdot p\left(\sigma^2 +\kappa^2+d^2 \right) \nonumber\\
&\ge h(1) +\sigma^2(n_1+n_2-p) + p^{-1/2+e_1}\cdot p\left(\sigma^2 +\kappa^2+d^2 \right) \ge g(1) \label{gA>g1}
\end{align}
under condition \eqref{choiceofpara0}.
This gives that
%\be\label{rough A-1}
$|\hat A-1|\le \delta$ for any small constant $\delta>0$.




%Hence it remains to consider the region $|A-1|\le \delta$ for a small enough constant $\delta>0$. 
To obtain the better bound \eqref{hatw_add1}, we study the function $h(A)$ more closely and find its minimizer of $h(A)$, denoted by $A^\star$. 
%Using Lemma \ref{largedeviation} again, we can simplify equation \eqref{revise_eq_val_mtl} as $\val(v)= N_2h(v) \cdot  \left( 1+\OO(p^{-1/2+\e})\right)$, where the function $ h$ is defined as
%	%We define the function
%	\begin{align}
%		h(v) =& \frac{\rho_1}{\rho_2}\left[d^2 +\left( v-1\right)^2\kappa^2\right]\cdot \tr\left[( v^2X_1^{\top}X_1 +X_2^{\top}X_2)^{-2} (X_2^{\top}X_2)^2\right] \nonumber \\
%		& +  v^2\left[d^2 +\left( v-1\right)^2\kappa^2\right]\cdot \tr\left[( v^2X_1^{\top}X_1 +X_2^{\top}X_2)^{-2} (X_1^{\top}X_1)^2\right] \nonumber\\
%		& + \left(\frac{\rho_1}{\rho_2} v^2 + 1\right)\sigma^2 \cdot \bigtr{(v^2X_1^{\top}X_1  + X_2^{\top}X_2)^{-1} }. \nonumber
%	\end{align}
%	%\cor Under the setting of Lemma \ref{prop_model_shift_tight},
%	Furthermore, the validation loss in equation \eqref{approxvalid} reduces to
%	\be\label{boundv-w}
%		g(v)=\left[N_2 h(v) + (N_1+N_2)\sigma^2\right]\cdot \left( 1+\OO(p^{-(1-\e_0)/2+\e})\right)\quad \text{w.h.p.}
%	\ee
%	for any constant $\e>0$. %\nc Thus for the following discussions, it suffices to focus on the behavior of $h (v)$.
%Let $\hat w$ be the minimizer of $h(v)$. Our proof consists of the following two steps.
%\squishlist
%	\item First, we show that $\hat{w}$ is close to $1$.
%	\item Second, with equation \eqref{boundv-w} we show that $\hat v$ is close to $\hat w$. Then we plug $\hat{v}$ into $\te(\hat{\beta}_2^{\MTL})$ to get equation \eqref{simple1}.
%\squishend
%%For the minimizer $\hat w$ of $\val(w)$, we have a similar result as in Proposition \ref{thm_cov_shift}.
%For the first step, we will prove the following result.
%To be consistent with the notation $\hat w$, we change the name of the argument to $w$ in the proof. 
First it is easy to observe that $h(A)< h(-A)$ for $A> 0$. 
%$h(1)=\OO(pd^2 + \sigma^2)$ and $ h(w)\gtrsim p\kappa^2 \gg h(1)$ for $w\ge 2$ or $w\le 1/2$.  Hence it suffices to consider the case $1/2\le w\le 2$.
Then we consider the case $A\ge 1$. We write
\begin{align*}
h(A):= [(1-A^{-1})^2\kappa^2 + (1+A^{-2})d^2/2] \cdot p^{-1} \tr\left[\left([(X^{(2)})^\top X^{(2)}]^{-1} + A^{-2} [(X^{(1)})^\top X^{(1)}]^{-1}\right)^{-1}\right].
\end{align*}
Notice that
$$\tr\left[\left([(X^{(2)})^\top X^{(2)}]^{-1} + A^{-2} [(X^{(1)})^\top X^{(1)}]^{-1}\right)^{-1}\right]$$
is an increasing functions in $A$. Hence taking derivative of $h(w)$ with respect to $w$, we obtain that
\begin{align}\label{h'(A)1}
h'(A) \ge \left[2(1-A^{-1})\frac{\kappa^2}{A^2} - \frac{d^2}{A^3}\right] \cdot p^{-1} \tr\left[\left([(X^{(2)})^\top X^{(2)}]^{-1} + A^{-2} [(X^{(1)})^\top X^{(1)}]^{-1}\right)^{-1}\right] \ge 0,
\end{align}
as long as $A \ge 1+d^2/(2\kappa^2)$.  
%\begin{align}
%	h(w) =&  \frac{\rho_1}{\rho_2}\left[\frac{d^2}{w^4} +\frac{\left( w-1\right)^2}{w^4}\kappa^2\right]\cdot \tr\left[( X_1^{\top}X_1 +w^{-2}X_2^{\top}X_2)^{-2} (X_2^{\top}X_2)^2\right] \nonumber \\
%	& + \left[\frac{d^2}{w^2} +\frac{\left( w-1\right)^2}{w^2}\kappa^2\right]\cdot \tr\left[( X_1^{\top}X_1 +w^{-2}X_2^{\top}X_2)^{-2} (X_1^{\top}X_1)^2\right] \nonumber\\
%			& +  \frac{\rho_1}{\rho_2}\sigma^2  \cdot \bigtr{(X_1^{\top}X_1  + w^{-2}X_2^{\top}X_2)^{-1} }+ \sigma^2 \cdot \bigtr{(w^2X_1^{\top}X_1  + X_2^{\top}X_2)^{-1} }.\nonumber%+  \sigma^2n_2 \cdot \bigtr{(X_1^{\top}X_1  + w^{-2}X_2^{\top}X_2)^{-1} }. \nonumber
%\end{align}
%Notice that
%$$\tr\left[( X_1^{\top}X_1 +w^{-2}X_2^{\top}X_2)^{-2} (X_i^{\top}X_i)^2\right],\ \ i=1,2, \quad \text{and} \quad \bigtr{(X_1^{\top}X_1  + w^{-2}X_2^{\top}X_2)^{-1} }$$
%are increasing functions in $w$. Hence taking derivative of $h(w)$ with respect to $w$, we obtain that
%\begin{align*}
%h'(w) \ge & \frac{\rho_1}{\rho_2}\left[ \frac{2(w-1)(2-w)}{w^5}\kappa^2 - \frac{4d^2}{w^5}\right] \tr\left[( X_1^{\top}X_1 +w^{-2}X_2^{\top}X_2)^{-2} (X_2^{\top}X_2)^2\right]   \\
% &+ \left[\frac{2\left( w-1\right)}{w^3}\kappa^2 - \frac{2d^2}{w^3} \right]\cdot \tr\left[( X_1^{\top}X_1 +w^{-2}X_2^{\top}X_2)^{-2} (X_1^{\top}X_1)^2\right] \nonumber\\
%		&- 2  \frac{\sigma^2}{w^3} \cdot \bigtr{(X_1^{\top}X_1 + w^{-2}X_2^{\top}X_2)^{-2} X_1^\top X_1  } =  \bigtr{(X_1^{\top}X_1 + w^{-2}X_2^{\top}X_2)^{-2} \cal A },
%\end{align*}
%where the matrix $\cal A$ is defined as 
%\begin{align*}
%\cal A :&= \frac{\rho_1}{\rho_2}\left[ \frac{2(w-1)(2-w)}{w^5}\kappa^2 - \frac{4d^2}{w^5}\right](X_2^{\top}X_2)^2 + \left[\frac{2\left( w-1\right)}{w^3}\kappa^2 - \frac{2d^2}{w^3} \right](X_1^{\top}X_1)^2 - 2 \frac{\sigma^2}{w^3}X_1^\top X_1.
%\end{align*}
%Using the estimate Fact equation \eqref{lem_minv}, we get that $\cal A$ is lower bounded as
%\begin{align*}
%\cal A \succeq&~ - \frac{4d^2}{w^5}n_1n_2 (\al_+(\rho_2)+\oo(1))^2 + \left[\frac{2\left( w-1\right)}{w^3}\kappa^2 - \frac{2d^2}{w^3} \right]n_1^2 (\al_-(\rho_1)-\oo(1))^2 \\
%&~ - 2 \frac{\sigma^2}{w^3}n_1(\al_{+}(\rho_1)+\oo(1)) \succ 0,
%\end{align*}
%as long as
%$$w> w_1:=1 +\frac{d^2}{\kappa^2}+ \frac{\sigma^2}{n_1\kappa^2}\frac{\al_{+}(\rho_1)+\oo(1)}{\al_{-}^2(\rho_1)} + \frac{2d^2}{\kappa^2}\frac{\rho_2(\al_+^2(\rho_2)+\oo(1))}{\rho_1\al_-^2(\rho_1) }.$$
%Hence $h'(w)>0$ on $(w_1,\infty)$, %i.e. $h(w)$ is strictly increasing for $w>w_1$. This 
%which gives $\hat w\le w_1$.
Next we consider the case $1-\delta\le A\le 1$. Notice that
$$\tr\left[\left(A^2[(X^{(2)})^\top X^{(2)}]^{-1} +  [(X^{(1)})^\top X^{(1)}]^{-1}\right)^{-1}\right].$$
is a decreasing functions in $A$. Hence taking derivative of $h(A)$, we obtain that
\begin{align}\label{h'(A)2}
	h'(A)\le [-2(1-A)\kappa^2 + Ad^2] \cdot p^{-1} \tr\left[\left(A^2[(X^{(2)})^\top X^{(2)}]^{-1} + [(X^{(1)})^\top X^{(1)}]^{-1}\right)^{-1}\right]\le 0,
\end{align}
as long as $A \le 1-d^2/(2\kappa^2)$. In sum, we see that the minimizer $A^\star$ must satisfy
$$1-d^2/(2\kappa^2)\le A^\star \le 1+d^2/(2\kappa^2).$$ 

%Taking derivative of $h(w)$, we obtain that
%\begin{align}
%	h'(w) \le& \frac{\rho_1}{\rho_2} \left[2\left( w-1\right) \kappa^2\right]\cdot \tr\left[( w^2X_1^{\top}X_1 +X_2^{\top}X_2)^{-2} (X_2^{\top}X_2)^2\right] \nonumber \\
%	& +  \left[2wd^2 +2w\left( w-1\right)(2w-1)\kappa^2\right]\cdot \tr\left[( w^2X_1^{\top}X_1 +X_2^{\top}X_2)^{-2} (X_1^{\top}X_1)^2\right] \nonumber\\
%	& + \frac{\rho_1}{\rho_2}(2 w\sigma^2) \cdot \bigtr{(w^2X_1^{\top}X_1 + X_2^{\top}X_2)^{-2} X_2^\top X_2  }= \frac{\rho_1}{\rho_2} \bigtr{(w^2X_1^{\top}X_1  + X_2^{\top}X_2)^{-1} \cal B} , \nonumber
%\end{align}
%where the matrix $\cal B$ is
%$$\cal B= 2\left( w-1\right) \kappa^2  (X_2^{\top}X_2)^2+\frac{\rho_2}{\rho_1}\left[2wd^2 +2w\left( w-1\right)(2w-1)\kappa^2\right](X_1^{\top}X_1)^2 + 2 w\sigma^2 X_2^\top X_2 .$$
%Using Fact \ref{lem_minv}, we get that $\cal B$ is upper bounded as
%\begin{align*}
%\cal B \preceq - 2(1-w)\kappa^2 n_2^2 (\al_-(\rho_2) -\oo(1))^2 +2w d^2 n_1n_2 (\al_+(\rho_1) +\oo(1))^2 + 2w\sigma^2 n_2 (\al_+(\rho_2)+\oo(1)) \prec 0,
%\end{align*}
%as long as
%$$w< w_2:=1 -   \frac{d^2}{\kappa^2}\frac{\rho_1(\al_+(\rho_1) +\oo(1))^2}{\rho_2 \al_-^2(\rho_2) } -  \frac{\sigma^2}{n_2\kappa^2}\frac{\al_{+}(\rho_2)+\oo(1)}{\al_{-}^2(\rho_2)} .$$
%Hence $h'(w)<0$ on $[0,w_2)$, %i.e. $h(w)$ is strictly decreasing for $w<w_2$. This 
%which gives $\hat w\ge w_2$.

%In sum, we have shown that $w_2\le w\le w_1$. Together with
%$$\max\{|w_1 -1|, |w_2 -1|\} =\OO\left(\frac{d^2}{\kappa^2} + \frac{\sigma^2}{p\kappa^2}\right),$$
%we conclude equation \eqref{hatw_add1}.


%For the rest of this section, we choose parameters that satisfy the following relations: \be\label{choiceofpara}
%pd^2 \sim \sigma^2 \sim 1,\quad p^{-1+c_0} \sigma^2 \le \kappa^2  \le p^{-\e_0-c_0}\sigma^2 ,
%\ee
%for some small constant $c_0>0$. 

Now we are ready to prove equation \eqref{hatw_add1}. First using equations \eqref{op_uplow} and  \eqref{h'(A)1}, we obtain that for $1+d^2/\kappa^2 \le A\le 1+\delta$,
$$h'(A)\ge \frac{2(A-1)\kappa^2 - d^2}{A^3} \cdot p^{-1} \tr\left[\left([(X^{(2)})^\top X^{(2)}]^{-1} + A^{-2} [(X^{(1)})^\top X^{(1)}]^{-1}\right)^{-1}\right] \gtrsim p\left(d^2 + |A-1|\kappa^2\right).$$
Thus we get that if $1+2d^2/\kappa^2 +p^{-1/4+e_2}\le A\le 1+\delta$,
\begin{align*}
h(A)-h(1)\ge h(A)-h(1+d^2/\kappa^2) \ge \int_{1+d^2/\kappa^2}^A h'(x)\dd x \gtrsim p\kappa^2 \cdot |A-1|^2  \gtrsim p\frac{d^2}{\kappa^2} + p^{-1/2+2e_2}\cdot p\kappa^2.
\end{align*}
Then if we choose the constant $c$ in Claim \ref{claim_largedev1} such that $e_1<2e_2$, we have 
$$ h(A)-h(1)\gtrsim  p^{-1/2+2e_2}\cdot p\kappa^2 \gg p^{-1/2+e_1}\cdot p\left(\sigma^2 +\kappa^2+d^2 \right)$$
under equation \eqref{choiceofpara0}, which implies the inequality \eqref{gA>g1}. Second using equations \eqref{op_uplow} and  \eqref{h'(A)2}, we obtain that for $1-\delta \le A \le 1-d^2/\kappa^2$,
$$- h'(A)\ge  [2(1-A)\kappa^2 - Ad^2] \cdot p^{-1} \tr\left[\left(A^2[(X^{(2)})^\top X^{(2)}]^{-1} + [(X^{(1)})^\top X^{(1)}]^{-1}\right)^{-1}\right]  \gtrsim p\left(d^2 + |A-1|\kappa^2\right).$$
Then with a similar argument as above, we can get that if $1-2d^2/\kappa^2 - p^{-\frac12c_\infty}\le A\le 1-\delta$, then 
$$ h(A)-h(1) \gg p^{-1/2+e_1}\cdot p\left(\sigma^2 +\kappa^2+d^2 \right)$$
under equation \eqref{choiceofpara0}, which implies the inequality \eqref{gA>g1}. In sum, we obtain that the minimizer $\hat A$ must satisfy equation \eqref{hatw_add1}.
%Next we prove the following estimate on the optimizer $\hat A$: with high probability,
%%\begin{lemma}
%%For the isotropic model, we have
%\be\label{hatv_add1}
%|\hat v - 1|= \OO\left(\cal E\right), \quad \cal E:=\frac{d^2}{\kappa^2} + \frac{\sigma^2}{p\kappa^2} + p^{-1/2 + \e_0 /2+ 2\e}.
%\ee
%%\end{lemma}
%%\begin{proof}
% In fact, from the proof of Claim \ref{lem_hat_v} above, one can check that if $C\cal E \le |w-1| \le 2C\cal E$ for a large enough constant $C>1$, then $|h'(w)|\gtrsim \sigma^2$. Moreover, under equation \eqref{choiceofpara} we have
%$$h(w) =\OO(\sigma^2),\quad \text{for}\quad   |w-1|\le 2C\cal E.$$
%Thus we obtain that for $|w-1|\ge 2C\cal E$,
%$$\left|h(w) - h(\hat w)\right|\ge |h(w)-\min\{h(w_1),h(w_2)\}|\gtrsim \sigma^2 \cal E \gtrsim \cal E \cdot h(\hat w),$$
%which leads to $g(w) > g(\hat w)$ with high probability by equation \eqref{boundv-w}. Thus $w$ cannot be a minimizer of $g(v)$, and we must have $|\hat v - 1|\le 2C\cal E$. %Together with equation \eqref{hatw_add1}, we conclude equation \eqref{hatv_add1}.
%%\end{proof}
%
%{\cor require $\kappa^2\gg d^2 + p^{-1/2+c}\sigma^2$}
\end{proof}

\paragraph{Calculating the prediction loss.} Now we calculate the predication loss using the optimal $\hat B$ in equation \eqref{eq_Bhat_2task} and optimal $\hat A=\hat A_1/\hat A_2$:
\begin{align}
L(\hat{\beta}_2^{\MTL}) &=\left|\Sigma_2^{1/2} \left( \hat B \hat A_2 - \beta_2\right)\right| \nonumber\\
&=  \left\| (\Sigma^{(2)})^{1/2}\hat \Sigma^{-1} \left[(X^{(1)})^\top X^{(1)} (\hat A\beta_1-\hat A^2\beta_2)+ (X^{(2)})^\top \epsilon^{(2)} + \hat A   (X^{(1)})^\top \epsilon^{(1)} \right]\right\|^2,\label{Lbeta_HPS}
\end{align}
where we denoted
$$\hat \Sigma:= \hat A^2 (X^{(1)})^\top X^{(1)}  + (X^{(2)})^\top X^{(2)} .$$
To illustrate the idea of bias-variance decomposition, we assume that $\hat A$ is independent of $\epsilon^{(1)}$, $\epsilon^{(2)}$, $\beta_1$ and $\beta_2$ for now. Then we can calculate the partial expectation of $L(\hat{\beta}_2^{\MTL})$ over the randomness in $\epsilon_1$ and $\epsilon_2$:
 \begin{align}
	&\exarg{\epsilon^{(1)}, \epsilon^{(2)}}{L(\hat{\beta}_2^{\MTL}) \mid X_1, X_2,\beta_1,\beta_2} \nonumber \\
&	=  \bignorm{(\Sigma^{(2)})^{1/2}\hat\Sigma^{-1} (X^{(1)})^{\top}X^{(1)} (\hat A \beta_1 - \hat A^2\beta_2)}^2 \nonumber\\ %\label{eq_bias_2task} 
			& + \sigma^2\cdot \bigtr{ X^{(2)} \hat\Sigma^{-1}\Sigma^{(2)}\hat\Sigma^{-1} (X^{(2)})^\top}+ \sigma^2\cdot \hat A^2 \bigtr{ X^{(1)} \hat\Sigma^{-1}\Sigma^{(2)}\hat\Sigma^{-1} (X^{(1)})^\top}\nonumber\\
			%+\sigma^2\cdot \bigtr{\Sigma_2(\frac{W_1^2}{W_2^2} X_1^{\top}X_1 + X_2^{\top}X_2)^{-1} } \nonumber\\ 
			&= \bignorm{(\Sigma^{(2)})^{1/2}\hat\Sigma^{-1} (X^{(1)})^{\top}X^{(1)} (\hat A \beta_1 - \hat A^2\beta_2)}^2 +\sigma^2\cdot \tr\left(\Sigma^{(2)}\hat \Sigma^{-1}\right).\nonumber
			%\label{eq_var_2task}
\end{align}
This gives a bias-variance decomposition of the MTL prediction loss: the first term on the right-hand side is the bias term, while the second term is the variance term. For the random-effects model, we further take the partial expectation over $\beta_1$ and $\beta_2$: 
 \begin{align}
	\exarg{\epsilon^{(1)}, \epsilon^{(2)},\beta_1,\beta_2}{L(\hat{\beta}_2^{\MTL}) \mid X_1, X_2}&	=  \left[(\hat A-1)^2 \kappa^2 + (\hat A^2+1)\frac{d^2}2\right] \cdot p^{-1}\bigtr{ (X^{(1)})^{\top}X^{(1)} \hat\Sigma^{-1} \Sigma^{(2)} \hat\Sigma^{-1} (X^{(1)})^{\top}X^{(1)} }\nonumber \\
	&+\sigma^2\cdot \tr\left(\Sigma^{(2)}\hat \Sigma^{-1}\right).
			\label{eq_var_2task}
\end{align}
One can compare the above expression with the predication loss for STL, that is,
\begin{align}
	\exarg{\epsilon_2}{L(\hat{\beta}_2^{\STL}) \mid X_2} = \sigma^2 \cdot \bigtr{\Sigma_2 (X_2^{\top} X_2)^{-1}}, \label{eq_var_stl}
\end{align}
we observe that while the bias of $\hat{\beta}_2^{\MTL}$ is always larger than that of $\hat{\beta}_2^{\STL}$, which is zero, the variance of $\hat{\beta}_2^{\MTL}$ is always lower than that of $\hat{\beta}_2^{\STL}$.\footnote{To see why this is true, we apply the Woodbury matrix identity over the second term on the right-hand side of equation \eqref{eq_var_2task} and use the fact that for the product of two PSD matrices, its trace is always nonnegative.}
In other words, training both tasks together helps predict the target task by reducing variance while incurring a bias.
Therefore, whether multi-task learning outperforms single-task learning is determined by the bias-variance decomposition!

%Plugging $Y^{(1)}$ and $Y^{(2)}$ in equation \eqref{model_YvsX} into equation \eqref{eq_Bhat_2task}, we can rewrite that 
% \begin{align}
%		\hat{B}= \frac{1}{A_2}\left[\beta_2 + \left(\frac{A_1^2}{A_2^2}(X^{(1)})^{\top}X^{(1)} +  (X^{(2)})^{\top}X^{(2)}\right)^{-1}\bigbrace{ (X^{(1)})^{\top}X_1\left(\frac{A_1}{A_2}\beta_1 - \frac{A_1^2}{A_2^2} \beta_2\right) + \left(\frac{A_1}{A_2} (X^{(1)})^{\top}\varepsilon_1 + (X^{(2)})^{\top}\varepsilon_2\right)}\right]. \label{eq_Bhat_2task2} %\\
%		%&= (B^\star A ^{\top}) (A A^{\top})^{-1} + (X^{\top}X)^{-1}X^{\top}   \bigbrace{\sum_{j=1}^t \varepsilon_i A_i^{\top}} (A  A^{\top})^{-1}.
%	\end{align}

In general, for $\hat A$ as  the minimizer of $g(A)$, it depends on $\epsilon^{(1)}$, $\epsilon^{(2)}$, $\beta_1$ and $\beta_2$. However, using the concentration estimates in Lemma \ref{largedeviation}, we can still obtain the expression on the right-hand side of \eqref{eq_var_2task} up to a small error. In particular, our proof will not use Lemma \ref{lem_hat_v}.

%control the difference between $g(A)$ and its partial expectation. 
 \begin{claim}\label{claim_largedev2}
 In the setting of Proposition \ref{cor_MTL_loss}, if $\hat A=\OO(1)$, then we have that with high probability,
$$\left| L(\hat{\beta}_2^{\MTL})- \cal L(\hat A)\right| \le p^{-1/2+e}\cdot p\left(\sigma^2 +\kappa^2 +d^2 \right),$$
for any small constant $e>0$. Here we introduced the function
\begin{align*}
	\cal L(\hat A)	:=  \left[(\hat A-1)^2 \kappa^2 + (\hat A^2+1)\frac{d^2}2\right] \cdot p^{-1}\bigtr{ (X^{(1)})^{\top}X^{(1)} \hat\Sigma^{-1} \Sigma^{(2)} \hat\Sigma^{-1} (X^{(1)})^{\top}X^{(1)} } +\sigma^2\cdot \tr\left(\Sigma^{(2)}\hat \Sigma^{-1}\right).
\end{align*}
 \end{claim}
 \begin{proof}
 The proof of this claim is very similar to the one for Claim \ref{claim_largedev1}, where the only difference is that $\hat A$ now may depend on $\epsilon^{(1)}$, $\epsilon^{(2)}$, $\beta_1$ and $\beta_2$. In this case, we can still use  Lemma \ref{largedeviation} to conclude the proof because $\hat A$ can be pulled out as a coefficient. Here we only give an outline of the proof without writing down all the details. We expand $L(\hat{\beta}_2^{\MTL}) $ in \eqref{Lbeta_HPS} as:
 \begin{align*}
L(\hat{\beta}_2^{\MTL}) &\cor  =  \left\| (\Sigma^{(2)})^{1/2}\hat \Sigma^{-1} \left[(X^{(1)})^\top X^{(1)} (\hat A\beta_1-\hat A^2\beta_2)+ (X^{(2)})^\top \epsilon^{(2)} + \hat A   (X^{(1)})^\top \epsilon^{(1)} \right]\right\|^2\\
&=h_1(A) + h_2(A) +h_3(A) + 2h_4(A) + 2h_5(A) + 2h_6(A) ,
\end{align*}
where

 \end{proof}
 
In the specific setting in Proposition \ref{cor_MTL_loss}, we can further simplify $\cal L(\hat A)$ using Claim \ref{lem_hat_v} and $\Sigma^{(1)}=\Sigma^{(2)}=\id_p$.
 
\begin{claim}
In the setting of Proposition \ref{cor_MTL_loss}, we have that with high probability,
%\be\label{choiceofpara}
%	\frac{pd^2}{\sigma^2} = \OO(1),\quad p^{-1+c_0} \le \frac{\kappa^2}{\sigma^2 }  \le p^{-\e_0-c_0},
%\ee
%where $\e_0$ is the constant as appeared in equation \eqref{approxvalid}. Then we have
\begin{align}
&\left|\cal L(\hat A)- d^2\cdot {p}^{-1}\tr\left[\Sigma(1)^{-2} \left((X^{(1)})^\top X^{(1)} \right)^2\right] -\sigma^2  \bigtr{\Sigma(1)^{-1}  }\right| \nonumber\\
&\lesssim  \frac{d^4 + \sigma^2 d^2}{\kappa^2}+p^{-1/2+2e}\kappa^2 +  p^{-1/4+e} (\sigma^2 +d^2) , \label{simple1}
\end{align}
for any small constant $e>0$, where we recall that $\Sigma(1)= (X^{(1)})^\top X^{(1)}  + (X^{(2)})^\top X^{(2)}$.
\end{claim}
\begin{proof}
Using Lemma \ref{lem_hat_v} and Fact \ref{lem_minv} (iii), we can obtain that with high probability,
\begin{align}
&\|\Sigma(1)^{-1}-\hat\Sigma^{-1}\| \le |\hat A-1|\|\Sigma(1)^{-1}\| \| (X^{(1)})^\top X^{(1)}\|\|\hat\Sigma^{-1}\| \label{est111}\\
&\lesssim \left(\frac{d^2}{\kappa^2} + p^{-1/4+e}\right)\frac{(\sqrt{n_1}+\sqrt{p})^{2}+n_1p^{-c_\varphi}}{ \left[(\sqrt{n_2}-\sqrt{p})^{2}-n_2p^{-c_\varphi}  + (\sqrt{n_1}-\sqrt{p})^{2}-n_1p^{-c_\varphi}\right]^2  }\lesssim p^{-1}\left(\frac{d^2}{\kappa^2} + p^{-1/4+e}\right),\nonumber
\end{align}
for any small constant $e>0$. With similar arguments, we can get that with high probability,
\begin{align}\label{est222}
&\left\|\left(\Sigma(1)^{-2}-\hat\Sigma^{-2}\right)\left((X^{(1)})^\top X^{(1)} \right)^2\right\| \lesssim  \frac{d^2}{\kappa^2} + p^{-1/4+e} .
\end{align}
Moreover, using Fact \ref{lem_minv} (iii) we can bound that with high probability,
$$\bignorm{ (X^{(1)})^{\top}X^{(1)} \hat\Sigma^{-1} \Sigma^{(2)} \hat\Sigma^{-1} (X^{(1)})^{\top}X^{(1)}}\lesssim 1.$$
which further implies that with high probability,
\be\label{est333}\bigtr{ (X^{(1)})^{\top}X^{(1)} \hat\Sigma^{-1} \Sigma^{(2)} \hat\Sigma^{-1} (X^{(1)})^{\top}X^{(1)}}\lesssim p.\ee
Now we bound the left-hand side of \eqref{simple1} as
\begin{align*}
&\left|\cal L(\hat A)-\frac{d^2}{p}\tr\left[\Sigma(1)^{-2} \left((X^{(1)})^\top X^{(1)} \right)^2\right] -\sigma^2  \bigtr{\Sigma(1)^{-1}  }\right| \\
&\lesssim \left(|\hat A-1|^2 \kappa^2 + |\hat A-1|d^2\right)\cdot p^{-1}\bigtr{ (X^{(1)})^{\top}X^{(1)} \hat\Sigma^{-1} \Sigma^{(2)} \hat\Sigma^{-1} (X^{(1)})^{\top}X^{(1)} } \\
&+ \frac{d^2}{p}\left|\tr\left[\left(\Sigma(1)^{-2}-\hat\Sigma^{-2}\right) \left((X^{(1)})^\top X^{(1)} \right)^2\right]\right|+ \sigma^2  \left|\bigtr{\Sigma(1)^{-1}-\hat\Sigma^{-1}  }\right|.
\end{align*}
Applying the estimates \eqref{hatw_add1}, \eqref{est111}, \eqref{est222} and \eqref{est333} to the three terms on the right-hand side, we can conclude \eqref{simple1}.
% Inserting equation \eqref{hatv_add1} into equation \eqref{eq_te_mtl_2task} and applying Lemma  \ref{largedeviation} to $(\beta_1-\hat v\beta_s)$, we get that w.h.p.,
%\begin{align}
%\te(\hat{\beta}_2^{\MTL})&=(1+\OO(\cal E))\cdot \left[d^2 + \OO\left(\cal E^2 \kappa^2\right)\right] \tr\left[(X_1^{\top}X_1 +X_2^{\top}X_2)^{-2} (X_1^{\top}X_1)^2\right]\nonumber\\
%&+(1+\OO(\cal E))\cdot \sigma^2  \bigtr{(X_1^{\top}X_1  + X_2^{\top}X_2)^{-1} }. \label{1Liso}
%\end{align}
%In order to study the phenomenon of bias-variance trade-off, we need the bias term with $d^2$ and the variance term with $\sigma^2$ to be of the same order. Using Fact equation \eqref{lem_minv}, we have that
%$$\tr\left[(X_1^{\top}X_1 +X_2^{\top}X_2)^{-2} (X_1^{\top}X_1)^2\right] \sim p,\quad \bigtr{(X_1^{\top}X_1  + X_2^{\top}X_2)^{-1} } \sim 1.$$
%Hence we need to choose that $p d^2 \sim \sigma^2$. On the other hand, we want the error term $\cal E^2 \kappa^2$ to be much smaller than $d^2$, which leads to the condition $p^{-1+\e_0+4\e}\kappa^2  \ll d^2 \ll \kappa^2$. The above considerations lead to the choices of parameters in equation \eqref{choiceofpara}, under which we can simplify equation \eqref{1Liso} to equation \eqref{simple1}.
%\begin{align}
%\te(\hat{\beta}_2^{\MTL})&=(1+\OO(n^{-\e}))\cdot d^2 \tr\left[(X_1^{\top}X_1 +X_2^{\top}X_2)^{-2} (X_1^{\top}X_1)^2\right] \nonumber \\
%&+(1+\OO(n^{-\e}))\cdot \sigma^2  \bigtr{(X_1^{\top}X_1  + X_2^{\top}X_2)^{-1} }  \label{simple1}
%\end{align}
%w.h.p. for some constant $\e>0$.
\end{proof}
  
% \todo{switch the $a_3$ and $a_4$}
\paragraph{Calculating the variance and bias terms.} Finally, we calculate the variance term $\sigma^2  \bigtr{\Sigma(1)^{-1}  }$ and the bias term $d^2\cdot {p}^{-1}\tr\left[\Sigma(1)^{-2} \left((X^{(1)})^\top X^{(1)} \right)^2\right]$ using Theorem \ref{thm_main_RMT}, which will conclude Proposition \ref{cor_MTL_loss}.
%For the isotropic model in Section \ref{sec_similarity}, we actually have an easier and sharper bound than Lemma \ref{thm_model_shift}. %as follows.
%\begin{lemma}\label{prop_model_shift_tight}
%		In the setting of Lemma \ref{thm_model_shift}, assume that $\Sigma_1 =\id$,
%		$\beta_2$ is a random vector that has i.i.d. entries with mean $0$, variance $\kappa^2$ and finite moments up to any order, and $(\beta_1-\beta_2)$ is a random vector that is independent of $\beta_2$ and has i.i.d. entries with mean $0$, variance $d^2$ and finite moments up to any order. Denote
%		$\Delta^\star_{\bias} := \bigbrace{(1 - \hat{v})^2 \kappa^2 + d^2} \bigtr{\Pi(\hat v)}$, where $\Pi$ was defined in equation \eqref{defnpihat}.	Then we have with high probability,
%		\begin{align*}
%			\te(\hat{\beta}_t^{\MTL}) \le \te(\hat{\beta}_t^{\STL})\quad \text{ when: }\ \  & \Delta_{\vari} \ge  (\al_+^2(\rho_1)+\oo(1))\cdot  \Delta^\star_{\bias} \ , \\
%			\te(\hat{\beta}_t^{\MTL}) \ge \te(\hat{\beta}_t^{\STL})\quad \text{ when: }\ \  & \Delta_{\vari} \le (\al_-^2(\rho_1)-\oo(1))\cdot  \Delta^\star_{\bias}\ ,
%		\end{align*}
%		where $\Delta_{\vari}$ was defined in equation \eqref{Deltavarv}.
%\end{lemma}

\begin{proof}[Proof of Proposition \ref{cor_MTL_loss}]
%The proof of Lemma \ref{prop_model_shift_tight} is similar to that for Lemma \ref{thm_model_shift}, except that we can replace equation \eqref{bounddelta-} with a tighter bound. We only describe the main difference in the following proof.
%For any fixed $v\in \R$, $\beta_1 - v\beta_2$ is a random vector with i.i.d. entries with mean $0$ and variance $(1-v)^2\kappa^2 + d^2$. Then using the concentration result, Lemma \ref{largedeviation}, we get that for any constant $\e>0$, 
%\begin{align}
%&\left|\delta_{\bias}(v)- [(1-v)^2\kappa^2 + d^2]\tr (\cal K^\top \cal K)\right| \nonumber\\
%&=  \left|(\beta_1 - {v} \beta_2)^\top \cal K^\top \cal K (\beta_1 - {v} \beta_2) - [(1-v)^2\kappa^2 + d^2]\tr (\cal K^\top \cal K)\right| \nonumber\\
%&\le p^\e [(1-v)^2\kappa^2 + d^2] \left\{\tr\left[(\cal K^\top \cal K)^2\right]\right\}^{1/2} \lesssim p^{1/2+\e} [(1-v)^2\kappa^2 + d^2] \quad \text{w.h.p.},\label{anotherbeta}
%\end{align}
%where $\delta_{\bias}(v)$ was defined in equation \eqref{revise_deltabias}, $\cal K$ was defined as $\cal K:=v\Sigma_2^{1/2}({v}^2 X_1^{\top}X_1 + X_2^{\top}X_2)^{-1} X_1^{\top}X_1$, and in the last step we used $\|\cal K\|=\OO(1)$ by Fact \ref{lem_minv}. Now for $\tr (\cal K^\top \cal K)$, we rewrite it as
%\begin{align*}
%v^2 \tr \left[ ({v}^2 X_1^{\top}X_1 + X_2^{\top}X_2)^{-1}\Sigma_2({v}^2 X_1^{\top}X_1 + X_2^{\top}X_2)^{-1}(X_1^\top X_1)^2 \right].
%\end{align*}
%Bounding $(X_1^\top X_1)^2=(Z_1^\top Z_1)^2$ 
Using \eqref{lem_cov_shift_eq}, we obtain that 
\be\label{eq_var111}\bigtr{\Sigma(1)^{-1}  } = \tr\left[ \left((X^{(1)})^\top X^{(1)}  + (X^{(2)})^\top X^{(2)}\right)^{-1}\right]=\bigtr{\frac{(a_1\id_p+a_2\id_p)^{-1}}{n_{1}+n_2} }+\OO(p^{-c_\varphi})\ee
with high probability. Then solving equation \eqref{eq_a12extra} with $\lambda_i\equiv 1$, $1\le i\le p$, we get that with high probability,
	\begin{align}
		 a_1 = \frac{\rho_1(\rho_1 + \rho_2 - 1)}{(\rho_1 + \rho_2)^2} ,\quad
		& a_2 = \frac{\rho_2(\rho_1 + \rho_2 - 1)}{(\rho_1 + \rho_2)^2} . \label{simplesovlea12}
			\end{align}
Inserting it into \eqref{eq_var111}, we obtain that 
\be\label{eq_var112}\bigtr{\Sigma(1)^{-1}  } = \frac{p}{n_1+n_2} \cdot \frac{\rho_1+\rho_2}{\rho_1+\rho_2-1}+\OO(p^{-c_\varphi})=  \frac{1}{\rho_1+\rho_2-1}+\OO(p^{-c_\varphi})\ee
with high probability. 

On the other hand, using Fact \ref{lem_minv} (iii), we can bound that with high probability,
\begin{align}
 \frac{(\sqrt{n_1}-\sqrt{p})^4 \cdot (1- p^{-c_\varphi})}{p}\sum_{i=1}^p \left(\Sigma(1)^{-2}\right)_{ii}  &\le {p}^{-1}\tr\left[\Sigma(1)^{-2} \left((X^{(1)})^\top X^{(1)} \right)^2\right] \label{eq_bias111}\\ 
 &\le \frac{(\sqrt{n_1}+\sqrt{p})^4 \cdot (1+ p^{-c_\varphi})}{p}\sum_{i=1}^p \left(\Sigma(1)^{-2}\right)_{ii} .\nonumber
\end{align}
To obtain this inequality, we used  
$$ (\sqrt{n_1}-\sqrt{p})^4 \cdot (1-p^{-c_\varphi}) \prec \left((X^{(1)})^\top X^{(1)} \right)^2 \preceq (\sqrt{n_1}+\sqrt{p})^4 \cdot (1+ p^{-c_\varphi}) \quad \text{with high probability},$$
by Fact \ref{lem_minv} (iii), and the fact that for the product of two PSD matrices, its trace is always nonnegative.
Using \eqref{lem_cov_derv_eq} with $\Sigma^{(1)}=\Sigma^{(2)}=\Lambda=V=\id_p$ and $w$ being the $i$-th coordinate vector, we can calculate that 
\begin{align}\label{eq_bias112}
 \left(\Sigma(1)^{-2}\right)_{ii} = \frac{1}{(n_1+n_2)^2}\cdot \frac{a_3+a_4+1}{(a_1+a_2)^2} +\OO(p^{-c_\varphi})
\end{align}
with high probability. Solving equation \eqref{eq_a34extra} with $a_1, a_2$ in \eqref{simplesovlea12} and $\lambda_i\equiv 1$, $1\le i\le p$, we can obtain that
	\begin{align}
		% a_1 = \frac{\rho_1(\rho_1 + \rho_2 - 1)}{(\rho_1 + \rho_2)^2} ,\quad & a_2 = \frac{\rho_2(\rho_1 + \rho_2 - 1)}{(\rho_1 + \rho_2)^2} , \label{simplesovlea12}\\
		a_3 = \frac{\rho_1}{(\rho_1 + \rho_2)(\rho_1 + \rho_2 - 1)}, \quad
		&  a_4 = \frac{\rho_2}{(\rho_1 + \rho_2)(\rho_1 + \rho_2 - 1)}.\label{simplesovlea34}
	\end{align}
Inserting them into \eqref{eq_bias112}, we obtain that 
\begin{align*}%\label{eq_bias113}
 \left(\Sigma(1)^{-2}\right)_{ii} = \frac{1}{(n_1+n_2)^2}\cdot \frac{(\rho_1+\rho_2)^3}{(\rho_1+\rho_2-1)^3} +\OO(p^{-c_\varphi})
\end{align*}
with high probability. In fact, we need this estimate to hold simultaneously for all $1\le i \le p$ with high probability. For this purpose, we shall use \eqref{apply derivlocal} to get that 
\begin{align*}%\label{eq_bias113}
\left| \left(\Sigma(1)^{-2}\right)_{ii} - \frac{1}{(n_1+n_2)^2}\cdot \frac{(\rho_1+\rho_2)^3}{(\rho_1+\rho_2-1)^3}\right|\prec p^{-\frac{\varphi-4}{2\varphi}}
\end{align*}
on a high probability event that does not depend on $i$. Then using Fact \ref{lem_stodomin} (i), we obtain that 
\begin{align*} 
\left| \sum_{i=1}^p \left(\Sigma(1)^{-2}\right)_{ii} - \frac{p}{(n_1+n_2)^2}\cdot \frac{(\rho_1+\rho_2)^3}{(\rho_1+\rho_2-1)^3}\right|\prec p^{1-\frac{\varphi-4}{2\varphi}}
\end{align*}
with high probability, which by Definition \ref{stoch_domination} gives that
\begin{align}\label{eq_bias113}
\left| p^{-1}\sum_{i=1}^p \left(\Sigma(1)^{-2}\right)_{ii} - \frac{1}{(n_1+n_2)^2}\cdot \frac{(\rho_1+\rho_2)^3}{(\rho_1+\rho_2-1)^3}\right|\le p^{-c_\varphi}
\end{align}
with high probability. Inserting \eqref{eq_bias113} into \eqref{eq_bias111}. we get
\begin{align}
& \frac{(\sqrt{n_1}-\sqrt{p})^4 \cdot (1- p^{-c_\varphi}) - n_1^2 \cdot (1+\OO(p^{-c_\varphi}))}{(n_1+n_2)^2}\cdot  \frac{(\rho_1+\rho_2)^3}{(\rho_1+\rho_2-1)^3}\nonumber \\
&\le  {p}^{-1}\tr\left[\Sigma(1)^{-2} \left((X^{(1)})^\top X^{(1)} \right)^2\right] - \frac{n_1^2}{(n_1+n_2)^2}\cdot  \frac{(\rho_1+\rho_2)^3}{(\rho_1+\rho_2-1)^3} \nonumber\\
&  \le \frac{(\sqrt{n_1}+\sqrt{p})^4 \cdot (1+ p^{-c_\varphi}) - n_1^2 \cdot (1+\OO(p^{-c_\varphi}))}{(n_1+n_2)^2}\cdot  \frac{(\rho_1+\rho_2)^3}{(\rho_1+\rho_2-1)^3}\label{eq_bias114}
\end{align}
with high probability. 

Now plugging \eqref{eq_var112} and \eqref{eq_bias114} into \eqref{simple1}, we get
\begin{align}
\left|\cal L(\hat A)- -\frac{\sigma^2}{\rho_1+\rho_2-1} - d^2\cdot \frac{\rho_1^2 (\rho_1+\rho_2)}{(\rho_1 + \rho_2 - 1)^3}\right| \le   \left[\left( 1+\frac{1}{\sqrt{\rho_1}}\right)^4-1\right] d^2\cdot \frac{\rho_1^2 (\rho_1+\rho_2)}{(\rho_1 + \rho_2 - 1)^3} \nonumber\\
+C\left( p^{-c_\varphi} (\sigma^2 + d^2)+ \frac{d^4 + \sigma^2 d^2}{\kappa^2}+p^{-1/2+2e}\kappa^2 +  p^{-1/4+e} (\sigma^2 +d^2) \right) \label{simple1add}
\end{align}
with high probability for some constant $C>0$. Then using Claim \ref{claim_largedev2} and noticing that the constant $e>0$ can be arbitrarily small, we conclude \eqref{cor_MTL_error}.
\end{proof}
%\begin{align}
%				a_3 + a_4 = \frac{1}{n_1 + n_2}\sum_{i=1}^p \frac{1}{\lambda_i^2 a_1 + a_2},
%				a_3 + \frac{1}{n_1 + n_2} \sum_{i=1}^p \frac{\lambda_i^2 (a_1 a_4 - a_2 a_3)}{(\lambda_i^2 a_1 + a_2)^2} = \frac{1}{n_1 + n_2} \sum_{i=1}^p \frac{\lambda_i^2 a_1}{(\lambda_i^2 a_1 + a_2)^{2}}.
%%				\left(\frac{\rho_1}{a_1^{2}} -  b_2  \right)\cdot  a_3 -  b_1 \cdot  a_4 = b_1,\quad \left(\frac{\rho_2}{a_2^{2}}-  b_0\right)\cdot  a_4 - b_1 \cdot  a_3
%%				= b_0.
%			\end{align}
%\be\label{anotherwtbeta} \delta^\star_{\bias}(v) \cdot (\al_-^2(\rho_1)-\OO(p^{-1/2+\e})) \le  [(1-v)^2\kappa^2 + d^2]\tr (\cal K^\top \cal K) \le \delta^\star_{\bias}(v) \cdot  (\al_+^2(\rho_1)+\OO(p^{-1/2+\e})),\ee
%where
%$$\delta^\star_{\bias}(v):=n_1^2 v^2 [(1-v)^2\kappa^2 + d^2]\tr \left[ ({v}^2 X_1^{\top}X_1 + X_2^{\top}X_2)^{-1}\Sigma_2({v}^2 X_1^{\top}X_1 + X_2^{\top}X_2)^{-1}\right].$$
%Note that $\delta^\star_{\bias}(v) \sim p$ by Fact \ref{lem_minv}. Hence combining equation \eqref{anotherbeta} and equation \eqref{anotherwtbeta} we get
%\be\label{replaceest}
%  \delta^\star_{\bias}(v) \cdot (\al_-^2(\rho_1)-\OO(p^{-1/2+\e})) \le  \delta_{\bias}(v)\le \delta^\star_{\bias}(v) \cdot  (\al_+^2(\rho_1)+\OO(p^{-1/2+\e})).
%\ee
%Now we can replace the estimate equation \eqref{bounddelta-} with this stronger estimate, and repeat all the other parts of the proof of Lemma \ref{thm_model_shift} to conclude Lemma \ref{prop_model_shift_tight}. In particular, one can calculate $  \delta^\star_{\bias}(v)$ using Lemma \ref{lem_cov_derivative} and get the $  \Delta^\star_{\bias}(v)$ term. 

%In the setting of Proposition \ref{prop_dist_transition}, we have $M = \Sigma_1^{1/2}\Sigma_2^{-1/2} = \id$. Then solving equations equation \eqref{eq_a12extra} and equation \eqref{eq_a34extra} with $\lambda_i\equiv 1$, we get that
%	\begin{align}
%		 a_1 = \frac{\rho_1(\rho_1 + \rho_2 - 1)}{(\rho_1 + \rho_2)^2} ,\quad
%		& a_2 = \frac{\rho_2(\rho_1 + \rho_2 - 1)}{(\rho_1 + \rho_2)^2} , \label{simplesovlea12}\\
%		a_3 = \frac{\rho_1}{(\rho_1 + \rho_2)(\rho_1 + \rho_2 - 1)}, \quad
%		&  a_4 = \frac{\rho_2}{(\rho_1 + \rho_2)(\rho_1 + \rho_2 - 1)}.\label{simplesovlea34}
%	\end{align}
%	Using Lemma \ref{lem_minv} and Lemma \ref{lem_cov_shift}, we can track the reduction of variance from $\hat{\beta}_2^{\MTL}$ to $\hat{\beta}_2^{\STL}$ as 
%\be\label{Deltavar}
%\begin{split}
%\delta_{\vari}&:=\sigma^2  \bigtr{(X_2^{\top}X_2)^{-1} }  - (1+\OO(n^{-\e}))\cdot \sigma^2  \bigtr{(X_1^{\top}X_1  + X_2^{\top}X_2)^{-1} } =\Delta_{\vari}\cdot (1+\OO(n^{-\e})) 
%\end{split}
%\ee
%with high probability, where 
%	\begin{align*}
%		\Delta_{\vari} &\define \sigma^2 \bigbrace{\frac{1}{\rho_2 - 1} - \frac{1}{\rho_1 + \rho_2}\cdot\frac{1}{a_1+ a_2} } =\sigma^2  \cdot \frac{\rho_1}{(\rho_2-1)(\rho_1 + \rho_2 -1)}.
%	\end{align*}
%	%where we use equation \eqref{eq_a12} and Lemma \ref{lem_hat_v}.
%	Next for the model shift bias
%	$$\delta_{\bias}:=(1+\OO(n^{-\e}))\cdot d^2 \tr\left[(X_1^{\top}X_1 +X_2^{\top}X_2)^{-2} (X_1^{\top}X_1)^2\right], $$
%	we can get from Lemma \ref{prop_model_shift_tight} that
%\be\label{Deltabeta} 
%\al_-^2(\rho_1) - \oo(1)  \le \frac{\delta_{\bias}}{ \Delta_{\bias}} \le \al_+^2(\rho_1) +  \oo(1) , \ee
%	where 
%	$$\Delta_{\bias}:=pd^2 \cdot \frac{\rho_1^2}{(\rho_1+\rho_2)^2}  \frac{1 + a_3 + a_4}{(a_1 + a_2)^2}= pd^2 \cdot \frac{\rho_1^2 (\rho_1+\rho_2)}{(\rho_1 + \rho_2 - 1)^3}.$$
 
\iffalse

\subsection{Covariate Shift}\label{app_proof_33}
 
Finally, we prove Proposition \ref{prop_covariate}, which shows that $\te(\hat{\beta}_2^{\MTL})$ is minimized approximately when $M$ is a scalar matrix, provided that there is enough source data.

%\begin{proof}[Proof of Claim \ref{claim_covar_shift}]	
%	With the same arguments as in Claim \ref{lem_hat_v}, we can show that equation \eqref{hatv_add1} holds. Moreover, if the parameters are chosen such that $p^{-1+c_0} \sigma^2 \le \kappa^2  \le p^{-\e_0-c_0}\sigma^2$ as in equation \eqref{choiceofpara}, we can simplify $g(M_0)$ as
%\be \nonumber
%\begin{split}
%g(M_0)&=(1+\OO(p^{-\e}))\cdot \sigma^2  \bigtr{\Sigma_2(X_1^{\top}X_1  + X_2^{\top}X_2)^{-1} }  \quad \text{w.h.p.}
%\end{split}
%\ee
% for some constant $\e>0$. In fact, Lemma \ref{lem_hat_v} was proved assuming that $M=\id$, but its proof can be easily extended to the case with general $M\in \cal S_{\mu}$ by using that $ \mu_{\min}\le \lambda_p(M)\le \lambda_1(M)\le \mu_{\max}$. We omit the details here. 
% \end{proof}

\begin{proof}[Proof of Proposition \ref{prop_covariate}]
Denote the minimizer of $g$ by 
$$M_0:=\argmin_{M\in \cal S_{\mu}}g(M).$$ 
We now calculate $g(M_0)$. With the same arguments as in Claim \ref{lem_hat_v}, we can show that equation \eqref{hatv_add1} holds. Moreover, if the parameters are chosen such that $p^{-1+c_0} \sigma^2 \le \kappa^2  \le p^{-\e_0-c_0}\sigma^2$ as in equation \eqref{choiceofpara}, we can simplify $g(M_0)$ as
\be \nonumber
\begin{split}
g(M_0)&=(1+\OO(p^{-\e}))\cdot \sigma^2  \bigtr{\Sigma_2(X_1^{\top}X_1  + X_2^{\top}X_2)^{-1} }  \quad \text{w.h.p.}
\end{split}
\ee
 for some constant $\e>0$. In fact, Lemma \ref{lem_hat_v} was proved assuming that $M=\id$, but its proof can be easily extended to the case with general $M\in \cal S_{\mu}$ by using that $ \mu_{\min}\le \lambda_p(M)\le \lambda_1(M)\le \mu_{\max}$. We omit the details here. 

Now using Lemma \ref{lem_cov_shift}, we obtain that with high probability,
\begin{align}\label{gvar_extra}
g(M_0)= \frac{\sigma^2}{\rho_1+\rho_2}\cdot \frac1p\tr\left( \frac{1}{a_1(M_0)\cdot M_0^\top M_0 + a_2(M_0)}\right) \cdot \left(1 +\OO(p^{-\e})\right).
\end{align}
From equation \eqref{eq_a12extra}, it is easy to obtain the following estimates on $ a_1(M)$ and $a_2(M)$ for any $M\in \cal S_\mu$:
\be\label{est_a12extra}
\frac{\rho_1-1}{\rho_1+\rho_2} < a_1(M)<  \frac{\rho_1+\rho_2-1}{\rho_1+\rho_2},\quad a_2(M) < \frac{\rho_2}{\rho_1+\rho_2}.
\ee
Inserting equation \eqref{est_a12extra} into equation \eqref{gvar_extra} and using $ M_0^\top M_0\succeq \mu_{\min}^2$, we obtain that with high probability,
\begin{align}\label{approximateteM}
\left(1+\frac{\rho_2}{(\rho_1-1)\mu_{\min}^2}\right)^{-1}h(M_0) \cdot \left(1 - \OO(p^{-\e})\right) \le g(M_0) \le h(M_0) \cdot \left(1 +\OO(p^{-\e})\right),
\end{align}
where
$$h(M_0):=\frac{\sigma^2}{(\rho_1+\rho_2)a_1(M_0)}\cdot \frac1p\tr\left( \frac{1}{M_0^\top M_0}\right) .$$
%With these two bounds, we can easily conclude equation \eqref{approxteM}. 
%
%We have that the test error satisfies
%\be\label{approxteM}  te(M)\left(1 -  \frac{n_2}{n_1-p} \frac{1}{\lambda_p^2 + \frac{n_2}{n_1-p}}\right)  \le  \frac{\sigma^2}{n_1+n_2}\tr\left( \frac{1}{a_1M^\top M + a_2}\right) \le te(M),\ee
%where $\lambda_p$ is the smallest singular value of $p$ and
%$$te(M):= \frac{\sigma^2}{a_1(n_1+n_2)}\tr\left( \frac{1}{M^\top M}\right) .$$
%Moreover, for all $M$ satisfying equation \eqref{GMcons}, the minimum of $te(M)$ is attained when $M= a\id$.
By AM-GM inequality, we observe that 
$$\tr\left( \frac{1}{M^\top M}\right) = \sum_{i=1}^p\frac{1}{\lambda_i^2}$$
is minimized when $\lambda_1 = \cdots\lambda_p=\mu$ under the restriction $\prod_{i=1}^p\lambda_i\le \mu^p$. Hence we get that 
\be\label{AMGM} h(M_0) \le \frac{\sigma^2}{\mu^2 (\rho_1+\rho_2)a_1(M_0)}.\ee
On the other hand, when $M=\mu \id$, applying Lemma \ref{lem_cov_shift} we obtain that with high probability,
\begin{align}\label{gvar_extra2}
\begin{split}
g(\mu \id)&= \frac{\sigma^2}{\rho_1+\rho_2}\cdot \frac1p\tr\left( \frac{1}{\mu^2 a_1 (\mu\id) + a_2(\mu\id)}\right) \cdot \left(1 +\OO(p^{-\e})\right)\\
&\le \frac{\sigma^2}{\mu^2(\rho_1+\rho_2)a_1 (\mu\id)}.
\end{split}
\end{align}
Combining equation \eqref{est_a12extra}, equation \eqref{approximateteM}, equation \eqref{AMGM} and equation \eqref{gvar_extra2}, we conclude the proof.
%, we conclude that the sum $\sum_{i=1}^p\lambda_i^{-1}$ is smallest when $\lambda_1=\cdots=\lambda_p = a$.
\end{proof}
 
 
In the simulation of Figure \ref{fig_covariate}, we observe the following two phases as  $n_1 / p$ increases.
When $n_1 \le n_2$, having complementary covariance matrices leads to lower test error compared to the case when $\Sigma_1 = \Sigma_2$.
When $n_1 > n_2$, having complementary covariance matrices leads to higher test error compared to the case when $\Sigma_1 = \Sigma_2$.
We provide a theoretical justification now.

\begin{proof}[Theoretical justification of Example \ref{ex_complement}]
We denote the test error as $\te(\hat \beta_{2}^{\MTL},\lambda)$ in the setting where $M$ has $p/2$ singular values that are equal to $\lambda$ and $p/2$ singular values that are equal to $1 / \lambda$. Then equations in equation \eqref{eq_a12extra} become
\be\label{compleeq} a_1 + a_2 = 1 - \frac{1}{\rho_1 + \rho_2},  \ \ a_1 + \frac{1}{2(\rho_1 + \rho_2)}\cdot \bigbrace{\frac{a_1}{a_1 + \lambda^2 a_2} + \frac{a_1}{a_1 + \lambda^{-2} a_2}} = \frac{\rho_1}{\rho_1 + \rho_2}. \ee
%It's not hard to verify that there is only one valid solution $(a_1,a_2)$ to equation \eqref{compleeq}. 
After solving these, with equation \eqref{gvar_extra} we get that w.h.p.,
\be\label{testcomple}
 \te(\hat \beta_{2}^{\MTL},\lambda)= \frac{\sigma^2}{2(\rho_1 + \rho_2)}(1+\OO(p^{-\e}))\cdot f(\lambda) ,\quad f(\lambda): = \frac{1}{\lambda^{-2}{a_1} + a_2} + \frac{1}{\lambda^2a_1 + a_2}.\ee

%First we notice that the curves in Figure  \ref{fig_model_shift_phasetrans} (c) all cross at the point $n_1=n_2$. In fact, if $n_1=n_2$, then it is easy to observe that $a_1=a_2=(1-\gamma)/2$ is the solution to equation \eqref{compleeq}, where we denote $ \gamma=p/(n_1+n_2)$. Then for any $\lambda$, the test error in equation \eqref{testcomple} takes the value
%$$te(\lambda)= \frac{\gamma}{2}\frac{1}{(1-\gamma)/2}=\frac{p}{n_1+n_2-p}.$$

%Second, from Figure \ref{fig_te_complement} we observe that the complementary cases with $\lambda>1$ is better than the case without covariate shift (i.e. $M=\id$ case) when $n_1<n_2$. On the other hand, if we have enough source task data such that $n_1>n_2$, then it is always better to have no covariate shift.
We now study the behavior of $f$ as $\lambda$ changes. 
%This phenomenon can be also explained using our theory. 
We abbreviate $\gamma:=(\rho_1 + \rho_2)^{-1}$. Then with equation \eqref{compleeq}, we can rewrite
$$f(\lambda)= \frac{1}{\lambda^{-2}{a_1} + (1-\gamma - a_1)} + \frac{1}{\lambda^2a_1 + (1-\gamma - a_1)}.$$
Then we can compute that
\begin{align*}
f(\lambda) - f(1)&= \frac{ \lambda^2-1}{1-\gamma} a_1\cdot \bigbrace{  \frac{1}{ -a_1(\lambda^2-1)+(1-\gamma)\lambda^2 } - \frac{1}{a_1(\lambda^2-1) + (1-\gamma)}} \\
&= \frac{(\lambda^2-1)^2}{1-\gamma}  a_1\cdot  \frac{2a_1 - (1-\gamma) }{[-a_1(\lambda^2-1)+(1-\gamma)\lambda^2 ][a_1(\lambda^2-1) + (1-\gamma)]} .
\end{align*}
From this expressions, we observe the following behaviors.
\begin{itemize}
\item[(i)] If $n_1>n_2$, we have $a_1>(1-\gamma)/2$ (because $a_1>a_2$ as observed from the equation \eqref{compleeq}). Hence $f(\lambda)>f(1)$, which gives $\te(\hat \beta_{2}^{\MTL},\lambda)>\te(\hat \beta_{2}^{\MTL},1)$.

\item[(ii)] If $n_1< n_2$, we have $a_1< (1-\gamma)/2$. Hence $f(\lambda)< f(1)$, which gives $\te(\hat \beta_{2}^{\MTL},\lambda)<\te(\hat \beta_{2}^{\MTL},1)$. 

\item[(iii)] If $n_1=n_2$, we have $f(\lambda)=f(1)=2/(1-\gamma)$, which means $\te(\hat \beta_{2}^{\MTL},\lambda)$ and $\te(\hat \beta_{2}^{\MTL},1)$ are roughly the same. %, which explains why the curves in Figure  \ref{fig_model_shift_phasetrans} (c) all cross at the point $n_1=n_2$.
\end{itemize}
This also partially justifies Proposition \ref{prop_covariate}. %Theorem  the observations in Figure \ref{fig_model_shift_phasetrans} (c), 
\end{proof}
\fi
 


\iffalse
\subsection{Task Similarity}\label{app_proof_31}

With equation \eqref{simple1} and Lemma \ref{prop_model_shift_tight}, we can prove Proposition \ref{prop_dist_transition}, which gives a transition threshold with respect to the ratio between the model bias and the noise level. %With slight abuse of notations, we shall write $\hat a_i$ and $\hat b_k$ as $a_i$ and $b_k$ throughout the proof. 

\begin{proof}[Proof of Proposition \ref{prop_dist_transition}]
	In the setting of Proposition \ref{prop_dist_transition}, we have $M = \Sigma_1^{1/2}\Sigma_2^{-1/2} = \id$. Then solving equations equation \eqref{eq_a12extra} and equation \eqref{eq_a34extra} with $\lambda_i\equiv 1$, we get that
	\begin{align}
		 a_1 = \frac{\rho_1(\rho_1 + \rho_2 - 1)}{(\rho_1 + \rho_2)^2} ,\quad
		& a_2 = \frac{\rho_2(\rho_1 + \rho_2 - 1)}{(\rho_1 + \rho_2)^2} , \label{simplesovlea12}\\
		a_3 = \frac{\rho_1}{(\rho_1 + \rho_2)(\rho_1 + \rho_2 - 1)}, \quad
		&  a_4 = \frac{\rho_2}{(\rho_1 + \rho_2)(\rho_1 + \rho_2 - 1)}.\label{simplesovlea34}
	\end{align}
	Using Lemma \ref{lem_minv} and Lemma \ref{lem_cov_shift}, we can track the reduction of variance from $\hat{\beta}_2^{\MTL}$ to $\hat{\beta}_2^{\STL}$ as 
\be\label{Deltavar}
\begin{split}
\delta_{\vari}&:=\sigma^2  \bigtr{(X_2^{\top}X_2)^{-1} }  - (1+\OO(n^{-\e}))\cdot \sigma^2  \bigtr{(X_1^{\top}X_1  + X_2^{\top}X_2)^{-1} } =\Delta_{\vari}\cdot (1+\OO(n^{-\e})) 
\end{split}
\ee
with high probability, where 
	\begin{align*}
		\Delta_{\vari} &\define \sigma^2 \bigbrace{\frac{1}{\rho_2 - 1} - \frac{1}{\rho_1 + \rho_2}\cdot\frac{1}{a_1+ a_2} } =\sigma^2  \cdot \frac{\rho_1}{(\rho_2-1)(\rho_1 + \rho_2 -1)}.
	\end{align*}
	%where we use equation \eqref{eq_a12} and Lemma \ref{lem_hat_v}.
	Next for the model shift bias
	$$\delta_{\bias}:=(1+\OO(n^{-\e}))\cdot d^2 \tr\left[(X_1^{\top}X_1 +X_2^{\top}X_2)^{-2} (X_1^{\top}X_1)^2\right], $$
	we can get from Lemma \ref{prop_model_shift_tight} that
\be\label{Deltabeta} 
\al_-^2(\rho_1) - \oo(1)  \le \frac{\delta_{\bias}}{ \Delta_{\bias}} \le \al_+^2(\rho_1) +  \oo(1) , \ee
	where 
	$$\Delta_{\bias}:=pd^2 \cdot \frac{\rho_1^2}{(\rho_1+\rho_2)^2}  \frac{1 + a_3 + a_4}{(a_1 + a_2)^2}= pd^2 \cdot \frac{\rho_1^2 (\rho_1+\rho_2)}{(\rho_1 + \rho_2 - 1)^3}.$$	
%\begin{align*}
%		\Delta_{\bias} &\define \hat{v}^2 \bignorm{(\hat{v}^2 X_1^{\top}X_1 + X_2^{\top}X_2)^{-1} X_1^{\top}X_1 (\beta_s - \hat{v}\beta_t)}^2 \\
%		&= d^2 \cdot \bignormFro{(\hat{v}^2 X_1^{\top}X_1 + X_2^{\top}X_2)^{-1}X_1^{\top}X_1}^2 + \bigo{p^{-1/2 + \varepsilon}d^2}
%	\end{align*}
%	Using Lemma \ref{prop_model_shift_tight}, we get an upper and lower bound on $\Delta_{\bias}$ as
%	\be\label{Deltabeta} \bigbrace{1 - \sqrt{\frac{1}{c_1}}}^4 \le \Delta_{\bias} / \bigbrace{p\cdot d^2 \cdot \frac{c_1^2}{(c_1+c_2)^2} \cdot \frac{1 + a_3 + a_4}{(a_1 + a_2)^2} + \bigo{p^{1/2+\e}d^2} } \le \bigbrace{1 + \sqrt{\frac{1}{c_1}}}^4. \ee	
%	Hence we obtain that
%	\begin{align*}
%		\frac{1 + a_3 + a_4}{(a_1 + a_2)^2}
%		= \frac{(c_1 + c_2)^3}{(c_1 + c_2 - 1)^3} + \bigo{p^{-1/2+\varepsilon}}.
%	\end{align*}
Note that by equation \eqref{simple1}, we have
 \be\label{var-beta}\te(\hat{\beta}_2^{\STL})-\te(\hat{\beta}_2^{\MTL}) =\delta_{\vari} - \delta_{\bias} .\ee
Then we can track its sign using equation \eqref{Deltavar} and equation \eqref{Deltabeta}.

\vspace{5pt}

\noindent{\bf Positive transfer.} With equation \eqref{Deltavar} and equation \eqref{Deltabeta}, we conclude that if
\be\label{upper101}pd^2 \cdot \frac{\rho_1^2 (\rho_1+\rho_2)}{(\rho_1 + \rho_2 - 1)^3} \cdot \left( \al_+^2(\rho_1) +  \oo(1) \right) < \sigma^2  \cdot \frac{\rho_1}{(\rho_2-1)(\rho_1 + \rho_2 -1)},\ee
then $\delta_{\vari} > \delta_{\bias}$, which implies $\te(\hat{\beta}_2^{\MTL})<\te(\hat{\beta}_2^{\STL})$. We can simplify equation \eqref{upper101} to
	\be\label{pos1}  \frac{pd^2}{\sigma^2}  <   \Phi(\rho_1, \rho_2)\cdot \left( \al_+^2(\rho_1) +  \oo(1) \right)^{-1}, \ee
Since $ \Psi(\beta_1,\beta_2)=pd^2/\sigma^2$ and $\nu\ge \al_+^2(\rho_1) +  \oo(1) $, it gives the first statement of Proposition \ref{prop_dist_transition}.	

 \vspace{5pt}
	
\noindent{\bf Negative transfer.} On the other hand, if
\be\label{upper102}pd^2 \cdot \frac{\rho_1^2 (\rho_1+\rho_2)}{(\rho_1 + \rho_2 - 1)^3} \cdot \left( \al_-^2(\rho_1) -  \oo(1) \right) > \sigma^2  \cdot \frac{\rho_1}{(\rho_2-1)(\rho_1 + \rho_2 -1)},\ee
	we have that $\delta_{\vari} < \delta_{\bias}$, which implies $\te(\hat{\beta}_2^{\MTL})>\te(\hat{\beta}_2^{\STL})$. We can simplify equation \eqref{upper102} to
	\be\label{neg1} \frac{ p d^2}{\sigma^2}  >  \Phi(\rho_1, \rho_2) \cdot \left( \al_-^2(\rho_1) -  \oo(1) \right)^{-1}, \ee
which gives the second statement of Proposition \ref{prop_dist_transition}.
\end{proof}







\subsection{Sample Ratio} \label{app_proof_32}
We first prove Proposition \ref{prop_data_size}, which describes the effect of source/task data ratio on the information transfer. 

\begin{proof}[Proof of Proposition \ref{prop_data_size}]
Following the proof of Proposition \ref{prop_dist_transition}, we know that $\te(\hat{\beta}_2^{\MTL})< \te(\hat{\beta}_2^{\STL})$ w.h.p. if equation \eqref{pos1} holds, while $\te(\hat{\beta}_2^{\MTL})>\te(\hat{\beta}_2^{\STL})$ w.h.p. if equation \eqref{neg1} holds. 
%\be\label{pos1}
%pd^2  \cdot \frac{\rho_1 (\rho_1 + \rho_2) (\rho_2 - 1)}{(\rho_1 + \rho_2 - 1)^2}\bigbrace{1 + \sqrt{\frac{1}{\rho_1}}}^4 <  \sigma^2  \bigbrace{1 - \oo(1)},\ee
%and we get from equation \eqref{upper102} that $\te(\hat{\beta}_t^{\MTL})>\te(\hat{\beta}_t^{\STL})$ w.h.p. if
%\be\label{neg1}pd^2  \cdot \frac{\rho_1 (\rho_1 + \rho_2) (\rho_2 - 1)}{(\rho_1 + \rho_2 - 1)^2}\bigbrace{1 - \sqrt{\frac{1}{\rho_1}}}^4 >  \sigma^2  \bigbrace{1 + \oo(1)}.\ee
%then we have $\te(\hat{\beta}_t^{\MTL})> \te(\hat{\beta}_t^{\STL})$ w.h.p..

We first explain the meaning of the condition 
\be\label{explainpsi}\Psi(\beta_1,\beta_2)>2/(\rho_2-1).\ee 
%For the first statement of Proposition \ref{prop_data_size}, we 
Notice that the function
$$ \Phi(\rho_1, \rho_2)=\frac{(\rho_1 + \rho_2 - 1)^2}{\rho_1 (\rho_1 + \rho_2) (\rho_2 - 1)}=\frac{1}{\rho_2-1} \left(1 +\frac{\rho_2-2}{\rho_1}+\frac{1}{\rho_1(\rho_1+\rho_2)}\right)$$
is strictly decreasing with respect to $\rho_1$ as long as $\rho_2> 2$, and $ \Phi(\rho_1, \rho_2)$ converges to $(\rho_2-1)^{-1}$ as $\rho_1\to \infty$. Since $\left( \al_-^2(\rho_1) -  \oo(1) \right)^{-1} < 2$ for $\rho_1>40$,  equation \eqref{explainpsi} implies that equation \eqref{neg1} holds for large enough $\rho_1$. Moreover, the transition from positive transfer when $\rho_1$ is small to negative transfer when $\rho_1$ is large is described by the two bounds in Proposition \ref{prop_data_size}, which we will show now. 
% $$ \Psi(\beta_1,\beta_2) \ge \bigbrace{1 + \sqrt{\frac{1}{\rho_1}}}^{-4}\frac{\sigma^2}{\rho_2-1}\left(1 -\oo(1)\right),$$ 
% then equation \eqref{pos1} holds, which shows that  $\te(\hat{\beta}_t^{\MTL}) < \te(\hat{\beta}_t^{\STL})$ holds w.h.p.. Plugging into $\rho_1>50$, we obtain the first statement. ...... Note that for $\rho_1>40$, we have 
%$\gamma_+^{-1}$
%The two bounds follows directly from equation \eqref{pos1} and equation \eqref{neg1}. 
In the following proof, we will use the following trivial inequalities 
\be\label{trivialphi}
\frac{(\rho_2-1) \rho_1 }{\rho_1+\rho_2-2}\cdot \left( 1-\frac{1}{(\rho_1+\rho_2-2)^2}\right) \le  \frac1{\Phi(\rho_1, \rho_2)} \le  \frac{(\rho_2-1) \rho_1 }{\rho_1+\rho_2-2}.
\ee

%This shows that equation \eqref{neg1} holds as long as $p$ is large enough, and hence $\te(\hat{\beta}_t^{\MTL})> \te(\hat{\beta}_t^{\STL})$ holds. 
%
%If $c_1 > \frac{(c_2-2) \sigma^2}{(1-a^{-1/2})^4(1-(a+c_2-2)^{-2})(c_2 - 1) pd^2 - \sigma^2}$, we have
%Suppose that
%$$pd^2 > (1 - a^{-1/2})^{-4}\frac{\sigma^2}{c_2-1}.$$ 

\noindent{\bf Positive transfer.} With equation \eqref{trivialphi}, we see that equation \eqref{pos1} is implied by the following inequality:
\begin{align}\label{pos1solv}
 &  \Psi(\beta_1,\beta_2) \cdot \frac{(\rho_2-1) \rho_1 }{\rho_1+\rho_2-2}   < \left( \al_+^2(\rho_1) +  \oo(1) \right)^{-1} .
 \end{align}
 Then we can solve equation \eqref{pos1solv} to get
 \be\label{addconstraint}\rho_1 < \frac{\rho_2-2}{\Psi(\beta_1,\beta_2) \cdot (\rho_2 - 1)\left( \al_+^2(\rho_1) +  \oo(1) \right) - 1}  .\ee
 %$$\rho_1 > \frac{(\rho_2-2) \sigma^2}{(1 - {\rho_1}^{-0.5})^4 (\rho_2 - 3) pd^2 - \sigma^2}$$
 This gives the first statement of Proposition \ref{prop_data_size} using $\nu\ge \al_+^2(\rho_1) +  \oo(1) $. %Plugging into $\rho_1>40$ and $\rho_2>500$, we conclude the upper bound for $\rho_1$ in order for positive transfer to hold.
 
 
 Note that if we require the RHS of equation \eqref{addconstraint} to be larger than $40$, that is, equation \eqref{addconstraint} is not a null condition, then with equation \eqref{explainpsi} we get
 $$ \rho_2 - 2>\left[ 2\left( \al_+^2(\rho_1) +  \oo(1) \right)  -1\right]\rho_1 .$$
Plugging into $\rho_1>40$, we get $\rho_2 \ge 106$. This gives a constraint on $\rho_2$. 
 
%On the other hand, if $c_1 < \frac{(c_2-2)\sigma^2}{(1+a^{-1/2})^4(c_2 - 1) pd^2 - \sigma^2}$, then we have 
% \begin{align*}
% pd^2 \cdot \frac{c_1(c_1+c_2)(c_2-1)}{(c_1+c_2-1)^2}  \cdot \bigbrace{1 + \sqrt{\frac{1}{c_1}}}^4 < \bigbrace{1 + \sqrt{\frac{1}{c_1}}}^4 \cdot \frac{pd^2(c_2-1) c_1 }{c_1+(c_2-2)}  < \sigma^2\cdot (1-\oo(1)).
% \end{align*}
% This shows that equation \eqref{pos1} holds as long as $p$ is large enough, and hence $\te(\hat{\beta}_t^{\MTL})< \te(\hat{\beta}_t^{\STL})$ holds.

\vspace{5pt}

 \noindent{\bf Negative transfer.} 
With equation \eqref{trivialphi}, we see that equation \eqref{neg1} is implied by the following inequality:
\begin{align}\label{neg1solv}
 & \Psi(\beta_1,\beta_2)  \frac{(\rho_2-1) \rho_1 }{\rho_1+\rho_2-2}\left( 1-\frac{1}{(\rho_1+\rho_2-2)^2}\right) >  \Psi(\beta_1,\beta_2) \frac{(\rho_2-1.5) \rho_1 }{\rho_1+\rho_2-2}> \left( \al_-^2(\rho_1) -  \oo(1) \right)^{-1} .
 \end{align}
 where we used $(1-(\rho_1+\rho_2-2)^{-2})(\rho_2 - 1) > \rho_2-1.5$ for $\rho_1>40$ and $\rho_2>110$. Then we can solve equation \eqref{neg1solv} to get that
 \be\label{addconstaint2}
 \rho_1 > \frac{\rho_2-2 }{\Psi(\beta_1, \beta_2) \cdot  (\rho_2 - 1.5)\left( \al_-^2(\rho_1) -  \oo(1) \right)  - 1}  ,
 \ee
which gives the second statement of Proposition \ref{prop_data_size}.
We remark that condition equation \eqref{explainpsi} implies $\Psi(\beta_1, \beta_2) \cdot  (\rho_2 - 1.5)\left( \al_-^2(\rho_1) -  \oo(1) \right)>1$, so equation \eqref{addconstaint2} does not give a trivial bound. 
 \end{proof}
 





%\todo{add a transition example}

\iffalse
Note that for the case of $k$ tasks with the same covariates, since there is no covariate shift and the data ratio is always equal to one, the main factor is model distance.

\paragraph{A precise bound when there is no model shift.}
As Proposition \ref{prop_monotone} shows, if $\beta_s$ and $\beta_t$ are equal, then adding the source task dataset always helps learn the target task.
The goal of this section is to understand how covariate shift affects the rate of transfer. \todo{add conceptual msg}

%The key quantity is to look at:
%The estimator using the source and target together from minimizing equation \eqref{eq_mtl_basic} is
%\[ \hat{\beta}_{s,t} = (X_1^{\top} X_1 + X_2^{\top} X_2)^{-1} (X_1^{\top}Y_1 + X_2^{\top}Y_2)\]
%The estimation error of $\hat{\beta}_{s,t}$ is
%\begin{align}\label{eq_two_task}
%  \err(\hat{\beta}_{s,t}) = \sigma^2 \cdot \tr[(X_1^{\top}X_1 + X_2^{\top} X_2)^{-1}].
%\end{align}
%The estimation error using the target alone is
%\begin{align}\label{eq_target_task}
%	\err(\hat{\beta}_t) = \sigma^2 \cdot \tr[(X_2^{\top} X_2)^{-1}].
%\end{align}
%The improvement of estimation error from adding the source task is then given by
%$\err(\hat{\beta}_t) - \err(\hat{\beta}_{s,t})$.
%For the test error on the target task, the improvement from adding the source task is
%\[ \te(\hat{\beta}_t) - \te(\hat{\beta}_{s,t}) = \sigma^2\cdot\bigtr{\bigbrace{(X_2^{\top}X_2)^{-1} - (X_1^{\top}X_1 + X_2^{\top}X_2)^{-1}}\cdot\Sigma_2}. \]

%We calculate the amount of improvement by comparing equation \eqref{eq_two_task} to equation \eqref{eq_target_task}.
A simple observation here is that when $\beta_s = \beta_t$, the optimal $\hat{w}$ for minimizing equation \eqref{eq_te_mtl} is equal to $1$.
Based on this observation, we can get a more precise result than Theorem \ref{thm_model_shift} on the improvement of adding the source task data that only depends on the covariance matrices $\Sigma_1, \Sigma_2$ and the number of data points $n_1, n_2$.



\begin{proposition}[Transfer rate without model shift]\label{thm_cov_shift}
Suppose $\beta_s = \beta_t$ and $\|\beta_t\|_2^2\sim p\sigma^2$ (i.e. the $l^2$-norm of the vector $\beta_t$ is of the same order as that of the error vector). Assume that the condition numbers of $\Sigma_1$, $\Sigma_2$ and $M:=\Sigma_1^{1/2}\Sigma_2^{-1/2}$ are all bounded by a constant $C>0$. Then we have that the optimal ratio for $W_1/W_2$ in equation \eqref{eq_te_mtl} satisfies
	$$1\le \hat{w} \le 1+\OO(p^{-1}).$$%is $\hat{w} = 1 \pm \bigo{\frac 1 {n_1+n_2}}$ \todo{(figure out the constants)}
	%where $M:=\Sigma_1^{1/2}\Sigma_2^{-1/2}$.
Moreover, we have
	\begin{align}\label{tehatw1}
		%\err(\hat{\beta}^{\TL}_{s,t}) &= \sigma^2 \cdot \bigtr{\frac 1 {(n_1 + n_2)a_1\Sigma_1 + (n_1 + n_2)a_2\Sigma_2}} \\
		\te(\hat{\beta}^{\TL}_{t}) &= \sigma^2 \cdot \bigtr{\bigbrace{(n_1 + n_2)a_1 M^\top M  + (n_1 + n_2)a_2\id}^{-1}} \cdot \left(1+ \bigo{p^{-1}}\right),
	\end{align}
where $a_1, a_2$ are the solutions to equations equation \eqref{eq_a2}. %\cor $w_0$ is close to 1 if the signal strength $\beta_t$ is much larger than the noise strength \nc
\end{proposition}


\begin{proof}
We abbreviate $\val(w_2\hat{B}(w)):=\val(w)$. Note that $\val(w)\le \val(-w)$ for $w\ge 0$. Hence we have $\hat w\ge 0$. Moreover, we notice that $\val (w) < \val (1)$ for all $0\le w < 1$. Thus we have $\hat w\ge 1$. It suffices to consider the case with $w> 1$. Under the assumption on $\beta_s$ and $\beta_t$, we can write
\begin{align}
	\val(w) =&~  \left( 1-\frac1w\right)^2 \left\|( M^\top Z_1^{\top}Z_1 M + w^{-2}Z_2^{\top}Z_2)^{-1} M^\top Z_1^{\top} Z_1 \Sigma_1^{1/2}\beta_t \right\|^2 \nonumber \\
			&~ + \frac{\sigma^2}{w^2} \cdot \bigtr{( M^\top Z_1^{\top}Z_1 M + w^{-2}Z_2^{\top}Z_2)^{-1} }. \nonumber
\end{align}
Since
\begin{align*}
&\left\|( M^\top Z_1^{\top}Z_1 M + w^{-2}Z_2^{\top}Z_2)^{-1} M^\top Z_1^{\top} Z_1 \Sigma_1^{1/2}\beta_t \right\|^2 \\
&= \tr \left[ ( M^\top Z_1^{\top}Z_1 M + w^{-2}Z_2^{\top}Z_2)^{-2}M^\top Z_1^{\top} Z_1 \Sigma_1^{1/2}\beta_t\beta_t^\top \Sigma_1^{1/2}Z_1^{\top} Z_1  M\right]
\end{align*}
is increasing with respect to $w$, then the derivative of $\val(w)$ can be bounded from below as
\begin{align*}
\val'(w) \ge &~ 2\frac{w-1}{w^3} \left\|( M^\top Z_1^{\top}Z_1 M + w^{-2}Z_2^{\top}Z_2)^{-1} M^\top Z_1^{\top} Z_1 \Sigma_1^{1/2}\beta_t \right\|^2   \\
			&~ - 2 \frac{\sigma^2}{w^3} \cdot \bigtr{(M^\top Z_1^{\top}Z_1 M +w^{-2}  Z_2^{\top}Z_2)^{-1} M^\top Z_1^{\top}Z_1 M (M^\top Z_1^{\top}Z_1 M + w^{-2} Z_2^{\top}Z_2)^{-1}} \\
\ge &~ 2\frac{w-1}{w^3} \left\|( M^\top Z_1^{\top}Z_1 M +  Z_2^{\top}Z_2)^{-1} M^\top Z_1^{\top} Z_1 \Sigma_1^{1/2}\beta_t \right\|^2   - 2 \frac{\sigma^2}{w^3} \cdot \bigtr{(M^\top Z_1^{\top}Z_1 M)^{-1}}.
			%\\ =& ~ 2\frac{d^2}{w^3} \tr\left[( M^\top Z_1^{\top}Z_1 M + w^{-2}Z_2^{\top}Z_2)^{-1} M^\top Z_1^{\top} \left[\left( w-1\right)\left(Z_1 \Sigma_1 Z_1^{\top}\right) - \frac{\sigma^2}{d^2}\id \right] Z_1 M ( M^\top Z_1^{\top}Z_1 M + w^{-2}Z_2^{\top}Z_2)^{-1}\right]  .
\end{align*}
Hence $\val'(w)\ge 0$ if $w>1+ \e_0$, where
$$\e_0:= \frac{\sigma^2  \bigtr{(M^\top Z_1^{\top}Z_1 M)^{-1}} }{\left\|( M^\top Z_1^{\top}Z_1 M + Z_2^{\top}Z_2)^{-1} M^\top Z_1^{\top} Z_1 \Sigma_1^{1/2}\beta_t \right\|^2}.$$
In other words, $\val(w)$ is strictly increasing function on $[1+\e_0,\infty]$. Thus we get that $\hat w$ satisfies
\be\label{hatw1}1\le w \le 1+\e_0.\ee Using Fact \ref{lem_minv}, we get that
$$\e_0=\OO(\sigma^2/\|\beta_t\|_2^2)=\OO(p^{-1}).$$

Finally, plugging equation \eqref{hatw1} into the expression $\te(\hat{\beta}^{\TL}_{t}) $, we obtain equation \eqref{tehatw1}.
%$$1\le \hat{w} \le w_0:=1 +\frac{\sigma^2  \bigtr{(M^\top Z_1^{\top}Z_1 M)^{-1}} }{\left\|( M^\top Z_1^{\top}Z_1 M + Z_2^{\top}Z_2)^{-1} M^\top Z_1^{\top} Z_1 \Sigma_1^{1/2}\beta_t \right\|^2},$$
\end{proof}



As a remark, we see that Proposition \ref{prop_monotone} follows from Theorem \ref{thm_cov_shift}.
The amount of reduction on test error for the target task is given as
	\begin{align*}
%		\err(\hat{\beta}_t) - \err(\hat{\beta}_{s,t})
%		&= \sigma^2 p \cdot \bigtr{\frac 1 {(n_2 - p) \Sigma_2} - \frac 1 {(n_1 + n_2)a_1 \Sigma_1 + (n_1 + n_2)a_2 \Sigma_2}}, \\
		\te(\hat{\beta}_t) - \te(\hat{\beta}_{s,t})
		&= \sigma^2 \cdot \bigbrace{\frac p {n_2 - p} -  \bigtr{\bigbrace{(n_1 + n_2)a_1\Sigma_2^{-1/2}\Sigma_1\Sigma_2^{-1/2} + (n_1 + n_2)a_2\id}^{-1}}}.
	\end{align*}
Because
\begin{align*}
	\te(\hat{\beta}_{s,t}) \le \te(\hat{\beta}_t)
	\Leftarrow~ & (n_2 - p)\Sigma_2 \preceq (n_1 + n_2) a_1 \Sigma_1 + (n_1 + n_2)a_2 \Sigma_2 \\
	\Leftrightarrow~ & \zeroMatrix \preceq (n_1 + n_2) a_1 \Sigma_1 + (n_1 - (n_1 + n_2)\cdot a_1) \Sigma_2,
\end{align*}
which is true since $a_1 \le n_1 / (n_1 + n_2)$ by equation \eqref{eq_a2}.
The proof for $\te(\hat{\beta}_{s,t}) \le \te(\hat{\beta}_t)$ follows by multiplying $\Sigma_2^{-1/2}$ on both sides of the inequalities above.

\medskip
\fi


%\textbf{Remark.} Furthermore, as a function of $c_1$ over the range $[??, \infty]$, the maximum of $\te(\hat{\beta}_t^{\STL}) - \te(\hat{\beta}_t^{\MTL})$ is attained when $c_1 = {c_2\sigma^2}/{\max(2(c_2 - 1)pd^2 -\sigma^2, 0)}$. {\cor (cannot get this because we only have some bounds. If we let $c_1\to \infty$, then the curve of $\te(\hat{\beta}_t^{\STL}) - \te(\hat{\beta}_t^{\MTL})$ already becomes flat, and it is meaningless to discuss the minimum of this function at this point?)}



\iffalse
\subsection{Label Denoising}

The proof of Proposition \ref{prop_var_transition} is similar to the above proof of Proposition \ref{prop_dist_transition}.
%The details can be found in Appendix \ref{app_proof_data}.


\begin{proof}[Proof of Proposition \ref{prop_var_transition}]
In the setting of Proposition \ref{prop_var_transition}, the test loss is given by equation \eqref{eq_te_mtl_2task}.
%the validation loss and the test error become
%\begin{align*}
%		\val(\hat{B}; w_1, w_2)
%	&=  n_1 \cdot \bignorm{\Sigma_1^{1/2}\left(\frac{w_1^2}{w_2^2} X_1^{\top}X_1 + X_2^{\top}X_2\right)^{-1}X_2^{\top}X_2\left (\beta_s - \frac{w_1}{w_2}\beta_t\right)}^2 \nonumber \\
%		&+ n_1 \sigma^2 \cdot \frac{w_1^2}{w_2^2} \bigtr{\left(\frac{w_1^2}{w_2^2}  X_1^{\top}X_1 + X_2^{\top}X_2\right)^{-2} \left(\sigma_1^2 \frac{w_1^2}{w_2^2}  X_1^{\top}X_1 + \sigma_2^2  X_2^{\top}X_2\right)} \nonumber \\
%		&+ n_2 \cdot \frac{w_1^2}{w_2^2}\bignorm{\Sigma_2^{1/2}\left(\frac{w_1^2}{w_2^2} X_1^{\top}X_1 + X_2^{\top}X_2\right)^{-1}X_1^{\top}X_1\left(\beta_s - \frac{w_1}{w_2}\beta_t\right)}^2 \nonumber \\
%		&+ n_2 \sigma^2 \cdot \bigtr{\left(\frac{w_1^2}{w_2^2} X_1^{\top}X_1 + X_2^{\top}X_2\right)^{-2} \left(\sigma_1^2 \frac{w_1^2}{w_2^2}  X_1^{\top}X_1 + \sigma_2^2  X_2^{\top}X_2\right)},
%\end{align*}
%and
%where $\hat v=\hat{w_1}/\hat{w_2}$ is the global minimizer of $\val(\hat{B}; w_1, w_2)$.
For the isotropic model, using again the concentration result, Lemma \ref{largedeviation}, we can rewrite $\te(\hat{\beta}_2^{\MTL})$ as
\begin{align*}
	\te(\hat{\beta}_2^{\MTL}) &=~ \hat{v}^2 \left[d^2 +\left( \hat v-1\right)^2\kappa^2\right]\bigtr{ (\hat{v}^2 X_1^{\top}X_1 + X_2^{\top}X_2)^{-2} (X_1^{\top}X_1)^2} \cdot \left(1+\OO(p^{-1/2+\e})\right)\nonumber \\
	& + \sigma_2^2 \cdot \bigtr{(\hat{v}^2 X_1^{\top}X_1 + X_2^{\top}X_2)^{-1} } + (\sigma_1^2 -\sigma_2^2)  \cdot \bigtr{(\hat{v}^2 X_1^{\top}X_1 + X_2^{\top}X_2)^{-2} \hat v^2X_1^\top X_1}
\end{align*}
with high probability for any constant $\e>0$.



In the current setting, we can also show that  equation \eqref{hatv_add1}  holds for $\hat v$.
Since the proof is almost the same as the one for Claim \ref{lem_hat_v}, we omit the details.
%$$|\hat w-1|=\OO(p^{-1}).$$
%as in equation \eqref{hatw_add1}.
%We omit the details of the proof, since it is almost the same as the one in the proof of Lemma \ref{lem_hat_v}.
Thus under equation \eqref{choiceofpara}, $\te(\hat{\beta}_2^{\MTL}) $ can be simplified as in equation \eqref{simple1}:
\be \label{simple2}
\begin{split}
\te(\hat{\beta}_2^{\MTL})&=(1+\OO(n^{-\e}))\cdot d^2 \tr\left[(X_1^{\top}X_1 +X_2^{\top}X_2)^{-2} (X_1^{\top}X_1)^2\right] \\
&+(1+\OO(n^{-\e}))\cdot \sigma_2^2  \bigtr{(X_1^{\top}X_1  + X_2^{\top}X_2)^{-1} }  \\
&+(1+\OO(n^{-\e}))\cdot (\sigma_1^2-\sigma_2^2)  \bigtr{(X_1^{\top}X_1  + X_2^{\top}X_2)^{-2} X_1^\top X_1} ,
\end{split}
\ee
with high probability for some small constant $\e>0$. Then we can write
$$ \te(\hat{\beta}_2^{\STL})-\te(\hat{\beta}_2^{\MTL}) =\delta_{\vari} - \delta_{\bias} - \delta_{\vari}^{(2)},$$
where
$$\delta_{\vari}:=\sigma_2^2  \bigtr{(X_2^{\top}X_2)^{-1} }  - (1+\OO(n^{-\e}))\cdot \sigma_2^2  \bigtr{(X_1^{\top}X_1  + X_2^{\top}X_2)^{-1} }$$
satisfies equation \eqref{Deltavar} with $\sigma^2$ replaced by $\sigma_2^2$;
$$\delta_{\bias}:=(1+\OO(n^{-\e}))\cdot d^2 \tr\left[(X_1^{\top}X_1 +X_2^{\top}X_2)^{-2} (X_1^{\top}X_1)^2\right]$$
satisfies equation \eqref{Deltabeta}; $\delta_{\vari}^{(2)}$ is defined as
\begin{align*}
	\delta_{\vari}^{(2)}:=(1+\OO(n^{-\e}))\cdot (\sigma_1^2 -\sigma_2^2) \bigtr{(X_1^{\top}X_1 + X_2^{\top}X_2)^{-2} X_1^\top X_1} .
\end{align*}
%$$	\te(\hat{\beta}_t^{\MTL}) = \left[\Delta_{\bias} + \Delta_{diff}+  \sigma_2^2 \cdot \tr(X_1^{\top}X_1 + X_2^{\top}X_2)^{-1} \right]\cdot \left(1+\OO(p^{-1/2})\right), $$
%where $\Delta_{\bias}:= \sigma_2^2 \cdot \bigtr{(\hat{v}^2 X_1^{\top}X_1 + X_2^{\top}X_2)^{-1} }$ has
To estimate this new term $\delta_{\vari}^{(2)}$, we use the same arguments as in the proof of Lemma \ref{prop_model_shift_tight}: we first replace $X_1^\top X_1$ with $n_1\id$ up to a small error using Fact equation \eqref{lem_minv}, and then apply Lemma \ref{lem_cov_derivative} to calcualte $\bigtr{(X_1^{\top}X_1 + X_2^{\top}X_2)^{-2}}$. This process leads to the following estimates on $\delta_{\vari}^{(2)}$:
\be\label{Deltavar2}
\al_-(\rho_1) - \oo(1)  \le  \frac{\delta_{\vari}^{(2)}}{ \Delta_{\vari}^{(2)}} \le \al_+(\rho_1) +  \oo(1) , \ee
where
$$ \Delta_{\vari}^{(2)}:=(\sigma_1^2 -\sigma_2^2) \frac{\rho_1 (\rho_1+\rho_2)}{(\rho_1+\rho_2-1)^3}.$$
Next we compare $\delta_{\vari}$ with $\delta_{\bias} + \delta_{\vari}^{(2)}$. Our main focus is to see how the extra $\delta_{\vari}^{(2)}$ affects the information transfer in this case.

 Note that the condition $pd^2< \frac{1}{2} {\sigma_2^2}  \cdot \Phi(\rho_1, \rho_2)$ for $\rho_1 > 40$ gives that $\delta_{\vari}>\delta_{\bias} $ by Proposition \ref{prop_dist_transition}. Hence if $\sigma_1^2\le \sigma_2^2$, then $\delta_{\vari}^{(2)}<0$ and we always have $\delta_{\vari} > \delta_{\bias}+ \delta_{\vari}^{(2)}$, which gives $\te(\hat{\beta}_2^{\MTL})<\te(\hat{\beta}_2^{\STL})$.  It remains to consider the case $\sigma_1^2 \ge \sigma_2^2$.

\vspace{5pt}

\noindent{\bf Positive transfer.} By equation \eqref{Deltavar}, equation \eqref{Deltabeta} and equation \eqref{Deltavar2}, if the following inequality holds,
\be\label{cond sigma1}
\begin{split}
&\sigma_2^2  \cdot \frac{\rho_1}{(\rho_2-1)(\rho_1 + \rho_2 -1)} \cdot (1-\oo(1)) \\
&>pd^2 \cdot \frac{\rho_1^2 (\rho_1+\rho_2)}{(\rho_1 + \rho_2 - 1)^3}\al_+^2(\rho_1)  + (\sigma_1^2 -\sigma_2^2)\cdot \frac{\rho_1 (\rho_1+\rho_2)}{(\rho_1+\rho_2-1)^3} \al_+(\rho_1)  ,
\end{split}
\ee
then we have $\delta_{\vari} > \delta_{\bias} + \delta_{\vari}^{(2)}$ w.h.p., which gives $\te(\hat{\beta}_2^{\MTL})<\te(\hat{\beta}_2^{\STL})$. We can solve equation \eqref{cond sigma1} to get
\begin{align*}
\sigma_1^2 < - pd^2\cdot \rho_1 \al_+(\rho_1) +\sigma_2^2 \left[ 1+ \rho_1\Phi(\rho_1, \rho_2) \al_+^{-1}(\rho_1)\right]\cdot (1-\oo(1)).
\end{align*}
This proves the first claim of Proposition \ref{prop_var_transition} using $\nu\ge \al_+^2(\rho_1) +  \oo(1) $.

\vspace{5pt}


\noindent{\bf Negative transfer.} On the other hand, if the following inequality holds,
\be\label{cond sigma12}
\begin{split}
&\sigma_2^2 \cdot \frac{\rho_1}{(\rho_2-1)(\rho_1 + \rho_2 -1)}\cdot \left(1 + \oo(1)\right) \\
&< pd^2 \cdot \frac{\rho_1^2 (\rho_1+\rho_2)}{(\rho_1 + \rho_2 - 1)^3}\al_-^2(\rho_1)  + (\sigma_1^2 -\sigma_2^2)\cdot \frac{\rho_1 (\rho_1+\rho_2)}{(\rho_1+\rho_2-1)^3} \al_-(\rho_1) ,
\end{split}
\ee
then we have $\delta_{\vari} < \delta_{\bias} + \delta_{\vari}^{(2)}$ w.h.p., which gives $\te(\hat{\beta}_t^{\MTL})>\te(\hat{\beta}_t^{\STL})$. We can solve equation \eqref{cond sigma12} to get
\begin{align*}
\sigma_1^2 > - pd^2 \cdot \rho_1 \al_-(\rho_1) +\sigma_2^2 \left[ 1+ \rho_1\Phi(\rho_1, \rho_2)\al_-^{-1}(\rho_1)\right]\cdot (1+\oo(1)).
\end{align*}
This proves the second claim of Proposition \ref{prop_var_transition}.
\end{proof}
\fi

\subsection{Labeling Efficiency}

Next we prove Proposition \ref{prop_data_efficiency}, which gives precise upper and lower bounds on the data efficiency ratio for Taskonomy.

\begin{proof}[Proof of Proposition \ref{prop_data_efficiency}]
Suppose we have reduced number of datapoints---$x n_1$ for task 1 and $x n_2$ for task 2 with $n_1=\rho_1 p$ and $n_2=\rho_2 p$. Then all the results in the proof of Proposition \ref{prop_dist_transition} still hold, except that we need to replace $(\rho_1,\rho_2)$ with $(x\rho_1,x\rho_2)$. More precisely, we have
	\begin{align*}
		 a_1 = \frac{\rho_1(x\rho_1 + x\rho_2 - 1)}{x(\rho_1 + \rho_2)^2} ,\quad
		& a_2 = \frac{\rho_2(x\rho_1 + x\rho_2 - 1)}{x(\rho_1 + \rho_2)^2} ,  \\
		 a_3 = \frac{\rho_2}{(\rho_1 + \rho_2)(x\rho_1 + x\rho_2 - 1)}, \quad
		& a_4 = \frac{\rho_1}{(\rho_1 + \rho_2)(x\rho_1 + x\rho_2 - 1)}.
	\end{align*}
Then with Lemma \ref{lem_cov_shift} we can obtain that with high probability,
	\be\label{reduceproof1}
\begin{split}
\te_i(\hat \beta_i^{\MTL}(x)) =  \frac{\sigma^2}{x (\rho_1+\rho_2) - 1}\left(1+ \oo(1)\right)+ \delta^{(i)}_{\bias}, \quad i=1,2,
\end{split}
\ee
 where as in equation \eqref{Deltabeta}, the model shift biases $\delta^{(i)}_{\bias}$ satisfy that with high probability,
\be\nonumber %\label{Deltabeta}
\al_-^2(\al\rho_i) - \oo(1)  \le {\delta_{\bias}^{(i)}}/{ \Delta_{\bias}^{(i)}} \le \al_+^2(\al\rho_i) +  \oo(1) , \quad i=1,2.\ee
Here $ \Delta^{(i)}_{\bias}$ are defined as
\be \nonumber
\Delta^{(i)}_{\bias} := pd^2 \frac{(x\rho_i)^2\cdot x (\rho_1+\rho_2)}{[x ( \rho_1+\rho_2) - 1]^3} ,\quad i=1,2,.
\ee
On the other hand, by Lemma \ref{lem_minv} we have that w.h.p.,
\be\label{reduceproof2}
\te_i(\hat{\beta}_i^{\STL}) = \frac{\sigma^2}{\rho_i -1} \left( 1+ \oo(1)\right),\quad i=1,2.
\ee
Comparing equation \eqref{reduceproof1} and equation \eqref{reduceproof2}, we immediately obtain the lower bound $x^\star\ge x_l $.
%where
%$$\al_l:=\frac1{\rho_1+\rho_2}\left(\frac{2}{\frac1{\rho_1-1}+\frac1{\rho_2 -1}}+1\right) \ge \frac{\min(\rho_1,\rho_2)}{\rho_1+\rho_2}.$$
In fact, if $x< x_l$, then we have that
$$ \frac{2\sigma^2}{x (\rho_1+\rho_2) - 1} > \frac{\sigma^2}{\rho_1-1}+\frac{\sigma^2}{\rho_2-1}.$$
This shows that $\te_1(\hat{\beta}_1^{\MTL}(x)) + \te_2(\hat{\beta}_2^{\MTL}(x))$ is larger than $\te_1(\hat{\beta}_1^{\STL}) + \te_2(\hat{\beta}_2^{\STL})$ even if we do not take into account the model shift biases $ \delta^{(i)}_{\bias}$.

Then we try to obtain an upper bound on $x^\star$. In the following discussions, we only consider $x$ such that $x> x_l$. In particular, we have $x\rho > x_l \rho \ge \min(\rho_1,\rho_2)$, where we abbreviated $\rho:=\rho_1+\rho_2$.
%Comparing equation \eqref{reduceproof1} and equation \eqref{reduceproof2}, we observe that $\te_2(\hat \beta(\al))  \ge \te_2(\hat{\beta}_t^{\STL}) $ for $\al \le 1/2-\oo(1)$, which gives $\al^* \ge 1/2 - \oo(1)$.
%\noindent{\bf The upper bound.}
From equation \eqref{reduceproof1} and equation \eqref{reduceproof2}, we see that $x^\star\le x$ if $x$ satisfies
\be\nonumber
\begin{split}
&(1+\oo(1)) \cdot \sum_{i=1}^2 pd^2 \frac{(x\rho_i)^2\cdot x\rho }{(x\rho - 1)^3}  \bigbrace{1 + \sqrt{\frac{1}{x \rho_i}}}^4 \le \frac{\sigma^2}{\rho_1 -1}+\frac{\sigma^2}{\rho_2 -1} - \frac{2\sigma^2}{x\rho  - 1} .
\end{split}
\ee
We rewrite the inequality as
\be
\begin{split}\label{solval1}
  (1+\oo(1)) \cdot \frac{\Psi(\beta_1,\beta_2)}{[1 - (x\rho)^{-1}]^3} \sum_{i=1}^2 \bigbrace{\sqrt{\frac{\rho_i}{\rho}} + \sqrt{\frac{1}{x\rho}}}^4\le \frac{1}{\rho_1 -1}+\frac{1}{\rho_2 -1} - \frac{2}{x \rho  - 1}.
\end{split}
\ee
With $x\rho\ge \min(\rho_1,\rho_2)\ge 9$, we can get the simple bound
$$\frac{1+\oo(1)}{[1 - (x\rho)^{-1}]^3} \sum_{i=1}^2 \bigbrace{\sqrt{\frac{\rho_i}{\rho}} + \sqrt{\frac{1}{x\rho}}}^4 < 5. $$
Inserting it into equation \eqref{solval1}, we can solve for the upper bound in equation \eqref{eq_uplowx}.
%where we abbreviated $\rho:=\rho_1+\rho_2$.
\end{proof}
\begin{remark}
We can get better bounds if the values of $\rho_1$ and $\rho_2$ increase. For example, if we consider the case $\min(\rho_1,\rho_2)\ge 100$, then with some basic calculations, one can check that in this case
$$ \frac{1}{[1 - (x \rho)^{-1}]^3} \sum_{i=1}^2 \bigbrace{\sqrt{\frac{\rho_i}{\rho}} + \sqrt{\frac{1}{x \rho}}}^4 <  \frac{\rho_1^2 + \rho_2^2}{\rho^2} + 0.52.$$
Thus the following inequality implies equation \eqref{solval1}:
\be\nonumber %\label{solval1add}
\begin{split}
&\left(  \frac{\rho_1^2 + \rho_2^2}{\rho^2} + 0.52\right) \Psi(\beta_1,\beta_2) < \frac{1}{\rho_1 -1}+\frac{1}{\rho_2 -1} - \frac{2 }{x\rho - 1} ,
\end{split}
\ee
%In particular, if
%\be\nonumber
%\begin{split}
%&\left(  \frac{\rho_1^2 + \rho_2^2}{(\rho_1+\rho_2)^2} + 0.32\right) pd^2 < \frac{\sigma^2}{\rho_1 -1}+\frac{\sigma^2}{\rho_2 -1} - \frac{2\sigma^2}{(\rho_1+\rho_2) - 1} ,
%\end{split}
%\ee
%that is, we have positive transfer when using all the data, then
from which we can solve for the following upper bound on $x^\star$:
\begin{align*}
x^\star &<  \frac1{\rho} \frac{2 }{\frac{1}{\rho_1-1}+\frac1{\rho_2-1}  -  \left(  \frac{\rho_1^2 + \rho_2^2}{\rho^2} + 0.52\right)\Psi(\beta_1,\beta_2)}+ \frac1\rho .
%\\
%& < \frac1{\rho_1+\rho_2}\left[\frac{2 }{\frac{1}{\rho_1}+\frac1{\rho_2}  - \left(  \frac{\rho_1^2 + \rho_2^2}{(\rho_1+\rho_2)^2} + \frac13\right)\frac{pd^2}{\sigma^2}}+ 1\right] .
\end{align*}
Similarly, we can get a better lower bound.
%\noindent{\bf The lower bound.}
From equation \eqref{reduceproof1} and equation \eqref{reduceproof2}, we see that $x^\star\ge x$ if $x$ satisfies
\be\label{solval2}
\begin{split}
&(1-\oo(1)) \cdot  \frac{\Psi(\beta_1,\beta_2)}{[1 - (x \rho)^{-1}]^3} \sum_{i=1}^2 \bigbrace{\sqrt{\frac{\rho_i}{\rho}} - \sqrt{\frac{1}{x \rho}}}^4 \ge \frac{1}{\rho_1 -1}+\frac{1}{\rho_2 -1} - \frac{2 }{x \rho - 1} .
\end{split}
\ee
%We then follow similar arguments as the above proof for the upper bound.
Then in the case $\min(\rho_1,\rho_2)\ge 100$, with some basic calculations, one can show that the sum on the left-hand side of equation \eqref{solval2} satisfies
$$ \frac{1}{[1 - (x \rho)^{-1}]^3} \sum_{i=1}^2 \bigbrace{\sqrt{\frac{\rho_i}{\rho}} - \sqrt{\frac{1}{x\rho}}}^4 >  \frac{\rho_1^2 + \rho_2^2}{\rho^2} -0.33 .$$
Thus the following inequality implies equation \eqref{solval2}:
\be\label{solval2add}
\begin{split}
&\left( \frac{\rho_1^2 + \rho_2^2}{\rho^2} -0.33\right) \Psi(\beta_1,\beta_2)  > \frac{1}{\rho_1 -1}+\frac{1}{\rho_2 -1} - \frac{2 }{x\rho - 1} ,
\end{split}
\ee
%There are two cases: if
%\be\nonumber
%\begin{split}
%&\left( \frac{\rho_1^2 + \rho_2^2}{(\rho_1+\rho_2)^2} -0.26\right) pd^2 \ge \frac{\sigma^2}{\rho_1 -1}+\frac{\sigma^2}{\rho_2 -1},
%\end{split}
%\ee
%then we have negative transfer for all choice of $0\le \al \le 1$; otherwise,
from which we can solve for the following lower bound on $x^\star$:
\begin{align*}
x^\star &>  \frac1\rho \frac{2 }{\frac{1}{\rho_1-1}+\frac1{\rho_2-1}  - \left( \frac{\rho_1^2 + \rho_2^2}{\rho^2} -0.33\right) \Psi(\beta_1,\beta_2)}+ \frac1\rho .
\end{align*}
This gives a lower bound above $x_l$.
\end{remark}
\fi
