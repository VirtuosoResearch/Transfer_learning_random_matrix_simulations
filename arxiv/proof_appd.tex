\section{Proofs of the Bias and Variance Bounds} \label{sec_maintools}

First, we give the basic assumption for our main objects---the random matrices $X_i$, $i=1,2$.

\begin{assumption}[Moment assumptions]\label{assm_secA1}
We will consider $n\times p$ random matrices of the form $X=Z\Sigma^{1/2}$, where $\Sigma$  is a $p\times p$ deterministic positive definite symmetric matrix, and $Z=(z_{ij})$ is an $n\times p$ random matrix with real i.i.d. entries with mean zero and variance one. Note that the rows of $X$ are i.i.d. centered random vectors with covariance matrix $\Sigma$. For simplicity, we assume that all the moments of $z_{ij}$ exists, that is, for any fixed $k\in \N$, there exists a constant $C_k>0$ such that
\begin{equation}\label{assmAhigh}
\mathbb{E} |z_{ij}|^k \le C_k ,\quad 1\le i \le n, \ \ 1\le j \le p.
\end{equation}
 We assume that $n=\rho p$ for some fixed constant $\rho>1$. Without loss of generality, after a rescaling we can assume that the norm of $\Sigma$ is bounded by a constant $C>0$. Moreover, we assume that $\Sigma$ is well-conditioned: $\kappa(\Sigma)\le C$, where $\kappa(\cdot)$ denotes the condition number.
\end{assumption}
Here we have assumed \eqref{assmAhigh} solely for simplicity of representation. If the entries of $Z$ only have finite $a$-th moment for some $a>4$, then all the results below still hold except that we need to replace $\OO(p^{-\frac12+\e})$ with $\OO( p^{-\frac12+\frac2a +\epsilon})$ in some error bounds.
We will not get deeper into this issue in this section, but refer the reader to Corollary \ref{main_cor} in Section \ref{sec locallaw1}.


Then we make the following assumptions on the data models.
\begin{assumption}[Linear regression model]\label{assm_secA2}
For some fixed $t\in \N$, let $Y_i = X_i\beta_i + \varepsilon_i$, $1\le i \le t$, be independent data models, where $X_i$, $\beta_i$ and $\varepsilon_i$ are also independent of each other. Suppose that $X_i=Z_i\Sigma_i^{1/2}\in \R^{n_i\times p}$ satisfy Assumption \ref{assm_secA1} with $\rho_i:=n_i/p>1$ being fixed constants.
$\e_i\in \R^{n_i}$ are random vectors with i.i.d. entries with mean zero, variance $\sigma_i^2$ and all moments as in \eqref{assmAhigh}.
\end{assumption}


The main goal of this section is to prove Lemma \ref{lem_cov_shift} and Lemma \ref{lem_cov_derivative}.
%In this section, we will work under the assumptions of Lemma \ref{lem_cov_shift}. However, 
In random matrix theory, it is much more convenient to rescale the matrices $Z_1$ and $Z_2$ such that their entries have variance $n^{-1}$, where $n:=n_1+n_2$. The advantage of this scaling is that the singular eigenvalues of $Z_1$ and $Z_2$ all lie in a bounded support that does grow with $n$. For reader's convenience, we now restate the setting with rescaled $Z_1$ and $Z_2$, and introduce a couple of other notations. 

%\textbf{Basic setting.}
%We denote the two sample covariance matrices by $\mathcal Q_1:=X_1^\top X_1$ and $\cal Q_2:= X_2^\top X_2$. 
We assume that $Z_1=(z^{(1)}_{ij})$ and $Z_2=(z^{(2)}_{ij})$ are $n_1\times p$ and $n_2\times p$ random matrices with i.i.d. entries satisfying
\begin{equation}\label{assm1}
\mathbb{E} z^{(k)}_{ij} =0, \ \quad \ \mathbb{E} \vert z^{(k)}_{ij} \vert^2  =n^{-1},\quad k=1,2. 
\end{equation}
%where we denote $n:=n_1+n_2$. Here we have chosen the scaling that is more standard in the random matrix theory literature---under this $n^{-1/2}$ scaling, the eigenvalues of $\cal Q_1$ and $\cal Q_2$ are all of order 1. 
Moreover, we assume that the fourth moments exist:
\be \label{conditionA2}
\mathbb{E} \vert \sqrt{n}z^{(\al)}_{ij} \vert^4  \le C
\ee
for some constant $C>0$. Let $0<\tau<1$ be a small constant. 
%In this paper, we regard $N$ as the fundamental (large) parameter and $n \equiv n(N)$ as depending on $N$. 
We assume that the aspect ratios $\rho_1= n_1/p$ and $\rho_2=n_2/p$ satisfy that 
\be\label{assm2}
0\le \rho_1 \le \tau^{-1}, \quad 1+\tau \le \rho_{2} \le \tau^{-1}.
\ee
Here the lower bound $1+\tau\le \rho_2$ is to ensure that the sample covariance matrix $X_2^\top X_2$ is non-singular with high probability; see Lemma \ref{SxxSyy} below. We assume that $\Sig_1$ and $\Sig_2$ have eigendecompositions
\be\label{eigen}\Sig_1= O_1\Lambda_1 O_1^\top, \ \ \Sig_2= O_2\Lambda_2 O_2^\top,\ \ \Lambda_1=\text{diag}(\si_1^{(1)}, \ldots, \si^{(1)}_p), \ \ \Lambda_2=\text{diag}( \si^{(2)}_1, \ldots, \si^{(2)}_p),
\ee
where the eigenvalues satisfy that
\begin{equation}\label{assm3}
\tau^{-1}\ge \si^{(1)}_1 \ge \si^{(1)}_2 \ge \ldots \ge \si^{(1)}_p \ge 0, \quad \tau^{-1}\ge  \si^{(2)}_1 \ge \si^{(2)}_2 \ge \ldots \ge \si^{(2)}_p \ge \tau,
\ee
such that $\Sigma_1$ and $\Sigma_2$ are both well-conditioned. We assume that $M=\Sig_1^{1/2} \Sig_2^{-1/2}$ has a singular value decomposition
\be\label{eigen2}
M= U\Lambda V^\top, \quad \Lambda=\text{diag}( \lambda_1, \ldots, \lambda_p),
\ee
where by \eqref{assm3} we have % the singular values satisfy that
\begin{equation}\label{assm32}
\tau \le \lambda_p \le \lambda_1 \le \tau^{-1} .%, \quad \max\left\{\pi_A^{(n)}([0,\tau]), \pi_B^{(n)}([0,\tau])\right\} \le 1 - \tau .
\end{equation}
%The first condition means that the operator norms of $A$ and $B$ are bounded by $\tau^{-1}$, and the second condition means that the spectrums of $A$ and $B$ do not concentrate at zero.
We summarize our basic assumptions here for future reference. Note that this assumption is in accordance with the assumptions of Lemma \ref{lem_cov_shift}, except that we rescale the entries of $Z_1$ and $Z_2$  and allow $\rho_1$ to be smaller than 1.
\begin{assumption}\label{assm_big1}
We assume that $Z_1$ and $Z_2$ are independent $n_1\times p$ and $n_2\times p$ random matrices with real $i.i.d.$ entries satisfying \eqref{assm1} and \eqref{conditionA2}, $\Sigma_1$ and $\Sigma_2$ are deterministic non-negative definite symmetric matrices satisfying \eqref{eigen}-\eqref{assm32}, and $d_{1,2}$ satisfy \eqref{assm2}.
 %and (\ref{assm2}). We assume that $T$ is an $M\times M$ deterministic diagonal matrix satisfying (\ref{simple_assumption}) and (\ref{assm3}).  
\end{assumption}


%\subsection{Notations and basic tools}



%Before giving the main proof, we first introduce some notations and tools. 

%Following the notations in \cite{EKYY,EKYY1}, we will use the following definition to characterize events of high probability.
%
%\begin{definition}[High probability event] \label{high_prob}
%Define
%\begin{equation}\label{def_phi}
%\varphi:=(\log N)^{\log \log N}.
%\end{equation} 
%We say that an $N$-dependent event $\Omega$ holds with {\it{$\xi$-high probability}} if there exist constant $c,C>0$ independent of $N$, such that
%\begin{equation}
%\mathbb{P}(\Omega) \geq 1-N^{C} \exp(- c\varphi^{\xi}),  \label{D25}
%\end{equation}
%for all sufficiently large $N$. For simplicity, for the case $\xi=1$, we just say {\it{high probability}}. Note that if (\ref{D25}) holds, then $\mathbb P(\Omega) \ge 1 - \exp(-c'\varphi^{\xi})$ for any constant $0\le c' <c$. 
%\end{definition}
%\begin{remark}
%For any $c, C>0$, there exists a $0< c^{\prime}<c$, such that $N^{C} \exp(-c \varphi^{\xi}) \leq \exp(-c^{\prime} \varphi^{\xi})$. Hence, if
%\begin{equation}
%\mathbb{P}(\Omega) \geq 1- \exp(-c \varphi^{\xi})
%\end{equation}
%$\Omega$ is also a $\xi$-high probability event.
%\end{remark}



For simplicity of notations, we will use the following notion of stochastic domination, which was first introduced in \cite{Average_fluc} and subsequently used in many works on random matrix theory. It greatly simplifies the presentation of the results and their proofs by systematizing statements of the form ``$\xi$ is bounded by $\zeta$ with high probability up to a small power of $n$".

\begin{definition}[Stochastic domination]\label{stoch_domination}
%\begin{itemize}
%\item[(i)] 
Let $\xi\equiv \xi^{(n)}$ and $\zeta\equiv \zeta^{(n)}$ be two $n$-dependent random variables.  
\begin{itemize}
\item[(i)] We say $\xi$ is stochastically dominated by $\zeta$, denoted by $\xi\prec \zeta$ or $\xi=\OO_\prec(\zeta)$, if for any (small) constant $\epsilon>0$ and (large) constant $D>0$, 
\[ \bbP\left(|\xi| >n^\epsilon |\zeta|\right)\le n^{-D}\]
for large enough $n\ge n_0(\epsilon, D)$. Moreover, if $\xi$ and $\zeta$ depend on a parameter $u$, then we say $\xi$ is stochastically dominated by $\zeta$ uniformly in $u\in \cal U$,  if for any constants $\e,D>0$, 
\[\sup_{u\in \cal U}\bbP\left(|\xi(u)|>n^\epsilon |\zeta(u)|\right)\le n^{-D}\]
for large enough $n\ge n_0(\epsilon, D)$.

\item[(ii)] We say an event $\Xi$ holds with high probability if for any constant $D>0$, $\mathbb P(\Xi)\ge 1- n^{-D}$ for large enough $n$. We say $\Xi$ holds with high probability on an event $\Omega$ if for any constant $D>0$, $\mathbb P(\Omega\setminus \Xi)\le n^{-D}$ for large enough $n$.
\end{itemize}
%(i) Let 
%\[\xi=\left(\xi^{(n)}(u):n\in\bbN, u\in U^{(n)}\right),\hskip 10pt \zeta=\left(\zeta^{(n)}(u):n\in\bbN, u\in U^{(n)}\right)\]
%be two families of nonnegative random variables, where $U^{(n)}$ is a possibly $n$-dependent parameter set. We say $\xi$ is stochastically dominated by $\zeta$, uniformly in $u$, if for any fixed (small) $\epsilon>0$ and (large) $D>0$, 
%\[\sup_{u\in U^{(n)}}\bbP\left[\xi^{(n)}(u)>n^\epsilon\zeta^{(n)}(u)\right]\le n^{-D}\]
%for large enough $n\ge n_0(\epsilon, D)$, and we shall use the notation $\xi\prec\zeta$. 
%%Throughout this paper, the stochastic domination will always be uniform in all parameters that are not explicitly fixed (such as matrix indices, and $z$ that takes values in some compact set). 
%%Note that $N_0(\epsilon, D)$ may depend on quantities that are explicitly constant, such as $\tau$ in Assumption \ref{assm_big1} and \eqref{assm_gap}. 
%If for some complex family $\xi$ we have $|\xi|\prec\zeta$, then we will also write $\xi \prec \zeta$ or $\xi=\OO_\prec(\zeta)$.
%\item[(ii)] 
%(ii) We extend the definition of $\OO_\prec(\cdot)$ to matrices in the weak operator sense as follows. Let $A$ be a family of random matrices and $\zeta$ be a family of nonnegative random variables. Then $A=O_\prec(\zeta)$ means that $\left|\left\langle\mathbf v, A\mathbf w\right\rangle\right|\prec\zeta \| \mathbf v\|_2 \|\mathbf w\|_2 $ uniformly in any deterministic vectors $\mathbf v$ and $\mathbf w$. Here and throughout the following, whenever we say ``uniformly in any deterministic vectors", we mean that ``uniformly in any deterministic vectors belonging to a set of cardinality $N^{\OO(1)}$".
%\item[(iv)] 
%\end{itemize}
\end{definition}

Then we introduce the following bounded support condition.

\begin{definition}\label{defn_support}
We say a random matrix $Z$ satisfies the {\it{bounded support condition}} with $q$, if
\begin{equation}
\max_{i,j}\vert Z_{ij}\vert \prec q. \label{eq_support}
\end{equation}
Here $q\equiv q(n)$ is a deterministic parameter and usually satisfies $ n^{-{1}/{2}} \leq q \leq n^{- \phi} $ for some (small) constant $\phi>0$. Whenever (\ref{eq_support}) holds, we say that $X$ has support $q$. 
%Moreover, if the entries of $X$ satisfy (\ref{size_condition}), then $X$ trivially satisfies the bounded support condition with $q=N^{-\phi}$.
\end{definition}

Note that if the entries of $\sqrt{n}Z$ have finite moments up to any order as in \eqref{assmAhigh}, then using Markov's inequality one can show that $Z$ has bounded support $n^{-1/2}$. More generally, if the entries of $\sqrt{n}Z$ have finite $a$-th moment for some $a>4$, then by Markov's inequality and a simple union bound, it is easy to see that $Z$ has bounded support $q=n^{2/a-1/2}$ on a event with probability $1-\oo(1)$. Hence the bounded support condition allows us to relax the moment conditions using a simple cut-off argument; see Corollary \ref{main_cor} below.

%\begin{remark}
%Note that the Gaussian distribution satisfies the condition (\ref{eq_support}) with $q< N^{-\phi}$ for any $\phi<1/2$. We also remark that if (\ref{eq_support}) holds, then the event $\left\{\vert x_{ij}\vert \le q, \forall 1\le i \le M,1\le j \le N\right\}$ holds with $\xi$-high probability for any fixed $\xi>0$ according to Definition \ref{high_prob}. For this reason, the bad event $\left\{\vert x_{ij}\vert > q \text{ for some }i,j\right\}$ is negligible, and we will not consider the case it happens throughout the proof.
%\end{remark}

%Since $Z_1^\top Z_1$ (resp.\;$Z_2^\top Z_2$) is a standard sample covariance matrix, and it is well-known that its nonzero eigenvalues are all inside the support of the Marchenko-Pastur law $[(1-\sqrt{d_1})^2 , (1+\sqrt{d_1})^2]$ (resp. $[(1-\sqrt{d_2})^2 , (1+\sqrt{d_2})^2]$) with probability $1-\oo(1)$ \cite{No_outside}. In our proof, we shall need a slightly stronger probability bound, which is given by the following lemma. 

For $Z_1$ and $Z_2$ with bounded support $q$, we have the following estimates on their singular values. We have used it in our previous proofs; see \eqref{eq_isometric}.   

 
%Then we state the following lemma on the eigenvalues of $Z_1^\top Z_1$ and $Z_2^\top Z_2$, which are denoted as $\lambda_1 (Z_1^\top Z_1) \ge \cdots \ge \lambda_{p} (Z_1^\top Z_1)$ and $\lambda_1 (Z_2^\top Z_2) \ge \cdots \ge \lambda_p (Z_2^\top Z_2)$. We have used it in our previous proofs; see \eqref{eq_isometric}. 
 
\begin{lemma}\label{SxxSyy}
Suppose Assumption \ref{assm_big1} holds, and $Z_1,Z_2$ satisfy the bounded support condition \eqref{eq_support} for some deterministic parameter $q\equiv q(n)$ satisfying $ n^{-{1}/{2}} \leq q \leq n^{- \phi} $ for a constant $\phi>0$. Then for any constant $\e>0$, we have that with high probability,
\be\label{op rough1} %(1-\sqrt{d_1})^2 - \e \le  \lambda_p(Z_1^\top Z_1)   \le  
\lambda_1(Z_1^\top Z_1) \le \frac{(\sqrt{n_1}+\sqrt{p})^2}{n} + n^\e q,
\ee
and
\be\label{op rough2} 
\frac{(\sqrt{n_2}-\sqrt{p})^2}{n} -  n^\e q \le  \lambda_p (Z_2^\top Z_2)  \le  \lambda_1(Z_2^\top Z_2) \le \frac{(\sqrt{n_2}+\sqrt{p})^2}{n} +  n^\e q .
\ee
where $\lambda_i(Z_k^\top Z_k)$, $k=1,2$ and $i=1,\cdots,p$, is the $i$-th largest eigenvalue of $Z_k^\top Z_k$.
\end{lemma}
\begin{proof}
This lemma essentially follows from \cite[Theorem 2.10]{isotropic}, although the authors considered the case with $q \prec n^{-1/2}$ only. The results for larger $q$ follows from \cite[Lemma 3.12]{DY}, but only the bounds for largest eigenvalues are given there in order to avoid the issue with the smallest eigenvalue when $d_{2}$ is close to 1. However, under the assumption \eqref{assm2}, the lower bound for the smallest eigenvalue follows from the same arguments as in \cite{DY}. Hence we omit the details. 
\end{proof}

The rest of the proof is organized as follows. In Section \ref{sec locallaw1}, we introduce the concept of resolvents, and give an almost optimal convergent estimate on it---Theorem \ref{LEM_SMALL}. This estimate is conventionally called {\it local law} in random matrix theory literature. Based on Theorem \ref{LEM_SMALL}, we then complete the proof of Lemma \ref{lem_cov_shift} and Lemma \ref{lem_cov_derivative}. The proof of Theorem \ref{LEM_SMALL} is presented in Section \ref{sec_Gauss}.

\subsection{Resolvent and Local Law}\label{sec locallaw1}


Our main goal is to study the matrix inverse $(X_1^\top X_1+X_2^\top X_2)^{-1}$ for $X_1=\sqrt{n}Z_1\Sigma_1^{1/2}$ and $X_2=\sqrt{n}Z_2\Sigma_2^{1/2}$. 
%\begin{align*}
%(\cal Q_1+\cal Q_2)^{-1}=\left( \Sigma_1^{1/2}Z_1^\top Z_1\Sigma_1^{1/2}+\Sigma_2^{1/2}Z_2^\top Z_2\Sigma_2^{1/2}\right)^{-1} .
%\end{align*}
Using \eqref{eigen2}, we can rewrite it as
\be\label{eigen2extra}(X_1^\top X_1+X_2^\top X_2)^{-1}=n^{-1}\Sigma_2^{-1/2}V\left(   \Lambda U^\top Z_1^\top Z_1 U\Lambda  + V^\top Z_2^\top Z_2V\right)^{-1}V^\top\Sigma_2^{-1/2}.\ee
For this purpose, we shall study the following matrix for $z\in \C_+$, 
\be\label{mainG}
\cal G(z):=\left(   \Lambda U^\top Z_1^\top Z_1 U\Lambda  + V^\top Z_2^\top Z_2V -z\right)^{-1},\quad z\in \C_+,
\ee
which we shall refer to as resolvent (or Green's function).

Next we introduce a convenient self-adjoint linearization trick. 
%This idea dates back at least to Girko, see e.g., the works \cite{girko1975random,girko1985spectral} and references therein. 
It has been proved to be useful in studying the local laws of random matrices of the Gram type \cite{Anisotropic, AEK_Gram, XYY_circular}. We define the following $(p+n)\times (p+n)$ self-adjoint block matrix, which is a linear function of $Z_1$ and $Z_2$:
%\begin{definition}[Linearizing block matrix]\label{def_linearHG}%definiton of the Green function
%We define the $(n+N)\times (n+N)$ block matrix
 \begin{equation}\label{linearize_block}
   H \equiv H(Z_1,Z_2): = \left( {\begin{array}{*{20}c}
   { 0 } & \Lambda U^{\top}Z_1^\top & V^\top Z_2^\top  \\
   {Z_1 U\Lambda  } & {0} & 0 \\
   {Z_2V} & 0 & 0
   \end{array}} \right).
 \end{equation}
Then we define its resolvent as
 \begin{equation}\label{eqn_defG}
 G \equiv G (Z_1,Z_2,z):= \left[H(Z_1,Z_2)-\left( {\begin{array}{*{20}c}
   { z\id_{p}} & 0 & 0 \\
   0 & { \id_{n_1}}  & 0\\
      0 & 0  & { \id_{n_2}}\\
\end{array}} \right)\right]^{-1} , \quad z\in \mathbb C_+ .
 \end{equation}
%It is easy to verify that the eigenvalues $\lambda_1(H)\ge \ldots \ge \lambda_{n+N}(H)$ of $H$ are related to the ones of $\mathcal Q_1$ through
%\begin{equation}\label{Heigen}
%\lambda_i(H)=-\lambda_{n+N-i+1}(H)=\sqrt{\lambda_i\left(\mathcal Q_2\right)}, \ \ 1\le i \le n\wedge N, \quad \text{and}\quad \lambda_i(H)=0, \ \ n\wedge N + 1 \le i \le n\vee N.
%\end{equation}
%and
%$$\lambda_i(H)=0, \ \ n\wedge N + 1 \le i \le n\vee N.$$
%where we used the notations $n\wedge N:=\min\{N,M\}$ and $n\vee N:=\max\{N,M\}$. 
%\begin{definition}[Index sets]\label{def_index}
For simplicity of notations, we define the index sets
$$\cal I_1:=\llbracket 1,p\rrbracket, \quad  \cal I_2:=\llbracket p+1,p+n_1\rrbracket, \quad \cal I_3:=\llbracket p+n_1+1,p+n_1+n_2\rrbracket ,\quad \cal I:=\cal I_1\cup \cal I_2 \cup \cal I_3.$$
 We will consistently use the latin letters $i,j\in\sI_{1}$, greek letters $\mu,\nu\in\sI_{2}\cup \sI_{3}$, and $\fa,\fb\in \cal I$. Correspondingly, the indices of the matrices $Z_1$ and $Z_2$ are labelled as 
 $$Z_1= (z_{\mu i}:i\in \mathcal I_1, \mu \in \mathcal I_2), \quad Z_2= (z_{\nu i}:i\in \mathcal I_1, \nu \in \mathcal I_3).$$
%Moreover, we denote $\overline i:= i+p$ for $i\in \cal I_1$, $\overline j:= j-p$ for $j\in \cal I_2$, $\overline \mu : = \mu +n $ for $\mu \in \cal I_3$, and $\overline \nu : = \nu - n $ for $\nu \in \cal I_4$. 
%\end{definition}
%\begin{definition}[Resolvents]\label{resol_not}
%For $z = E+ \ii \eta \in \mathbb C_+,$ we define the resolvents $G(z)$. 
%Then we denote the $\cal I_\al \times \cal I_\beta$ block of $ G(z)$ by $ \cal G_{\al\beta}(z)$ for $\al,\beta=1,2,3$. Moreover,  
%Then we denote the $\cal I_1 \times \cal I_1$ block of $ G(z)$ by $ \cal G_{L}(z)$, the $\cal I_1 \times (\cal I_2 \cup \cal I_3)$ block by $\cal G_{LR}$, the $ (\cal I_2 \cup \cal I_3)\times \cal I_1$ block by $\cal G_{RL}$, and the $ (\cal I_2 \cup \cal I_3)\times (\cal I_2 \cup \cal I_3)$ block by $\cal G_R$. 
%We denote the $(\cal I_1\cup \cal I_2)\times (\cal I_1\cup \cal I_2)$ block of $ G(z)$ by $ \cal G_L(z)$, the $(\cal I_1\cup \cal I_2)\times (\cal I_3\cup \cal I_4)$ block by $ \cal G_{LR}(z)$, the $(\cal I_3\cup \cal I_4)\times (\cal I_1\cup \cal I_2)$ block  by $ \cal G_{RL}(z)$, and the $(\cal I_3\cup \cal I_4)\times (\cal I_3\cup \cal I_4)$ block by $ \cal G_R(z)$. 
%Recalling the notations in \eqref{def Sxy}, we define $\cal H:=S_{xx}^{-1/2}S_{xy}S_{yy}^{-1/2}$ and
%\be\label{Rxy}
%\begin{split}
% R_1(z):=(\cal C_{XY}-z)^{-1}&=(\cal H\cal H^T-z)^{-1}, \\
% R_2(z):=(\cal C_{YX}-z)^{-1}&=(\cal H^T\cal H-z)^{-1},  \quad m(z):= q^{-1}\tr  R_2(z).
% \end{split}
%\ee
%Note that we have $R_1\cal H = \cal HR_2$, $\cal H^T R_1 = R_2 \cal H^T $, and 
%\be\label{R12} \tr  R_1 = \tr  R_2 - \frac{p-q}{z}= q  m(z) - \frac{p-q}{z},\ee
%since $\cal C_{XY}$ has $(p-q)$ more zeros eigenvalues than $\cal C_{YX}$. 
%Finally, we can define ${\cal G}^b_L(z)$, ${\cal G}^b_R(z)$, $ m^b_\al(z)$, $\cal H^b$, $R_{1,2}^b$, etc.\;in the obvious way by replacing $Y$ with $\cal Y$. 
%for $\wt {\mathcal Q}_{1,2}$ as
%\begin{equation}\label{def_green}
%\mathcal G_1(X,z):=\left(\wt{\mathcal Q}_1(X) -z\right)^{-1} , \ \ \ \mathcal G_2 (X,z):=\left(\wt{\mathcal Q}_2(X)-z\right)^{-1} .
%\end{equation}
% We denote the ESD $\rho^{(n)}$ of $\wt {\mathcal Q}_{1}$ and its Stieltjes transform as
%\be\label{defn_m}
%\rho\equiv \rho^{(n)} := \frac{1}{n} \sum_{i=1}^n \delta_{\lambda_i(\wt{\mathcal Q}_1)},\quad m(z)\equiv m^{(n)}(z):=\int \frac{1}{x-z}\rho_{1}^{(n)}(dx)=\frac{1}{n} \mathrm{Tr} \, \mathcal G_1(z).
%\ee
%We also introduce the following quantities:
%$$m_1(z)\equiv m_1^{(n)}(z):= \frac{1}{N}\sum_{i=1}^n\sigma_i (\mathcal G_1)_{ii}(z) ,\quad m_2(z)\equiv m_2^{(n)}(x):=\frac{1}{N}\sum_{\mu=1}^N \wt\sigma_\mu (\mathcal G_2)_{\mu\mu}(z). $$
%\end{definition}
For simplicity, we abbreviate $W:=(\Lambda^\top U^\top Z_1^\top, V^\top Z_2^\top)$. By Schur complement formula, we can obtain that
 \begin{equation} \label{green2}
 G (z):=  \left( {\begin{array}{*{20}c}
   { \cal G (z)} & \cal G(z)W  \\
   W^\top \cal G(z) & z \cal G_R
\end{array}} \right)^{-1},\quad \cal G_R:=(W^\top W - z)^{-1} .
 \end{equation}
%one can find that %(recall \eqref{mainG})
%\be \label{green2}
%\begin{split}
%& \cal G_{L}=\left(WW^\top -z \right)^{-1} =\cal G ,\quad \cal G_{LR}=\cal G_{RL}^\top  = \cal G W , \quad \cal G_R= z\left(W^\top W - z\right)^{-1}.
%\end{split}
%\ee
(Here $R$ in the subindex of $\cal G_R$ means the lower-right block.) Thus a control of $G$ yields directly a control of the resolvent $\mathcal G$. We also introduce the following random quantities which are some partial traces and weighted partial traces:
\be\label{defm}
\begin{split} 
m(z) :=\frac1p\sum_{i \in \cal I_1}G_{ii}(z) ,\quad & m_1(z):=\frac1p\sum_{i \in \cal I_1}\lambda_i^2 G_{ii}(z),\\
 m_2(z):= \frac{1}{n_1}\sum_{\mu \in \cal I_2}G_{\mu\mu}(z) ,\quad & m_3(z):= \frac{1}{n_2}\sum_{\mu \in \cal I_3}G_{\mu\mu}(z) .
\end{split}
\ee
%Now using Schur complement formula, we can verify that the (recall \eqref{def_green})
%\begin{align} 
%G = \left( {\begin{array}{*{20}c}
%   { z\mathcal G_1} & \mathcal G_1 \Sig^{1/2} U^{*}X V\tilde \Sig^{1/2}  \\
%   {\tilde\Sig^{1/2}V^\topX^\top U\Sig^{1/2} \mathcal G_1} & { \mathcal G_2 }  \\
%\end{array}} \right) = \left( {\begin{array}{*{20}c}
%   { z\mathcal G_1} & \Sig^{1/2} U^{*}X V\tilde \Sig^{1/2} \mathcal G_2   \\
%   {\mathcal G_2}\tilde\Sig^{1/2}V^\topX^\top U\Sig^{1/2} & { \mathcal G_2 }  \\ 
%\end{array}} \right). \label{green2}
%\end{align}
%where $\mathcal G_{1,2}$ are defined in (\ref{def_green}). 
Our proof will use the spectral decomposition of $G$. Let 
\be\label{SVDW}W= \sum_{k = 1}^{p} {\sqrt {\mu_k} \xi_k } \zeta _{k}^\top ,\quad \mu_1\ge \mu_2 \ge \ldots \ge \mu_{p} \ge 0 =\mu_{p+1} = \ldots = \mu_{n},\ee
be a singular value decomposition of $W$, where
%$\lambda_1\ge \lambda_2 \ge \ldots \ge \lambda_{p} \ge 0 = \lambda_{p+1} = \ldots = \lambda_{n}$ are the eigenvalues, 
$\{\xi_{k}\}_{k=1}^{p}$ are the left-singular vectors, and $\{\zeta_{k}\}_{k=1}^{n}$ are the right-singular vectors.
%orthonormal bases of $\mathbb R^{\mathcal I_1}$ and $\mathbb R^{\mathcal I_2}$, respectively. 
Then using (\ref{green2}), we get that for $i,j\in \mathcal I_1$ and $\mu,\nu\in \mathcal I_2\cup \cal I_3$,
\be\label{spectral}
\begin{split}
& G_{ij} = \sum_{k = 1}^{p} \frac{\xi_k(i) \xi_k^\top(j)}{\mu_k-z}, \ \ G_{\mu\nu} = 
%z\sum_{k = 1}^{n} \frac{\zeta_k(\mu) \zeta_k^\top(\nu)}{\lambda_k-z}=
z\sum_{k = 1}^{n} \frac{\zeta_k(\mu) \zeta_k^\top(\nu)}{\mu_k-z} , \ \ G_{i\mu} = G_{\mu i}= \sum_{k = 1}^{p} \frac{\sqrt{\mu_k}\xi_k(i) \zeta_k^\top(\mu)}{\mu_k-z}.
%\\&   \quad G_{\mu i} = \sum_{k = 1}^{p} \frac{\sqrt{\lambda_k}\zeta_k(\mu) \xi_k^\top(i)}{\lambda_k-z}. 
\end{split}
\ee

%We denote the eigenvalues of $\mathcal Q_1$ and $\mathcal Q_2$ in descending order by $\lambda_1(\mathcal Q_1)\geq \ldots \geq \lambda_{p}(\mathcal Q_1)$ and $\lambda_1(\mathcal Q_2) \geq \ldots \geq \lambda_p(\mathcal Q_2)$. Since $\mathcal Q_1$ and $\mathcal Q_2$ share the same nonzero eigenvalues, we will for simplicity write $\lambda_j$, $1\le j \le N\wedge n$, to denote the $j$-th eigenvalue of both $\mathcal Q_1$ and $\mathcal Q_2$ without causing any confusion. 
 

%\subsection{Resolvents and limiting law}
%
%In this paper, we will study the eigenvalue statistics of $\mathcal Q_{1}$ and $\mathcal Q_2$ through their {\it{resolvents}} (or  {\it{Green's functions}}). It is equivalent to study the matrices 
%\be\label{Qtilde}
%\wt{\mathcal Q}_1(X):=\Sig^{1/2} U^{*}XBX^\topU\Sig^{1/2}, \quad \wt{\mathcal Q}_2(X):=\wt\Sig^{1/2}V^\topX^\top A X V\wt \Sig^{1/2}.
%\ee
%In this paper, we shall denote the upper half complex plane and the right half real line by 
%$$\mathbb C_+:=\{z\in \mathbb C: \im z>0\}, \quad \mathbb R_+:=[0,\infty).$$ %\quad  \mathbb R_*:=(0,\infty).$$
%
%\begin{definition}[Resolvents]\label{resol_not}
%For $z = E+ \ii \eta \in \mathbb C_+,$ we define the resolvents for $\wt {\mathcal Q}_{1,2}$ as
%\begin{equation}\label{def_green}
%\mathcal G_1(X,z):=\left(\wt{\mathcal Q}_1(X) -z\right)^{-1} , \ \ \ \mathcal G_2 (X,z):=\left(\wt{\mathcal Q}_2(X)-z\right)^{-1} .
%\end{equation}
% We denote the ESD $\rho^{(n)}$ of $\wt {\mathcal Q}_{1}$ and its Stieltjes transform as
%\be\label{defn_m}
%\rho\equiv \rho^{(n)} := \frac{1}{n} \sum_{i=1}^n \delta_{\lambda_i(\wt{\mathcal Q}_1)},\quad m(z)\equiv m^{(n)}(z):=\int \frac{1}{x-z}\rho_{1}^{(n)}(\dd x)=\frac{1}{n} \mathrm{Tr} \, \mathcal G_1(z).
%\ee
%We also introduce the following quantities:
%$$m_1(z)\equiv m_1^{(n)}(z):= \frac{1}{N}\sum_{i=1}^n\sigma_i (\mathcal G_1(z) )_{ii},\quad m_2(z)\equiv m_2^{(n)}(x):=\frac{1}{N}\sum_{\mu=1}^N \wt\sigma_\mu (\mathcal G_2(z) )_{\mu\mu}. $$
%
%%, \ \ \rho_{2}^{(n)} := \frac{1}{N} \sum_{i=1}^N \delta_{\lambda_i(\mathcal Q_2)}.$$
%%Then the Stieltjes transforms of $\rho_{1}$ is given by
%%\begin{align*}
%%& m_1^{(n)}(z):=\int \frac{1}{x-z}\rho_{1}^{(n)}(\dd x)=\frac{1}{n} \mathrm{Tr} \, \mathcal G_1(z). 
%%%& m_2^{(n)}(z):=\int \frac{1}{x-z}\rho_{2}^{(n)}(\dd x)=\frac{1}{N} \mathrm{Tr} \, \mathcal G_2(z). %\label{ST_m12}
%%\end{align*}
%%and
%%\begin{equation}
%%m_2^{(n)}(z):=\int \frac{1}{x-z}d\rho_{2}^{M}(x)=\frac{1}{N}\sum_{i=1}^N (\mathcal G_2)_{ii}(z)=\frac{1}{N} \mathrm{Tr} \, \mathcal G_2(z). \label{ST_m2}
%%\end{equation}
%%Similarly, we can also define $m_1(z)\equiv m_1^{(M)}(z):= M^{-1}\mathrm{Tr} \, \mathcal G_1(z)$. 
%\end{definition}


We now describe the asymptotic limit of $\cal G(z)$ as $n\to \infty$. First we define the deterministic limits of $(m_2(z), m_{3}(z))$, denoted by $(m_{2 c}(z),m_{3c}(z))$, as the unique solution to the following system of equations
\begin{equation}\label{selfomega}
\begin{split}
& \frac1{m_{2c}} = \frac{\gamma_n}p\sum_{i=1}^p \frac{\lambda_i^2}{  z+\lambda_i^2 r_1 m_{2c} +r_2 m_{3c}  } - 1 ,\  \ \frac1{m_{3c}} = \frac{\gamma_n}p\sum_{i=1}^p \frac{1 }{  z+\lambda_i^2 r_1 m_{2c} +  r_2 m_{3c}  }- 1 ,
\end{split}
\ee
such that $(m_{2 c}(z), m_{3c}(z))\in \C_+^2$ for $z\in \C_+$, where, for simplicity, we  introduced the following ratios 
\be\label{ratios}
 \gamma_n :=\frac{p}{n},\quad r_1\equiv r_1(n) :=\frac{n_1}{n},\quad r_2\equiv r_2(n) :=\frac{n_2}{n}.
\ee
We then define the matrix limit of $G(z)$ as
\be \label{defn_piw}
\Pi(z) := \begin{pmatrix} -(z+r_1 m_{2c}(z)\Lambda^2  +  r_2 m_{3c}(z))^{-1} & 0 & 0 \\ 0 &  m_{2c}(z)\id_{n_1} & 0 \\ 0 & 0 & m_{3c}(z)\id_{n_2}  \end{pmatrix}.\ee
In particular, the matrix limit of $\cal G(z)$ is given by $-(z+r_1 m_{2c}\Lambda^2 + r_2 m_{3c})^{-1}$. 

If $z=0$, then the equations \eqref{selfomega} in are reduced to 
\begin{equation}\label{selfomega0}
\begin{split}
&r_1x_2+r_2x_3= 1-\gamma_n,\quad    x_2 +\frac1n\sum_{i=1}^p \frac{\lambda_i^2 x_2}{ \lambda_i^2 r_1 x_2+ (1-\gamma_n - r_1x_2) }=1 .
\end{split}
\ee
where $x_2:=-m_{2c}(0)$ and $x_3:=-m_{3c}(0)$. Note that the function 
$$f(x_2):=x_2 +\frac1n\sum_{i=1}^p \frac{\lambda_i^2 x_2}{ \lambda_i^2 x_2+ (1-\gamma_n - r_1x_2) } $$
is a strictly increasing function on $[0,r_1^{-1}(1-\gamma_n)]$. Moreover, we have $f(0)=0<1$ and $f(r_1^{-1}(1-\gamma_n))=r_1^{-1}>1$. Hence by mean value theorem, there exists a unique solution $x_2 \in (0,r_1^{-1}(1-\gamma_n))$. Moreover, it is easy to check that $f'(a)=\OO(1)$ for $a\in [0,r_1^{-1}(1-\gamma_n)]$, and $f(1)>1$ if $1\le r_1^{-1}(1-\gamma_n)$. Hence there exists a constant $\tau>0$, such that 
\be\label{a23}
 \tau \le   x_2 \le \min\{ r_1^{-1}(1-\gamma_n) - \tau, 1-\tau\},\quad \tau<r_2 x_3\le 1 -\gamma_n - r_1\tau .
\ee 
For general $z$ around $z=0$, the existence and uniqueness of the solution $(m_{2 c}(z),m_{3c}(z))$ is given by the following lemma. %Moreover, we also include some basic estimates on it. %{\cor(say something about the previous work)}


\begin{lemma} \label{lem_mbehaviorw}
There exist constants $c_0, C_0>0$ depending only on $\tau$ in \eqref{assm2}, \eqref{assm3}, \eqref{assm32} and \eqref{a23} such that the following statements hold. 
%Fix any constants $c, C>0$. If \eqref{assm20} holds, then we have the following estimates.
%If $|z|\le c_0$, then 
There exists a unique solution to \eqref{selfomega} under the conditions
\be\label{prior1}
|z|\le c_0, \quad  |m_{2 c}(z) - m_{2 c}(0)| + |m_{3 c}(z) - m_{3 c}(0)|\le c_0.
\ee
Moreover, the solution satisfies
\be\label{Lipomega}
 |m_{2 c}(z) - m_{2 c}(0)| + |m_{3 c}(z) - m_{3 c}(0)| \le C_0|z|.
\ee
%and
%\be\label{Immcw}
%0\le \im m_{\al c}(z) \le C_0\eta ,\quad z= E+ \ii\eta \in \C_+, \ \ \al=2,3.
%\ee
%For $z\in \C_+ \cap \{z: c\le |z| \le C\}$, we have
%\begin{equation}\label{absmcw}
% \vert m_{3c}(z) \vert \sim 1,  \quad \left|z^{-1} - (m_{1c}(z)+m_{2c}(z)) + (z-1)m_{1c}(z)m_{2c}(z) \right|\sim 1,
% \ee 
% and
% \be\label{Immcw} 0\le \im m_{3c}(z) \sim \begin{cases}
%    {\eta}/{\sqrt{\kappa+\eta}}, & \text{ if } E \notin [\lambda_-,\lambda_+] \\
%    \sqrt{\kappa+\eta}, & \text{ if } E \in [\lambda_-,\lambda_+]\\
%  \end{cases}.
%\end{equation}
%\end{itemize}
%The above estimates also hold for $m_{1c}$, $m_{2c}(z)$, $m_{4c}(z)$ and $m_c(z)$. Finally, the estimates \eqref{absmcw}, \eqref{eq_mcomplexw} and \eqref{eq_mdiffw} hold for $h(z)$. 
\end{lemma}
The proof is a standard application of the contraction principle. For reader's convenience, we will include its proof in Appendix \ref{sec contract}. As a byproduct of the contraction mapping argument there, we also obtain the following stability result that will be used in the proof of Theorem \ref{LEM_SMALL}. 


\begin{lemma} \label{lem_stabw}
There exist constants $c_0, C_0>0$ depending only on $\tau$ in \eqref{assm2}, \eqref{assm3}, \eqref{assm32} and \eqref{a23} so that the self-consistent equations in \eqref{selfomega} are stable in the following sense. Suppose $|z|\le c_0$ and $m_{k}: \C_+\mapsto \C_+$, $k=2,3$, are analytic functions of $z$ such that 
\be \nonumber %\label{prior12}
|m_{2}(z) - m_{2 c}(0)| + |m_{3}(z) - m_{3 c}(0)|\le c_0.
\ee
Suppose they satisfy the system of equations
\begin{equation}\label{selfomegaerror}
\begin{split}
&\frac{1}{m_{2}} + 1 -\frac{\gamma_n}p\sum_{i=1}^p \frac{\lambda_i^2}{  z+\lambda_i^2r_1m_{2} +r_2 m_{3}  } =\cal E_2,\ \ \frac{1}{m_{3}} + 1 -\frac{\gamma_n}p\sum_{i=1}^p \frac{1 }{  z+\lambda_i^2 r_1m_{2} +  r_2m_{3}  }=\cal E_3,
\end{split}
\ee
for some (random) errors satisfying $  |\mathcal E_2| +  |\mathcal E_3| \le \delta(z),$
where $\delta(z)$ is a deterministic $z$-dependent function satisfying $\delta(z) \le (\log n)^{-1}.$ Then we have 
 \begin{equation}
  \left|m_2(z)-m_{2 c}(z)\right| +  \left|m_3(z)-m_{3 c}(z)\right|\le C_0\delta(z).\label{Stability1}
\end{equation}
\end{lemma}



%It was shown in \cite{Separable} that if $d_N \to d \in (0,\infty)$ and $\pi_A^{(n)}$, $\pi_B^{(n)}$ converge to certain probability distributions, then almost surely $\rho^{(n)}$ converges to a deterministic distributions $ \rho_{\infty}$. We now describe it through the Stieltjes transform
%$$m_{\infty}(z):=\int_{\mathbb R} \frac{\rho_{\infty}(\dd x)}{x-z}, \quad z \in \mathbb C_+.$$
%For any finite $N$ and $z\in \mathbb C_+$, we define $(m^{(n)}_{1c}(z),m^{(n)}_{2c}(z))\in \mathbb C_+^2$ as the unique solution to the system of self-consistent equations
%\begin{equation}\label{separa_m12}
%{m^{(n)}_{1c}(z)} = d_N \int\frac{x}{-z\left[1+xm^{(n)}_{2c}(z) \right]} \pi_A^{(n)}(\dd x), \quad  {m^{(n)}_{2c}(z)} =  \int\frac{x}{-z\left[1+xm^{(n)}_{1c}(z) \right]} \pi_B^{(n)}(\dd x).
%\end{equation}
%Then we define
%\begin{equation}\label{def_mc}
%m_c(z)\equiv m_c^{(n)}(z):= \int\frac{1}{-z\left[1+xm^{(n)}_{2c}(z) \right]} \pi_A^{(n)}(\dd x).
%\end{equation}
%It is easy to verify that $m_c^{(n)}(z)\in \mathbb C_+$ for $z\in \mathbb C_+$. Letting $\eta \downarrow 0$, we can obtain a probability measure $\rho_{c}^{(n)}$ with the inverse formula
%\begin{equation}\label{ST_inverse}
%\rho_{c}^{(n)}(E) = \lim_{\eta\downarrow 0} \frac{1}{\pi}\Im\, m^{(n)}_{c}(E+\ii \eta).
%\end{equation}
%If $d_N \to d \in (0,\infty)$ and $\pi_A^{(n)}$, $\pi_B^{(n)}$ converge to certain probability distributions, then $m_c^{(n)}$ also converges and we define
%$$m_{\infty}(z):=\lim_{N\to \infty} m_c^{(n)}(z), \ \ z \in \mathbb C_+.$$
%Letting $\eta \downarrow 0$, we can recover the asymptotic eigenvalue density $ \rho_{\infty}$ with
%\begin{equation}\label{ST_inverse}
%\rho_{\infty}(E) = \lim_{\eta\downarrow 0} \frac{1}{\pi}\Im\, m_{\infty}(E+\ii \eta).
%\end{equation}
%It is also easy to see that $\rho_\infty$ is the weak limit of $\rho_{c}^{(n)}$. 
%%The measure $ \rho_{\infty}$ is sometimes called the {\it{multiplicative free convolution}} of $\pi_A$, $\pi_B$ with the Marchenko-Pastur (MP) law (see e.g. \cite{AGZ,VDN}), i.e. $\pi_A \boxtimes \rho_{MP} \boxtimes \pi_B$, where $\rho_{MP}$ denotes the MP distribution. 
%
%The above definitions of $m_c^{(n)}$, $\rho_c^{(n)}$, $m_\infty$ and $\rho_\infty$ make sense due to the following theorem. Throughout the rest of this paper, we often omit the super-indices $(n)$ and $(N)$ from our notations. 
%
%%Throughout the rest of this paper, we will often omit the super-index $n$ or $N$ from our notations. 
%
%\begin{theorem} [Existence, uniqueness, and continuous density]
%For any $z\in \mathbb C_+$, there exists a unique solution $(m_{1c},m_{2c})\in \mathbb C_+^2$ to the systems of equations in (\ref{separa_m12}). The function $m_c$ in (\ref{def_mc}) is the Stieltjes transform of a probability measure $\mu_c$ supported on $\mathbb R_+$. Moreover, $\mu_c$ has a continuous derivative $\rho_c(x)$ on $(0,\infty)$, which is defined by \eqref{ST_inverse}.
%\end{theorem}
%\begin{proof}
%See {\cite[Theorem 1.2.1]{Zhang_thesis}}, {\cite[Theorem 2.4]{Hachem2007}} and {\cite[Theorem 3.1]{Separable_solution}}.
%\end{proof}
%
%
%
% 
%Now we go back to study the equations in (\ref{separa_m12}). If we define the function
%% Corresponding to the equation in (\ref{separa_m12}), we define the function 
%\begin{equation}\label{separable_MP}
%f(z,\al):=- \al + \int\frac{x}{-z+xd_N \int\frac{t}{1+t\al} \pi_A(\dd t)} \pi_B(\dd x) ,
%\end{equation}
%then $m_{2c}(z)$ can be characterized as the unique solution to the equation $f(z,\al)=0$ of $\al$ with $\Im \, \al> 0$, and $m_{1c}(z)$ is defined using the first equation in \eqref{separa_m12}.
%%as $$m_{1c}(z) = d_N \int\frac{x}{-z\left[1+xm_{2c}(z) \right]} \pi_A(\dd x).$$
%Moreover, $m_{1,2c}(z)$ are the Stieltjes transforms of densities $\rho_{1,2c}$:
%$$\rho_{1,2c}(E) = \lim_{\eta\downarrow 0} \frac{1}{\pi}\Im\, m_{1,2c}(E+\ii \eta).$$
%Then we have the following result.
%
%\begin{lemma}\label{lambdar}%[Support of the deformed MP law]
%The densities $\rho_{c}$ and $\rho_{1,2c}$ all have the same support on $(0,\infty)$, which is a union of intervals: %connected components:
%\begin{equation}\label{support_rho1c}
%{\rm{supp}} \, \rho_{c} \cap (0,\infty) ={\rm{supp}} \, \rho_{1,2c} \cap (0,\infty) = \bigcup_{k=1}^p [a_{2k}, a_{2k-1}] \cap (0,\infty),
%\end{equation}
%where $p\in \mathbb N$ depends only on $\pi_{A,B}$. Moreover, $(x,\al)=(a_k, m_{2c}(a_k))$ are the real solutions to the equations
%\begin{equation}
%f(x,\al)=0, \ \ \text{and} \ \ \frac{\partial f}{\partial \al}(x,\al) = 0. \label{equationEm2}
%\end{equation}
%Moreover, we have $m_{1c}(a_1) \in (-\wt \sigma_1^{-1}, 0)$ and $m_{2c}(a_1) \in (-\sigma_1^{-1}, 0)$. %Finally, under (\ref{assm2}) and (\ref{assm3}), we have $a_1 \le C$ for some constant $C>0$. 
%\end{lemma}
%\begin{proof}
%See Section 3 of \cite{Separable_solution}.
%\end{proof}
%
% %It is easy to observe that $b_k=m_{2c}(a_k)$ according to the definition of $f$. 
% We shall call $a_k$ the spectral edges. In particular, we will focus on the rightmost edge $\lambda_+ := a_1$. 
%%\begin{equation}\label{right_edge}
%%\lambda_+ := a_1 
%%\end{equation}
%%throughout the following.
%Now we make the following assumption: there exists a constant $\tau>0$ such that %{\color{red}(can we remove one of the conditions?)}
%\begin{equation}\label{assm_gap}
%1 + m_{1c}(\lambda_+) \wt \sigma_1 \ge \tau, \quad 1 + m_{2c}(\lambda_+) \sigma_1\ge \tau. %\quad \frac{\partial^2 f}{\partial m^2}\left(\lambda_+,m_{2c}(\lambda_+)\right) \ge \tau.
%\end{equation}
%This assumption guarantees a regular square-root behavior of the spectral densities $\rho_{1,2c}$ near $\lambda_+$ as shown by the following lemma.
%%(see Lemma \ref{lem_mbehavior} below), which is used in proving the local deformed MP law at the soft edge.
%
%\begin{lemma} \label{lambdar_sqrt}
%Under the assumptions \eqref{assm2}, \eqref{assm3} and \eqref{assm_gap}, there exist constants $a_{1,2}>0$ such that
%\be\label{sqroot3}
%\rho_{1,2c}(\lambda_+ - x) = a_{1,2} x^{1/2} + \OO(x), \quad x\downarrow 0,
%\ee
%and
%\be\label{sqroot4}
%\quad m_{1,2c}(z) = m_{1,2c}(\lambda_+) + \pi a_{1,2}(z-\lambda_+)^{1/2} + \OO(|z-\lambda_+|), \quad z\to \lambda_+ , \ \ \im z\ge 0.
%\ee
%The estimates \eqref{sqroot3} and \eqref{sqroot4} also hold for $\rho_c$ and $m_c$ with a different constant. 
%\end{lemma}
% 




%In particular, we shall denote
%\begin{equation}
%S(c_0,C_0,-\infty):= \left\{z=E+ \ii \eta: \lambda_r - c_0 \leq E \leq C_0 \lambda_r, 0 \leq \eta \leq 1 \right\}.
%\end{equation}
%We define the distance to the rightmost edge as
%\begin{equation}
%\kappa \equiv \kappa_E := \vert E -\lambda_r\vert , \ \ \text{for } z= E+\ii \eta.\label{KAPPA}
%\end{equation}

%Then we have the following lemma, which summarizes some basic properties of $m_{2c}$ and $\rho_{2c}$.
%%{\color{red}Discuss about the case \eqref{assm3extra}. }
%
%\begin{lemma}\label{lem_mbehavior}
%Suppose the assumptions \eqref{assm2}, \eqref{assm3} and \eqref{assm_gap} hold. Then
%there exists sufficiently small constant $\tilde c>0$ such that the following estimates hold:
%\begin{itemize}
%\item[(1)]
%\begin{equation}
%\rho_{1,2c}(x) \sim \sqrt{\lambda_r-x}, \quad \ \ \text{ for } x \in \left[\lambda_r - 2\tilde c,\lambda_r \right];\label{SQUAREROOT}
%\end{equation}
%\item[(2)] for $z =E+\ii \eta\in S(\tilde c,C_0,-\infty)$, 
%\begin{equation}\label{Immc}
%\vert m_{1,2c}(z) \vert \sim 1,  \quad  \im m_{1,2c}(z) \sim \begin{cases}
%    {\eta}/{\sqrt{\kappa+\eta}}, & \text{ if } E\geq \lambda_r \\
%    \sqrt{\kappa+\eta}, & \text{ if } E \le \lambda_r\\
%  \end{cases};
%\end{equation}
%%for $z = E+\ii \eta\in S(\tilde c,C_0,\omega)$;
%\item[(3)] there exists constant $\tau'>0$ such that
%\begin{equation}\label{Piii}
%\min_{\mu\in \mathcal I_2} \vert 1 + m_{1c}(z)\tilde \sigma_\mu \vert \ge \tau', \quad \min_{i\in \mathcal I_1} \vert 1 + m_{2c}(z)\sigma_i  \vert \ge \tau',
%\end{equation}
%for any $z \in S(\tilde c,C_0,-\infty)$.
%\end{itemize}
%The estimates \eqref{SQUAREROOT} and \eqref{Immc} also hold for $\rho_c$ and $m_c$. 
%\end{lemma}
%%and
%%\begin{equation}
%%  \operatorname{Im} m_{2c}(z) \sim \begin{cases}
%%    {\eta}/{\sqrt{\kappa+\eta}}, & E\geq \lambda_r \\
%%    \sqrt{\kappa+\eta}, & E \le \lambda_r\\
%%  \end{cases},  \label{SQUAREROOTBEHAVIOR}
%%\end{equation}
%\begin{proof}
%The estimate \eqref{SQUAREROOT} is already given by Lemma \ref{lambdar_sqrt}. The estimate \eqref{Immc} can be proved easily with \eqref{sqroot4}. 
%%The estimate \eqref{SQUAREROOT} for $\rho_c$ is already given by Lemma \ref{lambdar_sqrt}. The estimate \eqref{Immc} for $m_c$ follows from  \eqref{def_mc}, \eqref{Piii}, and \eqref{Immc} for $m_{2c}$.
%It remains to prove \eqref{Piii}. By assumption \eqref{assm_gap} and the fact $m_{2c}(\lambda_r) \in (-\sigma_1^{-1}, 0)$, we have
%$$\left| 1+ m_{2c}(\lambda_r) \sigma_i \right| \ge \tau,  \quad i\in \mathcal I_1.$$
%With \eqref{sqroot4}, we see that if $\kappa+\eta \le 2c_0$ for some sufficiently small constant $c_0>0$, then
%$$\left| 1+ m_{2c}(z)\sigma_k \right| \ge \tau/2.$$
%Then we consider the case with $E \ge \lambda_r + c_0$ and $\eta \le c_1$ for some constant $c_1>0$. In fact, for $\eta=0$ and $E> \lambda_r$, $m_{2c}(E)$ is real and it is easy to verify that $m_{2c}'(E)\ge 0$ using 
%the Stieltjes transform formula 
%%(\ref{Stj_app}). Applying (\ref{SQUAREROOT}) to the Stieltjes transform
%\begin{equation}\label{Stj_app}
%m_{2c}(z):=\int_{\mathbb R} \frac{\rho_{2c}(dx)}{x-z},
%\end{equation}
%Hence we have
%$$ 1+ \sigma_i m_{2c}(E)  \ge 1+ \sigma_i m_{2c}(\lambda_r ) \ge \tau, \ \ \text{ for }E\ge \lambda_r + c_0.$$
%Using (\ref{Stj_app}) again, we can get that 
%$$\left|\frac{\dd m_{2c}(z)}{ \dd z }\right| \le c_0^{-2}, \ \ \text{for } E\ge \lambda_r + c_0.$$ 
%Thus if $c_1$ is sufficiently small, we have
%$$\left| 1+ \sigma_k m_{2c}(E+\ii\eta) \right| \ge  \tau/2$$
%for $E\ge \lambda_r + c_0$ and $\eta \le c_1$. Finally, it remains to consider the case with $\eta \ge c_1$. Note that we have $|m_{2c}(z)| \sim \Im \, m_{2c}(z) \sim 1$ by (\ref{Immc}). If $\sigma_k \le \left|2m_{2c}(z)\right|^{-1}$, then $\left| 1+ \sigma_k m_{2c}(z) \right| \ge 1/2$. Otherwise, we have %Together with (\ref{Immc}), we get that
%$$\left| 1+ \sigma_k m_{2c}(z) \right| \ge \sigma_k \Im\, m_{2c}(z) \ge \frac{\Im\, m_{2c}(z)}{2 |m_{2c}(z)|}\gtrsim 1 .$$
%%for some constant $\tau'>0$. 
%In sum, we have proved the second estimate in \eqref{Piii}. The first estimate can be proved in a similar way. 
%\end{proof}

%Then we have the following estimates for $m_{2c}$:
%%\begin{lemma}[Lemma ]\label{lem_mbehavior}
%%and $\delta_N \le (\log N)^{-1}$, 
%%we have
%\begin{equation}\label{Immc}
%\vert m_{2c}(z) \vert \sim 1, \ \  \Im \, m_{2c}(z) \sim \begin{cases}
%    {\eta}/{\sqrt{\kappa+\eta}}, & \text{ if } E \notin \text{supp}\, \rho_{2c}\\
%    \sqrt{\kappa+\eta}, & \text{ if } E \in \text{supp}\, \rho_{2c}\\
%  \end{cases},
%\end{equation}
%%and 
%\begin{equation}\label{Piii}
%\max_{i\in \mathcal I_1} \vert (1 + m_{2c}(z)\sigma_i)^{-1} \vert = \OO(1).
%\end{equation}
%\end{lemma}

%\begin{remark}
%Recall that $a_k$ are the edges of the spectral density $\rho_{2c}$; see (\ref{support_rho1c}). Hence $\rho_{2c}(a_k)=0$, and we must have $a_k < \lambda_r - 2\tilde c$ for $2\le k \le 2p$. In particular, $S(c_0,C_0,\e)$ is away from all the other edges if we choose $c_0 \le \tilde c$. 
%\end{remark}

%\begin{definition} [Classical locations of eigenvalues]
%The classical location $\gamma_j$ of the $j$-th eigenvalue of $\mathcal Q_1$ is defined as
%\begin{equation}\label{gammaj}
%\gamma_j:=\sup_{x}\left\{\int_{x}^{+\infty} \rho_{c}(x)dx > \frac{j-1}{n}\right\}.
%\end{equation}
%In particular, we have $\gamma_1 = \lambda_r$.
%\end{definition}
%%\begin{remark}
%%If $\gamma_j$ lies in the bulk of $\rho_{2c}$, then by the positivity of $\rho_{2c}$ we can define $\gamma_j$ through the equation
%%\begin{equation*}
%%\int_{\gamma_j}^{+\infty} \rho_{2c}(x)dx = \frac{j-1}{N}.
%%\end{equation*}
%%We can also define the classical location of the $j$-th eigenvalue of $\mathcal Q_1$ by changing $\rho_{2c}$ to $\rho_{1c}$ and $(j-1)/{N}$ to $(j-1)/{M}$ in (\ref{gammaj}). By (\ref{def21}), this gives the same location as $\gamma_j$ for $j\le n\wedge N$.
%%\end{remark}
%
%In the rest of this section, we present some results that will be used in the proof of Theorem \ref{main_thm}. Their proofs will be given in subsequent sections. For any matrix $X$ satisfying Assumption \ref{assm_big1} and the tail condition (\ref{tail_cond}), we can construct a matrix $X^s$ that approximates $X$ with probability $1-\oo(1)$, and satisfies Assumption \ref{assm_big1}, the bounded support condition (\ref{eq_support}) with $q\le N^{-\phi}$ for some small constant $\phi>0$, and
%%${\bf x_3 }$:
%\begin{equation}\label{conditionA2}
%\mathbb{E}\vert  x^s_{ij} \vert^3 =\OO(N^{-{3}/{2}}), \quad   \mathbb{E} \vert  x^s_{ij} \vert^4  =\OO_\prec (N^{-2});
%\end{equation}
%see Section \ref{sec_cutoff} for the details. We will need the following local laws, eigenvalues rigidity, eigenvector delocalization, and edge universality results for separable covariance matrices with $X^s$.
%%with support $q\le N^{-\phi}$ and satisfying the condition (\ref{conditionA2}).
%
%%\cor ---------------------------------- (revise starting from here) ------------------ \nc
%
%We define the deterministic limit $\Pi$ of the resolvent $G$ in (\ref{eqn_defG}) as
%\begin{equation}\label{defn_pi}
%\Pi (z): = \left( {\begin{array}{*{20}c}
%   { -\left(1+m_{2c}(z)\Sigma \right)^{-1} } & 0  \\
%   0 & { - z^{-1} (1+m_{1c}(z)\tilde \Sigma )^{-1} }  \\
%\end{array}} \right) .
%\end{equation}
%Note that we have
%\be\label{mcPi}
%\frac1{nz}\sum_{i\in \mathcal I_1} \Pi_{ii} =m_c. 
%\ee
%Define the control parameters
%\begin{equation}\label{eq_defpsi}
%\Psi (z):= \sqrt {\frac{\Im \, m_{2c}(z)}{{N\eta }} } + \frac{1}{N\eta}.
%\end{equation}
%Note that by (\ref{Immc}) and (\ref{Piii}), we have
%\begin{equation}\label{psi12}
%\|\Pi\|=\OO(1), \quad \Psi \gtrsim N^{-1/2} , \quad \Psi^2 \lesssim (N\eta)^{-1}, \quad \Psi(z) \sim  \sqrt {\frac{\Im \, m_{1c}(z)}{{N\eta }} } + \frac{1}{N\eta},
%\end{equation}
%%and 
%%\begin{equation}\labelpsi12
%%\Psi(z) \sim  \sqrt {\frac{\Im \, m_{1c}(z)}{{N\eta }} } + \frac{1}{N\eta},
%%\end{equation}
%for $z\in S(\tilde c, C_0,-\infty)$. Now we are ready to state the local laws for $G(X,z)$. For the purpose of proving Theorem \ref{main_thm}, we shall relax the condition \eqref{assm_3rdmoment} a little bit. 

%\begin{definition}[Deterministic limit of $G$]
%We define the deterministic limit $\Pi$ of the Green function $G$ in (\ref{green2}) as
%\begin{equation}
%\Pi (z): = \left( {\begin{array}{*{20}c}
%   { -\left(1+m_{2c}(z)\Sigma \right)^{-1} } & 0  \\
%   0 & { - z^{-1} (1+m_{1c}(z)\tilde \Sigma )^{-1} }  \\
%\end{array}} \right) .
%\end{equation}
%%where $\Sigma$ is defined in (\ref{def_Sigma}).
%\end{definition}



%\begin{theorem} [Local laws]\label{LEM_SMALL} %[Results on covariance matrices with small support]
%
%Suppose Assumption \ref{assm_big1} and \eqref{assm_gap} hold. Suppose $X$ satisfies the bounded support condition (\ref{eq_support}) with $q\le N^{-\phi}$ for some constant $\phi>0$. Furthermore, suppose $X$ satisfies \eqref{conditionA2} and
%\be\label{assm_3moment}
%%\mathbb E x_{ij}^3=0,  
%\left|\mathbb E x_{ij}^3\right|\le b_N N^{-2}, \quad 1\le i \le n,\ \  1\le j \le N,
%\ee
%%and
%%\begin{equation}\label{conditionA4}
%%\mathbb{E}\vert x_{ij} \vert^3 \leq C N^{-{3}/{2}}, \quad \mathbb{E} \vert x_{ij} \vert^4  \prec N^{-2},  \quad 1\le i \le n, 1\le j \le N. %\mathbb{E}\vert x_{ij} \vert^3 \prec N^{-{3}/{2}}, \quad  
%%\end{equation}
%where $b_N$ is an $N$-dependent deterministic parameter satisfying $1 \leq b_N \le N^{1/2}$. Fix $C_0>1$ and let $c_0>0$ be a sufficiently small constant. Given any $\epsilon>0$, we define the domain
%\be \label{tildeS}
%\tilde S(c_0,C_0,\e):= S(c_0,C_0,\epsilon) \cap \left\{z = E+ \ii \eta: b_N \left(\Psi^2(z) + \frac{q}{N\eta}\right)\le N^{-\e}\right\}.
%\ee
%Then for any fixed $\e>0$, the following estimates hold. 
%\begin{itemize}
%\item[(1)] {\bf Anisotropic local law}: For any $z\in \tilde S(c_0,C_0,\epsilon)$ and deterministic unit vectors $\mathbf u, \mathbf v \in \mathbb C^{\mathcal I}$,
%\begin{equation}\label{aniso_law}
%\left| \langle \mathbf u, G(X,z) \mathbf v\rangle - \langle \mathbf u, \Pi (z)\mathbf v\rangle \right| \prec q+ \Psi(z).
%\end{equation}
%
%\item[(2)] {\bf Averaged local law}: For any $z \in \tilde S(c_0, C_0, \epsilon)$,  we have
%\begin{equation}
% \vert m(z)-m_{c}(z) \vert \prec q^2 + (N \eta)^{-1}. \label{aver_in1} %+ q^2 
%\end{equation}
%where $m$ is defined in \eqref{defn_m}. Moreover, outside of the spectrum we have the following stronger estimate
%\begin{equation}\label{aver_out1}
% | m(z)-m_{c}(z)|\prec q^2  + \frac{1}{N(\kappa +\eta)} + \frac{1}{(N\eta)^2\sqrt{\kappa +\eta}},
%\end{equation}
%uniformly in $z\in \tilde S(c_0,C_0,\epsilon)\cap \{z=E+\ii\eta: E\ge \lambda_r, N\eta\sqrt{\kappa + \eta} \ge N^\epsilon\}$, where $\kappa$ is defined in \eqref{KAPPA}. 
%\end{itemize}
%The above estimates are uniform in the spectral parameter $z$ and any set of deterministic vectors of cardinality $N^{\OO(1)}$. If $A$ or $B$ is diagonal, then \eqref{aniso_law}-\eqref{aver_out1} hold for $z\in S(c_0,C_0,\epsilon) $.
%\end{theorem}
%
%The following theorem gives that anisotropic local law for $\cal R(z,0)$.

In the following proof, we choose a sufficiently small constants $c_0>0$ such that Lemma \ref{lem_mbehaviorw} and Lemma \ref{lem_stabw} hold. Then we define a domain of the spectral parameter $z$ as
\begin{equation}
\mathbf D:= \left\{z=E+ \ii \eta \in \C_+: |z|\le (\log n)^{-1} \right\}. \label{SSET1}
\end{equation}
The following theorem gives an almost optimal estimate on the resolvent $G$, which is conventionally called the {\it anisotropic local law}. %They imply Lemma \ref{lem_cov_shift} and \eqref{derivative} immediately. 

\begin{theorem} \label{LEM_SMALL} %[Results on covariance matrices with small support]
Suppose Assumption \ref{assm_big1} holds, and $Z_1$ and $Z_2$ satisfy the bounded support condition \eqref{eq_support} for a deterministic parameter $q$ satisfying $ n^{-{1}/{2}} \leq q \leq n^{- \phi} $ for some constant $\phi>0$. Then there exists a sufficiently small constant $c_0>0$ such that the following  anisotropic local law holds uniformly for all $z\in \mathbf D$.
%\begin{itemize}
%\item[(1)] {\bf Anisotropic local law}: 
For any deterministic unit vectors $\mathbf u, \mathbf v \in \mathbb R^{p+n_1+n_2}$, we have
\begin{equation}\label{aniso_law}
\left| \mathbf u^\top (G(z)-\Pi(z)) \mathbf v \right|  \prec  q.
\end{equation}
%
%\item[(2)] {\bf Averaged local law}: We have
%\begin{equation}
% \left\vert {p}^{-1}\sum_{i\in \cal I_1}\left[G_{ii}(z)-\Pi_{ii}(z)\right] \right\vert \prec n^{-1}. \label{aver_in} %+ q^2 
%\end{equation}
%%where $m$ is defined in \eqref{defn_m}. Moreover, outside of the spectrum we have the following stronger estimate
%%\begin{equation}\label{aver_out1}
%% | m(z)-m_{c}(z)|\prec q^2  + \frac{1}{N(\kappa +\eta)} + \frac{1}{(N\eta)^2\sqrt{\kappa +\eta}},
%%\end{equation}
%%uniformly in $z\in \tilde S(c_0,C_0,\epsilon)\cap \{z=E+\ii\eta: E\ge \lambda_r, N\eta\sqrt{\kappa + \eta} \ge N^\epsilon\}$, where $\kappa$ is defined in \eqref{KAPPA}. 
%\end{itemize}
\end{theorem}
The proof of this theorem will be given in Section \ref{sec_Gauss}. We now use it to complete the proof of Lemma \ref{lem_cov_shift} and  Lemma \ref{lem_cov_derivative}. 

\begin{proof}[Proof of Lemma \ref{lem_cov_shift}]
In the setting of Lemma \ref{lem_cov_shift}, using \eqref{eigen2extra} and \eqref{mainG} we can write 
$$\cal R:= ( X_1^\top X_1 + X_2^\top X_2)^{-1}= n^{-1}\Sigma_2^{-1/2}V \cal G(0)V^\top\Sigma_2^{-1/2}.$$
%where $$\cal G(0)=\left(   \Lambda U^\top Z_1^\top Z_1 U\Lambda  + V^\top Z_2^\top Z_2V\right)^{-1}.$$
%%where $\Sigma_1$, $\Sigma_2$, $Z_1$ and $Z_2$ satisfy Assumption \ref{assm_big1}. H
%(Here the extra $n^{-1}$ is due to the choice of the scaling---in the setting of Lemma \ref{lem_cov_shift} the variances of the $Z_{1}$ and $Z_2$ entries are equal to 1, while they are taken to be $n^{-1}$ in the above expression.) 
%%As in \eqref{eigen2}, we assume that $M:=\wt\Sig_1^{1/2} \Sig_2^{-1/2}$ has singular value decomposition
%%\be\label{tildeM}
%%M= U\Lambda V^\top, \quad \Lambda=\text{diag}( \sigma, \ldots, \sigma_p).
%%\ee
%Then as in \eqref{eigen2extra}, we can rewrite $\cal R$ as
%$$\cal R= n^{-1}\Sigma_2^{-1/2}V \cal G(0)V^\top\Sigma_2^{-1/2},\quad \cal G(0)=\left(   \Lambda U^\top Z_1^\top Z_1 U\Lambda  + V^\top Z_2^\top Z_2V\right)^{-1}.$$
If the entries of $\sqrt{n}Z_1$ and $\sqrt{n}Z_2$ have arbitrarily high moments as in \eqref{assmAhigh}, then $Z_1$ and $Z_2$ have bounded support $q=n^{-1/2}$. Using Theorem \ref{LEM_SMALL}, %Now by Corollary \ref{main_cor}, 
we obtain that for any small constant $\e>0$, %with probability $1-\oo(1)$,
\be\label{G0Pi0}\max_{1\le i \le p}| (A\cal R - n^{-1}A \Sigma_2^{-1/2}V \Pi(0)V^\top\Sigma_2^{-1/2})_{ii}| \prec n^{-3/2}\|A\| ,\ee
where by \eqref{defn_piw}, %we have
$$\Pi(0)= -(r_1m_{2c}(0)\Lambda^2  +  r_2m_{3c}(0))^{-1}= (r_1x_2 V^\top M^\top M V +  r_2x_3)^{-1},$$
with $(x_2,x_3)$ satisfying \eqref{selfomega0}. Thus from \eqref{G0Pi0} we get that
$$ \tr (A\cal R) = n^{-1}\tr (r_1x_2  M^\top M  +  r_2x_3)^{-1} +\OO_\prec(n^{-1/2} \|A\|).$$
%with probability $1-\oo(1)$. 
This concludes \eqref{lem_cov_shift_eq} if we rename $(r_1x_2,r_2x_3)$ to $(a_1,a_2)$. 
%For \eqref{lem_cov_shift_eq}, it is a well-known result for inverse Whishart matrices {\color{red}(add some references)}. 

Note that if we set $n_1=0$ and $n_2=n$, then $a_1 = 0$ and $a_2 = (n_2-p) / n_2$ is the solution to \eqref{eq_a12extra}. This gives \eqref{XXA} using \eqref{lem_cov_shift_eq}. 
\end{proof}
 

\begin{proof}[Proof of Lemma \ref{lem_cov_derivative}]
In the setting of Lemma \ref{lem_cov_derivative}, we can write 
\begin{align*}
\Delta&:=n^2\bignorm{\Sigma_2^{1/2} ( X_1^{\top}X_1 + X_2^{\top}X_2)^{-1}\beta}^2 =\beta^\top \Sigma_2^{-1/2}  \left(M^\top Z_1^\top Z_1 M +  Z_2^\top Z_2 \right)^{-2}   \Sigma_2^{-1/2}\beta,
\end{align*}
%where $\wt\Sigma_1:= w^2 \Sigma_1$, $\Sigma_2$, $Z_1$ and $Z_2$ satisfy Assumption \ref{assm_big1} and $M:=\wt\Sig_1^{1/2} \Sig_2^{-1/2}$. 
where in the second step the $n^{2}$ factor disappeared due to the choice of scaling in \eqref{assm1}. 
%Again we assume that $M$ has the singular value decomposition \eqref{tildeM}. 
With \eqref{eigen2}, we can write the above expression as
$$\Delta= \bv^\top (\cal G^2)(0)  \bv,\quad \bv:=V^\top  \Sigma_2^{-1/2} \beta .$$ 
Note that $\cal G^2(0)=\partial_z \cal G|_{z=0}$. Now using Cauchy's integral formula and Theorem \ref{LEM_SMALL}, we get that %with probability $1-\oo(1)$, 
\be\label{apply derivlocal}
\begin{split}
  \bv^\top \cal G^2(0)\bv  = \frac{1}{2\pi \ii}\oint_{\cal C} \frac{ \bv^\top \cal G(z)\bv }{z^2}\dd z &=  \frac{1}{2\pi \ii}\oint_{\cal C} \frac{ \bv^\top\Pi(z)\bv}{z^2}\dd z +\OO_\prec(n^{-1/2}\|\beta\|^2) \\
  &=  \bv^\top \Pi'(0)\bv + \OO_\prec(n^{-1/2}\|\beta\|^2),
\end{split}
\ee
where $\cal C$ is the contour $\{z\in \C: |z| = (\log n)^{-1} \}$. Hence it remains to study the derivatives 
\be\label{dervPi}
\bv^\top \Pi'(0)\bv = \bv \frac{1+r_1m_{2c}'(0)\Lambda^2 + r_2m_{3c}'(0)}{(r_1m_{2c}(0)\Lambda^2 + r_2m_{3c}(0))^2}\bv,
\ee
where we need to calculate the derivatives $ m_{2 c}'(0)$ and $ m_{3 c}'(0)$. 

	
%Then we have
%$$  \left\|X_1^{\top}X_1 (\beta_s - \beta_t)  - \frac{n_1}{n}(\beta_s - \beta_t)\right\|_2 \le C \sqrt{\frac{p}{n}} \left\| (\beta_s - \beta_t)\right\|_2. $$
%\todo{(revise the following proof)} It remain to study the following expression
%\begin{align*}
%\frac{1}{X_1^{\top}X_1 + X_2^{\top}X_2}  \Sigma_2 \frac{1}{X_1^{\top}X_1 + X_2^{\top}X_2}  = \Sigma_2^{-1/2}\left(\frac{1}{A^T Z_1^{\top}Z_1 A + Z_2^{\top}Z_2} \right)^2 \Sigma_2^{-1/2} \\
%\stackrel{d}{=} \Sigma_2^{-1/2}V \left(\frac{1}{\Lambda Z_1^{\top}Z_1 \Lambda + Z_2^{\top}Z_2} \right)^2 V^T \Sigma_2^{-1/2},
%\end{align*}
%where
%\be  \label{eigen2000}
%A:=\Sigma_1^{1/2}\Sigma_2^{-1/2} = U\Lambda V^T ,\quad \Lambda=\text{diag}(\lambda_1, \cdots, \lambda_p).
%\ee
%Using
%$$\left(\frac{1}{\Lambda Z_1^{\top}Z_1 \Lambda + Z_2^{\top}Z_2} \right)^2 =\left. \frac{\dd }{\dd z}\right|_{z=0}\frac{1}{\Lambda Z_1^{\top}Z_1 \Lambda + Z_2^{\top}Z_2 - z} ,$$
%we  need to study the resolvent of
%$$G(z) = \left( \Lambda Z_1^{\top}Z_1 \Lambda + Z_2^{\top}Z_2 - z \right)^{-1}.$$
%Its local law can be studied as in previous subsection (be careful we need to switch the roles of $Z_1$ and $Z_2$). More precisely,  we have that
%$$ G(z) \approx \diag\left( \frac{1}{-z\left( 1+ m_3(z) + \lambda_i^2 m_4(z)\right)}\right)_{1\le i \le p}= \frac{1}{-z\left( 1+ m_3(z) + \Lambda^2 m_4(z)\right)} .$$
%Here $m_{3,4}(z)$ satisfy the following self-consistent equations
%%$$\frac{1}{G_{ii}} \approx -z \left( 1+m_3 + d_i^2m_4 \right), \quad \frac{1}{G_{\mu\mu}} = -z(1+m_1), \ \ \mu\in \cal I_1,\quad \frac{1}{G_{\nu\nu}} = -z(1+m_2), \ \ \nu\in \cal I_2,$$
%%$$m_1= \frac1n\sum_{i}G_{ii}, \quad m_2= \frac1n\sum_{i}d_i^2 G_{ii}, \quad m_3 = \frac1n\sum_{\mu\in \cal I_1} G_{\mu\mu},\quad m_4 = \frac1n\sum_{\mu\in \cal I_2} G_{\mu\mu}.$$
%\begin{align}\label{m34shift}
%\frac{n_2}{n}\frac1{m_3} = - z +\frac1n\sum_{i=1}^p \frac1{  1+m_3 + \lambda_i^2m_4  } ,\quad \frac{n_1}{n}\frac1{m_4} = - z +\frac1n\sum_{i=1}^p \frac{\lambda_i^2 }{  1+m_3 + \lambda_i^2m_4  } .
%\end{align}


Taking implicit differentiation of \eqref{selfomega}, we obtain that
\be \nonumber%\label{dotm34}
\begin{split}
 \frac{m'_{2c}(0) }{m_{2c}^2(0)}=   \frac1n\sum_{i=1}^p \frac{\lambda_i^2\left(1+\lambda_i^2 r_1m'_{2c}(0) + r_2 m'_{3c}(0)\right) }{  ( \lambda_i^2 r_1 m_{2c}(0) +r_2 m_{3c}(0))^2 } ,\quad \frac{m'_{3c}(0)}{m_{3c}^2(0)} &=  \frac1n\sum_{i=1}^p \frac{1+\lambda_i^2 r_1m'_{2c}(0) + r_2 m'_{3c}(0)}{ ( \lambda_i^2 r_1 m_{2c}(0) +r_2 m_{3c}(0))^2  } .
\end{split}
\ee
If we rename $(-r_1m_{2c}(0),-r_2m_{3c}(0))$ to $(a_1, a_2)$ and  $(r_2m_{3c}'(0),r_1m'_{2c}(0))$ to $(a_3 ,a_4)$, then these equations become
%\be%\label{dotm34}
%\begin{split}
% r_1 \frac1{a_1^2}a_4 &=   \frac1n\sum_{i=1}^p \frac{\lambda_i^2\left(1+ \lambda_i^2 a_4 + a_3\right) }{  ( \lambda_i^2 a_1 +a_2)^2 } ,\\
%r_2\frac1{a_2^2}a_3 &=  \frac1n\sum_{i=1}^p \frac{1+\lambda_i^2 a_4 + a_3}{ ( \lambda_i^2 a_1 +a_2)^2  } .
%\end{split}
%\ee
%We can solve the above equations to get $u'_3$ and $u'_4$. Then we have
%$$ G^2(z) \approx   \frac{1 +  u_3'(z) +\Lambda^2 u'_4(z)}{ \left( z+ u_3(z) + \Lambda^2  u_4(z)\right)^2}  $$
%{\cob in certain sense}.
%\cob $a_3\to a_2$, $a_4\to a_1$ \nc
%We now simplify the expressions for $z\to 0$ case. When $z\to 0$, we shall have
%$$u_3(z)= -  {a_3} + \OO(z), \quad u_4(z)= -  a_4 + \OO(z), \quad a_3,a_4 >0.$$
%For $z\to0$, the equations in \eqref{m34shift} are reduced to
%\begin{align}\label{m35shift}
%\frac{n_2}{n}\frac{1}{\todo{a_2}} = 1 +\frac1n\sum_{i=1}^p \frac{1}{\todo{a_2} + \lambda_i^2\todo{a_1}  } ,\quad \frac{n_1}{n}\frac1{\todo{a_1}} = 1 +\frac1n\sum_{i=1}^p \frac{\lambda_i^2 }{  \todo{a_2} + \lambda_i^2 \todo{a_1} }.
%\end{align}
%It is easy to see that these equations are equivalent to
%\begin{align} a_1 + a_2 = 1- \gamma_n, \quad a_1 +\frac1n\sum_{i=1}^p \frac{a_1}{a_1 + a_2/\lambda_i^2}=\frac{n_1}{n}  .\end{align}
%The equations in \eqref{dotm34} reduce to
\be \label{dotm34red}
\begin{split}
\left(\frac{r_2}{a_2^2}- \frac1n\sum_{i=1}^p \frac{1 }{ (   \lambda_i^2a_1+a_2   )^2  }\right) a_3 -  \left(\frac1n\sum_{i=1}^p \frac{  \lambda_i^2 }{ (   \lambda_i^2a_1+a_2   )^2  }\right)a_4 =  \frac1n\sum_{i=1}^p \frac{1 }{ (   \lambda_i^2a_1+a_2   )^2  } ,\\
 %\frac{n_1}{n}\frac1{a_4^2}b_4 =   \frac1n\sum_{i=1}^p \frac{\lambda_i^2\left(1+x_3 + \lambda_i^2 b_4\right) }{  (a_3 + \lambda_i^2a_4)^2  } , \\
 \left(  \frac{r_1}{a_1^2} -  \frac1n\sum_{i=1}^p \frac{\lambda_i^4   }{  (\lambda_i^2a_1+a_2)^2  }\right)a_4 -\left( \frac1n\sum_{i=1}^p \frac{\lambda_i^2  }{  (\lambda_i^2a_1+a_2  )^2  }\right)a_3 =   \frac1n\sum_{i=1}^p \frac{\lambda_i^2 }{  (\lambda_i^2a_1+a_2 )^2  } ,
\end{split}
\ee
which are equivalent to \eqref{eq_a34extra}. 
Then by \eqref{apply derivlocal} and \eqref{dervPi}, we get
\begin{align*}
\Delta&=\beta^\top \Sigma_2^{-1/2} V  \frac{1 + a_3+ a_4\Lambda^2 }{(a_1\Lambda^2 + a_2)^2}  V^\top  \Sigma_2^{-1/2} \beta =\beta^\top \Sigma_2^{-1/2} \frac{1 + a_3+ a_4M^\top M }{(a_1M^\top M + a_2)^2} \Sigma_2^{-1/2} \beta,
\end{align*}
where we used $M^\top M= V\Lambda^2 V^\top$ in the second step. 
%where we denote $x_3:=u_3'(0)$ and $b_4:=u_4'(0)$. Thus we have
%$$\left(\frac{1}{\Lambda Z_1^{\top}Z_1 \Lambda + Z_2^{\top}Z_2} \right)^2  = G^2(0) \approx   \frac{ 1 + x_3 + \Lambda^2  b_4}{\left( a_3 + \Lambda^2  a_4\right)^2} .$$
%This gives that
%\be\label{derivative}
%\begin{split}
%& \frac{1}{X_1^{\top}X_1 + X_2^{\top}X_2}  \Sigma_2 \frac{1}{X_1^{\top}X_1 + X_2^{\top}X_2} =\Sigma_2^{-1/2}V \left(\frac{1}{\Lambda Z_1^{\top}Z_1 \Lambda + Z_2^{\top}Z_2} \right)^2 V^T \Sigma_2^{-1/2}\\
%& \approx  \Sigma_2^{-1/2}V \frac{ 1 + x_3 + \Lambda^2  b_4}{\left( a_3 + \Lambda^2  a_4\right)^2}V^T \Sigma_2^{-1/2}= \Sigma_2^{-1/2}  \frac{ 1 + x_3 +  b_4 A^\top A }{\left( a_3 +   a_4 A^\top A\right)^2}\Sigma_2^{-1/2} .
%\end{split}
%\ee
%Hence we have
%\begin{align*}
%& (\beta_s - \beta_t)^{\top}X_1^{\top}X_1 \frac{1}{X_1^{\top}X_1 + X_2^{\top}X_2}  \Sigma_2 \frac{1}{X_1^{\top}X_1 + X_2^{\top}X_2} X_1^{\top}X_1 (\beta_s - \beta_t) \\
% & \approx (\beta_s - \beta_t)^{\top}\Sigma_1 \Sigma_2^{-1/2}  \frac{ 1 + a_3 +  a_4 A^\top A }{\left( a_2 +   a_1 A^\top A\right)^2}\Sigma_2^{-1/2} \Sigma_1(\beta_s - \beta_t).
%\end{align*}
This concludes Lemma \ref{lem_cov_derivative}.
\end{proof}


Using a simple cutoff argument, it is easy to obtain from Theorem \ref{LEM_SMALL} the following corollary under weaker moment assumptions. 

\begin{corollary}\label{main_cor}
Suppose Assumption \ref{assm_big1} holds. Moreover, assume that the entries of $Z_1$ and $Z_2$ are i.i.d. random variables satisfying \eqref{assm1} and  
\be\label{condition_4e} 
\max_{i,j}\mathbb{E}  |\sqrt{n} z^{(k)}_{ij} | ^{a}  =\OO(1),  \quad k=1,2,
\ee 
for some fixed $a>4$. Then \eqref{aniso_law} holds for $q= n^{2/a-1/2}$ on an event with probability $1-\oo(1)$.
\end{corollary}
\begin{proof}%[Proof of Corollary \ref{main_cor}]
Fix any sufficiently small constant $\e>0$. We choose $q= n^{-c_a +\e}$ with $c_a:=1/2-2/a $. Then we introduce the truncated matrices $\wt Z_1$ and $\wt Z_2$, with entries
$$ \wt z^{(k)}_{ij}:= \mathbf 1\left( |\wt z^{(k)}_{ij}|\le q \right)\cdot z^{(k)}_{ij},\quad k=1,2.$$
%where, for convenience, we again use $j$ to denote the column index of $X$. 
By the moment conditions (\ref{condition_4e}) and a simple union bound, we have
\begin{equation}\label{XneX}
\mathbb P(\wt Z_1 = Z_1,  \wt Z_2 = Z_2) =1-\OO ( n^{-a\e}).
\end{equation}
Using (\ref{condition_4e}) and integration by parts, it is easy to verify that %we can get that
%\begin{align*}
%\mathbb E  |z^{(\al)}_{ij}|1_{|z^{(\al)}_{ij}|> q} =\OO(n^{-2-\e}), \quad \mathbb E |z^{(\al)}_{ij}|^2 1_{|z^{(\al)}_{ij}|> q} =\OO(n^{-2-\e}), \quad \al=1,2,
%\end{align*}
%which imply that
\be\label{meanshif}
|\mathbb E  \tilde z^{(k)}_{ij}| =\OO(n^{-2-\e}), \quad  \mathbb E |\tilde z^{(k)}_{ij}|^2 = n^{-1} + \OO(n^{-2-\e}), \quad k=1,2.
\ee
%and
%$$\left| \mathbb E\tilde x_{ij}^2\right| =O( n^{-2-\omega/2}), \ \ \text{if $x_{ij}$ is complex.} $$
%Moreover, we trivially have
%$$\mathbb E  |\tilde z^{(\al)}_{ij}|^4 \le \mathbb E  |z^{(\al)}_{ij}|^4 =\OO(n^{-2}), \quad \al=1,2.$$
Then we can centralize and rescale $\wt Z_1$ and $\wt Z_2$ as
$ \wh Z_k :=(\wt Z_k - \E \wt Z_k)/{(\E|\wt z^{(k)}_{11}|^2)^{1/2}},$ $k=1,2.$ 
Now $\wh Z_1$ and $\wh Z_2$ satisfy the assumptions in Theorem \ref{LEM_SMALL} with $q= n^{-c_a +\e}$, and \eqref{aniso_law} gives that
$$\left| \mathbf u^\top (G(\wh Z_1,\wh Z_2,z)-\Pi(z)) \mathbf v \right|  \prec  q.$$
Then using \eqref{meanshif} and \eqref{priorim} below, it is obtain that  
$$\left| \mathbf u^\top (G(\wh Z_1,\wh Z_2,z)-G(\wt Z_1,\wt Z_2,z)) \mathbf v \right|  \prec  n^{-1-\e},$$
where we also used the bound $\|\E \wt Z_k\|=\OO(n^{-1-\e})$, $k=1,2$, by \eqref{meanshif}. This shows that \eqref{aniso_law} also holds for $G(\wt Z_1,\wt Z_2,z)$ with $q= n^{-c_a +\e}$, which concludes the proof by \eqref{XneX} and the fact that $\e$ can be chosen arbitrarily small.
\end{proof}
With this corollary, we can easily extend Lemma \ref{lem_cov_shift} and Lemma \ref{lem_cov_derivative} to the case with weaker moment assumptions. Considering the length of this paper, we will not go into further details here. 

%In order for this argument to work, we need to assume 
%%that the third moments of the $X$ entries coincide with that of the Gaussian random variable, i.e. 
%\eqref{assm_3rdmoment}. Under the weaker condition \eqref{assm_3moment}, we cannot prove the local laws up to the optimal scale $\eta \gg N^{-1}$, but only up to the scale $\eta \gg \max\{\frac{qb_N}{N},\frac{\sqrt{b_N}}{N}\}$ near the edge. However, to prove the edge universality, we only need to have a good local law up to the scale $\eta \le N^{-2/3-\e}$, hence $b_N$ can take values up to $b_N \ll N^{1/3}$. (In the proof of Theorem \ref{main_thm} in Section \ref{sec_cutoff}, we will take $b_N=N^{-\e}$ for some small constant $\e>0$.) Finally, if $A$ or $B$ is diagonal, one can prove the local laws up to the optimal scale for all $b_N=\OO( N^{1/2})$ by using an improved comparison argument in \cite{Anisotropic}. 

%Then due to (\ref{match_moments}), we expect that $X$ has ``similar properties" as $\tilde X$, so that these results also hold for $X$. This will be proved with a Green function comparison method, that is, we expand the Green functions with $X$ in terms of Green functions with $\tilde X$ using resolvent expansions, and then estimate the relevant error terms. \cor see Section \ref{comparison} for more details.
%
%we will make use of the results in Theorem \ref{LEM_SMALL}-Lemma \ref{lem_smallcomp} for separable covariance matrices with small support. 
%
%\cor (say something ...) From Theorem \ref{LEM_SMALL}-Lemma \ref{lem_smallcomp}, we see that Theorems \ref{thm_largerigidity}, \ref{lem_comparison} and Lemma \ref{thm_largebound} hold for $\tilde X$. Then due to (\ref{match_moments}), we expect that $X$ has ``similar properties" as $\tilde X$, so that these results also hold for $X$. This will be proved with a Green function comparison method, that is, we expand the Green functions with $X$ in terms of Green functions with $\tilde X$ using resolvent expansions, and then estimate the relevant error terms. \cor see Section \ref{comparison} for more details. \nc


\subsection{Proof of the Anisotropic Local Law}\label{sec_Gauss}

%We divide the proof of Theorem \ref{LEM_SMALL} into two steps. We first prove Theorem \ref{LEM_SMALL} in the special case where $X$ is Gaussian. Then we use a self-consistent comparison arguments developed in \cite{Anisotropic} to prove Theorem \ref{LEM_SMALL} in the general case. 

The main difficulty in the proof of Theorem \ref{LEM_SMALL} is due to the fact that the entries of $Y_1=Z_1 U\Lambda  $ and $Y_2={Z_2V}$ are not independent. However, notice that if the entries of $Z_1\equiv Z_1^{Gauss}$ and $Z_2\equiv Z_2^{Gauss}$ are i.i.d. Gaussian, then by the rotational invariance of the multivariate Gaussian distribution, we have
$$Z_1^{Gauss} U\Lambda \stackrel{d}{=} Z_1^{Gauss} \Lambda, \quad Z_2^{Gauss} V \stackrel{d}{=} Z_2^{Gauss} ,$$
where ``$\stackrel{d}{=}$" means ``equal in distribution". In this case, the problem is reduced to proving the anisotropic local law for $G$ with $U=\id$ and $V=\id$, such that the entries of $Y_1$ and $Y_2$ are independent. This problem can be handled using the standard resolvent methods as in e.g. \cite{isotropic,yang2019spiked,PY}. To go from the Gaussian case to the general $X$ case, we will adopt a continuous self-consistent comparison argument developed in \cite{Anisotropic}. 

%As discussed above, we first prove Theorem \ref{LEM_SMALL} for the case $U=\id$ and $V=\id$, which will imply the local laws in the Gaussian $Z_1$ and $Z_2$ case. Thus 

We now consider the case $U=\id$ and $V=\id$, where we need to deal with the following resolvent:
 \begin{equation}\label{eqn_comparison1}
 G_0 (z):= \left( {\begin{array}{*{20}c}
   { -z \id_{p}} & \Lambda Z_1^\top & Z_2^\top  \\
   {Z_1 \Lambda  } & {-\id_{n_1}} & 0 \\
   {Z_2} & 0 & -\id_{n_2}
   \end{array}} \right) ^{-1} , \quad z\in \mathbb C_+ .
 \end{equation}
%separable covariance matrices of the form $\Sig^{1/2} X \tilde \Sig X^\top \Sig^{1/2}$, which will imply the local laws in the Gaussian $X$ case. Thus in this section, we deal with the following resolvent:
%\begin{equation}\label{eqn_comparison1}
%G(X,z) {=}  \left[\left( {\begin{array}{*{20}c}
%   { 0 } & \Sig^{1/2} X \tilde \Sig^{1/2}   \\
%   {\tilde\Sig^{1/2}X^\top\Sig^{1/2} } & {0}  \\
%   \end{array}} \right)-\left( {\begin{array}{*{20}c}
%   { I_{n\times n}} & 0  \\
%   0 & { zI_{N\times N}}  \\
%\end{array}} \right)\right]^{-1}
%\end{equation}
%with $X$ satisfying \eqref{eq_support} with $q=N^{-1/2}$.
%we choose the entries of $X$ to be $i.i.d.$ Gaussian due to the following reason. If $X=X^{Gauss}$ is Gaussian, then $U^\top X^{Gauss} V \stackrel{d}{=} X^{Gauss}$. Thus for the resolvent $G$ defined in in (\ref{eqn_defG}), we have
%\begin{equation}\label{eqn_comparison1}
%G(X,z) \stackrel{d}{=}  \left[\left( {\begin{array}{*{20}c}
%   { 0 } & \Sig^{1/2} X \tilde \Sig^{1/2}   \\
%   {\tilde\Sig^{1/2}X^\top\Sig^{1/2} } & {0}  \\
%   \end{array}} \right)-\left( {\begin{array}{*{20}c}
%   { I_{n\times n}} & 0  \\
%   0 & { zI_{N\times N}}  \\
%\end{array}} \right)\right]^{-1}
%\end{equation}
%provided that $X$ is Gaussian. In particular, the entries of $\Sig^{1/2} X^{Gauss}\tilde \Sig^{1/2}$ are independent and satisfies the bounded support condition \eqref{eq_support} with $q=N^{-1/2}$, which make the direct proof of Theorem \ref{LEM_SMALL} possible using the methods in \cite{isotropic}. 
We will prove the following local law on $G_0$. 

\begin{proposition}\label{prop_diagonal}
Suppose Assumption \ref{assm_big1} holds, and $Z_1$ and $Z_2$ satisfy the bounded support condition \eqref{eq_support} with $q= n^{-1/2}$. Suppose $U$ and $V$ are identity. Then the estimate \eqref{aniso_law} holds for $G_0(z)$.
%Suppose $X$ satisfies the bounded support condition (\ref{eq_support}) with $q= N^{-1/2}$. Suppose $A$ and $B$ are diagonal, i.e. $U=I_{n\times n}$ and $V=I_{N\times N}$. Fix $C_0>1$ and let $c_0>0$ be a sufficiently small constant. Then for any fixed $\epsilon>0$, the following estimates hold. %there exist constants $C_1>0$ and $\xi_1 \ge 3$ such that the following events hold with $\xi_1$-high probability:
%\begin{itemize}
%\item[(1)] {\bf Anisotropic local law}:  For any $z\in S(c_0,C_0,\epsilon)$ and deterministic unit vectors $\mathbf u, \mathbf v \in \mathbb C^{\mathcal I}$,
%\begin{equation}\label{aniso_diagonal}
%\left| \langle \mathbf u, G(X,z) \mathbf v\rangle - \langle \mathbf u, \Pi (z)\mathbf v\rangle \right| \prec \Psi(z).
%\end{equation}
%
%\item[(2)] {\bf Averaged local law}: We have %{\bf Local deformed MP law}:
%\begin{equation}\label{aver_diagonal}
% | m(z)-m_{c}(z)|\prec ({N\eta})^{-1}
%\end{equation}
%for any $z\in S(c_0,C_0,\epsilon)$, and 
%%Moreover, outside of the spectrum we have the following stronger averaged local law (recall \eqref{KAPPA})
%\begin{equation}\label{aver_out}
% | m(z)-m_{c}(z)|\prec \frac{1}{N(\kappa +\eta)} + \frac{1}{(N\eta)^2\sqrt{\kappa +\eta}},
%\end{equation}
%for any $z\in S(c_0,C_0,\epsilon)\cap \{z=E+\ii\eta: E\ge \lambda_r, N\eta\sqrt{\kappa + \eta} \ge N^\epsilon\}$. 
%\end{itemize}
%Both of the above estimates are uniform in the spectral parameter $z$ and the deterministic vectors $\mathbf u, \mathbf v$.
\end{proposition}
%The proof Proposition \ref{prop_diagonal} is similar to the previous proof of the local laws, such as \cite{isotropic, DY, Anisotropic, yang2019spiked}. Thus instead of giving all the details, we only describe briefly the proof. In particular, we shall focus on the key self-consistent equation argument, which is (almost) the only part that departs significantly from the previous proof in e.g. \cite{isotropic}. 
This section is organized as follows. In Section \ref{sec tools}, we collect some basic estimates and resolvent identities that will be used in the proof of Theorem \ref{LEM_SMALL} and Proposition \ref{prop_diagonal}. Then in Section \ref{sec entry} we give the proof of Proposition \ref{prop_diagonal}, which concludes Theorem \ref{LEM_SMALL} when $Z_1$ and $Z_2$ have i.i.d. Gaussian entries. In Section \ref{sec_comparison}, we describe how to extend the result in Theorem \ref{LEM_SMALL} from the Gaussian case to the case where the entries of $Z_1$ and $Z_2$ are generally distributed. Finally, in Section \ref{sec contract}, we give the proof of Lemma \ref{lem_mbehaviorw} and Lemma \ref{lem_stabw}. In the proof, we always denote the spectral parameter by $z=E+\ii\eta$. 



\subsubsection{Basic Estimates}\label{sec tools}
The estimates in this section work for general $G$, that is, we do not require $U$ and $V$ to be identity.
First with Lemma \ref{SxxSyy}, we can obtain the following a priori estimate on the resolvent $G(z)$ for $z\in \mathbf D$ (recall \eqref{SSET1}).



\begin{lemma}\label{lemm apri}
Suppose the assumptions of Lemma \ref{SxxSyy} holds. Then there exists a constant $C>0$ such that the following estimates hold uniformly in $z,z'\in \mathbf D$ with high probability: 
\be\label{priorim}
\|G(z)\| \le C,%|\im \langle \bu,\cal R  (z,0)\bv\rangle| \le C\eta, 
\ee
and for any deterministic unit vectors $\mathbf u, \mathbf v \in \mathbb R^{p+n}$,
\be\label{priordiff} 
\left| \bu^\top \left[G  (z) - G(z')\right]\bv \right| \le C|z-z'|.
\ee
\end{lemma}
\begin{proof}
As in \eqref{SVDW}, $\{\mu_k\}_{1\le k \le p}$ are the eigenvalues of $WW^\top$. By Lemma \ref{SxxSyy} and the assumption \eqref{assm2}, we obtain that 
%\be\label{lambdap} 
$\mu_p \ge \lambda_p(Z_2^\top Z_2) \ge c_\tau $ for some constant $c_\tau>0$ depending on $\tau$. This further implies the estimate that
$ \inf_{z\in \mathbf D}\min_{1\le k \le p}|\mu_k-z| \ge c_\tau/2.$
Then together with \eqref{spectral}, we obtain \eqref{priorim} and \eqref{priordiff}.
\end{proof}

%Note that by \eqref{priordiff} and local law \eqref{aniso_outstrong}, we have the rough bound
%\be\label{roughinitial}
%\max_{z\in \mathbf D} \max_{\fa , \fb\in \cal I}|\cal R_{\fa \fb} (z)- \Pi_{\fa\fb}(\theta_l)|\le C(\log n)^{-1} 
%\ee
%with high probability. 

%For simplicity, we denote $Y:=\Sig^{1/2} X \tilde \Sig^{1/2}$. 

The following lemma collects basic properties of stochastic domination $\prec$, which will be used tacitly in the proof.

\begin{lemma}[Lemma 3.2 in \cite{isotropic}]\label{lem_stodomin}
Let $\xi$ and $\zeta$ be two families of nonnegative random variables depending on some parameters $u\in \cal U$ or $v\in \cal V$.
\begin{itemize}
\item[(i)] Suppose that $\xi (u,v)\prec \zeta(u,v)$ uniformly in $u\in \cal U$ and $v\in \cal V$. If $|\cal V|\le n^C$ for some constant $C>0$, then $\sum_{v\in \cal V} \xi(u,v) \prec \sum_{v\in \cal V} \zeta(u,v)$ uniformly in $u$.

\item[(ii)] If $\xi_1 (u)\prec \zeta_1(u)$ and $\xi_2 (u)\prec  \zeta_2(u)$ uniformly in $u\in \cal U$, then $\xi_1(u)\xi_2(u) \prec \zeta_1(u) \zeta_2(u)$ uniformly in $u\in \cal U$.

\item[(iii)] Suppose that $\Psi(u)\ge n^{-C}$ is a family of deterministic parameters and $\xi(u)$ satisfies $\mathbb E\xi(u)^2 \le n^C$. If $\xi(u)\prec \Psi(u)$ uniformly in $u$, then we also have $\mathbb E\xi(u) \prec \Psi(u)$ uniformly in $u$.
\end{itemize}
\end{lemma}

Now we introduce the concept of minors, which are defined by removing certain rows and columns of the matrix $H$.

\begin{definition}[Minors]\label{defnMinor}
For any $ (p+n)\times (p+n)$ matrix $\cal A$ and index subset $\mathbb T \subseteq \mathcal I$, we define the minor $\cal A^{(\mathbb T)}:=(\cal A_{\fa\fb}:\fa,\fb \in \mathcal I\setminus \mathbb T)$ as the $ (p+n-|\mathbb T|)\times (p+n-|\mathbb T|)$ matrix obtained by removing all rows and columns indexed by $\mathbb T$. Note that we keep the names of indices when defining $\cal A^{(\mathbb T)}$, i.e. $(\cal A^{(\mathbb{T})})_{\fa\fb}= \cal A_{\fa\fb}$ for $\fa,\fb \notin \mathbb{{T}}$. Correspondingly, we define the resolvent minor as (recall \eqref{green2})
\begin{align*}
G^{(\mathbb T)}:&=\left[\left(H - \left( {\begin{array}{*{20}c}
   { zI_{p}} & 0  \\
   0 & { I_{n}}  \\
\end{array}} \right)\right)^{(\mathbb T)}\right]^{-1} = \left( {\begin{array}{*{20}c}
   { \mathcal G^{(\mathbb T)}} & \mathcal G^{(\mathbb T)} W^{(\mathbb T)}  \\
   {\left(W^{(\mathbb T)}\right)^\top\mathcal G^{(\mathbb T)}} & { \mathcal G_R^{(\mathbb T)} }  \\
\end{array}} \right)  ,
%= \left( {\begin{array}{*{20}c}
%   { z\mathcal G_1^{(\mathbb T)}} & Y^{(\mathbb T)}\mathcal G_2^{(\mathbb T)}   \\
%   {\mathcal G_2^{(\mathbb T)}}\left(Y^{(\mathbb T)}\right)^\top & { \mathcal G_2^{(\mathbb T)} }  \\
%\end{array}} \right),
\end{align*}
and define the partial traces $m^{(\mathbb T)}$ and $m^{(\mathbb T)}_k$, $k=1,2,3,$ by replacing $G$ with $G^{(\mathbb T)}$ in \eqref{defm}.
%$$m_1^{(\mathbb T)}:=\frac{1}{Nz}\sum_{i\notin \mathbb T}\sigma_i G_{ii}^{(\mathbb T)},\ \ m_2^{(\mathbb T)}:= \frac{1}{N}\sum_{\mu \notin \mathbb T}\tilde \sigma_\mu G_{\mu\mu}^{(\mathbb T)}.$$ 
%\be\label{m123}
%\begin{split} 
%m^{(\mathbb T)} := \frac1p \sum_{i \in \cal I_1}G^{(\mathbb T)}_{ii}(z),\quad & m^{(\mathbb T)}_1:=\frac1p\sum^{(\mathbb T)}_{i \in \cal I_1}\lambda_i^2 G^{(\mathbb T)}_{ii}(z),\\
% m_2^{(\mathbb T)}(z):=  \frac1{n_1} \sum_{\mu\in \cal I_2} G_{\mu\mu}^{(\mathbb T)}(z),\quad  &m_3^{(\mathbb T)}(z):=  \frac1{n_2} \sum_{\mu\in \cal I_3} G_{\mu\mu}^{(\mathbb T)}(z),
%\end{split}
%\ee
%where we abbreviated that $\sum_{a}^{(\mathbb T)} := \sum_{a\notin \mathbb T} $. 
For convenience, we will adopt the convention that for any minor $\cal A^{(\mathbb T)}$ defined as above, $\cal A^{(\mathbb T)}_{\fa\fb} = 0$ if $\fa \in \mathbb T$ or $\fb \in \mathbb T$. Moreover, we will abbreviate $(\{\fa\})\equiv (\fa)$ and $(\{\fa, \fb\})\equiv (\fa\fb)$ for $\fa,\fb\in \cal I$.
\end{definition}

%\begin{definition} [Minor of matrix] For a $M \times N$ matrix $X$, $\ \mathbb{T}$ is a subset of  $\ \{1,2,\cdots,N\}$, we define $X^{\{\mathbb{T}\}}$ as the $M \times (N- \vert \mathbb{T} \vert)$ minor of matrix $X$ by deleting the $i$-th($i \in \mathbb{T}$) columns of $X$. We will keep the name of index of $X$ for $X^{\{\mathbb{T} \}}$, namely,
%$(X^{\{\mathbb{T}\}})_{ij}=\mathbf{1}_{ \{j \notin \mathbb{{T}}\}} X_{ij}$. 
%\end{definition}

The following simple resolvent identities are important tools for our proof. 
\begin{lemma}\label{lemm_resolvent}
We have the following resolvent identities. 
\begin{itemize}
\item[(i)]
For $i\in \mathcal I_1$ and $\mu\in \mathcal I_2\cup \cal I_3$, we have
\begin{equation}
\frac{1}{G_{ii}} =  - z - \left( {WG^{\left( i \right)} W^\top} \right)_{ii} ,\quad  \frac{1}{{G_{\mu \mu } }} =  - 1  - \left( {W^\top  G^{\left( \mu  \right)} W} \right)_{\mu \mu }.\label{resolvent2}
\end{equation}

 \item[(ii)]
 For $i\in \mathcal I_1$, $\mu \in \mathcal I_2\cup \cal I_3$, $\fa\in \cal I\setminus \{i\}$ and $\fb\in \cal I\setminus \{ \mu\}$, we have
\begin{equation}
G_{i\fa}   = -G_{ii}  \left( WG^{\left( {i} \right)} \right)_{i\fa},\quad  G_{\mu \fb }  = - G_{\mu \mu }  \left( W^\top  G^{\left( {\mu } \right)}  \right)_{\mu \fb }. \label{resolvent3}
\end{equation}
%For $i\in \mathcal I_1$ and $\mu\in \mathcal I_2$, we have
%\begin{equation}\label{resolvent6}
%\begin{split}
% G_{i\mu } =  -G_{ii}  \left( WG^{\left( {i} \right)} \right)_{i\mu}, \quad G_{\mu i}= -G_{\mu\mu}\left( W^\top G^{(\mu)}\right)_{\mu i}. %\left( { - Y_{i\mu }  +  {\left( {YG^{\left( {i\mu } \right)} Y} \right)_{i\mu } } } \right) . %, \ \  G_{\mu i}  = G_{\mu \mu } G_{ii}^{\left( \mu  \right)} \left( { - Y_{\mu i}^\top  + \left( {Y^\top  G^{\left( {\mu i} \right)} Y^\top  } \right)_{\mu i} } \right).
%\end{split}
%\end{equation}

 \item[(iii)]
 For $\fa \in \mathcal I$ and $\fb, \mathfrak c \in \mathcal I \setminus \{\fa\}$,
\begin{equation}
G_{\fb\mathfrak c}^{\left( \fa \right)}  = G_{\fb\mathfrak c}  - \frac{G_{\fb\fa} G_{\fa\mathfrak c}}{G_{\fa\fa}}, \quad  \frac{1}{{G_{\fb\fb} }} = \frac{1}{{G_{\fb\fb}^{(\fa)} }} - \frac{{G_{\fb\fa} G_{\fa\fb} }}{{G_{\fb\fb} G_{\fb\fb}^{(\fa)} G_{\fa\fa} }}. \label{resolvent8}
\end{equation}
%and
%\begin{equation}
%\frac{1}{{G_{ss} }} = \frac{1}{{G_{ss}^{(r)} }} - \frac{{G_{sr} G_{rs} }}{{G_{ss} G_{ss}^{(r)} G_{rr} }}.
%\label{resolvent9}
%\end{equation}
% \item[(iv)]
%All of the above identities hold for $G^{(\mathbb T)}$ instead of $G$ for $\mathbb T \subset \mathcal I$, and in the case where $A$ and $B$ are not diagonal.
\end{itemize}
\end{lemma}
\begin{proof}
All these identities can be proved directly using Schur's complement formula. The reader can also refer to, for example, \cite[Lemma 4.4]{Anisotropic}.
\end{proof}

%\begin{lemma}\label{Ward_id}
%%Fix constants $c_0,C_0>0$. The following estimates hold uniformly for all $z\in S(c_0,C_0,a)$ for any $a\in \mathbb R$:
%%\begin{equation}
%%\left\| G \right\| \le C\eta ^{ - 1} ,\ \ \left\| {\partial _z G} \right\| \le C\eta ^{ - 2}. \label{eq_gbound}
%%\end{equation}
%%Furthermore, 
%We have the following identities:
%\begin{align}
%& \sum_{i \in \mathcal I_1 }  \left| {G_{j i} } \right|^2 = \sum_{i \in \mathcal I_1 }  \left| {G_{ij} } \right|^2  = \frac{|z|^2}{\eta}\Im\left(\frac{G_{jj}}{z}\right) ,  \label{eq_gsq2} \\
%& \sum_{\mu  \in \mathcal I_2 } {\left| {G_{\nu \mu } } \right|^2 } = \sum_{\mu  \in \mathcal I_2 } {\left| {G_{\mu \nu} } \right|^2 }  = \frac{{\Im \, G_{\nu\nu} }}{\eta}, \label{eq_gsq1}\\ 
%& \sum_{i \in \mathcal I_1 } {\left| {G_{\mu i} } \right|^2 } = \sum_{i \in \mathcal I_1 } {\left| {G_{i\mu} } \right|^2 } = {G}_{\mu \mu}  + \frac{\bar z}{\eta} \Im \, G_{\mu\mu} , \label{eq_gsq3} \\ 
%&\sum_{\mu \in \mathcal I_2 } {\left| {G_{i \mu} } \right|^2 } = \sum_{\mu \in \mathcal I_2 } {\left| {G_{\mu i} } \right|^2 } =  \frac{{G}_{ii}}{z}  + \frac{\bar z}{\eta} \Im\left(\frac{{G_{ii} }}{z}\right) . \label{eq_gsq4} 
% \end{align}
%All of the above estimates remain true for $G^{(\mathbb T)}$ instead of $G$ for any $\mathbb T \subseteq \mathcal I$, and in the case where $A$ and $B$ are not diagonal.
%%Finally, suppose $\{\mathbf v_{i}\}_{i=1}^{N}$ and $\{\mathbf w_{\mu}\}_{\mu=1}^{N}$ are orthonormal bases of $\mathbb C^{\mathcal I_1}$ and $\mathbb C^{\mathcal I_2}$, respectively, then the above estimates remain true if we replace $\mathbf e_i$ with $\mathbf v_i$ and $\mathbf e_\mu$ with $\mathbf v_{\mu}$.
%\label{lemma_Im}
%\end{lemma}
%\begin{proof}
%These estimates and identities can be proved through simple calculations with (\ref{green2}), (\ref{spectral1}) and (\ref{spectral2}). We refer the reader to \cite[Lemma 4.6]{Anisotropic} and \cite[Lemma 3.5]{XYY_circular}.
%\end{proof}
%


%The following lemma gives large deviation bounds for bounded supported random variables. 
%It constitutes the main difference between our proof and the one in \cite{KY2}, where the authors used a large deviation bound for random variables with arbitrarily high moments.

\begin{lemma}[Lemma 3.8 of \cite{EKYY1}]\label{largedeviation}
Let $(x_i)$, $(y_j)$ be independent families of centered and independent random variables, and $(A_i)$, $(B_{ij})$ be families of deterministic complex numbers. Suppose the entries $x_i$, $y_j$ have variances at most $n^{-1}$ and satisfy the bounded support condition (\ref{eq_support}) with $q\le n^{-\phi}$ for some constant $\phi>0$. Then we have the following bounds:
%for any fixed $\xi>0$, the following bounds hold with $\xi$-high probability:
\begin{align*}
\Big\vert \sum_i A_i x_i \Big\vert \prec  q \max_{i} \vert A_i \vert+ \frac{1}{\sqrt{n}}\Big(\sum_i |A_i|^2 \Big)^{1/2} , \quad &\Big\vert \sum_{i,j} x_i B_{ij} y_j \Big\vert \prec q^2 B_d  + qB_o + \frac{1}{n}\Big(\sum_{i\ne j} |B_{ij}|^2\Big)^{{1}/{2}}, \\
\Big\vert \sum_{i} \bar x_i B_{ii} x_i - \sum_{i} (\mathbb E|x_i|^2) B_{ii}  \Big\vert  \prec q B_d   , \quad &\Big\vert \sum_{i\ne j} \bar x_i B_{ij} x_j \Big\vert  \prec qB_o + \frac{1}{n}\Big(\sum_{i\ne j} |B_{ij}|^2\Big)^{{1}/{2}} ,
\end{align*}
where $B_d:=\max_{i} |B_{ii} |$ and $B_o:= \max_{i\ne j} |B_{ij}|.$ Moreover, if all the moments of $\sqrt{n}x_i$ and $\sqrt{n}y_j$ exist in the sense of \eqref{assmAhigh}, then we have stronger bounds
\begin{align*}
\Big\vert \sum_i A_i x_i \Big\vert \prec  \frac{1}{\sqrt{n}}\Big(\sum_i |A_i|^2 \Big)^{1/2} , \quad  & \Big\vert \sum_{i,j} x_i B_{ij} y_j \Big\vert \prec  \frac{1}{n}\Big(\sum_{i\ne j} |B_{ij}|^2\Big)^{{1}/{2}} , \\
 \Big\vert \sum_{i} \bar x_i B_{ii} x_i - \sum_{i} (\mathbb E|x_i|^2) B_{ii}  \Big\vert  \prec \frac1{n}\Big( \sum_i |B_{ii} |^2\Big)^{1/2}  ,\quad & \Big\vert \sum_{i\ne j} \bar x_i B_{ij} x_j \Big\vert  \prec  \frac{1}{n}\Big(\sum_{i\ne j} |B_{ij}|^2\Big)^{{1}/{2}} .
\end{align*}
\end{lemma}  



%Finally, we have the following lemma, which is a consequence of the Assumption \ref{assm_big2}.
%\begin{lemma}\label{lem_assm3}
%There exists constants $c_0, \tau' >0$ such that 
%\begin{equation}\label{Piii}
%|1+m_{2c}(z)\sigma_k |\ge \tau',
%\end{equation}
%for all $z \in S(c_0,C_0,C_1)$ and $1\le k \le M$.
%\end{lemma}
%\begin{proof}
%By Assumption \ref{assm_big2} and the fact $m_{2c}(\lambda_r) \in (-\sigma_1^{-1}, 0)$, we have
%$$\left| 1+ m_{2c}(\lambda_r) \sigma_k \right| \ge \tau,  \ \ 1\le k \le M.$$
%Applying (\ref{SQUAREROOT}) to the Stieltjes transform
%\begin{equation}\label{Stj_app}
%m_{2c}(z):=\int_{\mathbb R} \frac{\rho_{2c}(dx)}{x-z},
%\end{equation}
%one can verify that $m_{2c}(z) \sim \sqrt{z-\lambda_r}$ for $z$ close to $\lambda_r$. Hence if $\kappa+\eta \le 2c_0$ for some sufficiently small constant $c_0>0$, we have
%$$\left| 1+ m_{2c}(z)\sigma_k \right| \ge \tau/2.$$
%Then we consider the case with $E-\lambda_r \ge c_0$ and $\eta \le c_1$ for some constant $c_1>0$. In fact, for $\eta=0$ and $E\ge \lambda_r + c_0$, $m_{2c}(E)$ is real and it is easy to verify that $m_{2c}'(E)\ge 0$ using the formula (\ref{Stj_app}). Hence we have
%$$\left| 1+ \sigma_k m_{2c}(E) \right| \ge \left| 1+ \sigma_k m_{2c}(\lambda_r + c_0) \right| \ge \tau/2, \ \ \text{ for }E\ge \lambda_r + c_0.$$
%Using (\ref{Stj_app}) again, we can get that 
%$$\left|\frac{dm_{2c}(z)}{ d z }\right| \le c_0^{-2}, \ \ \text{for } E\ge \lambda_r + c_0.$$ 
%So if $c_1$ is sufficiently small, we have
%$$\left| 1+ \sigma_k m_{2c}(E+\ii\eta) \right| \ge \frac{1}{2}\left| 1+ \sigma_k m_{2c}(E) \right| \ge \tau/4$$
%for $E\ge \lambda_r + c_0$ and $\eta \le c_1$. Finally, it remains to consider the case with $\eta \ge c_1$. If $\sigma_k \le \left|2m_{2c}(z)\right|^{-1}$, then we have $\left| 1+ \sigma_k m_{2c}(z) \right| \ge 1/2$. Otherwise, we have $\Im \, m_{2c}(z) \sim 1$ by (\ref{SQUAREROOTBEHAVIOR}). Together with (\ref{Immc}), we get that
%$$\left| 1+ \sigma_k m_{2c}(z) \right| \ge \sigma_k \Im\, m_{2c}(z) \ge \frac{\Im\, m_{2c}(z)}{2 m_{2c}(z)} \ge \tau' $$
%for some constant $\tau'>0$.
%\end{proof}

\subsubsection{Entrywise Local Law}\label{sec entry}

%For the proof of Proposition \ref{prop_diagonal}, it is convenient to introduce the following random control parameters.
%
%\begin{definition}[Control parameters]
%We define the random errors
%\begin{equation}\label{eqn_randomerror}
%\Lambda : = \mathop {\max }\limits_{a,b \in \mathcal I} \left| {\left( {G - \Pi } \right)_{ab} } \right|,\ \ \Lambda _o : = \mathop {\max }\limits_{a \ne b \in \mathcal I} \left| {G_{ab} } \right|, \ \ \theta:= |m_1-m_{1c}| +  |m_2-m_{2c}| ,
%\end{equation}
%%Moreover, we define 
%and the random control parameter (recall $\Psi$ defined in \eqref{eq_defpsi})
%\begin{equation}\label{eq_defpsitheta}
%\Psi _\theta  : = \sqrt {\frac{{\Im \, m_{2c}  + \theta }}{{N\eta }}} + \frac{1}{N\eta}.
%\end{equation}
%%and the deterministic control parameter
%%\begin{equation}\label{eq_defpsi}
%%\Psi := \sqrt {\frac{\Im\, m_{2c} }{{N\eta }} } + \frac{1}{N\eta}.
%%\end{equation}
%\end{definition}

The main goal of this subsection is to prove the following entrywise local law, Lemma \ref{prop_entry}. The anisotropic local law \eqref{aniso_law} (in the setting Proposition \ref{prop_diagonal}) then follows from it combined with a polynomialization method as we will explain later. Recall that under the assumptions of Proposition \ref{prop_diagonal}, we have $q=n^{-1/2}$ and
\be\label{diagW}W= (\Lambda Z_1^\top, Z_2^\top).\ee

\begin{lemma}\label{prop_entry}
Suppose the assumptions of Proposition \ref{prop_diagonal} hold. %Fix $C_0>0$ and let $c_0>0$ be a sufficiently small constant. 
Then the following estimate holds uniformly in $z\in \mathbf D$: %there exist constants 
\begin{equation}\label{entry_diagonal}
\max_{\fa,\fb\in \cal I}\left| (G_0)_{\fa\fb}(z)  - \Pi_{\fa\fb} (z) \right| \prec n^{-1/2}.
\end{equation} 
\end{lemma}

%We fix a $\xi_1\ge 3$ throughout this section. 
%Our goal is to prove that $G$ is close to $\Pi$ in the sense of entrywise and averaged local laws. Hence it is convenient to introduce the following random control parameters.

%\begin{definition}[Control parameters]
%We define the entrywise and averaged errors
%\begin{equation}\label{eqn_randomerror}
%\Lambda : = \mathop {\max }\limits_{a,b \in \mathcal I} \left| {\left( {G - \Pi } \right)_{ab} } \right|,\ \ \Lambda _o : = \mathop {\max }\limits_{a \ne b \in \mathcal I} \left| {G_{ab} } \right|, \ \ \theta:= |m_2-m_{2c}| .
%\end{equation}
%Moreover, we define the random control parameter
%\begin{equation}\label{eq_defpsitheta}
%\Psi _\theta  : = \sqrt {\frac{{\Im \, m_{2c}  + \theta }}{{N\eta }}} + \frac{1}{N\eta},
%\end{equation}
%and the deterministic control parameter
%\begin{equation}\label{eq_defpsi}
%\Psi := \sqrt {\frac{\Im\, m_{2c} }{{N\eta }} } + \frac{1}{N\eta}.
%\end{equation}
%\end{definition}
\begin{proof}
The proof of Lemma \ref{prop_entry} is divided into three steps. For simplicity, we will denote $G\equiv G_0$ in the following proof, while keeping in mind that $W$ takes the form in \eqref{diagW}.

\vspace{5pt}

\noindent{\bf Step 1: Large deviation estimates.} In this step, we prove some (almost) optimal large deviation estimates on the off-diagonal entries of $G$, and on the following $Z$ variables. In analogy to \cite[Section 3]{EKYY1} and \cite[Section 5]{Anisotropic}, we introduce the $Z$ variables {\cor(start here)}
\begin{equation*}
  Z_{\fa}^{(\mathbb T)}:=(1-\mathbb E_{\fa})\big(G_{\fa\fa}^{(\mathbb T)}\big)^{-1}, \quad \fa\notin \mathbb T,
\end{equation*}
where $\mathbb E_{\fa}[\cdot]:=\mathbb E[\cdot\mid H^{(\fa)}],$ i.e. it is the partial expectation over the randomness of the $\fa$-th row and column of $H$. Using (\ref{resolvent2}), we get that for $i \in \cal I_1$, $\mu\in \cal I_2$ and $\nu\in \cal I_3$,
\begin{align}
&Z_i =  \lambda_i^2 \sum_{\mu ,\nu\in \mathcal I_2}  G^{(i)}_{\mu\nu} \left(\frac{1}{n}\delta_{\mu\nu} - z_{\mu i}z_{\nu i}\right)+ \sum_{\mu ,\nu\in \mathcal I_3}  G^{(i)}_{\mu\nu} \left(\frac{1}{n}\delta_{\mu\nu} - z_{\mu i}z_{\nu i}\right),  \label{Zi}\\
& Z_\mu= \sum_{i,j \in \mathcal I_1}  {\sigma_i \sigma_j}G^{(\mu)}_{ij} \left(\frac{1}{n} \delta_{ij} - z_{\mu i}z_{\mu j}\right), \quad Z_\nu = \sum_{i,j \in \mathcal I_1} G^{(\nu)}_{ij} \left(\frac{1}{n} \delta_{ij} - z_{\nu i}z_{\nu j}\right).\label{Zmu} 
\end{align}
For simplicity, we introduce the random error
%\begin{equation} \nonumber % \label{eqn_randomerror}
%\begin{split}
$ \Lambda _o : = % \max_{\fa \ne \fb } \left|  G_{\fa\fb}   \right| +  
 \max_{\fa \ne \fb } \left|  G_{\fa\fa}^{-1}G_{\fa\fb}   \right| .$
%\end{split}
%\end{equation}
The following lemma gives the desired large deviations estimates on $\Lambda_o$ and the $Z$ variables.

\begin{lemma}\label{Z_lemma}
Suppose the assumptions in Proposition \ref{prop_diagonal} hold. 
%Let $c_0>0$ be a sufficiently small constant and fix $C_0, \epsilon >0$. Define the $z$-dependent event $\Xi(z):=\{\Lambda(z) \le (\log N)^{-1}\}$. Then there exists constant $C>0$ such that 
Then the following estimates hold uniformly for all $z\in \mathbf D$:
\begin{align}
\Lambda_o + \max_{\fa\in \cal I} |Z_{\fa}|  \prec n^{-1/2}. \label{Zestimate1}
\end{align}
%and 
%\begin{align}
%{\mathbf 1}\left(\eta \ge 1 \right)\left(\Lambda_o + |Z_{a}|\right)\prec \Psi_\theta. \label{Zestimate2}
%\end{align}
%Moreover, then for $z\in S(C_0)$,
%\begin{align}
%1(\Xi)\Lambda_0 \le C(\log N)^{2\xi}\left(q+\Psi_\theta\right) \label{offestimate}
%\end{align}
%holds with $\xi$-high probability.
\end{lemma}
\begin{proof}
%Suppose $\Xi$ holds, then we have $|G_{\fa\fa} -\Pi|\lesssim (\log N)^{-1}$ on $\Xi$, and  
Note that for any $\fa\in \cal I$, $H^{(\fa)}$ and $G^{(\fa)}$ also satisfies the assumptions for Lemma \ref{lemm apri}. Hence \eqref{priorim} and \eqref{priordiff} also hold for $G^{(\fa)}$. Now applying Lemma \ref{largedeviation} to (\ref{Zi}) and \eqref{Zmu}, and using the a priori bound \eqref{priorim}, we get that for any $i\in \cal I_1$, %on $\Xi$,
\begin{equation}\nonumber%\label{estimate_Zi}
\begin{split}
\left| Z_{i}\right|&\lesssim \sum_{\al=2}^3 \left| \sum_{\mu ,\nu\in \mathcal I_{\al}}  G^{(i)}_{\mu\nu} \left(\frac{1}{n}\delta_{\mu\nu} - z_{\mu i}z_{\nu i}\right)\right| \prec n^{-1/2}+ \frac{1}{n} \Big( \sum_{\mu,\nu \in \cal I_2\cup \cal I_3 }  {\left| G_{\mu\nu}^{(i)}  \right|^2 }  \Big)^{1/2} \prec n^{-1/2} , 
%+ \frac{1}{\sqrt{n}} \left( \frac{1}{n}\sum_{\mu \in \cal I_{\al+2}}  {\left(\cal R^{(i)} J_{\al+2} (\cal R^{(i)})^*\right)_{\mu\mu} }  \right)^{1/2} \prec n^{-1/2},
%\\& \prec  \phi_n + \frac{1}{n} \left[ \sum_{\mu\in \cal I_3} \left( 1+ \frac{ \im \left( U(z) G_{[\mu\mu]}^{(i)} \right)_{11} }{\eta}   \right]  \right)^{1/2} ,%= q+ \frac{1}{N}\left( {\sum_\mu \frac{\wt \sigma_\mu  \im G_{\mu\mu}^{(i)} }{\eta} } \right)^{1/2}= q + \sqrt { \frac{ \Im\, m_2^{(i)}  } {N\eta} },
\end{split}
\end{equation}
where in the last step we used \eqref{priorim} to get that for any $\mu$,
\be\label{GG*}\sum_{\nu \in \cal I_2\cup \cal I_3 }  \left| G_{\mu\nu}^{(i)}  \right|^2\le \sum_{\fa \in \cal I }  \left| G_{\mu\fa}^{(i)}  \right|^2 =\left[G^{(i)}(G^{(i)})^* \right]_{\mu\mu} =\OO(1).\ee
Similarly, applying Lemma \ref{largedeviation} to $Z_{\mu}$ and $Z_\nu$ in (\ref{Zmu}) and using \eqref{priorim}, we obtain the same bound.
%\begin{equation}\label{estimate_Zmu} \|Z_{[\mu]}\|\prec \frac{1}{n} \left( \sum_{i,j \in \cal I_{1}\cup \cal I_2}  {\left| \cal R_{ij}^{[\mu]}  \right|^2 }  \right)^{1/2} =  \frac{1}{\sqrt{n}} \left(  \frac{1}{n}\sum_{i \in \cal I_{1}\cup \cal I_2}  {\left(\cal R^{[\mu]} (J_{1}+J_2) (\cal R^{[\mu]})^*\right)_{ii} } \right)^{1/2} \prec n^{-1/2}.\ee
%This completes the proof of \eqref{Zestimate1}.

we have
\begin{equation}
G_{i\fa}   = -G_{ii}  \left( WG^{\left( {i} \right)} \right)_{i\fa},\quad  G_{\mu \fb }  = - G_{\mu \mu }  \left( W^\top  G^{\left( {\mu } \right)}  \right)_{\mu \fb }.  
\end{equation}

Then we prove the off-diagonal estimate on $\Lambda_o$. For $i\in \mathcal I_1$ and $\fa\in \cal I\setminus \{i\}$, using \eqref{resolvent3}, Lemma \ref{largedeviation} and \eqref{priorim}, we obtain that 
\begin{align*}
& \left|G_{ii}^{-1}G_{i\fa}\right| \prec n^{-1/2}+\frac{1}{\sqrt{n}} \Big( \sum_{\mu \in \cal I_2\cup \cal I_3}  {\left| G_{\mu \fa}^{(i)}  \right|^2 }  \Big)^{1/2} \prec n^{-1/2}. 
%\\& \left|G_{\mu\mu}^{-1} G_{\mu\fb} \right| \prec n^{-1/2}+  \frac{1}{\sqrt n} \left( \sum_{i \in \cal I_{1}}  {\left| G_{i\fb}^{(\mu)}  \right|^2 }  \right)^{1/2} \prec n^{-1/2}.
 \end{align*}
 We have a similar estimate for $\left|G_{\mu\mu}^{-1} G_{\mu\fb} \right|$ with $\mu \in \mathcal I_2\cup \cal I_3$ and $\fb\in \cal I\setminus \{ \mu\}$.  
%For $i\in \cal I_1\cup \cal I_2$ and $\mu \in \mathcal I_3$, using \eqref{resolvent3}, Lemma \ref{largedeviation} and \eqref{priorim}, we obtain that  
%$$ \left| G_{ii}^{-1}G_{i\mu} \right|+ \left| G_{\mu\mu}^{-1}G_{\mu i} \right| \prec n^{-1/2} + \frac{1}{\sqrt n} \left( \sum_{\nu \in \cal I_{2}\cup \cal I_3 }  {\left|G^{(i)}_{\nu\mu}  \right|^2 }  \right)^{1/2} + \frac{1}{\sqrt n} \left( \sum_{j \in \cal I_{1} }  {\left|G^{(\mu)}_{ji}  \right|^2 }  \right)^{1/2} \prec   n^{-1/2}.$$
Thus we obtain that $\Lambda_o\prec n^{-1/2}$, which concludes \eqref{Zestimate1}.
\end{proof}

Note that comibining \eqref{priorim} and \eqref{Zestimate1}, we immediately conclude \eqref{entry_diagonal} for $\fa\ne \fb$.


%{\cor
%\begin{lemma}
%Fix constants $c_0,C_0>0$. For any $\mathbb T \subseteq \mathcal I$ and $a\in \mathbb R$, the following bounds hold uniformly in $z\in S(c_0,C_0,a)$:
%\begin{equation}\label{m_T}
%\big| {m_1  - m_1^{\left( \mathbb T \right)} } \big| + \big| {m_2  - m_2^{\left( \mathbb T \right)} } \big| \le \frac{{C\left| \mathbb T \right|}}{{N\eta }}, %\ \ i= 1,2, 
%\end{equation}
%%and 
%%\begin{equation}\label{m11_T}
%%\left| {\frac{1}{N}\sum_{i=1}^M \sigma_i \left(G_{ii}^{(\mathbb T)} - G_{ii}\right)} \right| \le \frac{{C\left| \mathbb T \right|}}{{N\eta }}, %\ \ i= 1,2, 
%%\end{equation}
%where $C>0$ is a constant depending only on $\tau$.
%%where $C>0$ depends only on $C_0 \lambda_r$. %is an absolute constant. %depending only on the aspect ratio $d$.
%\end{lemma}
%\begin{proof}
%For $\mu\in\mathcal I_2$, we have
%\begin{align*}
%\left|m_2-m_2^{(\mu)}\right|& =\frac{1}{N}\left|\sum_{\nu\in\mathcal I_2}  \tilde \sigma_\nu\frac{G_{\nu\mu}G_{\mu\nu}}{G_{\mu\mu}}\right| \le \frac{C}{N|G_{\mu\mu}|} \sum_{\nu\in\mathcal I_2} |G_{\nu\mu}|^2 = \frac{C\Im\, G_{\mu\mu}}{N\eta |G_{\mu\mu}|} \le \frac{C}{N\eta}, % \label{rough_boundmi}
%\end{align*}
%where in the first step we used (\ref{resolvent8}), and in the second and third steps we used (\ref{eq_gsq1}). Similarly, using (\ref{resolvent8}) and (\ref{eq_gsq3}) we get
%\begin{align*}
%\left|m_2 -m_2^{(i)}\right| & = \frac{1}{N}\left|\sum_{\nu \in\mathcal I_2}\tilde \sigma_\nu\frac{G_{\nu i}G_{i\nu}}{G_{ii}}\right| \le \frac{C}{N|G_{ii}|} \left( \frac{{G}_{ii}}{z}  + \frac{\bar z}{\eta} \Im\left(\frac{{G_{ii} }}{z}\right)\right)   \le \frac{C}{N\eta}.
%\end{align*}
%Similarly, we can prove the same bounds for the $m_1$ case. Then (\ref{m_T}) can be proved by induction on the indices in $\mathbb T$. %The proof for (\ref{m11_T}) is similar except that one needs to use the assumption (\ref{assm3}).
%\end{proof}
%}

\vspace{5pt}

\noindent{\bf Step 2: Self-consistent equations.} This is the key step of the proof for Proposition \ref{prop_entry}, which derives approximate self-consistent equations safisfised by $m_2(z)$ and $m_3(z)$. More precisely, we will show that $(m_2(z),m_3(z))$ satisfies \eqref{selfomegaerror} for some small error $|\cal E_{2,3}|\prec n^{-1/2}$. Then in Step 3 we will apply Lemma \ref{lem_stabw} to show that $(m_2(z),m_3(z))$ is close to $(m_{2c}(z),m_{3c}(z))$. %---this will discussed .

We define the following $z$-dependent event 
%\be\label{Xiz}\Xi(z):=\left\{\max_{\fa\in \cal I}|G_{\fa\fa}(z)-\Pi_{\fa\fa}(z)| \le (\log n)^{-1/2}\right\}.\ee
\be\label{Xiz}\Xi(z):=\left\{ |m_{2}(z)-m_{2c}(z)| + |m_{3}(z)-m_{3c}(z)| \le (\log n)^{-1/2}\right\}.\ee
Note that by \eqref{Lipomega}, we have $|m_{2c}+x_2|\lesssim (\log n)^{-1}$ and $|m_{3c}+x_3|\lesssim (\log n)^{-1}$. Together with \eqref{selfomega}, \eqref{a23} and \eqref{assm32}, we obtain the following basic estimates 
\be\label{Gsim1}
 |m_{2c}| \sim |m_{3c}| \sim 1, \quad |z+\lambda_i^2 r_1m_{2c} + r_2 m_{3c}|\sim 1  ,\quad \left|1 + \gamma_n m_c\right|\sim  |1+\gamma_n m_{1c}|\sim 1  ,
 \ee
 uniformly in $z\in \mathbf D$, where we abbreviated
 $$m_c(z):=-\frac1p\sum_{i\in \cal I_1}\frac{1}{z+\lambda_i^2 r_1m_{2c} +r_2m_{3c}},\quad m_{1c}(z):=-\frac1p\sum_{i\in \cal I_1}\frac{\lambda_i^2}{z+\lambda_i^2 r_1m_{2c} +r_2m_{3c}}.$$
Plugging \eqref{Gsim1} into \eqref{defn_piw}, we get
\be
|\Pi_{\fa\fa}(z)| \sim 1 \ \ \text{uniformly in } z\in \mathbf D, \ \fa\in \cal I.
\ee 

%In particular, on $\Xi$ we have
%\be\label{Gsim1}
%\mathbf 1(\Xi)|G_{\fa\fa}(z)| \sim 1 .
%\ee 

Then we claim the following result.
%that $m_{2c}(z)$ is the solution to the equation $z=f(m)$ for $f$ defined in (\ref{deformed_MP2}).

\begin{lemma}\label{lemm_selfcons_weak}
Suppose the assumptions in Proposition \ref{prop_diagonal} hold. Then the following estimates hold uniformly in $z \in \mathbf D$: %with $\xi$-high probability:
%\begin{equation}
%{\mathbf 1}(\eta \ge 1)\left| f(z, m_2) \right|\prec N^{-1/2}, \quad {\mathbf 1}(\eta\ge 1)\left|  m_1(z) - d_N \int\frac{x}{-z\left[1+xm_{2}(z) \right]} \pi_A^{(n)}(dx)\right| \prec N^{-1/2}, \label{selfcons_lemm2}
%\end{equation}
%and
%\begin{equation}
%{\mathbf 1}(\Xi)\left| f(z, m_2) \right| \prec \Psi_\theta, \quad {\mathbf 1}(\Xi)\left|  m_1(z) - d_N \int\frac{x}{-z\left[1+xm_{2}(z) \right]} \pi_A^{(n)}(dx)\right| \prec \Psi_\theta, \label{selfcons_lemm}
%\end{equation}
\begin{equation} \label{selfcons_lemm}
\begin{split}
&{\mathbf 1}(\Xi) \left|\frac{1}{m_{2}} + 1 -\frac1n\sum_{i=1}^p \frac{\lambda_i^2}{  z+\lambda_i^2r_1m_{2} + r_2m_{3}  } \right|\prec n^{-1/2},\\
&  {\mathbf 1}(\Xi)\left|\frac{1}{m_{3}} + 1 -\frac1n\sum_{i=1}^p \frac{1 }{  z+\lambda_i^2 r_1m_{2} +  r_2m_{3}  }\right|\prec n^{-1/2}.
\end{split}
\ee
%Moreover, we have the finer estimates
%\begin{equation}\label{selfcons_improved}
%\begin{split}
%&{\mathbf 1}(\Xi) \left|\frac{r_1}{m_{2}} + 1 -\frac1n\sum_{i=1}^p \frac{\lambda_i^2}{  z+\lambda_i^2m_{2} + m_{3}  } \right|\prec {\mathbf 1}(\Xi)\left( \left|[Z]_1\right|+ \left|[Z]_2\right| \right) + n^{-1},\\
%&  {\mathbf 1}(\Xi)\left|\frac{r_2}{m_{3}} + 1 -\frac1n\sum_{i=1}^p \frac{1 }{  z+\lambda_i^2 m_{2} +  m_{3}  }\right|\prec {\mathbf 1}(\Xi)\left(\left|[Z]\right|+ \left|[Z]_3\right|\right) + n^{-1} 
%  %\\ {\mathbf 1}(\Xi)\left|f(z, m_2)\right| \prec  {\mathbf 1}(\Xi)\left(\left|[Z]_1\right| + \left|[Z]_2\right|\right) + \Psi^2_\theta, 
%\end{split}
%\end{equation}
%%and
%%\be\label{selfcons_improved2}
%%{\mathbf 1}(\Xi)\left|  m_1(z) - d_N \int\frac{x}{-z\left[1+xm_{2}(z) \right]} \pi_A^{(n)}(dx)\right| \prec {\mathbf 1}(\Xi)\left|[Z]_1\right| + \Psi^2_\theta,
%%\ee
% where
%\begin{equation}\label{def_Zaver}
%[Z]:=\frac{1}{n}\sum_{i\in \mathcal I_1}\frac{Z_i}{\left(z + \lambda_i^2 r_1 m_2+r_2 m_3\right)^2}, \quad [Z]_1:=\frac{1}{n}\sum_{i\in \mathcal I_1}\frac{\lambda_i^2 Z_i}{\left(z + \lambda_i^2 r_1 m_2+r_2 m_3\right)^2}, \ \ [Z]_\al:=\frac{1}{n}\sum_{\mu \in \mathcal I_\al}  {Z_\mu} ,\ \ \al=2,3  .
%\end{equation}
\end{lemma}

\begin{proof}
 By (\ref{resolvent2}), (\ref{Zi}) and (\ref{Zmu}), we obtain that for $i \in \cal I_1$, $\mu \in \cal I_2$ and $\nu\in \cal I_3$,
\begin{align}
\frac{1}{{G_{ii} }}&=  - z - \frac{\lambda_i^2}{n} \sum_{\mu\in \mathcal I_2} G^{\left( i \right)}_{\mu\mu}- \frac{1}{n} \sum_{\mu\in \mathcal I_3} G^{\left( i \right)}_{\mu\mu} + Z_i =  - z - \lambda_i^2 r_1m_2 - r_2m_3 + \epsilon_i,  \label{self_Gii}\\
\frac{1}{{G_{\mu\mu} }}&=  - 1 - \frac{1}{n} \sum_{i\in \mathcal I_1}\lambda_i^2 G^{\left(\mu\right)}_{ii}+ Z_{\mu} =  - 1  - \gamma_n m_1 + \epsilon_\mu,  \label{self_Gmu1}\\
\frac{1}{{G_{\nu\nu} }}&=  - 1 - \frac{1}{n} \sum_{i\in \mathcal I_1} G^{\left(\nu\right)}_{ii}+ Z_{\nu} =   - 1 - \gamma_n m + \epsilon_\nu,  \label{self_Gmu2}
\end{align}
where we recall Definition \ref{defnMinor}, and %\eqref{m123}, and 
$$\epsilon_i := Z_i + \sigma_ir_1\left(m_2 - m_2^{(i)}\right) + r_2\left(m_3-m_3^{(i)}\right) ,\quad \epsilon_\mu := \begin{cases}Z_{\mu} + \gamma_n(m_1-m_1^{(\mu)}) , \ &\text{if } \mu \in \cal I_2\\ Z_{\mu} +\gamma_n(m-m^{(\mu)}) , \ &\text{if } \mu \in \cal I_3\end{cases}.$$
By (\ref{resolvent8}) we can bound that
\begin{equation}\nonumber
  |m_2 - m_2^{(i)}| \le   \frac1{n_1}\sum_{\mu\in \mathcal I_2}  \left|\frac{G_{\mu i} G_{i\mu}}{G_{ii}}\right|  \prec n^{-1},
\end{equation}
where we used (\ref{Zestimate1}) in the second step. Similarly, we can get that 
\be\label{higherr} |m - m^{(\mu)}| + |m_1-m_1^{(\mu)}| + |m_2 - m_2^{(i)}| +  |m_3 - m_3^{(i)}| \prec n^{-1}  \ee
for any $i\in \cal I_1$ and $\mu \in \cal I_{2}\cup \cal I_3$. Together with (\ref{Zestimate1}), we obtain that for all $i$ and $\mu$,
\begin{equation}\label{erri}
 |\epsilon_i | + |\epsilon_\mu|  \prec n^{-1/2}.
\end{equation}
%with $\xi $-high probability.
%Similarly, we can also get that
%\begin{equation}\label{self_Gmu}
%\frac{1}{{G_{\mu\mu} }}=  - z - \frac{1}{N} \sum_{i\in \mathcal I_1}\sigma_i G^{\left( \mu\right)}_{ii}+ Z_{\mu} =  - z - \frac{1}{N} \sum_{i\in \mathcal I_1}\sigma_i G_{ii} + \epsilon_\mu,
%\end{equation}
%where 
%$$\epsilon_\mu := Z_{\mu} + \frac{1}{N} \sum_{i\in \mathcal I_1}\sigma_i \left(G^{\left( \mu\right)}_{ii}-G_{ii}\right).$$
%Moreover, we have

With \eqref{Gsim1} and the definition of $\Xi$, we get that $\mathbf 1(\Xi)|z + \lambda_i^2 r_1m_2+r_2m_3|\sim1$. Hence using (\ref{self_Gii}), \eqref{erri} and \eqref{Zestimate1}, we obtain that
\be\label{Gmumu0}
\mathbf 1(\Xi)G_{ii}=\mathbf 1(\Xi)\left[-\frac{1}{z + \lambda_i^2 r_1 m_2+r_2 m_3} +\OO_\prec\left(n^{-1/2}\right)\right].
\ee
Plugging it into the definitions of $m$ and $m_1$ in \eqref{defm}, we get
\begin{align}
\mathbf 1(\Xi)m&=\mathbf 1(\Xi)\left[-\frac1p\sum_{i\in \cal I_1}\frac{1}{z + \lambda_i^2 r_1 m_2+r_2 m_3}  +\OO_\prec\left(n^{-1/2}\right)\right],\label{Gmumu} \\
\mathbf 1(\Xi)m_1&=\mathbf 1(\Xi)\left[-\frac1p\sum_{i\in \cal I_1}\frac{\lambda_i^2}{z + \lambda_i^2 r_1 m_2+r_2 m_3}   +\OO_\prec\left(n^{-1/2}\right)\right]. \label{Gmumu2}
\end{align}
As a byproduct, we obtain from these two estimates that  
\be\label{Gsim11}
\mathbf 1(\Xi)\left(|m-m_{c}| +|m_1-m_{1c}| \right)\lesssim (\log n)^{-1/2}, \quad \text{with high probability}. 
\ee
Together with \eqref{Gsim1}, we get that %from \eqref{self_Gmu1} and \eqref{self_Gmu2} that
\be\label{Gsim2}
|1+\gamma_nm_1|\sim 1, \quad |1+\gamma_nm|\sim 1, \quad \text{with high probability on } \Xi.
\ee
Now using \eqref{self_Gmu1}, \eqref{self_Gmu2}, \eqref{erri}, \eqref{Zestimate1} and \eqref{Gsim2}, we obtain that for $\mu \in \cal I_2$ and $\nu \in \cal I_3,$ %can obtain that with high probability,
\begin{align}
\mathbf 1(\Xi)\left(G_{\mu\mu}+\frac{1}{1 + \gamma_nm_1}\right) &=\OO_\prec(n^{-1/2}) , \quad \mathbf 1(\Xi)\left(G_{\nu\nu}+\frac{1}{1 + \gamma_nm}\right)&= \OO_\prec(n^{-1/2}) .\label{Gii0} 
%\\ \quad \nu \in \cal I_3.\label{Gii1}
\end{align}
Taking average over $\mu\in \cal I_2$ and $\nu\in \cal I_3$, we get that with high probability,
\be\label{Gii}
\begin{split}
& \mathbf 1(\Xi)\left(m_2+\frac{1}{1 + \gamma_n m_1}\right)  = \OO_\prec\left(n^{-1/2}\right) ,\quad  \mathbf 1(\Xi)\left(m_3+\frac{1}{1 +\gamma_n  m}\right)= \OO_\prec\left(n^{-1/2}\right) .
\end{split}
\ee
%which further implies
%\be\label{Gii}
% \mathbf 1(\Xi)\left(\frac{1}{m_2} + 1 + \gamma_nm_1\right) \prec  n^{-1/2} ,\quad \mathbf 1(\Xi)\left(\frac{1}{m_3} + 1 + \gamma_nm\right) \prec  n^{-1/2} .
%\ee
Finally, plugging \eqref{Gmumu} and \eqref{Gmumu2} into \eqref{Gii}, we conclude (\ref{selfcons_lemm}). 
\end{proof}

%The following lemma gives the stability of the equation $ f(z,m)=0$. Roughly speaking, it states that if $f(z, m_{2}(z))$ is small and $m_2(\tilde z)-m_{2c}(\tilde z)$ is small for $\Im\, \tilde z \ge \Im\, z$, then $m_{2}(z)-m_{2c}(z)$ is small. For an arbitrary $z\in S(c_0,C_0, \e)$, we define the discrete set
%\begin{align*}%\label{eqn_def_L}
%L(w):=\{z\}\cup \{z'\in S(c_0,C_0, \e): \text{Re}\, z' = \text{Re}\, z, \text{Im}\, z'\in [\text{Im}\, z, 1]\cap (N^{-10}\mathbb N)\} .
%\end{align*}
%Thus, if $\text{Im}\, z \ge 1$, then $L(z)=\{z\}$; if $\text{Im}\, z<1$, then $L(z)$ is a 1-dimensional lattice with spacing $N^{-10}$ plus the point $z$. Obviously, we have $|L(z)|\le N^{10}$. %The following lemma is stated as Definition 5.4 of \cite{KY2} %and Lemma 4.5 of \cite{BEKYY}.
%
%\begin{lemma}\label{stability}
%Let $c_0>0$ be a sufficiently small constant and fix $C_0,\epsilon>0$. The self-consistent equation $f(z,m)=0$ is stable on $S(c_0,C_0, \epsilon)$ in the following sense. Suppose the $z$-dependent function $\delta$ satisfies $N^{-2} \le \delta(z) \le (\log N)^{-1}$ for $z\in S(c_0,C_0, \epsilon)$ and that $\delta$ is Lipschitz continuous with Lipschitz constant $\le N^2$. Suppose moreover that for each fixed $E$, the function $\eta \mapsto \delta(E+\ii\eta)$ is non-increasing for $\eta>0$. Suppose that $u_2: S(c_0,C_0,\epsilon)\to \mathbb C$ is the Stieltjes transform of a probability measure. Let $z\in S(c_0,C_0,\epsilon)$ and suppose that for all $z'\in L(z)$ we have 
%\begin{equation}\label{Stability0}
%\left| f(z, u_2)\right| \le \delta(z).
%\end{equation}
%Then we have
%\begin{equation}
%\left|u_2(z)-m_{2c}(z)\right|\le \frac{C\delta}{\sqrt{\kappa+\eta+\delta}},\label{Stability1}
%\end{equation}
%for some constant $C>0$ independent of $z$ and $N$, where $\kappa$ is defined in (\ref{KAPPA}). 
%%Similarly, the self-consistent equation $\mathcal D_2$ in (\ref{def_D12}) is also stable on $S(C_1)$.
%\end{lemma}
%\begin{proof}
%This lemma can proved with the same method as in e.g. \cite[Lemma 4.5]{isotropic} and \cite[Appendix A.2]{Anisotropic}. The only input is Lemma \ref{lambdar_sqrt}. 
%\end{proof}

%\vspace{5pt}

\noindent{\bf Step 3: $\Xi$ holds with high probability.} In this step, we show that the event $\Xi(z)$ in fact holds with high probability for all $z\in \mathbf D$. Once we have proved this fact, then applying Lemma \ref{lem_stabw} to \eqref{selfcons_lemm} immediately shows that $(m_2(z),m_3(z))$ is equal to $(m_{2c}(z),m_{3c}(z))$ up to an error of order $n^{-1/2}$. 

We claim that it suffices to show
\be\label{Xiz0}
|m_{2}(0)-m_{2c}(0)| + |m_{3}(0)-m_{3c}(0)| \prec n^{-1/2}.
\ee
 Once we know \eqref{Xiz0}, then by \eqref{Lipomega} and \eqref{priordiff}, we get $\max_{\al=2}^3|m_{\al c}(z)-m_{\al c}(0)|=\OO((\log n)^{-1})$ and $\max_{\al=2}^3|m_{\al }(z)-m_{\al }(0)|=\OO((\log n)^{-1})$ with high probability for all $z\in \mathbf D$. Together with \eqref{Xiz0}, we obtain that 
\be\label{roughh1} 
\sup_{z\in \mathbf D} \left(|m_{2}(z)-m_{2c}(z)| + |m_{3}(z)-m_{3c}(z)|\right)  \lesssim (\log n)^{-1} \quad \text{with high probability},\ee
and %Moreover, with this estimate and \eqref{Lipomega}, we conclude that 
\be\label{roughh2} \sup_{z\in \mathbf D} \left( |m_{2}(z)-m_{2 c}(0)|+ |m_{3}(z)-m_{3c}(0)|\right) \lesssim (\log n)^{-1} \quad \text{with high probability}.\ee
The condition \eqref{roughh1} shows that $\Xi$ holds with high probability, and the condition \eqref{roughh2} verifies the condition \eqref{prior1} of Lemma \ref{lem_stabw}. Then applying Lemma \ref{lem_stabw} to \eqref{selfcons_lemm}, we obtain that
\be\label{Xizz}
|m_{2}(z)-m_{2 c}(z)|+ |m_{3}(z)-m_{3 c}(z)| \prec n^{-1/2} 
\ee
for all $z\in \mathbf D$. Plugging \eqref{Xizz} into \eqref{self_Gii}-\eqref{self_Gmu2}, we get the diagonal estimate
\be\label{diagest}
\max_{\fa\in \cal I}|G_{\fa\fa}(z)-\Pi_{\fa\fa}(z)| \prec n^{-1/2}. 
\ee
Together with the off-diagonal estimate in \eqref{Zestimate1}, we conclude \eqref{entry_diagonal}.\end{proof}
Now we give the proof of \eqref{Xiz0}.
%It remains to show that \eqref{Xiz0} holds.% In the following lemma, we pick $0=0$.
%\begin{lemma}
%Under the assumptions of Proposition \ref{prop_diagonal}, the estimate \eqref{Xiz0} holds.
%\end{lemma}
\begin{proof}[Proof of \eqref{Xiz0}]
By \eqref{spectral}, we get
$$ m(0)=\frac1p\sum_{i\in \cal I_1}G_{ii}(0) = \frac1p\sum_{k = 1}^{p} \frac{|\xi_k(i)|^2 }{\lambda_k} \ge \lambda_1^{-1} \gtrsim 1. $$
Similarly, we can also get that $m_1(0)$ is positive and has size $m_1(0)\sim 1$. Hence we have 
$$1+\gamma_n m_1(0)\sim 1,\quad 1+\gamma_n m_1(0)\sim 1.$$ 
Together with \eqref{self_Gmu1}, \eqref{self_Gmu2} and \eqref{erri}, we obtain that \eqref{Gii} holds at $z=0$ even without the indicator function $\mathbf 1(\Xi)$. Furthermore, it gives that 
$$  \left|\lambda_i^2 r_1m_2(0)+r_2m_3(0)\right|=\left|\frac{\lambda_i^2 r_1}{ 1+\gamma_n m_1(0)} +\frac{r_2}{1+\gamma_n m(0)}+ \OO_\prec (n^{-1/2})\right| \sim 1$$
with high probability. Then using (\ref{self_Gii}) and \eqref{erri}, we obtain that \eqref{Gmumu} and \eqref{Gmumu2} hold at $z=0$ even without the indicator function $\mathbf 1(\Xi)$. Finally, plugging \eqref{Gmumu} and \eqref{Gmumu2} into \eqref{Gii}, we conclude \eqref{selfcons_lemm} holds at $z=0$, that is,
\begin{equation} \label{selfcons_lemma222}
\begin{split}
& \left|\frac{1}{m_{2}(0)} + 1 -\frac1n\sum_{i=1}^p \frac{\lambda_i^2}{ \lambda_i^2r_1m_{2}(0) + r_2m_{3}(0)  } \right|\prec n^{-1/2},\\ 
&\left|\frac{1}{m_{3}(0)} + 1 -\frac1n\sum_{i=1}^p \frac{1 }{ \lambda_i^2 r_2m_{2}(0) + r_2 m_{3}(0)  }\right|\prec n^{-1/2}.
\end{split}
\ee

Denoting $\omega_{2}=-m_{2c}(0)$ and $\omega_{3}=-m_{2c}(0)$. By \eqref{Gii}, we have
$$\omega_2= \frac{1}{1+\gamma_n m_{1}(0)} +\OO_\prec(n^{-1/2}),\quad \omega_3= \frac{1}{1+\gamma_n m(0)}+\OO_\prec(n^{-1/2}).$$ 
Hence there exists a sufficiently small constant $c>0$ such that 
\be\label{omega12} c \le  \omega_2 \le 1, \quad  c\le \omega_3\le 1, \quad \text{with high probability}.\ee
%This shows that we can choose a sufficiently small constant $c_1$ such that 
%\be \nonumber %\label{omega12}
%c \le  \omega_2 \le 1 - \gamma_n - c,\quad c \le  \omega_3 \le 1 - \gamma_n - c , \quad \text{with high probability},
%\ee
%where we also used $1-\gamma_n - r_{1,2}\gtrsim 1$ by \eqref{assm2}. 
Also one can verify from \eqref{selfcons_lemma222} that $(\omega_2,\omega_3)$ satisfy approximately the same equations as \eqref{selfomega0}:
\be\label{selfcons_lemm000}
\begin{split}
r_1\omega_2+r_2\omega_3 = 1-\gamma_n + \OO_\prec (n^{-1/2}),\quad f(\omega_2)=1 + \OO_\prec (n^{-1/2}).
\end{split}
\ee
The first equation and \eqref{omega12} together implies that $\omega_2 \in [0,r_1^{-1}(1-\gamma_n)]$ with high probability. Since $f$ is strictly increasing and has bounded derivatives on $[0,r_1^{-1}(1-\gamma_n)]$, by basic calculus the second equation in \eqref{selfcons_lemm000} gives that $|\omega_2-x_2|\prec n^{-1/2}$. Together with the first equation in \eqref{selfcons_lemm000}, we get $|\omega_3-x_3|\prec n^{-1/2}$. This concludes \eqref{Xiz0}.
\end{proof}
%\begin{lemma}[Weak entrywise local law]\label{alem_weak} 
%Let $c_0>0$ be a sufficiently small constant and fix $C_0,\epsilon>0$. Then we have %there exists $C>0$ such that with $\xi$-high probability,
%\begin{equation} \label{localweakm}
%\Lambda(z) \prec (N\eta)^{-1/4},
%\end{equation}
%uniformly in $z \in S(c_0,C_0,\epsilon)$.
%\end{lemma}
%\begin{proof}
%One can prove this lemma using a continuity argument as in e.g. \cite[Section 4.1]{isotropic}, \cite[Section 5.3]{Semicircle} or \cite[Section 3.6]{EKYY1}. The key inputs are Lemmas \ref{Z_lemma}-\ref{stability}, and the estimates (\ref{average_L})-(\ref{diag_L}) in the $\eta \ge 1$ case. All the other parts of the proof are essentially the same. 
%\end{proof}

%This lemma concludes \eqref{Xiz0}, and as explained above, concludes the proof of Lemma \ref{prop_entry}. 
%It remains to show that \eqref{aver_in} holds.
%
%\vspace{10pt}
%
%\noindent{\bf Step 4: Fluctuation averaging.} By \eqref{Gmumu} and using $\Pi_{ii}=-(z + \lambda_i^2 m_{2c}+m_{3c})^{-1}$, to show \eqref{aver_in} it suffices to prove that
%\be \label{avergoal} 
%|m_{2}-m_{2c}|+|m_{3}-m_{3c}|+|[Z]| \prec n^{-1}.
%\ee
%For this purpose, we need stronger bounds on $[Z]_1$ and $[Z]_2$ in (\ref{selfcons_improved}). They follow from the following {\it{fluctuation averaging lemma}}. 
%
%\begin{lemma}[Fluctuation averaging] \label{abstractdecoupling}
%%Suppose $\Phi$ and $\Phi_o$ are positive, $N$-dependent deterministic functions on $S(c_0,C_0,\epsilon)$ satisfying $N^{-1/2} \le \Phi, \Phi_o \le N^{-c}$ for some constant $c>0$. Suppose moreover that $\Lambda \prec \Phi$ and $\Lambda_o \prec \Phi_o$. Then for all $z \in S(c_0,C_0,\epsilon)$ we have
%Under the assumptions in Proposition \ref{prop_diagonal}, we have for all $z\in \mathbf D$.
%\begin{equation}\label{flucaver_ZZ}
%|[Z]|+|[Z_1]|+|[Z_2]|+|[Z_3]| \prec n^{-1}.
%\end{equation}
%%Fix a constant $\xi>0$. Suppose $q\le \varphi^{-5\xi}$ and that there exists $\tilde S\subseteq S(c_0,C_0,L)$ with $L\ge 18\xi$ such that with $\xi$-high probability,
%%\begin{equation} 
%%\Lambda(z) \le \gamma(z) \text{ for } z\in \tilde S,
%%\end{equation}
%%where $\gamma$ is a deterministic function satisfying $\gamma(z)\le \varphi^{-\xi}$. Then we have that with $(\xi-\tau_N)$-high probability,
%%\begin{equation}
%%\left|[Z]_1(z)\right|+ \left|[Z]_2(z)\right| \le \varphi^{18\xi} \left(q^2 + \frac{1}{(N\eta)^2} + \frac{\Im \, m_{2c}(z) + \gamma(z)}{N\eta} \right),
%%\end{equation}
%%for $z\in \tilde S$, where $\tau_N:=2/\log \log N$. 
%\end{lemma}
%\begin{proof}
%The proof is the same as the one for Lemma 5.6 of \cite{Anisotropic}.
%\end{proof}
%Now we give the proof of Proposition \ref{prop_entry}.
%
%%Fix $c_0,C_0>0$, $\xi> 3$ and set 
%%$$L:=120\xi, \ \ \tilde \xi:= 2/\log 2 + \xi.$$
%%Hence we have $\tilde \xi \le 2\xi$ and $L\ge 60\tilde \xi$. Then to prove (\ref{DIAGONAL}), it suffices to prove 
%%\begin{equation}\label{goal_law1}
%%\bigcap_{z \in S(c_0,C_0,L)} \left\{ \Lambda(z) \leq C\varphi^{20\tilde \xi}\left(q+ \sqrt{\frac{\operatorname{Im} m_{2c}(z) }{N \eta}}+ \frac{1}{N\eta}\right) \right\},
%%\end{equation}
%%with $\xi$-high probability. %For notational convenience, we shall denote $m_c:=m_{1c}+m_{2c}$. 
%
%By Lemma \ref{alem_weak}, the event $\Xi$ holds with high probability. Then by Lemma \ref{alem_weak} and Lemma \ref{Z_lemma}, we can take
%\be\label{initial_phio}
%\Phi_o = \sqrt{\frac{\im m_{2c} + (N\eta)^{-1/4}}{N\eta}} + \frac{1}{N\eta},\quad \Phi= \frac{1}{(N\eta)^{1/4}},
%\ee
% in Lemma \ref{abstractdecoupling}. 
%%have that
%%$\Lambda\prec |w|^{-3/8}(N\eta)^{-1/4}$. Therefore  $\theta\prec |w|^{-3/8}(N\eta)^{-1/4}$ and
%%$$\Lambda_o\prec\Psi_\theta\prec\sqrt{\frac{\Im(m_{1c}+m_{2c})+|w|^{-3/8}(N\eta)^{-1/4}}{N\eta}}, $$
%%where we use $|w|^{-1/2}(N\eta)^{-3/8}\ge (N\eta)^{-1}$ by the definition (\ref{eq_domainD}) of $\bD$.
%%Lemma \ref{fluc_aver} then gives 
%%\[\Phi_o = |w|^{1/2}\sqrt{\frac{\Im(m_{1c}+m_{2c})+|w|^{-3/8}(N\eta)^{-1/4}}{N\eta}},\ \ \ \Phi=\left(\frac{|w|^{1/2}}{N\eta}\right)^{1/4}\]
%%$\|[Z]\| + \|\langle Z \rangle\|\prec |w|^{-1/2}\Phi_o^2.$ 
%Then (\ref{selfcons_improved}) gives
%$$|f(z,m_2)| \prec\frac{ \im m_{2c} + (N\eta)^{-1/4}}{N\eta}.$$
%Using Lemma \ref{stability}, we get
%\be\label{m2}
%|m_2-m_{2c}|\prec\frac{\im m_{2c}}{N\eta\sqrt{\kappa+\eta}}+\frac{1}{(N\eta)^{5/8}} \prec \frac{1}{(N\eta)^{5/8}} ,
%\ee
%where we used $\im m_{2c}=\OO(\sqrt{\kappa+\eta})$ by \eqref{Immc} in the second step. With (\ref{selfcons_improved2}) and \eqref{m2}, we get the same bound for $m_1$, which gives
%\be\label{m1}
%\theta \prec {(N\eta)^{-5/8}} ,
%\ee
%Then using Lemma \ref{Z_lemma} and (\ref{m1}), we obtain that
%\begin{align}\label{1iteration}
%\Lambda_o \prec  \sqrt{\frac{\im m_{2c} + (N\eta)^{-5/8}}{N\eta}} + \frac{1}{N\eta}
%\end{align}
%uniformly in $z\in S(c_0,C_0,\epsilon)$, which is a better bound than the one in (\ref{initial_phio}). Taking the RHS of \eqref{1iteration} as the new $\Phi_o$, we can obtain an even better bound for $\Lambda_o$. Iterating the above arguments, we get the bound
%$$\theta \prec \left({N\eta}\right)^{-\sum_{k=1}^l 2^{-k} - 2^{-l-2} }$$
%after $l$ iterations. This implies %the averaged local law
%\be\label{aver_proof}
%\theta\prec(N\eta)^{-1}
%\ee
%since $l$ can be arbitrarily large. Now with \eqref{aver_proof}, Lemma \ref{Z_lemma}, \eqref{Gii0} and \eqref{Gmumu0}, we can obtain \eqref{entry_diagonal}. 
%%that
%%$$\Lambda(z)\prec \Psi(z),$$
%%which proves Proposition \ref{prop_entry}.
%
%\end{proof}
%
%
%\subsection{Proof of Proposition \ref{prop_diagonal}}
%
%Plugging \eqref{flucaver_ZZ} into \eqref{selfcons_improved}, we get 
%\begin{equation} \nonumber
%\begin{split}
%&  \left|\frac{r_1}{m_{2}} + 1 -\frac1n\sum_{i=1}^p \frac{\lambda_i^2}{  z+\lambda_i^2m_{2} + m_{3}  } \right|\prec   n^{-1},\quad \left|\frac{r_2}{m_{3}} + 1 -\frac1n\sum_{i=1}^p \frac{1 }{  z+\lambda_i^2 m_{2} +  m_{3}  }\right|\prec  n^{-1} .
%  %\\ {\mathbf 1}(\Xi)\left|f(z, m_2)\right| \prec  {\mathbf 1}(\Xi)\left(\left|[Z]_1\right| + \left|[Z]_2\right|\right) + \Psi^2_\theta, 
%\end{split}
%\end{equation}
%Then using Lemma \ref{lem_stabw}, we get $|m_{2}-m_{2c}|+|m_{3}-m_{3c}|\prec n^{-1}$, which concludes \eqref{avergoal}. This concludes the proof of \eqref{aver_in}, and hence Lemma \ref{prop_entry}.


With Lemma \ref{prop_entry}, we can complete the proof of Proposition \ref{prop_diagonal}.

\begin{proof}[Proof of Proposition \ref{prop_diagonal}]
With (\ref{entry_diagonal}), one can use the polynomialization method in \cite[Section 5]{isotropic} to get the anisotropic local law (\ref{aniso_law}) for $G_0$ with $q=n^{-1/2}$. The proof is exactly the same, except for some minor differences in notations, so we omit the details.
\end{proof}



\subsubsection{Anisotropic Local Law}\label{sec_comparison}

%Following the above discussions, we divide the proof of Theorem \ref{LEM_SMALL} into two steps. In Section \ref{sec_Gauss}, we give the proof for separable covariance matrices of the form $\Sig^{1/2} X \tilde \Sig X^\top \Sig^{1/2}$, which implies the local laws in the Gaussian $X$ case. In Section \ref{sec_comparison}, we apply the self-consistent comparison argument in \cite{Anisotropic} to extend the result to the general $X$ case. Compared with \cite{Anisotropic}, there are two differences in our setting: (1) the support of $X$ in Theorem \ref{LEM_SMALL} is $q=\OO(N^{-\phi})$ for some constant $0<\phi \le 1/2$, while \cite{Anisotropic} dealt with $X$ with smaller support $q=\OO(N^{-1/2})$; (2) one has $B=I$ in \cite{Anisotropic}, which simplifies the proof a little bit.

In this subsection, we finish the proof of Theorem \ref{LEM_SMALL} for a general $X$ satisfying the bounded support condition (\ref{eq_support}) with $q\le n^{-\phi}$ for some constant $\phi>0$. 
%As remarked at the beginning of Section \ref{sec_Gauss},
Proposition \ref{prop_diagonal} implies that \eqref{aniso_law} holds for Gaussian $Z_1^{Gauss}$ and $Z_2^{Gauss}$ as discussed before. Thus the basic idea is to prove that for $Z_1$ and $Z_2$ satisfying the assumptions in Theorem \ref{LEM_SMALL}, %we have
\begin{equation*}%\label{Gaussian_starting}
 \mathbf u^\top  \left( G(Z,z) -  G(Z^{Gauss}, z)\right) \mathbf v  \prec q 
\end{equation*}
for any deterministic unit vectors $\mathbf u,\mathbf v\in{\mathbb C}^{\mathcal I}$ and $z\in \mathbf D$. Here we abbreviated $Z:=\begin{pmatrix}Z_1\\ Z_2\end{pmatrix}$ and $Z^{Gauss}:=\begin{pmatrix}Z^{Gauss}_1\\ Z^{Gauss}_2\end{pmatrix}$.
%Now similar to Lemma \ref{lemma_Im}, we can prove the following estimates for $\mathcal G$.
%
%\begin{lemma}\label{lem_comp_gbound}
%For $i\in \mathcal I_1$ and $\mu\in \mathcal I_2$, we define $\mathbf u_i=U^\top \mathbf e_i  \in \mathbb C^{\mathcal I_1}$ and $\mathbf v_\mu=V^\top \mathbf e_\mu  \in \mathbb C^{\mathcal I_2}$, i.e. $\mathbf u_i$ is the $i$-th row vector of $U$ and $\mathbf v_\mu$ is the $\mu$-th row vector of $V$. Let $\mathbf x \in \mathbb C^{\mathcal I_1}$ and $\mathbf y \in \mathbb C^{\mathcal I_2}$. Then we have %for some constant $C>0$,
%  \begin{align}
% & \sum_{i \in \mathcal I_1 }  \left| {G_{\mathbf x \mathbf u_i} } \right|^2  =\sum_{i \in \mathcal I_1 }  \left| {G_{ \mathbf u_i \mathbf x} } \right|^2  = \frac{|z|^2}{\eta}\im\left(\frac{ G_{\mathbf x\mathbf x}}{z}\right) , \label{eq_sgsq2} \\
%& \sum_{\mu  \in \mathcal I_2 } {\left| {G_{\mathbf y \mathbf v_\mu } } \right|^2 }=\sum_{\mu  \in \mathcal I_2 } {\left| {G_{\mathbf v_\mu \mathbf y } } \right|^2 }  = \frac{{\im G_{\mathbf y\mathbf y} }}{\eta }, \label{eq_sgsq1}\\ 
%& \sum_{i \in \mathcal I_1 } {\left| {G_{\mathbf y \mathbf u_i} } \right|^2 } =\sum_{i \in \mathcal I_1 } {\left| {G_{ \mathbf u_i \mathbf y} } \right|^2 } = {G}_{\mathbf y\mathbf y}  +\frac{\bar z}{\eta} \im G_{\mathbf y\mathbf y}  , \label{eq_sgsq3} \\
%& \sum_{\mu \in \mathcal I_2 } {\left| {G_{\mathbf x \mathbf v_\mu} } \right|^2 }= \sum_{\mu \in \mathcal I_2 } {\left| {G_{\mathbf v_\mu \mathbf x } } \right|^2 }= \frac{G_{\mathbf x\mathbf x}}{z}  + \frac{\bar z}{\eta} \im \left(\frac{G_{\mathbf x\mathbf x}}{z}\right) .\label{eq_sgsq4}
% \end{align}
% All of the above estimates remain true for $G^{(\mathbb T)}$ instead of $G$ for any $\mathbb T \subseteq \mathcal I$. 
%\end{lemma}
%%\begin{proof}
%%The proof is almost the same as the proof of Lemma \ref{lemma_Im}, except that we use
%%$$ \sum_{i\in \mathcal I_1}\mathbf v_i \mathbf v_i^\dag = V_1 V_1^\dag = I_{N\times N}.$$
%%\end{proof}
%\begin{proof}
%We only prove \eqref{eq_sgsq1} and \eqref{eq_sgsq3}. The proof for \eqref{eq_sgsq2} and \eqref{eq_sgsq4} is very similar. With  \eqref{spectral1}, we get that
%\begin{align}\label{middle}
%\sum_{\mu  \in \mathcal I_2 } {\left| {G_{\mathbf y \mathbf v_\mu } } \right|^2 } =& \sum_{\mu  \in \mathcal I_2 } \left\langle \mathbf y,G {\mathbf v_\mu  } \right\rangle \left\langle {\mathbf v_\mu}, G^\dag \mathbf y \right\rangle  = \sum_{k = 1}^N {\frac{{\left| {\left\langle {\mathbf y,\zeta _k } \right\rangle } \right|^2  }}{{\left( {\lambda _k  - E} \right)^2  + \eta ^2 }} }   =\frac{{\im  G_{\mathbf y\mathbf y} }}{\eta }.
%\end{align}
%For simplicity, we denote $Y:=\Sig^{1/2} U^{*}X V\tilde \Sig^{1/2}$. Then with \eqref{green2} and \eqref{spectral2}, we get that
%\begin{align*}
% \sum_{i \in \mathcal I_1 } {\left| {G_{\mathbf y\mathbf u_i} } \right|^2 } =  \left( {{\mathcal G_2} Y^\dag Y \mathcal G_2^\dag  } \right)_{\mathbf y\mathbf y}=  \left( {{\mathcal G_2} \left(Y^\dag Y-\bar z\right) \mathcal G_2^\dag  } \right)_{\mathbf y\mathbf y} + \bar z \left( {{\mathcal G_2} \mathcal G_2^\dag  } \right)_{\mathbf y\mathbf y} =  {G}_{\mathbf y\mathbf y}  +\frac{\bar z}{\eta} \im G_{\mathbf y\mathbf y}  ,
% \end{align*}
% where we used $\mathcal G_2^\dag= \left(Y^\dag Y-\bar z\right)^{-1}$ and \eqref{middle} in the last step.
%\end{proof}
%
%
%%\subsection{Bootstrapping on the spectral scale}
%%\begin{subsection}{Self-consistent comparison}\label{subsection_selfcomp}
%Our proof basically follows the arguments in \cite[Section 7]{Anisotropic} with some modifications. Thus we will not give all the details. We first focus on proving the anisotropic local law \eqref{aniso_law}, and the proof of \eqref{aver_in1}-\eqref{aver_out1} will be given at the end of this section. By polarization, to prove \eqref{aniso_law} it suffices to prove that %the following bound:
% \begin{equation}\label{goal_ani2}
%\left\langle \mathbf v, \left(G(X,z)- \Pi(z)\right) \mathbf v \right\rangle \prec q+\Psi(z)
%\end{equation}
%uniformly in $z\in \tilde S(c_0,C_0,\e)$ and any deterministic unit vector $ \mathbf v\in{\mathbb C}^{\mathcal I}$. In fact, we can obtain the more general bound \eqref{aniso_law}
%%\begin{equation*}%\label{goal_ani}
%%\left\langle \mathbf u, \left(G(X,z) - \Pi(z)\right) \mathbf v \right\rangle \prec \Psi(z)
%%\end{equation*}
%by applying (\ref{goal_ani2}) to the vectors $\mathbf u + \mathbf v$ and $\mathbf u + i\mathbf v$, respectively.
%
%%\begin{proposition}\label{comparison_prop}
%%Suppose the assumptions of Theorem \ref{LEM_SMALL} hold.  Fix ${\left| z \right|^2 } \le 1 - \tau$ and suppose that the assumptions of Theorem \ref{law_wideT} hold. If (\ref{assm_3rdmoment}) holds or $\eta \ge N^{-1/2+\zeta}|m_{2c}|^{-1}$, then for any regular domain $\mathbf S \subseteq \mathbf D$,
%% \begin{equation}\label{goal_ani2}
%%\left\langle \mathbf v, \left( G(w)-\Pi(w)\right) \mathbf v \right\rangle \prec \Psi(z)
%%\end{equation}
%%uniformly in $w\in \bS$ and any deterministic unit vectors $ \mathbf v\in{\mathbb C}^{\mathcal I}$.
%%\end{proposition}
%
%%We first assume that (\ref{assm_3rdmoment}) holds. Then we will show how to modify the arguments to prove the $\eta \ge N^{-1/2+\zeta}|m_{2c}|^{-1}$ case.
%The proof consists of a bootstrap argument from larger scales to smaller scales in multiplicative increments of $N^{-\delta}$, where
%\begin{equation}
% \delta \in\left(0,\frac{\min\{\epsilon,\phi\}}{2C_a}\right). \label{assm_comp_delta}
%\end{equation}
%Here $\e>0$ is the constant in $\tilde S(c_0,C_0,\e)$, $\phi>0$ is a constant such that $q\le N^{-\phi}$, $C_a> 0$ is an absolute constant that will be chosen large enough in the proof. For any $\eta\ge N^{-1+\e}$, we define
%\begin{equation}\label{eq_comp_eta}
%\eta_l:=\eta N^{\delta l} \text{ for } \ l=0,...,L-1,\ \ \ \eta_L:=1.
%\end{equation}
%where
%%\begin{equation}\label{eq_comp_L}
%$L\equiv L(\eta):=\max\left\{l\in\mathbb N|\ \eta N^{\delta(l-1)}<1\right\}.$
%%\end{equation}
%%through
%% \begin{equation}\label{eq_comp_eta}
%%  \eta_l:=\eta N^{\delta l}\ \ l=0,...,L-1,\ \ \ \eta_L:=1.
%% \end{equation}
%Note that $L\le \delta^{-1}$.
%
%By (\ref{eq_gbound}), the function $z\mapsto G(z)- \Pi(z)$ is Lipschitz continuous in $\tilde S(c_0,C_0,\e)$ with Lipschitz constant bounded by $N^2$. Thus to prove (\ref{goal_ani2}) for all $z\in \tilde S(c_0,C_0,\e)$, it suffices to show that (\ref{goal_ani2}) holds for all $z$ in some discrete but sufficiently dense subset ${\mathbf S} \subset \tilde S(c_0,C_0,\e)$. We will use the following discretized domain $\bS$.
%\begin{definition}
%Let $\mathbf S$ be an $N^{-10}$-net of $\tilde S(c_0,C_0,\e)$ such that $ |\mathbf S |\le N^{20}$ and
%\[E+\ii\eta\in\mathbf S\Rightarrow E+\ii\eta_l\in\mathbf S\text{ for }l=1,...,L(\eta).\]
%\end{definition}
%
%The bootstrapping is formulated in terms of two scale-dependent properties ($\bA_m$) and ($\bC_m$) defined on the subsets
%\[\mathbf S_m:=\left\{z\in\mathbf S\mid\text{Im} \, z\ge N^{-\delta m}\right\}.\]
%${(\bA_m)}$ For all $z\in\mathbf S_m$, all deterministic unit vectors $\mathbf x \in \mathbb C^{\mathcal I_1}$ and $\mathbf y \in \mathbb C^{\mathcal I_2}$, and all $X$ satisfying the assumptions in Theorem \ref{LEM_SMALL}, we have
%\begin{equation}\label{eq_comp_Am}
% \im \left(\frac{G_{\mathbf x\mathbf x}(z)}{z}\right) + \im G_{\mathbf y\mathbf y}(z)\prec \im m_{2c}(z) +N^{C_a\delta}(q+\Psi(z)).
%\end{equation}
%${(\bC_m)}$ For all $z\in\mathbf S_m$, all deterministic unit vector $\mathbf v\in \mathbb C^{\mathcal I}$, and all $X$ satisfying the assumptions in Theorem \ref{LEM_SMALL}, %(\ref{assm1})-(\ref{assm2}), 
%we have
%\begin{equation}\label{eq_comp_Cm}
% \left|G_{\mathbf v\mathbf v}(z)-\Pi_{\mathbf v\mathbf v}(z)\right|\prec N^{C_a\delta}(q+\Psi(z)).
%\end{equation}
%%The bootstrapping is started by the following result
%%\begin{lemma}\label{lemm_boot0}
%It is trivial to see that ${(\mathbf A_0)}$ holds by \eqref{eq_gbound} and \eqref{Immc}. Moreover, it is easy to observe the following result.
%%\end{lemma}
%%\begin{proof}
%% By Lemma \ref{lemma_Im} and the assumption (\ref{assm3}), we have for $w\in\widehat\bS_0$,
%% \[\text{Im} G_{\mathbf{vv}}(w)\le C |w|^{1/2}\left\|G(w)\right\| \le \frac{C}{\eta}\le C |w|^{1/2}\Im \left[m_{1c}(w)+m_{2c}(w)\right],\]
%%where we use (\ref{estimate1_bulk}) for $w\in{\widehat\bS}_0$.
%%\end{proof}
%
%\begin{lemma}\label{lemm_boot2}
%For any $m$, property ${(\mathbf C_m)}$ implies property $(\mathbf A_m)$.
%\end{lemma}
%\begin{proof}
%By \eqref{Immc}, \eqref{Piii} and the definition of $\Pi$ in \eqref{defn_pi}, it is easy to get that 
%$$\im \left(\frac{\Pi_{\mathbf x\mathbf x}(z)}{z}\right) + \im \Pi_{\mathbf y\mathbf y}(z)\lesssim \im m_{2c}(z) ,$$
%%$\im \Pi_{\bv\bv}=\OO(\im m_{2c})$, 
%which finishes the proof.
%%Suppose property $(\mathbf C_m)$ holds. By (\ref{def_PiPhi}), we have
%%\begin{align*}
%%\widetilde \Pi_{\mathbf v\mathbf v} = |w|^{1/2} \left\langle \mathbf v, \overline T^\dag \Pi \overline T \mathbf v \right\rangle = |w|^{1/2} \left(\Pi_d \right)_{ {\mathbf u} {\mathbf u}} ,
%%\end{align*}
%%where $ \mathbf u = \bar T  \mathbf v.$ Now (\ref{estimate_PiImw}) implies
%%\begin{equation}\label{eqn_ImPi}
%%\Im\, \widetilde \Pi_{\mathbf v\mathbf v} \le C \Im(m_{1c}+m_{2c}),
%%\end{equation}
%%and further
%%$$\text{Im} G_{\mathbf{vv}}(w)\le\text{Im}\, \Pi_{\mathbf {vv}}+\left| G_{\mathbf{vv}}(w)-\Pi_{\mathbf{vv}}(w)\right|\prec|w|^{1/2}\text{Im}\left[m_{1c}(w)+m_{2c}(w)\right]+N^{C_a\delta}\Psi(z).$$
%%Thus the property $(\mathbf A_m)$ follows.
%\end{proof}
%
%The key step is the following induction result.
%\begin{lemma}\label{lemm_boot}
%For any $1\le m\le \delta^{-1}$, property $(\mathbf A_{m-1})$ implies property $(\mathbf C_m)$.
%\end{lemma}
%
%Combining Lemmas \ref{lemm_boot2} and \ref{lemm_boot}, we conclude that (\ref{eq_comp_Cm}) holds for all $w\in\mathbf S$. Since $\delta$ can be chosen arbitrarily small under the condition (\ref{assm_comp_delta}), we conclude that (\ref{goal_ani2}) holds for all $w\in\mathbf S$, and \eqref{aniso_law} follows for all $z\in \tilde S(c_0,C_0,\e)$. What remains now is the proof of Lemma \ref{lemm_boot}. Denote
%\begin{equation}\label{eq_comp_F(X)}
% F_{\mathbf v}(X,z):=\left|G_{\mathbf{vv}}(X,z)-\Pi_{\mathbf {vv}}(z)\right|.
%\end{equation}
%By Markov's inequality, it suffices to prove the following lemma.
%\begin{lemma}\label{lemm_comp_0}
% Fix $p\in \mathbb N$ and $m\le \delta^{-1}$. Suppose that the assumptions of Theorem \ref{LEM_SMALL} and property $(\mathbf A_{m-1})$ hold. Then we have
% \begin{equation}
%  \mathbb EF_{\mathbf v}^p(X,z)\le\left[ N^{C_a\delta}\left(q+\Psi(z)\right)\right]^p
% \end{equation}
% for all $z\in{\mathbf S}_m$ and any deterministic unit vector $\mathbf v$.
%\end{lemma}
%In the rest of this section, we focus on proving Lemma \ref{lemm_comp_0}. 
%%\begin{subsubsection}{Rough bound}
%First, in order to make use of the assumption $(\mathbf A_{m-1})$, which has spectral parameters in $\mathbf S_{m-1}$, to get some estimates for $G$ with spectral parameters in $\mathbf S_{m}$, we shall use the following rough bounds for $ G_{\mathbf{xy}}$.
%
%\begin{lemma}\label{lemm_comp_1}
%For any $z=E+\ii\eta\in\mathbf S$ and unit vectors $\mathbf x,\mathbf y\in \mathbb C^{\mathcal I}$,  we have %{\cor need to revise}
%\begin{align*}
%\left|G_{\mathbf x\mathbf y}(z)-\Pi_{\mathbf x\mathbf y}(z)\right|\prec & N^{2\delta}\sum_{l=1}^{L(\eta)} \left[\im \left(\frac{G_{\mathbf x_1\mathbf x_1}(E+\ii\eta_l)}{E+\ii\eta_l}\right)+\im G_{\mathbf x_2\mathbf x_2}(E+\ii\eta_l) \right.\\
%& \left. +\im \left(\frac{G_{\mathbf y_1\mathbf y_1}(E+\ii\eta_l)}{E+\ii\eta_l}\right)+\im G_{\mathbf y_2\mathbf y_2}(E+\ii\eta_l)\right]+1,
%\end{align*}
%where $\mathbf x=\left( {\begin{array}{*{20}c}
%   {\mathbf x}_1   \\
%   {\mathbf x}_2 \\
%   \end{array}} \right)$ and $\mathbf y=\left( {\begin{array}{*{20}c}
%   {\mathbf y}_1   \\
%   {\mathbf y}_2 \\
%   \end{array}} \right)$ for ${\mathbf x}_1,{\mathbf y}_1\in\mathbb C^{\mathcal I_1}$ and ${\mathbf x}_2,{\mathbf y}_2\in\mathbb C^{\mathcal I_2}$, and $\eta_l$ is defined in (\ref{eq_comp_eta}).
%%recall that $L(\eta)$ and $\eta_l$ are defined in $(\ref{eq_comp_L})$ and $(\ref{eq_comp_eta})$.
%\end{lemma}
%\begin{proof} The proof is the same as the one for \cite[Lemma 7.12]{Anisotropic}.\end{proof}
%%\begin{proof}
%%By (\ref{estimate_Piw12}) and the definition of $\widetilde \Pi$ in (\ref{def_PiPhi}), we get that $\left|\Pi_{\mathbf x\mathbf y}\right| \le |\mathbf x||\mathbf y|.$ Thus it suffices to estimate $|\mathcal G_{\mathbf{xy}}|$. By the definition of $\mathcal G$ in (\ref{def_mathcalg}), we see that $\mathcal G_{\mathbf{xy}}=R_{\mathbf{\bar x \bar y}}$ for $R:=|w|^{1/2}G$ and $\bar {\mathbf u}:=\overline T\mathbf u$ for $\mathbf u\in\{\mathbf x, \mathbf y\}$.
%%Using the singular value decomposition (\ref{singular_rep}) we get that
%%\begin{equation}\label{eqn_roughbound1}
%%\left|R_{\bar{\mathbf x}_1\bar{\mathbf y}_1}\right|=\left|\inprod{\bar{\mathbf x}_1,|w|^{1/2} \sum \limits_{ k=1 }^{ N } \frac { \xi_k \xi_k ^{ \dag  } }{ \lambda _{ k }-w } \bar{\mathbf y}_1}\right|\le |w|^{1/2} \sum \limits_{ k=1 }^{ N }\frac{|\inprod{\bar{\mathbf x}_1,\xi_k}|^2}{2\left|\lambda_k-w\right|}+ |w|^{1/2} \sum \limits_{ k=1 }^{ N }\frac{|\inprod{\bar{\mathbf y}_1,\xi_k}|^2}{2\left|\lambda_k-w\right|},
%%\end{equation}
%%and
%%\begin{equation}\label{eqn_roughbound2}
%%\left|R_{\bar{\mathbf x}_1\bar{\mathbf y}_2}\right| = \left|\inprod{\bar{\mathbf x}_1,w^{-1/2}|w|^{1/2} \sum_{k=1}^{ N } \frac { \sqrt{\lambda_k}\xi_k \zeta_{\bar k}^\dag }{\lambda_k-w } \bar{\mathbf y}_2}\right|\le \sum \limits_{ k=1 }^{ N }\frac{\sqrt{\lambda_k}|\inprod{\bar{\mathbf x}_1,\xi_k}|^2}{2\left|\lambda_k-w\right|}+\sum \limits_{ k=1 }^{ N }\frac{\sqrt{\lambda_k}|\inprod{\bar{\mathbf y}_2,\zeta_{\bar k}}|^2}{2\left|\lambda_k-w\right|}.
%%\end{equation}
%%
%%%where the second step is by
%%%\begin{align*}
%%% \frac{\sqrt{\lambda_k} }{|\lambda^k-w|}=&\sqrt{\frac{\lambda_k}{\lambda_k^2+E^2+\eta^2-2E\lambda_k}}\\
%%% \le &\sqrt{\frac{\lambda_k}{2\lambda_k\sqrt{E^2+\eta^2}-2E\lambda_k}}\\
%%% =&\sqrt{\frac{\sqrt{E^2+\eta^2}+E}{2\eta^2}}\le \frac{|w|^{1/2}}{\eta}.
%%%\end{align*}
%%%{\color{red} \[R(w)=\begin{pmatrix} |w|^{1/2} \sum \limits_{ k=1 }^{ N }\frac { \xi_k \xi_k ^{ \dag  } }{ \lambda _{ k }-w }  & w^{-1/2}|w|^{1/2}\sum \limits_{ k=1 }^{ N } \frac { \sqrt \lambda_k\xi_k \zeta_k ^{ \dag  } }{ \lambda _{ k }-w }  \\ w^{-1/2}|w|^{1/2}\sum \limits_{ k=1 }^{ N } \frac { \sqrt\lambda_k\zeta_k \xi_k ^{ \dag  } }{ \lambda _{ k }-w }  & |w|^{1/2}\sum \limits_{ k=1 }^{ N } \frac { \zeta_k \zeta_k ^{ \dag  } }{ \lambda _{ k }-w }  \end{pmatrix}.\]}
%%Recall the notation $\eta_l$ in (\ref{eq_comp_eta}), define the subsets of indices
%%\[U_l=\{k\mid\eta_{l-1}\le|\lambda_k-E|<\eta_l\},\ \ l=0,...,L+1,\]
%%where we set $\eta_{-1}=0$ and $\eta_{L+1}=\infty$. Now we split the summation in (\ref{eqn_roughbound2}) according to $U_l$. For $l=1,...,L$ we have
%%\begin{align*}
%% \sum \limits_{ k\in U_l }\frac{\sqrt{\lambda_k}|\inprod{\bar{\mathbf x}_1,\xi_k}|^2}{2\left|\lambda_k-w\right|}\le& \sum_{k\in U_l}\frac{\sqrt{E+\eta_l}|\inprod{\bar{\mathbf x}_1,\xi_k}|^2\eta_l}{2(\lambda_k-E)^2} \le \sum_{k\in U_l}\frac{({2E^2+2\eta_l^2})^{1/4}|\inprod{\bar{\mathbf x}_1,\xi_k}|^2\eta_l}{(\lambda_k-E)^2+\eta_{l-1}^2}\\
%% \le& 2^{1/4}N^{2\delta}\sum_{k\in U_l}\frac{|{E+\ii\eta_l}|^{1/2}|\inprod{\bar{\mathbf x}_1,\xi_k}|^2\eta_l}{(\lambda_k-E)^2+\eta_{l}^2} = 2^{1/4}N^{2\delta}\text{Im}\, R_{\bar{\mathbf x}_1\bar{\mathbf x}_1}\left(E+\ii\eta_l\right);
%%\end{align*}
%%for $l=0$,
%%\begin{align*}
%% \sum \limits_{ k\in U_0 }\frac{\sqrt{\lambda_k}|\inprod{\bar{\mathbf x}_1,\xi_k}|^2}{2\left|\lambda_k-w\right|}\le& \sum_{k\in U_0}\frac{\sqrt{E+\eta}|\inprod{\bar{\mathbf x}_1,\xi_k}|^2 \sqrt{2}\eta}{2\left[(\lambda_k-E)^2+\eta^2\right]} \le 2^{-1/4}N^{2\delta}\sum_{k\in U_0}\frac{|{E+\ii\eta_1}|^{1/2}|\inprod{\bar{\mathbf x}_1,\xi_k}|^2\eta_1}{(\lambda_k-E)^2+\eta_1^2}\\
%% =&2^{-1/4}N^{\delta}\text{Im}\, R_{\bar{\mathbf x}_1\bar{\mathbf x}_1}\left(E+\ii\eta_1\right);
%%\end{align*}
%%and for $l={L+1}$,
%%\begin{align*}
%% \sum \limits_{ k\in U_{L+1} }\frac{\sqrt{\lambda_k}\inprod{\bar{\mathbf x}_1,\xi_k}^2}{2\left|\lambda_k-w\right|} & \le \sum_{k\in U_{L+1}}\frac{\sqrt{\lambda_k}|\inprod{\bar{\mathbf x}_1,\xi_k}|^2 \sqrt{2}|\lambda_k-E|}{2\left[(\lambda_k-E)^2+\eta_L^2\right]} \prec \sum_{k\in U_{L+1}}\frac{|\inprod{\bar{\mathbf x}_1,\xi_k}|^2\eta_L}{(\lambda_k-E)^2+\eta_{L}^2} \\
%% & \le C \text{Im}\, R_{\bar{\mathbf x}_1\bar{\mathbf x}_1}\left(E+\ii\eta_L\right),
%%\end{align*}
%%where in the second step we used that $\lambda_k\prec 1$, which follows from (\ref{norm_upperbound}). Combining the above estimates we get that
%%\[\sum \limits_{ k=1 }^{ N }\frac{\sqrt{\lambda_k}|\inprod{\bar{\mathbf x}_1,\xi_k}|^2}{2\left|\lambda_k-w\right|}\prec N^{2\delta}\sum_{l=1}^{L}\text{Im}\, R_{\bar{\mathbf x}_1\bar{\mathbf x}_1}(E+\ii\eta_l).\]
%%Similarly, we can prove that
%%\[\sum \limits_{ k=1 }^{ N }\frac{\sqrt{\lambda_k}|\inprod{\bar{\mathbf y}_2,\zeta_k}|^2}{\left|\lambda_k-w\right|}\prec N^{2\delta}\sum_{l=1}^{L}\text{Im}\, R_{\bar{\mathbf y}_2\bar{\mathbf y}_2}(E+\ii\eta_l).\]
%%Since $|E+\ii\eta|\le|E+\ii\eta_l|$, we immediately get
%%\[|w|^{1/2}\sum \limits_{ k=1 }^{ N }\frac{\inprod{\bar{\mathbf u}_1,\xi_k}^2}{\left|\lambda_k-w\right|}\prec N^{2\delta}\sum_{l=1}^{L}\text{Im}\, R_{\bar{\mathbf u}_1\bar{\mathbf u}_1}(E+\ii\eta_l)\]
%%for $\mathbf u \in \{\mathbf x, \mathbf y\}$.
%%%and
%%%\[|w|^{1/2}\sum \limits_{ k=1 }^{ N }\frac{\inprod{\bar{\mathbf y}_1,\xi_k}^2}{\left|\lambda_k-w\right|}\prec N^{2\delta}\sum_{l=1}^{L}\text{Im}R_{\bar{\mathbf y}_1\bar{\mathbf y}_1}(E+\ii\eta_l)\]
%%Plugging into (\ref{eqn_roughbound1}) and (\ref{eqn_roughbound2}), we get that
%%\[|R_{\bar{\mathbf x}_1\bar{\mathbf y}_1}|+|R_{\bar{\mathbf x}_1\bar{\mathbf y}_2}|\prec N^{2\delta}\sum_{l=1}^{L(\eta)}\left[\text{Im}\, R_{\bar{\mathbf x}_1\bar{\mathbf x}_1}(E+\ii\eta_l)+\text{Im}\, R_{\bar{\mathbf y}_1\bar{\mathbf y}_1}(E+\ii\eta_l)+\text{Im}\, R_{\bar{\mathbf y}_2\bar{\mathbf y}_2}(E+\ii\eta_l)\right].\]
%%We can bound $|R_{\bar{\mathbf x}_2 \bar{\mathbf y}_1}|$ and $|R_{\bar{\mathbf x}_2\bar{\mathbf y}_2}|$ in a similar way. This concludes the proof.
%%\end{proof}
%
%Recall that for a given family of random matrices $A$, we use $A=O_\prec(\zeta)$ to mean $\left|\left\langle\mathbf v, A\mathbf w\right\rangle\right|\prec\zeta \| \mathbf v\|_2 \|\mathbf w\|_2 $ uniformly in any deterministic vectors $\mathbf v$ and $\mathbf w$ (see Definition \ref{stoch_domination} (ii)).
%
%\begin{lemma}\label{lemm_comp_2}
%Suppose $(\mathbf A_{m-1})$ holds, then
% \begin{equation}\label{eq_comp_apbound}
%  G(z)-\Pi(z)=\OO_{\prec}(N^{2\delta}),
% \end{equation}
% and
%%for all $w\in \mathbf S_m$. Moreover for any unit vector $\mathbf v$ we have
%\begin{equation}\label{eq_comp_apbound2}
%\im \left(\frac{G_{\mathbf x\mathbf x}(z)}{z}\right) + \im G_{\mathbf y\mathbf y}(z) \prec N^{2\delta}\left[ \im m_{2c}(z)+N^{C_a\delta}(q+\Psi(z))\right],
%\end{equation}
% for all $z \in \mathbf S_m$ and any deterministic unit vectors $\mathbf x \in \mathbb C^{\mathcal I_1}$ and $\mathbf y \in \mathbb C^{\mathcal I_2}$.
%\end{lemma}
%\begin{proof} The proof is the same as the one for \cite[Lemma 7.13]{Anisotropic}.\end{proof}
%%\begin{proof}
%%Let $z=E+\ii\eta \in \mathbf S_m$. Then $E+\ii\eta_l \in \mathbf S_{m-1}$ for $l=1,\ldots, L(\eta)$, and (\ref{eq_comp_Am}) gives $\im G_{\mathbf v\mathbf v}(z)\prec 1.$ The estimate (\ref{eq_comp_apbound}) now follows immediately from Lemma \ref{lemm_comp_1}. To prove (\ref{eq_comp_apbound2}), we remark that if $s(w)$ is the Stieltjes transform of any positive integrable function on $\mathbb R$, the map $\eta \mapsto \eta\Im\, s(E+\ii\eta)$ is nondecreasing and the map $\eta \mapsto \eta^{-1} \im s(E+\ii\eta)$ is nonincreasing. We apply them to $\im G_{\mathbf v\mathbf v}(E+\ii\eta)$ and $\im m_{2c}(E+\ii\eta)$ to get for $z_1=E+\ii\eta_1\in \mathbf S_{m-1}$,
%%\begin{align*}
%%\im G_{\mathbf v\mathbf v}(w) & \le N^{\delta}\frac{|w|^{1/2}}{|w_1|^{1/2}}\im G_{\mathbf v\mathbf v}(w_1)\prec N^{\delta}\left[|w|^{1/2}\im\left(m_{1c}(w_1)+m_{2c}(w_1)\right)+N^{C_a\delta}\frac{|w|^{1/2}}{|w_1|^{1/2}}\Phi(w_1)\right] \\
%%& \le N^{2\delta}\left[|w|^{1/2}\im\left(m_{1c}(w)+m_{2c}(w)\right)+N^{C_a\delta}\Psi(z)\right],
%%\end{align*}
%%where we used $\Psi(z):=|w|^{1/2}\Psi(w)$ and the fact that $\eta \mapsto \Psi(E+\ii\eta)$ is nonincreasing, which is clear from the definition (\ref{eq_defpsi}).
%%\end{proof}
%%\end{subsubsection}
We prove the above statement using a continuous comparison argument introduced in \cite{Anisotropic}. %to prove Lemma \ref{lemm_comp_0}. 
The proof is similar to the ones in Sections 7-8 of \cite{Anisotropic}, so we only give a rough description of the basic idea, without writing down all the details. 
%We divide the proof into three subsections. In Sections \ref{subsec_interp}-\ref{section_words}, we prove Lemma \ref{lemm_comp_0} under the condition 


 


%\begin{subsection}{Interpolation and expansion} \label{subsec_interp}
%The self-consistent comparison is performed with the following interpolation.
\begin{definition}[Interpolation]
%Introduce the notation $X^0:=X^{Gauss}$ and $X^1:=X$. We define the interpolation matrix $X^\theta$ by setting
%\begin{equation}
% X^\theta_{i\mu}:=\chi^\theta_{i\mu} X^1+(1-\chi^\theta_{i\mu})X^0, \ \ \theta\in [0,1],
%\end{equation}
%for $i\in \mathcal I_1$ and $\mu\in \mathcal I_2$ (recall the Definition \ref{def_indexsets}). Here $(\chi^\theta_{i\mu})$ is a family of i.i.d Bernoulli random variables, independent of $X^0$ and $X^1$, satisfying $\bbP(\chi_{i\mu}^\theta=1)=\theta$ and $\bbP(\chi_{i\mu}^\theta=0)=1-\theta$.
We denote $Z^0:=Z^{Gauss}$ and $Z^1:=Z$. Let $\rho_{\mu i}^0$ and $\rho_{\mu i}^1$ be the laws of $Z_{\mu i}^0$ and $Z_{\mu i}^1$, respectively. For $\theta\in [0,1]$, we define the interpolated law
$\rho_{\mu i}^\theta := (1-\theta)\rho_{\mu i}^0+\theta\rho_{\mu i}^1.$ We shall work on the probability space consisting of triples $(Z^0,Z^\theta, Z^1)$ of independent $n\times p$ random matrices, where the matrix $Z^\theta=(Z_{\mu i}^\theta)$ has law
\begin{equation}\label{law_interpol}
\prod_{i\in \mathcal I_1}\prod_{\mu\in \mathcal I_2\cup \cal I_3} \rho_{\mu i}^\theta(\dd Z_{\mu i}^\theta).
\end{equation}
For $\lambda \in \mathbb R$, $i\in \mathcal I_1$ and $\mu\in \mathcal I_2\cup \cal I_3$, we define the matrix $Z_{(\mu i)}^{\theta,\lambda}$ through
\[\left(Z_{(\mu i)}^{\theta,\lambda}\right)_{\nu j}:=\begin{cases}Z_{\mu i}^{\theta}, &\text{ if }(j,\nu)\ne (i,\mu)\\ \lambda, &\text{ if }(j,\nu)=(i,\mu)\end{cases}.\]
We also introduce the matrices $G^{\theta}(z):=G\left(Z^{\theta},z\right),\ \ \ G^{\theta, \lambda}_{(\mu i)}(z):=G\left(Z_{(\mu i)}^{\theta,\lambda},z\right).$
%according to (\ref{def_mathcalg}) and the Definition \ref{def_linearHG}.
\end{definition}

We shall prove %Lemma \ref{lemm_comp_0} 
\eqref{aniso_law} through interpolation matrices $Z^\theta$ between $Z^0$ and $Z^1$. We have see that \eqref{aniso_law} holds for $Z^0$ by Proposition \ref{prop_diagonal}. %as remarked at the beginning of Section \ref{sec_Gauss}.
%\begin{lemma}\label{Gaussian_case}
%Lemma \ref{lemm_comp_0} holds if $X=X^0$.
%\end{lemma}
%\begin{proof}
%As remarked above (\ref{Gaussian_starting}), the anisotropic law (\ref{goal_ani}) holds for $X^0$, i.e. $F_{\mathbf v}^p(X^0,w)\prec \Phi^p$. Now to apply (iii) of Lemma \ref{lem_stodomin}, we need an upper bound $\mathbb E\left(F_{\mathbf v}^p(X^0,w)\right)^2 \le N^{C_p}$ for some constant $C_p$. This follows easily from (\ref{eq_gbound}) and
%(\ref{estimate_Piw12}).
%\end{proof}
Using (\ref{law_interpol}) and fundamental calculus, we get the following basic interpolation formula:
%\begin{lemma}\label{lemm_comp_3}
for $F:\mathbb R^{n \times p}\rightarrow \mathbb C$,
\begin{equation}\label{basic_interp}
\frac{\dd}{\dd\theta}\mathbb E F(Z^\theta)=\sum_{i\in\mathcal I_1}\sum_{\mu\in\mathcal I_2\cup \cal I_3}\left[\mathbb E F\left(Z^{\theta,Z_{\mu i}^1}_{(\mu i)}\right)-\mathbb E F\left(Z^{\theta,Z_{\mu i}^0}_{(\mu i)}\right)\right]
\end{equation}
 provided all the expectations exist.
%\end{lemma}
We shall apply \eqref{basic_interp} to $F(Z):=F_{\bu\mathbf v}^s(Z,z)$ for (large) $s\in 2\N$ and $F_{\mathbf u\mathbf v}(Z,z)$ defined as
\begin{equation*}%\label{eq_comp_F(X)}
 F_{\bu\mathbf v}(Z,z):=\left|\mathbf u^\top \left(G (Z,z)-\Pi(z)\right)\mathbf v\right|.
\end{equation*}
%Here for simplicity of notations, we introduce the following notation of generalized entries: for $\mathbf u,\mathbf v \in \mathbb R^{\mathcal I}$, we shall denote $
%G_{\mathbf{uv}}:= \mathbf u^\top G\mathbf v   . %\quad G_{a\mathbf{w}}:=\langle \mathbf e_a,G\mathbf w\rangle,
%$
%Moreover, we shall abbreviate $G_{\mathbf{u}\fa}:= G_{\bu\mathbf e_{\fa}}$ for $\fa\in \mathcal I$, where $\mathbf e_{\fa}$ is the standard unit vector along $\fa$-th axis. Given any vector $\mathbf u\in \mathbb \R^{\mathcal I_{1,2,3}}$, we always identify it with its natural embedding in $\mathbb R^{\mathcal I}$. 
%%$\left( {\begin{array}{*{20}c}
%%   {\mathbf x}  \\
%%   0 \\
%%\end{array}} \right)$ and $\left( {\begin{array}{*{20}c}
%%   0  \\
%%   \mathbf y \\
%%\end{array}} \right)$ in $\mathbb C^{\mathcal I}$.
%The exact meanings will be clear from the context. 
The main part of the proof is to show the following self-consistent estimate for the right-hand side of (\ref{basic_interp}) for any fixed $s\in 2\N$ and constant $\e>0$:
%\begin{lemma}\label{lemm_comp_4}
 %Fix $p\in 2\mathbb N$ and $m\le \delta^{-1}$. Suppose (\ref{3moment}) and $\mathbf{(A_{m-1})}$ hold, then we have
 \begin{equation}\label{lemm_comp_4}
  \sum_{i\in\mathcal I_1}\sum_{\mu\in\mathcal I_2\cup \cal I_3}\left[\mathbb EF_{\bu\mathbf v}^s\left(Z^{\theta,Z_{\mu i}^1}_{(\mu i)},z\right)-\mathbb EF_{\bu\mathbf v}^s\left(Z^{\theta,Z_{\mu i}^0}_{(\mu i)},z\right)\right]=
  \OO\left((n^\e q)^{s}+\E F_{\bu\mathbf v}^s\left(Z^{\theta},z\right) \right)
 \end{equation}
 for all $\theta\in[0,1]$. If \eqref{lemm_comp_4} holds, then combining \eqref{basic_interp} with a Gr\"onwall's argument we obtain that for any fixed $s\in 2\N$ and constant $\e>0$: 
 $$\E\left|G_{\mathbf{uv}}(Z^1,z)-\Pi_{\mathbf {uv}}(z)\right|^p  \le (n^\e q)^{p}.$$
Together with Markov's inequality, we conclude \eqref{aniso_law}. 
%we can conclude Lemma \ref{lemm_comp_0} and hence \eqref{goal_ani2}. %Theorem \ref{LEM_SMALL}.%Proposition \ref{comparison_prop}.
%In order to prove \eqref{lemm_comp_4}, we compare $Z^{\theta,Z_{\mu i}^0}_{(\mu i)}$ and $Z^{\theta,Z_{\mu i}^1}_{(\mu i)}$ via a common $Z^{\theta,0}_{(\mu i)}$, i.e. we will prove that
%\begin{equation}\label{lemm_comp_5}
%\sum_{i\in\mathcal I_1}\sum_{\mu\in\mathcal I_2\cup \cal I_3}\left[\mathbb EF_{\bu\mathbf v}^p\left(Z^{\theta,Z_{\mu i}^a}_{(\mu i)},z\right)-\mathbb EF_{\mathbf v}^p\left(Z^{\theta,0}_{(\mu i)},z\right)\right]= \OO\left((n^\e q)^{p}+\E F_{\bu\mathbf v}^p\left(Z^{\theta},z\right) \right)
% \end{equation}
%for all $a\in \{0,1\}$ and $\theta\in[0,1]$. 
Underlying the proof of (\ref{lemm_comp_4}) is an expansion approach, which is very similar to the ones for Lemma 7.10 of \cite{Anisotropic} and Lemma 6.11 of \cite{yang2019spiked}. So we omit the details.

%This follows from an improved self-consistent comparison argument for sample covariance matrices in \cite[Section 8]{Anisotropic}. The argument for our case is almost the same except for some notational differences, so we omit the details. 

%-------------remove-------------
%%Throughout the rest of the proof, 
%%During the proof, we always assume that $(\mathbf A_{m-1})$ holds. Also the rest of the proof is performed at a fixed $z\in \mathbf S_m$. 
%We define the $\mathcal I \times \mathcal I$ matrix $\Delta_{(\mu i)}^\lambda$ as
%\begin{equation}\label{deltaimu}
%\Delta_{(\mu i)}^{\lambda} :=\lambda \left( {\begin{array}{*{20}c}
%   { 0 } &   \mathbf u_i^{(\mu)} \mathbf e_\mu^\top     \\
%   {\mathbf e_\mu (\mathbf u_i^{(\mu)})^\top } & {0}  \\
%   \end{array}} \right),%   \lambda \mathbf u_i \delta_{is}\delta_{\mu t}+\lambda\delta_{it}\delta_{\mu s}, \ \  i\in \mathcal I_1, \mu\in \mathcal I_2,
%\end{equation}
%where we denote $\bu_i^{(\mu)}:=\Lambda U\mathbf e_i$ if $\mu \in \cal I_2$ and $\bu_i^{(\mu)}:=V\mathbf e_i$ if $\mu \in \cal I_3$. Then by the definition of $H$ in \eqref{linearize_block}), we have for any $\lambda,\lambda'\in \mathbb R$ and $K\in \mathbb N$,
%\begin{equation}\label{eq_comp_expansion}
%G_{(i \mu)}^{\theta,\lambda'} = G_{(\mu i)}^{\theta,\lambda}+\sum_{k=1}^{K}  G_{(\mu i)}^{\theta,\lambda}\left( \Delta_{(\mu i)}^{\lambda-\lambda'} G_{(\mu i)}^{\theta,\lambda}\right)^k+ G_{(\mu i)}^{\theta,\lambda'}\left(\Delta_{(\mu i)}^{\lambda-\lambda'} G_{(\mu i)}^{\theta,\lambda}\right)^{K+1}.
%\end{equation}
%%In the remainder of this section, we prove (\ref{lemm_comp_5}) for $u=1$.
%%where
%%$\overline V:=\begin{pmatrix}V_1 & 0\\ 0 & I\end{pmatrix}$ and $\alpha:=\frac{w^{1/2}}{|w|^{1/2}}.$
%%The following result provides a priori bounds for the entries of $G_{(\mu i)}^{\theta,\lambda}$.
%Using this expansion and the a priori bound \eqref{priorim}, it is easy to prove the following estimate: if $y$ is a random variable satisfying $|y|\prec q$, then
% \begin{equation}\label{comp_eq_apriori}
%   G_{(\mu i)}^{\theta,y}=\OO (1),\quad i\in\sI_1, \ \mu\in\sI_2 \cup \cal I_3,
% \end{equation}
% with high probability.
%% for all $i\in\sI^M_1$ and $\mu\in\sI_2$.
%%\end{lemma}
%%\begin{proof} The proof is the same as the one for \cite[Lemma 7.14]{Anisotropic}. \end{proof}
%%\begin{proof}
%% It suffices to show that $G_{(\mu i)}^{\theta,y}=O_{\prec}(N^{2\delta})$ since $\|\Pi\|=\OO(1)$ by \eqref{Piii}. By assumption ($\mathbf A_{m-1}$), the Lemma \ref{lemm_comp_2} holds for the matrix ensemble $X^\theta$ (since it satisfies (\ref{assm1})-(\ref{assm2})). In particular $G^{\theta,X_{i\mu}^u}_{(\mu i)}=O_\prec(N^{2\delta})$. Now we apply the expansion (\ref{eq_comp_expansion}) with $\lambda:=X_{i\mu}^\theta$, $\lambda':=y$ and large enough $K$ such that $K(2\delta-\phi)\le -2$. Using $|\lambda-\lambda'|\prec q$, it is easy to estimate all the terms in (\ref{eq_comp_expansion}) using Lemma \ref{lemm_comp_2} except the rest term.
%%To handle the rest term, we use the rough bound $G_{(\mu i)}^{\theta,\lambda'}\prec N$ coming from a simple modification of (\ref{eq_gbound}).
%%\end{proof}
%
%In the following proof, for simplicity of notations, we introduce $f_{(\mu i)}(\lambda):=F_{\mathbf v}^p(Z_{(\mu i)}^{\theta, \lambda})$. We use $f_{(\mu i)}^{(r)}$ to denote the $r$-th derivative of $f_{(\mu i)}$. By \eqref{comp_eq_apriori}, it is easy to see that for any fixed $r\in\bbN$, $ f_{(\mu i)}^{(r)}(y) =\OO(1)$ with high probability for any random variable $y$ satisfying $|y|\prec q$. 
%% With Lemma \ref{lemm_comp_6} and (\ref{eq_comp_expansion}), it is easy to prove the following result.
%%\begin{lemma}
%%Suppose that $y$ is a random variable satisfying $|y|\prec q$. Then for fixed $r\in\bbN$,
%%  \begin{equation}
%%  \left|f_{(\mu i)}^{(r)}(y)\right|\prec N^{2\delta(r+p)}.
%% \end{equation}
%%\end{lemma}
%%Then in Section \ref{subsec_3moment}, we show how to relax \eqref{3moment} to \eqref{assm_3moment} for $z\in \tilde S(c_0,C_0,\e)$.
%Then the Taylor expansion of $f_{(\mu i)}$ gives
%\begin{equation}\label{eq_comp_taylor}
%f_{(\mu i)}(y)=\sum_{r=0}^{p+4}\frac{y^r}{r!}f^{(r)}_{(\mu i)}(0)+\OO_\prec\left( q^{p+4}\right),
%\end{equation}
%%provided $C_a$ is chosen large enough in (\ref{assm_comp_delta}). 
%Therefore we have for $a\in\{0,1\}$,
%\begin{align}
%&\mathbb EF_{\mathbf v}^p\left(Z^{\theta,Z_{\mu i}^a}_{(\mu i)}\right)-\mathbb EF_{\mathbf v}^p\left(Z^{\theta,0}_{(\mu i)}\right)=\bbE\left[f_{(\mu i)}\left(Z_{i\mu}^a\right)-f_{(\mu i)}(0)\right]\nonumber\\
%& =\bbE f_{(\mu i)}(0)+\frac{1}{2n}\bbE f_{(\mu i)}^{(2)}(0)+\sum_{r=4}^{p+4}\frac{1}{r!}\bbE f^{(r)}_{(\mu i)}(0)\bbE\left(Z_{i\mu}^a\right)^r+\OO_\prec(q^{p+4}). \label{taylor1}
%\end{align}
%Here to illustrate the idea in a more concise way, we assume the extra condition 
%\be\label{3moment}
%%\mathbb E x_{ij}^3=0,  
%\mathbb E (Z^1_{\mu i})^3=0, \quad 1\le \mu \le n,\ \  1\le i \le p.
%\ee
%Hence the $r=3$ term in the Taylor expansion vanishes. However, this is not necessary as we will explain at the end of the proof.
%
%%where we used that $Z_{\mu i}^u$ has vanishing first and third moments and its variance is $1/N$. (Note that this is the only place where we need the condition \eqref{3moment}.) 
%
%
%By \eqref{conditionA2} and the bounded support condition, we have
%\be\label{moment-4}
%\left|\bbE\left(Z_{i\mu}^a\right)^r\right| \prec n^{-2}q^{r-4} , \quad r \ge 4.
%\ee
%Thus to show (\ref{lemm_comp_5}), we only need to prove for $r=4,5,...,p+4$,
%\begin{equation}\label{eq_comp_est}
%n^{-2}q^{r-4}\sum_{i\in\mathcal I_1}\sum_{\mu\in\mathcal I_2\cup \cal I_3}\left|\bbE f^{(r)}_{(\mu i)}(0)\right|=\OO\left(\left(n^\e q\right)^p+\mathbb EF_{\bu\mathbf v}^p(Z^\theta,z)\right).\end{equation}
%%where we used (\ref{assm2}). 
%In order to get a self-consistent estimate in terms of the matrix $Z^\theta$ on the right-hand side of (\ref{eq_comp_est}), we want to replace $Z^{\theta,0}_{(\mu i)}$ in $f_{(\mu i)}(0)=F_{\bu\mathbf v}^p(Z_{(\mu i)}^{\theta, 0})$ with $Z^\theta = Z_{(\mu i)}^{\theta, Z_{\mu i}^\theta}$. %We have the following lemma.
%\begin{lemma}
%Suppose that
%\begin{equation}\label{eq_comp_selfest}
%n^{-2}q^{r-4}\sum_{i\in\mathcal I_1}\sum_{\mu\in\mathcal I_2 \cup \cal I_3}\left|\bbE f^{(r)}_{(\mu i)}(Z_{i\mu}^\theta)\right|=\OO\left(\left(n^\e q\right)^p+\mathbb EF_{\mathbf v}^p(X^\theta,z)\right)
%\end{equation}
%holds for $r=4,...,4p+4$. Then (\ref{eq_comp_est}) holds for $r=4,...,4p+4$.
%\end{lemma}
%\begin{proof}
%The proof is the same as the one for \cite[Lemma 7.16]{Anisotropic}.
%%We abbreviate $ f_{(\mu i)}\equiv f$ and $X_{i\mu}^\theta \equiv \xi$. Then with (\ref{eq_comp_taylor}) we can get
%%\begin{equation}\label{eq_comp_taylor2}
%%\E f^{(l)}(0)=\E f^{(l)}(\xi)-\sum_{k=1}^{4p+4-l}\E f^{(l+k)}(0)\frac{\E \xi^k}{k!}+\OO_\prec(q^{p+4-l}).
%%\end{equation}
%%The estimate \eqref{eq_comp_est} then follows from a repeated application of (\ref{eq_comp_taylor2}).  Fix $r=4,...,4p+4$. Using (\ref{eq_comp_taylor2}), we get
%%\begin{align*}
%%\mathbb E f^{(r)}(0)&=\mathbb E f^{(r)}(\xi) - \sum_{k_1\ge 1} \mathbf 1(r+k_1 \le 4p +4)\mathbb E f^{(r+k_1)}(0) \frac{\mathbb E\xi^{k_1}}{k_1!}+\OO_\prec(q^{p+4-r}) \\
%%&=\mathbb E f^{(r)}(\xi) - \sum_{k_1\ge 1} \mathbf 1(r+k_1 \le 4p +4)\mathbb E f^{(r+k_1)}(\xi) \frac{\mathbb E\xi^{k_1}}{k_1!} \\
%%&+\sum_{k_1,k_2\ge 1} \mathbf 1(r+k_1+k_2 \le 4p +4)\mathbb E f^{(r+k_1+k_2)}(0) \frac{\mathbb E\xi^{k_1}}{k_1!} \frac{\mathbb E\xi^{k_2}}{k_2!} + \OO_\prec(q^{p+4-r}) \\
%%&=\cdots=\sum_{t=0}^{4p+4-r}(-1)^t \sum_{k_1,\cdots, k_t \ge 1}\mathbf 1\left(r+\sum_{j=1}^t k_j \le 4p +4\right)\mathbb E f^{(r+\sum_{j=1}^t k_j)}(\xi)\prod_{j=1}^t \frac{\mathbb E\xi^{k_j}}{k_j!} + \OO_\prec(q^{p+4-r}).
%%\end{align*}
%%The lemma now follows easily by using \eqref{moment-4}.
%\end{proof}
%%\end{subsection}
%
%%\begin{subsection}{Conclusion of the proof with words}\label{section_words}
%
%
%
%What remains now is to prove (\ref{eq_comp_selfest}). For simplicity of notations, we shall abbreviate $Z^\theta \equiv Z$. %for the remainder of the proof. 
%For any $k\in \N$, we denote
%\[A_{\mu i}(k):= \left(\frac{\partial}{\partial Z_{\mu i}}\right)^k \left( G_{\mathbf u\mathbf v}-\Pi_{\mathbf u\mathbf v}\right).\]
%The derivative on the right-hand side can be calculated using the expansion \eqref{eq_comp_expansion}. In particular, it is easy to verify that it satisfies the following bound 
% \begin{equation}\label{eq_comp_A2}
% |A_{\mu i}(k)|\prec \begin{cases}(\mathcal R_i^{(\mu)})^2+\mathcal R_\mu^2 , \ & \text{if } k \ge 2 \\ 
% \mathcal R_i^{(\mu)}\mathcal R_\mu , \ & \text{if } k = 1 \end{cases},
%  \end{equation}
%where for $i\in \cal I_1$ and $\mu \in \cal I_2\cup \cal I_3$, we denote
%\begin{equation}\label{eq_comp_Rs}
%\mathcal R_i^{(\mu)}:=|G_{\mathbf u \bu_i^{(\mu)}}|+|G_{ \mathbf v \bu_i^{(\mu)}}|,\quad \mathcal R_\mu:=|G_{\mathbf u \mu}|+|G_{ \mathbf v \mu}|.
%\end{equation}
%%for $a\in \sI$, where $\mathbf w_i:= \Sigma^{1/2} \bu_i$ for $i\in\sI_1$ and $\mathbf w_\mu:=\tilde \Sigma^{1/2} \bv_\mu$ for $\mu\in\sI_2$.
%%\begin{definition}[Words]\label{def_comp_words}
%%Given $i\in \mathcal I_1$ and $\mu\in \mathcal I_2\cup \cal I_3$. Let $\sW$ be the set of words of even length in two letters $\{\mathbf i, {\mu}\}$. We denote the length of a word $w\in\sW$ by $2m(w)$ with $m(w)\in \mathbb N$. We use bold symbols to denote the letters of words. For instance, $w=\mathbf t_1\mathbf s_2\mathbf t_2\mathbf s_3\cdots\mathbf t_r\mathbf s_{r+1}$ denotes a word of length $2r$.
%%Define $\sW_r:=\{w\in \mathcal W: m(w)=r\}$ to be the set of words of length $2r$, and such that
%%%We require that 
%%each word $w\in \sW_r$ satisfies that $\mathbf t_l\mathbf s_{l+1}\in\{\mathbf i{}{\mu},{}{\mu}\mathbf i\}$ for all $1\le l\le r$.
%%
%%Next we assign to each letter $*$ a value $[*]$ through $[\mathbf i]:=\Sigma \bu_i$, $[{} {\mu}]:=\tilde \Sigma \mathbf v_\mu,$ where $\mathbf u_i$ and $\bv_\mu$ are defined in Lemma \ref{lem_comp_gbound} and are regarded as summation indices. Note that it is important to distinguish the abstract letter from its value, which is a summation index. Finally, to each word $w$ we assign a random variable $A_{\mathbf v, i, \mu}(w)$ as follows. If $m(w)=0$ we define
%% $$A_{\mathbf v, i, \mu}(w):=G_{\mathbf v\mathbf v}-\Pi_{\mathbf v\mathbf v}.$$
%% If $m(w)\ge 1$, say $w=\mathbf t_1\mathbf s_2\mathbf t_2\mathbf s_3\cdots\mathbf t_r\mathbf s_{r+1}$, we define
%% \begin{equation}\label{eq_comp_A(W)}
%% A_{\mathbf v, i, \mu}(w):=G_{\bv[\mathbf t_1]} G_{[\mathbf s_2][\mathbf t_2]}\cdots G_{[\mathbf s_r][\mathbf t_r]} G_{[\mathbf s_{r+1}]\bv}.
%% \end{equation}
%%\end{definition}
%%Notice the words are constructed such that, by \eqref{deltaimu} and (\ref{eq_comp_expansion}) ,
%%\[\left(\frac{\partial}{\partial X_{i\mu}}\right)^r \left( G_{\mathbf v\mathbf v}-\Pi_{\mathbf v\mathbf v}\right)=(-1)^r r!\sum_{w\in \mathcal W_r} A_{\mathbf v, i, \mu}(w),\quad r\in \mathbb N,\]
%%with which we get that
%Then we can calculate the derivative
%\begin{align*}
%\left(\frac{\partial}{\partial Z_{\mu i}}\right)^r F_{\bu\bv}^p(Z)= \sum_{k_1+\cdots+k_p=r}\prod_{t=1}^{p/2} \left(A_{\mu i}(k_t)\overline{A_{\mu i}(k_{t+p/2})}\right).
%\end{align*}
%Then to prove (\ref{eq_comp_selfest}), it suffices to show that
%\begin{equation}
%n^{-2}q^{r-4}\sum_{i\in\mathcal I_1}\sum_{\mu\in\mathcal I_2\cup \cal I_3}\left|\bbE\prod_{t=1}^{p/2}A_{\mu i}(k_t)\overline{A_{\mu i}(k_{t+p/2})}\right|=\OO\left(\left(n^\e q\right)^p+\mathbb EF_{\bu\mathbf v}^p(Z,z)\right)\label{eq_comp_goal1}
%\end{equation}
%for  $4\le r\le p+4$ and $(k_1,\cdots,k_p)\in \N^p$ satisfying $k_1 +\cdots+k_p=r$. 
%%To avoid the unimportant notational complications associated with the complex conjugates, we will actually prove that
%%\begin{equation}\label{eq_comp_goal2}
%%N^{-2}q^{r-4}\sum_{i\in\mathcal I_1}\sum_{\mu\in\mathcal I_2}\left|\bbE\prod_{t=1}^{p}A_{\mathbf v, i, \mu}(w_t)\right|=\OO\left(\left(n^\e q\right)^p+\mathbb EF_{\bu\mathbf v}^p(Z,z)\right).
%%\end{equation}
%%The proof of $(\ref{eq_comp_goal1})$ is essentially the same but with slightly heavier notations. 
%Treating zero $k$'s separately (note $A_{\mu i}(0)=(G_{\mathbf u\mathbf v}-\Pi_{\mathbf u\mathbf v})$ by definition), we find that it suffices to prove
%\begin{equation}
%\label{eq_comp_goal3}n^{-2}q^{r-4}\sum_{i\in\mathcal I_1}\sum_{\mu\in\mathcal I_2\cup \cal I_3}\bbE|A_{\mu i}(0)|^{p-l}\prod_{t=1}^{l}\left|A_{\mu i}(k_t)\right|=\OO\left(\left(n^\e q\right)^p+\mathbb EF_{\bu\mathbf v}^p(Z,z)\right)
%\end{equation}
%for  $4\le r\le p+4$ and $1\le l \le p$. Here without loss of generality, we assume that $k_t=0$ for $l+1\le t \le p$, and $\sum_{t=1}^l k_t=r$ with $k_t \ge 1$ for $t\le l$.
%
%%To estimate (\ref{eq_comp_goal3}) we introduce the quantity
%%\begin{equation}\label{eq_comp_Rs}
%%\mathcal R_a:=|G_{\mathbf v \mathbf w_a}|+|G_{\mathbf w_a \mathbf v}|
%%\end{equation}
%%for $a\in \sI$, where $\mathbf w_i:= \Sigma^{1/2} \bu_i$ for $i\in\sI_1$ and $\mathbf w_\mu:=\tilde \Sigma^{1/2} \bv_\mu$ for $\mu\in\sI_2$.
%%
%%\begin{lemma}\label{lem_comp_A}
%%  For $w\in\sW$, we have the rough bound
%%  \begin{equation}
%%  |A_{\mathbf v, i, \mu}(w)|\prec N^{2\delta(m(w)+1)}.\label{eq_comp_A1}
%%  \end{equation}
%%  Furthermore, for $m(w)\ge 1$ we have
%%  \begin{equation}
%%  |A_{\mathbf v, i, \mu}(w)|\prec(\mathcal R_i^2+\mathcal R_\mu^2)N^{2\delta(m(w)-1)}.\label{eq_comp_A2}
%%  \end{equation}
%%  For $m(w)=1$, we have the better bound
%%  \begin{equation}
%%  |A_{\mathbf v, i, \mu}(w)|\prec \mathcal R_i\mathcal R_\mu.\label{eq_comp_A3}
%%  \end{equation}
%%\end{lemma}
%%\begin{proof}
%%The estimates (\ref{eq_comp_A1}) and (\ref{eq_comp_A2}) follow immediately from the rough bound (\ref{eq_comp_apbound}) and definition (\ref{eq_comp_A(W)}).  
%%%For (\ref{eq_comp_A2}), we break $A_{\mathbf v, i, \mu}(w)$ into $G_{\bv[\mathbf t_1]}(G_{[\mathbf s_2][\mathbf t_2]}\cdots G_{[\mathbf s_n][\mathbf t_n]})^{1/2}$ times $(G_{[\mathbf s_2][\mathbf t_2]}\cdots G_{[\mathbf s_n][\mathbf t_n]})^{1/2}G_{[\mathbf s_{n+1}]\bv}$ and use Cauchy-Schwarz inequality. 
%%The estimate (\ref{eq_comp_A3}) follows from the constraint $\mathbf t_1\ne\mathbf s_2$ in the definition (\ref{eq_comp_A(W)}).
%%\end{proof}
%Now we first consider the case $r\le 2l-2$. Then by pigeonhole principle, there exist at least two $k_t$'s with $k_t=1$. Therefore by \eqref{eq_comp_A2} we have
%\begin{equation}\label{eq_comp_r1}
%\prod_{t=1}^{l}\left|A_{\mu i}(k_t)\right| \prec  \one(r\ge 2l-1)\left[(\mathcal R_i^{(\mu)})^2+\mathcal R_\mu^2\right]+\one(r\le 2l-2)(\mathcal R_i^{(\mu)})^2\mathcal R_\mu^2 .
%\end{equation}
%%Let $\mathbf v=\left( {\begin{array}{*{20}c}
%%   {\mathbf v}_1   \\
%%   {\mathbf v}_2 \\
%%   \end{array}} \right)$ for ${\mathbf v}_1 \in\mathbb C^{\mathcal I_1}$ and ${\mathbf v}_2\in\mathbb C^{\mathcal I_2}$. 
%Using \eqref{priorim} and a similar argument as in \eqref{GG*}, we get that
%\begin{align}
%\sum_{i\in\sI_1}(\mathcal R_i^{(\mu)})^2 =\OO(1),\quad  \sum_{\mu\in\sI_2 \cup \cal I_3}\mathcal R_{\mu}^2 =\OO(1), \quad \text{with high probability}.\label{eq_comp_r2}
%\end{align}
%%where in the second step we used the two bounds in Lemma \ref{lemm_comp_2} and $\eta =\OO(\im m_{2c})$ by \eqref{Immc}, and in the last step the definition of $\Psi$ in \eqref{eq_defpsi}. Using the same method we can get
%%\begin{equation}\label{eq_comp_r3}
%%\frac{1}{N^2}\sum_{i\in\sI_1}\sum_{\mu\in\sI_2}\mathcal R_i^2\mathcal R_\mu^2\prec \left[N^{(C_a+2)\delta}\left(\Psi^2(z) + \frac{q}{N\eta}\right)\right]^2.
%%\end{equation}
%Using (\ref{eq_comp_r2}) and $n^{-1/2}\le q$, we get that %the left-hand side of (\ref{eq_comp_goal3}) is bounded by
%\begin{align*}
%n^{-2}q^{r-4}\sum_{i\in\mathcal I_1}\sum_{\mu\in\mathcal I_2\cup \cal I_3}|A_{\mu i}(0)|^{p-l}\prod_{t=1}^{l}\left|A_{\mu i}(k_t)\right| &\prec  q^{r-4} F_{\bu\bv}^{p-l}(Z)\left[\one(r\ge 2l-1) n^{-1} +\one(r\le 2l-2)n^{-2}\right] \\
%&\le  F_{\bu\bv}^{p-l}(Z)\left[\one(r\ge 2l-1)q^{r-2}+\one(r\le 2l-2)q^r\right].
%%\\ &\le \bbE F_{\bv}^{p-l}(X)\left[\one(r\ge 2l-1)\left(N^{C_a\delta/2+12\delta}(q+\Psi)\right)^{r-2}+\one(r\le 2l-2)\left(N^{C_a\delta/2+12\delta}(q+\Psi)\right)^r\right],
%\end{align*}
%%\[q^{m-4}N^{2\delta(n+l+2)}\bbE F_{\bv}^{p-l}(X)\left[\one(m\ge 2l-1)\left(N^{C_a\delta/2}(q+\Psi(z))\right)^2+\one(m\le 2l-2)\left(N^{C_a\delta/2}(q+\Psi(z))\right)^4\right].\]
%%Using $\Phi \gtrsim N^{-1/2}$, we find that the left hand side of (\ref{eq_comp_goal3}) is bounded by
%%\begin{align*}
%% & N^{2\delta(n+q+2)} \bbE F_{\bv}^{p-q}(X)\left(\one(m\ge 2l-1)\left(N^{C_0\delta/2}\Phi\right)^{n-2}+\one(m\le 2l-2)\left(N^{C_0\delta/2}\Phi\right)^n\right)\\
%% &\le \bbE F_{\bv}^{p-q}(X)\left(\one(m\ge 2l-1)\left(N^{C_0\delta/2+12\delta}\Phi\right)^{n-2}+\one(m\le 2l-2)\left(N^{C_0\delta/2+12\delta}\Phi\right)^n\right)
%%\end{align*}
%If $r\le 2l-2$, then we get $q^r\le q^l$ using the trivial inequality $r\ge l$. On the other hand, if $r\ge 4$ and $r\ge 2l-1$, then $r\ge l+2$ and we get $q^r\le q^{l+2}$. Therefore we conclude that %the left-hand side of $(\ref{eq_comp_goal3})$ is bounded by
%\begin{equation}\nonumber
%n^{-2}q^{r-4}\sum_{i\in\mathcal I_1}\sum_{\mu\in\mathcal I_2\cup \cal I_3} |A_{\mu i}(0)|^{p-l}\prod_{t=1}^{l}\left|A_{\mu i}(k_t)\right| \prec F_{\bu\bv}^{p-l}(Z) q^l.
%\end{equation}
%Now (\ref{eq_comp_goal3}) follows from H\"older's inequality. This concludes the proof of (\ref{eq_comp_selfest}), and hence of (\ref{lemm_comp_5}), and hence of \eqref{aniso_law}. 
%%This proves \eqref{goal_ani2}, and hence \eqref{aniso_law} under the condition \eqref{3moment}.
%
%
%%Proposition \ref{comparison_prop} under the assumption (\ref{assm_3rdmoment}).
%%the anisotropic local law in Theorem \ref{law_wideT}.
%%\end{subsection}
%%\end{subsection}
%
%
%
%%\subsection{Non-vanishing third moment}\label{subsec_3moment}
%%
%%In this subsection, we prove Lemma \ref{lemm_comp_0} under \eqref{assm_3moment} for $z\in \tilde S(c_0, C_0,\e)$. 
%%%In this case, we can verify that
%%%\begin{equation}\label{eq_comp_boundPhi}
%%%\Phi \le N^{-1/4-\zeta/2}.
%%%\end{equation}
%%Following the arguments in Sections \ref{subsec_interp}-\ref{section_words}, we see that it suffices to prove the estimate ($\ref{eq_comp_selfest}$) in the $r=3$ case. In other words, we need to prove the following lemma. 
%%\begin{lemma}\label{lemm_comparison_big}
%%Fix $p\in 2\mathbb N$ and $m \le \delta^{-1}$. Let $z\in {\mathbf S}_m $ and suppose $(\mathbf A_{m-1})$ holds. Then %we have
%%\begin{equation}\label{eq_comp_selfest_generalX}
%%b_N N^{-2}\sum_{i\in\mathcal I_1}\sum_{\mu\in\mathcal I_2}\left|\bbE f^{(3)}_{(\mu i)}(X_{i\mu}^\theta)\right|=\OO\left(\left[N^{C_a\delta} (q+\Psi)\right]^p+\mathbb EF_{\mathbf v}^p(X^\theta,z)\right).
%%\end{equation}
%%\end{lemma}
%%\begin{proof}
%%The main new ingredient of the proof is a further iteration step at a fixed $z$. Suppose
%%\begin{equation}\label{comp_geX_iteration}
%%G-\tilde\Pi=\OO_\prec(\Phi)
%%\end{equation}
%%for some deterministic parameter $\Phi\equiv \Phi_N$. By the a priori bound (\ref{eq_comp_apbound}), we can take $\Phi\le N^{2\delta}$. Assuming (\ref{comp_geX_iteration}), we shall prove a self-improving bound of the form
%%\begin{equation}\label{comp_geX_self-improving-bound}
%%b_N N^{-2}\sum_{i\in\mathcal I_1}\sum_{\mu\in\mathcal I_2}\left|\bbE f^{(3)}_{(\mu i)}(X_{i\mu}^\theta)\right|=\OO\left(\left[N^{C_a\delta} (q+\Psi)\right]^p+(N^{-\epsilon/2}\Phi)^p+\mathbb EF_{\mathbf v}^p(X^\theta,w)\right).
%%\end{equation}
%%Once (\ref{comp_geX_self-improving-bound}) is proved, we can use it iteratively to get an increasingly accurate bound 
%%for $\left|G_{\mathbf{vv}}(X,z)-\Pi_{\mathbf {vv}}(z)\right|$. After each step, we obtain a better bound (\ref{comp_geX_iteration}) with $\Phi$ reduced by $N^{-\e/2}$. Hence after $\OO(\e^{-1})$ many iterations we can get (\ref{eq_comp_selfest_generalX}).
%%
%%As in Section \ref{section_words}, to prove (\ref{comp_geX_self-improving-bound}) it suffices to show 
%%\begin{equation}\label{comp_geX_words}
%%b_N N^{-2}\left|\sum_{i\in\mathcal I_1}\sum_{\mu\in\mathcal I_2}A^{p-l}_{\mathbf v, i, \mu}(w_0)\prod_{t=1}^{l}A_{\mathbf v, i, \mu}(w_t)\right|\prec F_{\bv}^{p-l}(X)\left[N^{(C_0-1)\delta}(q+\Psi) + N^{-\e/2}\Phi\right]^l,
%%\end{equation}
%%which follows from the bound
%%\begin{equation}\label{comp_geX_words2}
%%b_N N^{-2}\left|\sum_{i\in\mathcal I_1}\sum_{\mu\in\mathcal I_2}\prod_{t=1}^{l}A_{\mathbf v, i, \mu}(w_t)\right|\prec \left[N^{(C_0-1)\delta}(q+\Psi) + N^{-\e/2}\Phi\right]^l.
%%\end{equation}
%%%Each of the three cases $l=1,\, 2,\, 3$ can be proved as in \cite[Lemma 12.7]{Anisotropic}, and we leave the details to the reader. This concludes Lemma \ref{lemm_comparison_big}.
%%%The rest of the proof is straightforward. 
%%We now list all the three cases with $l=1,\, 2,\, 3$, and discuss each case separately.  
%%
%%When $l = 1$, the single factor $A_{\mathbf v, i, \mu}(w_1)$ is of the form
%%\[ G_{\mathbf v[\mathbf t_1]} G_{[\mathbf s_2][\mathbf t_2]} G_{[\mathbf s_3][\mathbf t_3]} G_{[\mathbf s_4]\mathbf v}.\]
%%%\[\wt G_{[s][t]}= G_{[s][t]}-\wt\Pi_{[s][t]},\] 
%%Then we split it as
%%\begin{align}
%%G_{\mathbf v[\mathbf t_1]} G_{[\mathbf s_2][\mathbf t_2]} G_{[\mathbf s_3][\mathbf t_3]} G_{[\mathbf s_4]\mathbf v}
%%=& G_{\mathbf v[\mathbf t_1]} \Pi_{[\mathbf s_2][\mathbf t_2]} \Pi_{[\mathbf s_3][\mathbf t_3]} G_{[\mathbf s_4]\mathbf v} + G_{\mathbf v[\mathbf t_1]}\wt G_{[\mathbf s_2][\mathbf t_2]} \Pi_{[\mathbf s_3][\mathbf t_3]}G_{[\mathbf s_4]\mathbf v}\nonumber\\
%% + & G_{\mathbf v[\mathbf t_1]} \Pi_{[\mathbf s_2][\mathbf t_2]} \wt G_{[\mathbf s_3][\mathbf t_3]}  G_{[\mathbf s_4]\mathbf v}+ G_{\mathbf v[\mathbf t_1]}\wt G_{[\mathbf s_2][\mathbf t_2]}
%%\wt G_{[\mathbf s_3][\mathbf t_3]} G_{[\mathbf s_4]\mathbf v},\label{comp_geX_expG}
%%\end{align}
%%where we abbreviate $\wt G: = G - \Pi$. For the second term, we have
%%\begin{align}\label{term11}
%%b_N N^{-2}\sum_{i\in\mathcal I_1}\sum_{\mu\in\mathcal I_2}\left|G_{\mathbf v[\mathbf t_1]}\wt G_{[\mathbf s_2][\mathbf t_2]} \Pi_{[\mathbf s_3][\mathbf t_3]}G_{[\mathbf s_4]\mathbf v}\right|\prec b_N \Phi \cdot N^{(C_a+2)\delta}\left(\Psi^2 + \frac{q}{N\eta}\right)\prec N^{-\e/2}\Phi
%%\end{align}
%%provided $\delta$ is small enough, where we used (\ref{eq_comp_r2}), (\ref{comp_geX_iteration}) and the definition \eqref{tildeS}. The third and fourth term of (\ref{comp_geX_expG}) can be dealt with in a similar way. For the first term, when $[\mathbf t_1]=\mathbf w_i$ and $[\mathbf s_4]=\bw_\mu$, we have
%%\begin{align*}
%%& \Big|\sum_{i\in\mathcal I_1}\sum_{\mu\in\mathcal I_2} G_{\mathbf v \mathbf w_i} \Pi_{[\mathbf s_2][\mathbf t_2]}\Pi_{[\mathbf s_3][\mathbf t_3]} G_{\mathbf w_\mu \mathbf v}\Big| \prec N^{1+2\delta}\left(\sum_{\mu\in\mathcal I_2}| G_{\mathbf w_\mu\mathbf v}|^2\right)^{1/2}\prec N^{3/2+(C_a/2+3)\delta}(q+\Psi),
%%\end{align*}
%%where we used (\ref{eq_comp_r2}) and the fact that $\Pi$ is deterministic, such that the a priori bound (\ref{comp_eq_apriori}) gives
%%$$\Big|\sum_{i\in\mathcal I_1} G_{\mathbf v \mathbf w_i} \Pi_{[\mathbf s_2][\mathbf t_2]}\Pi_{[\mathbf s_3][\mathbf t_3]} \Big| \prec N^{1/2+2\delta} .$$
%%%Cauchy-Schwarz inequality, a priori bounds (\ref{comp_eq_apriori}) and (\ref{eq_comp_r2}), and $\|\sum_i\mathbf v_i\|\le \sqrt N$. 
%%If $[\mathbf t_1]=\mathbf w_\mu$ and $[\mathbf s_4]=\mathbf v_i$, the proof is similar. If $[\mathbf t_1]=[\mathbf s_4]$, then at least one of the terms $\Pi_{[\mathbf s_2][\mathbf t_2]}$ and $\Pi_{[\mathbf s_3][\mathbf t_3]}$ must be of the form $\Pi_{\mathbf w_i\mathbf w_\mu}$ or $\Pi_{\mathbf w_\mu\mathbf w_i}$, and hence we have
%%$$\sum_i|\Pi_{[\mathbf s_2][\mathbf t_2]}\Pi_{[\mathbf s_3][\mathbf t_3]}|=\OO(N^{1/2}) \quad\text{ or }\quad \sum_\mu | \Pi_{[\mathbf s_2][\mathbf t_2]} \Pi_{[\mathbf s_3][\mathbf t_3]}|=\OO(N^{1/2}).$$
%%Therefore using $(\ref{eq_comp_r2})$ and (\ref{tildeS}), we get
%%\begin{align*}
%%\Big|\sum_{i\in\mathcal I_1}\sum_{\mu\in\mathcal I_2}G_{\mathbf v[\mathbf t_1]} \Pi_{[\mathbf s_2][\mathbf t_2]}\Pi_{[\mathbf s_3][\mathbf t_3]}G_{[\mathbf s_4]\mathbf v}\Big|
%%& \prec N^{3/2+(C_a+2)\delta}\left(q^2 + \Psi^2 \right) \le  N^{3/2}(q+\Psi) .
%%\end{align*}
%%provided $\delta$ is small enough. 
%%%where we used $(\ref{eq_comp_r2})$ and (\ref{eq_comp_boundPhi}).
%%In sum, we obtain that
%%$$b_NN^{-2}\Big|\sum_{i\in\mathcal I_1}\sum_{\mu\in\mathcal I_2} G_{\mathbf v[\mathbf t_1]} \Pi_{[\mathbf s_2][\mathbf t_2]}\Pi_{[\mathbf s_3][\mathbf t_3]} G_{[\mathbf s_4]\mathbf v}\Big|\prec N^{(C_a-1)\delta}(q+\Psi)$$
%%provided that $C_a\ge 8$. Together with \eqref{term11}, this proves  (\ref{comp_geX_words2}) for $l=1$.
%%
%%When $l=2$, $\prod_{t=1}^2 A_{\mathbf v, i, \mu}(w_t)$ is of the form
%%\begin{align}
%% & G_{\mathbf v\mathbf w_i} G_{\mathbf w_\mu \mathbf v} G_{\mathbf v \mathbf w_i} G_{\mathbf w_\mu\mathbf w_\mu} G_{\mathbf w_i\mathbf v}, \quad   G_{\mathbf v\mathbf w_i} G_{\mathbf w_\mu \mathbf v} G_{\mathbf v \mathbf w_\mu} G_{\mathbf w_i \mathbf w_i} G_{\mathbf w_\mu\mathbf v}, \label{eqn_q21}  \\
%% &  G_{\mathbf v\mathbf w_i} G_{\mathbf w_\mu \mathbf v} G_{\mathbf v \mathbf w_i} G_{\mathbf w_\mu \mathbf w_i} G_{\mathbf w_\mu\mathbf v}, \quad   G_{\mathbf v\mathbf w_i} G_{\mathbf w_\mu \mathbf v} G_{\mathbf v \mathbf w_\mu} G_{\mathbf w_i \mathbf w_\mu} G_{\mathbf w_i\mathbf v},\label{eqn_q22}
%%\end{align}
%%or an expression obtained from one of these four by exchanging $\mathbf w_i$ and $\mathbf w_\mu$. The first expression in (\ref{eqn_q21}) can be estimated using (\ref{comp_eq_apriori}), (\ref{eq_comp_r2}) and (\ref{comp_geX_iteration}):
%%\begin{equation}
%%\label{q=2_1}\sum_\mu G_{\mathbf w_\mu \mathbf v} G_{\mathbf w_\mu\mathbf w_\mu}=\sum_\mu G_{\mathbf w_\mu \mathbf v}\wt G_{\mathbf w_\mu\mathbf w_\mu}+\sum_\mu  G_{\mathbf w_\mu \mathbf v}\Pi_{\mathbf w_\mu\mathbf w_\mu}=\OO_\prec\left[N^{1+(C_a/2+1)\delta}\Phi\left(\Psi^2 + \frac{q}{N\eta}\right)^{1/2}+ N^{1/2+2\delta}\right],
%%\end{equation}
%%and
%%\begin{equation}\label{q=2_2}
%%\Big|\sum_i G_{\mathbf v\mathbf w_i} G_{\mathbf v \mathbf w_i} G_{\mathbf w_i\mathbf v}\Big|\prec N^{1+(C_a+4)\delta} \left(\Psi^2 + \frac{q}{N\eta}\right).
%%\end{equation}
%%Combining \eqref{tildeS}, (\ref{q=2_1}) and (\ref{q=2_2}), we get that 
%%\[b_N N^{-2}\Big|\sum_i\sum_\mu G_{\mathbf v\mathbf w_i} G_{\mathbf w_\mu \mathbf v} G_{\mathbf v \mathbf w_i} G_{\mathbf w_\mu\mathbf w_\mu} G_{\mathbf w_i\mathbf v}\Big| \prec \left(N^{(C_a-1)\delta}(q+\Psi) + N^{-\e/2}\Phi\right)^2,\]
%%provided $\delta$ is small enough. The second expression in (\ref{eqn_q21}) can be estimated similarly. The first expression of (\ref{eqn_q22}) can be estimated using \eqref{tildeS}, (\ref{comp_eq_apriori}) and (\ref{eq_comp_r2}) by
%%\begin{equation*}
%%\begin{split}
%%b_N N^{-2}\left|\sum_i\sum_\mu  G_{\mathbf v\mathbf w_i} G_{\mathbf w_\mu \mathbf v} G_{\mathbf v \mathbf w_i} G_{\mathbf w_\mu \mathbf w_i} G_{\mathbf w_\mu\mathbf v}\right|& \prec b_N N^{-2+2\delta}\sum_i\sum_\mu\left| G_{\mathbf v\mathbf w_i}\right|^2\left| G_{\mathbf w_\mu\mathbf v}\right|^2 \\
%%& \prec b_N N^{(2C_0+6)\delta}\left( \Psi^2 +\frac{q}{N\eta}\right)^2 \le (q +\Psi)^2
%%\end{split}
%%\end{equation*}
%%for small enough $\delta$. The second expression in (\ref{eqn_q22}) is estimated similarly.  This proves (\ref{comp_geX_words2}) for $l=2$.
%%
%%When $l = 3$, $\prod_{t=1}^3 A_{\mathbf v, i, \mu}(w_t)$ is of the form 
%%$( G_{\mathbf v\mathbf w_i} G_{\mathbf w_\mu \mathbf v})^3$ or an expression obtained by exchanging $\mathbf w_i$ and $\mathbf w_\mu$ in some of the three factors. We use (\ref{eq_comp_r2}) and $\sum_i|\Pi_{\mathbf v\mathbf w_i}|^2 = \OO(1)$ to get that
%%\[\left|\sum_i( G_{\mathbf v\mathbf w_i})^3\right|\prec \sum_i|\wt G_{\mathbf v\mathbf w_i}|^3+\sum_i|\Pi_{\mathbf v\mathbf w_i}|^3\prec \Phi\sum_i \left(| G_{\mathbf v\mathbf w_i}|^2+|\Pi_{\mathbf v\mathbf w_i}|^2 \right)+1\prec N^{1+(C_0+2)\delta}\left(\Psi^2 +\frac{q}{N\eta}\right)\Phi+\Phi+1.\]
%%Now we conclude (\ref{comp_geX_words2}) for $l=3$ using \eqref{tildeS} and $N^{-1/2}=\OO( q+\Psi)$.
%%\end{proof}
%----------remove end----------------

%Finally, if the condition \eqref{3moment} does not hold, then there is also an $r=3$ term in the Taylor expansion \eqref{taylor1}:
%$$\frac{1}{6}\bbE f^{(3)}_{(\mu i)}(0)\bbE\left(Z_{i\mu}^a\right)^3.$$
%Note that $\bbE\left(Z_{i\mu}^a\right)^3$ is of order $n^{-3/2}$, while the sum over $i$ and $\mu$ in \eqref{lemm_comp_5} provides a factor $n^2$. In fact, $\bbE f^{(3)}_{(\mu i)}(0)$ will provide an extra $n^{-1/2}$ to compensate the remaining $n^{1/2}$ factor. This follows from an improved self-consistent comparison argument for sample covariance matrices in \cite[Section 8]{Anisotropic}. The argument for our case is almost the same except for some notational differences, so we omit the details. 



%\subsection{Weak averaged local law}\label{section_averageTX}

%We first prove the following weak averaged local law.
%
%\begin{lemma} \label{thm_largerigidity2}
%Suppose the assumptions in Theorem \ref{thm_largerigidity} hold. %Fix the constants $c_0$ and $C_0$ as given in Theorem \ref{LEM_SMALL}. 
%Then for any fixed $\epsilon>0$, we have
%%there exists constant $C_1>0$, depending only on $c_0$, $C_0$, $B$ and $\phi$, such that %with high probability we have
%\begin{equation}
% \vert m(z)-m_{c}(z) \vert \prec  q^2 + (N \eta)^{-1}, \label{NEWMPBOUNDS2}
%\end{equation}
%uniformly for all $z \in  S(c_0, C_0, \epsilon)$. Moreover, outside of the spectrum we have the following stronger averaged local law (recall \eqref{KAPPA})
%\begin{equation}\label{aver_out2}
% | m(z)-m_{c}(z)|\prec q^2   + \frac{1}{N(\kappa +\eta)} + \frac{1}{(N\eta)^2\sqrt{\kappa +\eta}},
%\end{equation}
%uniformly in $z\in S(c_0,C_0,\epsilon)\cap \{z=E+\ii\eta: E\ge \lambda_r, N\eta\sqrt{\kappa + \eta} \ge N^\epsilon\}$ for any constant $\epsilon>0$. If $A$ or $B$ is diagonal, then \eqref{NEWMPBOUNDS2} and \eqref{aver_out2} hold without the condition \eqref{assm_3moment}.
%\end{lemma}

%In this section, we prove the weak averaged local laws in \eqref{aver_in1} and \eqref{aver_out1}. %\begin{proof}
%The proof is similar to that for \eqref{aniso_law} in previous subsections, and we only explain the differences. Note that the bootstrapping argument is not necessary, since we already have a good a priori bound by \eqref{aniso_law}.
%%In this section we prove the averaged local law in Theorem \ref{law_wideT}. The anisotropic local law proved in the previous section gives a good a priori bound. 
%In analogy to (\ref{eq_comp_F(X)}), we define
%%\begin{align*}
%%\wt F(X,w) : &=|w|^{1/2} |m_2(w)-m_{2c}(w)|=|w|^{1/2}\left|\frac{1}{N}\sum_{\nu\in\sI_2}G_{\nu\nu}(w)-m_{2c}(w)\right|\\
%%&=\left|\frac{1}{N}\sum_{\nu\in\sI_2} G_{\nu\nu}(w)-|w|^{1/2}m_{2c}(w)\right|.
%%\end{align*}
%\begin{align*}
%\wt F(X,z) : &= |m(z)-m_{c}(z)| =\left|\frac{1}{nz}\sum_{i\in\sI_1} \left(G_{ii}(X,z)- \Pi_{ii}(z)\right)\right|,
%\end{align*}
%where we used \eqref{mcPi}. 
%%$\Phi^2 =\OO(|w|^{1/2}/{(N\eta)})$, 
%%it suffices to prove that $\wt F\prec (q+\Psi(z))^2$. 
%Moreover, by Proposition \ref{prop_diagonal}, \eqref{aver_in1} and \eqref{aver_out1} hold for Gaussian $X$ (without the $q^2$ term). For now, we assume \eqref{3moment} and prove the following stronger estimates:
%\begin{equation}
% \vert m(z)-m_{c}(z) \vert \prec (N \eta)^{-1} \label{aver_ins} %+ q^2 
%\end{equation}
%for $z\in S(c_0,C_0,\epsilon)$, and 
%\begin{equation}\label{aver_outs}
% | m(z)-m_{c}(z)|\prec \frac{q}{N\eta}  + \frac{1}{N(\kappa +\eta)} + \frac{1}{(N\eta)^2\sqrt{\kappa +\eta}},
%\end{equation}
%for $z\in S(c_0,C_0,\epsilon)\cap \{z=E+\ii\eta: E\ge \lambda_r, N\eta\sqrt{\kappa + \eta} \ge N^\epsilon\}$. At the end of this section, we will show how to relax \eqref{3moment} to \eqref{assm_3moment} for $z\in \tilde S(c_0,C_0,\e)$.
%
%Note that
%\be\label{psi2}
%\Psi^2(z) \lesssim \frac{1}{N\eta}, \quad \text{and} \quad \Psi^2(z) \lesssim \frac{1}{N(\kappa +\eta)} + \frac{1}{(N\eta)^2\sqrt{\kappa +\eta}} \ \text{ outside of the spectrum}.
%\ee
%Then following the argument in Section \ref{subsec_interp}, analogous to (\ref{eq_comp_selfest}), we only need to prove that
%\begin{equation}\label{eq_comp_selfestAvg}
%N^{-2}q^{r-4}\sum_{i\in\mathcal I_1}\sum_{\mu\in\mathcal I_2}\left|\bbE \left(\frac{\partial}{\partial X_{i\mu}}\right)^r\wt F^p(X)\right|=\OO\left(\left[N^{\delta}\left(\Psi^2+\frac{q}{N\eta}\right)\right]^p+\mathbb E\wt F^p(X)\right)
%\end{equation}
%for all $r=4,...,4p+4$, where $\delta>0$ is any positive constant. Analogous to (\ref{eq_comp_goal2}), it suffices to prove that for $r=4,...,4p+4$,
%\begin{equation}\label{eq_comp_goalAvg}
%N^{-2}q^{r-4}\sum_{i\in\mathcal I_1}\sum_{\mu\in\mathcal I_2}\left|\bbE\prod_{t=1}^{p}\left(\frac{1}{n}\sum_{j\in\sI_1}A_{ \mathbf e_j, i, \mu}(w_t)\right)\right|=\OO\left(\left[N^{\delta}\left(\Psi^2+\frac{q}{N\eta}\right)\right]^p+\mathbb E\wt F^p(X)\right)
%\end{equation}
%for $\sum_t m(w_t)=r$. 
%%The only difference in the definition of $A_{\mathbf v, i, \mu}(w)$ is that when $m(w)=0$, we define
%%\[A_{\mathbf v, i, \mu}(w):= G_{\mathbf v\mathbf v}-|w|^{1/2}m_{2c}.\]
%Similar to (\ref{eq_comp_Rs}) we define
%\begin{equation}\nonumber%\label{eq_comp_RsAvg}
%\mathcal R_{j, a}:=| G_{j \mathbf w_a}|+| G_{\mathbf w_a j}|.
%\end{equation}
%Using \eqref{aniso_law} and Lemma \ref{lem_comp_gbound}, similarly to \eqref{eq_comp_r2}, we get that
%\begin{equation}\label{eq_comp_r22}
%\begin{split}
% \frac{1}{n}\sum_{j\in\sI_1}\mathcal R_{j,a}^2 & \prec \frac{ \im \left(z^{-1}G_{\mathbf w_i\mathbf w_i}\right) + \im G_{\mathbf w_\mu\mathbf w_\mu} + \eta\left(\left| G_{\mathbf w_i\mathbf w_i} \right|+ \left| G_{\mathbf w_\mu \mathbf w_\mu} \right|\right)}{N\eta} \prec \Psi^2+\frac{q}{N\eta}.
% \end{split}
%\end{equation}
%Since $G=\OO_\prec(1)$ by \eqref{aniso_law}, we have 
%\begin{equation}\label{average_bound}
%\left|\frac{1}{n}\sum_{j\in\sI_1}A_{ \mathbf e_j, i, \mu}(w)\right|\prec \frac{1}{n}\sum_{j\in\sI_1}\left(\mathcal R_{j,i}^2+\mathcal R_{j,\mu}^2\right)\prec \Psi^2 +\frac{q}{N\eta} \quad \text{ for any $w$ such that }m(w)\ge 1.
%\end{equation}
%With (\ref{average_bound}), for any $r\ge 4$, the left-hand side of (\ref{eq_comp_goalAvg}) is bounded by
%\[\bbE\wt F^{p-l}(X)\left(\Psi^2+\frac{q}{N\eta}\right)^{l}.\]
%Applying Holder's inequality, we get \eqref{eq_comp_selfestAvg}, which completes the proof of \eqref{aver_ins} and \eqref{aver_outs} under \eqref{3moment}. %\cor about removing 3rd moment assumption \nc
%

%\end{proof}


%Then we prove the averaged local law for $z\in \tilde S(c_0,C_0,\e)$ under \eqref{assm_3moment}. By \eqref{psi2}, it suffices to prove 
%\begin{equation}\label{comp_avg_geX_self-improving-bound}
%b_N N^{-2}\left|\sum_{i\in\mathcal I_1}\sum_{\mu\in\mathcal I_2}\bbE \left(\frac{\partial}{\partial X_{i\mu}}\right)^3\wt F^p(X)\right|=\OO\left(\left[N^\delta (q^2 +\Psi^2)\right]^p + \left( \frac{N^{-\e/2}}{N\eta}\right)^p+\mathbb E\wt F^p(X)\right),
%\end{equation}
%for any constant $\delta>0$. Analogous to the arguments in Section \ref{subsec_3moment}, it reduces to showing that
%\begin{equation}\label{eq_comp_goalAvg_genX}
%b_N N^{-2}\left|\sum_{i\in\mathcal I_1}\sum_{\mu\in\mathcal I_2} \prod_{t=1}^{l}\left(\frac{1}{n}\sum_{j\in\sI_1}A_{ \mathbf e_j, i, \mu}(w_t)\right)\right|=\OO_\prec\left(\left(q^2+\Psi^2\right)^{l} + \left( \frac{N^{-\e/2}}{N\eta}\right)^l\right),
%\end{equation}
%where $l\in \{1,2,3\}$ is the number of words with nonzero length. Then we can discuss these three cases using a similar argument as in Section \ref{subsec_3moment}, with the only difference being that we now can use the anisotropic local law \eqref{aniso_law} instead of the a priori bounds \eqref{comp_eq_apriori}  and (\ref{comp_geX_iteration}). %As an example, we only give the proof for the case with $l=1$.
%
%%Again we can prove the three cases $q=1,\, 2,\, 3$ as in \cite[Lemma 12.8]{Anisotropic}, and we leave the details to the reader. This concludes the averaged local law. 
%%
%%Again we discuss the three cases $q=1,\, 2,\, 3$ separately. During the proof we tacitly use the anisotropic local law proved above.
%
%In the $l=1$ case, we first consider the expression $A_{ \mathbf e_j, i, \mu}(w_1) =  G_{j\mathbf w_i} G_{\mathbf w_\mu \mathbf w_\mu} G_{\mathbf w_i\mathbf w_i} G_{\mathbf w_\mu j}$. We have 
%\begin{equation}\nonumber
%\left|\sum_i G_{j\mathbf w_i} G_{\mathbf w_i\mathbf w_i}\right| \le \left|\sum_i G_{j\mathbf w_i} \Pi_{\mathbf w_i\mathbf w_i}\right| + \sum_i (q+\Psi)\left| G_{j\mathbf w_i} \right|\prec \sqrt N + N (q+\Psi) \left(\Psi^2 +\frac{q}{N\eta}\right)^{1/2},
%\end{equation}
%where we used \eqref{aniso_law} and \eqref{eq_comp_r2}.
%%since the leading term is $\sum_i\wt\Pi_{\nu\mathbf v_i}\wt\Pi_{\mathbf v_i\mathbf v_i}$. 
%Similarly, we also have
%\begin{equation}\nonumber
% \left|\sum_\mu G_{\mathbf w_\mu\mathbf w_\mu} G_{\mathbf w_\mu j}\right|\prec \left|\sum_\mu {\Pi}_{\mathbf w_\mu \mathbf w_\mu}  G_{\mathbf w_\mu j}\right|  + \left|\sum_\mu \tilde{ G}_{\mathbf w_\mu \mathbf w_\mu}  G_{\mathbf w_\mu j}\right|  \prec \sqrt N(q+\Psi)+N (q+\Psi) \left(\Psi^2 +\frac{q}{N\eta}\right)^{1/2},
%\end{equation} 
%where we also used $\Pi_{\mathbf w_\mu j}=0$ for any $\mu$ in the second step. Then with \eqref{tildeS}, we can see that the LHS of (\ref{eq_comp_goalAvg_genX}) is bounded by $\OO_\prec(q^2 + \Psi^2)$ in this case.
%%$$b_N N^{-2}\left|\sum_{i\in\mathcal I_1}\sum_{\mu\in\mathcal I_2}\left(\frac{1}{n}\sum_{j\in\sI_1}A_{ \mathbf e_j, i, \mu}(w_1)\right)\right| \prec q^2+\Psi^2 .$$
%For the case $A_{ \mathbf e_j, i, \mu}(w_1) =  G_{j\mathbf w_i} G_{\mathbf w_\mu\mathbf w_\mu} G_{\mathbf w_i \mathbf w_\mu} G_{\mathbf w_i j}$, we can estimate that
%$$\left|\sum_\mu  G_{\mathbf w_\mu\mathbf w_\mu} G_{\mathbf w_i \mathbf w_\mu} \right| \le \left|\sum_\mu  \Pi_{\mathbf w_\mu\mathbf w_\mu} G_{\mathbf w_i \mathbf w_\mu} \right| + \sum_\mu (q+\Psi)\left|G_{\mathbf w_i \mathbf w_\mu} \right| \prec \sqrt N + N (q+\Psi) \left(\Psi^2 +\frac{q}{N\eta}\right)^{1/2},$$
%and
%$$ \sum_i \left|G_{j\mathbf w_i} G_{\mathbf w_i j}\right|\prec N\left(\Psi^2 +\frac{q}{N\eta}\right).$$
%Thus in this case the LHS of (\ref{eq_comp_goalAvg_genX}) is also bounded by $\OO_\prec(q^2 + \Psi^2)$. The case $A_{ \mathbf e_j, i, \mu}(w_1) =  G_{j\mathbf w_i} G_{\mathbf w_\mu\mathbf w_i} G_{\mathbf w_\mu \mathbf w_\mu} G_{\mathbf w_i j}$ can be handled similarly. Finally in the case $A_{ \mathbf e_j, i, \mu}(w_1) =  G_{j\mathbf w_i} G_{\mathbf w_\mu\mathbf w_i} G_{\mathbf w_\mu \mathbf w_i} G_{\mathbf w_\mu j}$, we can estimate that
%$$ \left|\sum_{i,\mu}  G_{j\mathbf w_i} G_{\mathbf w_\mu\mathbf w_i} G_{\mathbf w_\mu \mathbf w_i} G_{\mathbf w_\mu j} \right| \prec  \sum_{i,\mu} \left(\left| G_{j\mathbf w_i}\right|^2 +\left| G_{\mathbf w_\mu j} \right|^2 \right) | G_{\mathbf w_\mu \mathbf w_i}|^2 \prec N^2 \left(\Psi^2 +\frac{q}{N\eta}\right)^2.$$
%Again in this case the LHS of (\ref{eq_comp_goalAvg_genX}) is bounded by $\OO_\prec(q^2 + \Psi^2)$. All the other expressions are obtained from these four by exchanging $\mathbf w_i$ and $\mathbf w_\mu$.
%
%In the $l=2$ case, $\prod_{t=1}^{2}\left(\frac{1}{n}\sum_{j\in\sI_1}A_{ \mathbf e_j, i, \mu}(w_t)\right)$ is of the forms
%\[\frac{1}{N^2}\sum_{j_1,j_2} G_{j_1\mathbf w_i} G_{\mathbf w_\mu j_1} G_{j_2 \mathbf w_i} G_{\mathbf w_\mu\mathbf w_\mu} G_{\mathbf w_i j_2}\quad \text{ or }\quad \frac{1}{N^2}\sum_{j_1,j_2} G_{j_1\mathbf w_i} G_{\mathbf w_\mu j_1} G_{j_2\mathbf w_i} G_{\mathbf w_\mu\mathbf w_i} G_{\mathbf w_\mu j_2},\]
%or an expression obtained from one of these terms by exchanging $\mathbf w_i$ and $\mathbf w_\mu$. These two expressions can be written as 
%\be\label{2terms}
%N^{-2}( G^{\times 2} )_{\mathbf w_\mu\mathbf w_i}(G^{\times 2})_{\mathbf w_i\mathbf w_i} G_{\mathbf w_\mu\mathbf w_\mu}, \quad N^{-2}( G^{\times 2})^2_{\mathbf w_\mu\mathbf w_i} G_{\mathbf w_\mu\mathbf w_i}, \quad G^{\times 2}:= G \begin{pmatrix}I_{\mathcal I_1 \times \mathcal I_1} & 0\\ 0 & 0\end{pmatrix} G.
%\ee
%For the second term, using \eqref{green2}, \eqref{spectral1} and recalling that $Y=\Sig^{1/2} U^{*}X V\tilde \Sig^{1/2}$, we can get that
%\begin{align}
%& \left|\frac{1}{N^2}\sum_{i,\mu} ( G^{\times 2})^2_{\mathbf w_\mu\mathbf w_i} G_{\mathbf w_\mu\mathbf w_i}\right| \le \frac{1}{N^2}\sum_{i,\mu} \left|( G^{\times 2})_{\mathbf w_\mu\mathbf w_i} \right|^2 = \frac{|z|^2}{N^2}\text{Tr}\left[(\mathcal G_1^{*})^2 YY^\top (\mathcal G_1)^2\right] \nonumber\\
%& =  \frac{|z|^2}{N^2}\text{Tr}\left[\mathcal G_1^{*} (\mathcal G_1)^2\right]  +  \frac{\bar z |z|^2}{N^2}\text{Tr}\left[(\mathcal G_1^{*})^2 (\mathcal G_1)^2\right]  \lesssim \frac{1}{N^2}\sum_k \frac{1}{\left[(\lambda_k-E)^2 +\eta^2\right]^{3/2}}+ \frac{1}{N^2}\sum_k \frac{1}{\left[(\lambda_k-E)^2 +\eta^2\right]^{2}} \nonumber\\
%& \lesssim \frac{1}{N\eta^3}\left(\frac1n \sum_k \frac{\eta}{(\lambda_k-E)^2 +\eta^2} \right) =\frac{\im m}{N\eta^3}\prec  \frac{\im m_c + q+\Psi}{N\eta^3} \lesssim \eta^{-2}\left(\Psi^2 +\frac{q}{N\eta}\right).\label{3term}
%\end{align}
%Using \eqref{aniso_law} and \eqref{eq_comp_r2}, it is easy to show that
%\be \label{3.5term}
%\left|\sum_{\mu}( G^{\times 2} )_{\mathbf w_\mu\mathbf w_i} \Pi_{\mathbf w_\mu\mathbf w_\mu}\right| \prec N^{3/2}\left( \Psi^2 + \frac{q}{N\eta}\right),\quad \text{ and } \quad \left|(G^{\times 2})_{\mathbf x\mathbf y} \right| \prec N\left( \Psi^2 + \frac{q}{N\eta}\right), \ee
%for any deterministic unit vectors $\mathbf x$, $\mathbf y$. Thus for the first term in \eqref{2terms}, we have
%\begin{align}
%\left|\frac{1}{N^2}\sum_{i,\mu}( G^{\times 2} )_{\mathbf w_\mu\mathbf w_i}(G^{\times 2})_{\mathbf w_i\mathbf w_i} G_{\mathbf w_\mu\mathbf w_\mu}\right| & \le \left|\frac{1}{N^2}\sum_{i,\mu}( G^{\times 2} )_{\mathbf w_\mu\mathbf w_i}(G^{\times 2})_{\mathbf w_i\mathbf w_i} \tilde G_{\mathbf w_\mu\mathbf w_\mu}\right| + \left|\frac{1}{N^2}\sum_{i,\mu}( G^{\times 2} )_{\mathbf w_\mu\mathbf w_i}(G^{\times 2})_{\mathbf w_i\mathbf w_i} \Pi_{\mathbf w_\mu\mathbf w_\mu}\right| \nonumber\\
%& \prec N(q+\Psi)\left( \Psi^2 + \frac{q}{N\eta}\right)\left(\frac{1}{N^2}\sum_{i,\mu}\left|( G^{\times 2} )_{\mathbf w_\mu\mathbf w_i}\right|^2\right)^{1/2} +N^{3/2}\left( \Psi^2 + \frac{q}{N\eta}\right)^2 \nonumber\\
%& \prec N\eta^{-1}(q+\Psi)\left( \Psi^2 + \frac{q}{N\eta}\right)^{3/2} +N^{3/2}\left( \Psi^2 + \frac{q}{N\eta}\right)^2,\label{4term}
%\end{align}
%where in the last step we used the bound in \eqref{3term}. Now using \eqref{3term}, \eqref{4term} and \eqref{tildeS}, we get
%$$b_N N^{-2}\left|\sum_{i\in\mathcal I_1}\sum_{\mu\in\mathcal I_2} \prod_{t=1}^{2}\left(\frac{1}{n}\sum_{j\in\sI_1}A_{ \mathbf e_j, i, \mu}(w_t)\right)\right| \prec \left(q^2+\Psi^2\right)^{2} + \left( \frac{N^{-\e/2}}{N\eta}\right)^2 .$$%\left( q^2 + \Psi^2 + \frac{N^{-\e/2}}{N\eta}\right)^2.$$
%
%Finally, in the $l=3$ case, $\prod_{t=1}^{3}\left(\frac{1}{N}\sum_{j\in\sI_1}A_{ \mathbf e_j, i, \mu}(w_t)\right)$ is of the form 
%${N^{-3}}( G^{\times 2})^3_{\mathbf w_i\mathbf w_\mu}$, or an expression obtained by exchanging $\mathbf w_i$ and $\mathbf w_\mu$ in some of the three factors. Using \eqref{3.5term} and the bound in \eqref{3term}, we can estimate that %Using $\|\mathcal G_2\|\prec 1$, we can estimate $\left|\sum_i( G_2)^3_{\mathbf v_i\mu}\right|\prec 1$. Therefore 
%$$\frac{1}{N^3}\left|\sum_{i,\mu}( G^{\times 2})^3_{\mathbf w_i\mathbf w_\mu}\right| \prec \left( \Psi^2 + \frac{q}{N\eta}\right)\frac{1}{N^2}\sum_{i,\mu}\left|( G^{\times 2} )_{\mathbf w_\mu\mathbf w_i}\right|^2 \prec \eta^{-2}\left(\Psi^2 +\frac{q}{N\eta}\right)^2,$$
%Then the LHS of (\ref{eq_comp_goalAvg_genX}) is bounded by 
%$$O_\prec\left(\left(q^2 + \Psi^2\right) \left(\frac{N^{-\e/2}}{N\eta}\right)^2\right).$$
%
%Combining the above three cases, we conclude \eqref{comp_avg_geX_self-improving-bound}, which finishes the proof of \eqref{aver_in1} and \eqref{aver_out1}. %under \eqref{assm_3moment}.

%If $A$ or $B$ is diagonal, then by the remark at the end of Section \ref{subsec_3moment}, the anisotropic local law \eqref{aniso_law} holds for all $z\in S(c_0,C_0,\e)$ even in the case with $b_N=N^{1/2}$ in \eqref{assm_3moment}. Then with \eqref{aniso_law} and the self-consistent comparison argument in \cite[Section 9]{Anisotropic}, we can prove \eqref{aver_in1} and \eqref{aver_out1} for $z\in S(c_0,C_0,\e)$. Again most of the arguments are the same as the ones in \cite[Section 9]{Anisotropic}, hence we omit the details. 

\subsubsection{Proofs of the Limiting Equations}\label{sec contract}
Finally, we give the proof of Lemma \ref{lem_mbehaviorw} and Lemma \ref{lem_stabw} using the contraction principle. 
 %We first prove the existence and continuity of the solutions to \eqref{falvww}. 
 \begin{proof}[Proof of Lemma \ref{lem_mbehaviorw}]
 One can check that the equations in \eqref{selfomega} are equivalent to the following ones:
\begin{equation}\label{selfalter}
r_1m_{2c}=-(1-\gamma_n) - r_2m_{3c} - z\left( m_{3c}^{-1}+1\right),\quad g_z(m_{3c}(z))=1, 
\ee
where
$$g_z(m_{3c}):= - m_{3c} +\frac1n\sum_{i=1}^p \frac{m_{3c} }{  z -\lambda_i^2(1-\gamma_n)+ (1 - \lambda_i^2) r_2m_{3c} - \lambda_i^2 z\left(  m_{3c}^{-1}+1\right) }.$$
We first show that there exists a unique solution $m_{3c}(z)$ to the equation $g_z(m_{3c}(z))=1$ under the conditions in \eqref{prior1}, and the solution satisfies \eqref{Lipomega}. Now we abbreviate $\e(z):= m_{3 c}(z) - m_{3 c}(0)$, and from \eqref{selfalter} we obtain that 
\begin{equation} \nonumber
0=\left[g_z(m_{3c}(z)) -  g_0(m_{3c}(0)) -g_z'(m_{3c}(0))\e(z)\right] + g_z'(m_{3c}(0))\e(z),
\ee
which implies
\be\nonumber%\label{selfomega1}
 \e(z) =- \frac{ g_z(m_{3c}(0)) - g_0(m_{3c}(0)) }{g_z'(m_{3c}(0))}- \frac{ g_z(m_{3c}(0)+\e(z)) -  g_z(m_{3c}(0))-g_z'(m_{3c}(0))\e(z)}{g_z'(m_{3c}(0))}.
 \ee
%By \eqref{selfomega}, we obtain the self-consistent equations for $(m_{2c}(0),m_{3c}(0))$ and $(m_{2c}(z),m_{3c}(z))$:
% \begin{equation}\label{selfomega1}
%\begin{split}
%& \frac{r_1}{m_{2c}(0) }=- 1 + \frac1n\sum_{i=1}^p \frac{\lambda_i^2}{\lambda_i^2m_{2c}(0) + m_{3c} (0) },\quad \frac{r_2}{m_{3c} (0)}=  - 1 +\frac1n\sum_{i=1}^p \frac{1 }{ \lambda_i^2 m_{2c} (0)+  m_{3c}(0)  }.
%\end{split}
%\ee
%Subtract \eqref{selfomega1} from \eqref{selfomega}, we get that
% \begin{equation}\label{selfomega3}
%\begin{split}
%& \e_{2} \frac{r_1}{(m_{2c} +\e_2) m_{2c}}=  \frac1n\sum_{i=1}^p \frac{\lambda_i^2 (z+\lambda_i^2 \e_2 +\e_3)}{(\lambda_i^2m_{2c}  + m_{3c})(z+\lambda_i^2(m_{2c}+\e_2)  + (m_{3c}+\e_3) ) },\\
%&  \e_3 \frac{r_2}{(m_{3c}+\e_3) m_{3c}}=  \frac1n\sum_{i=1}^p \frac{z+\lambda_i^2 \e_2 +\e_3}{(\lambda_i^2m_{2c}  + m_{3c})(z+\lambda_i^2(m_{2c}+\e_2)  + (m_{3c}+\e_3) ) }.
%\end{split}
%\ee
Inspired by this equation, we define iteratively a sequence ${\e}^{(k)} \in \C$ such that ${\e}^{(0)}=0$, and 
\be\label{selfomega2}
 \e^{(k+1)} =- \frac{g_z(m_{3c}(0)) - g_0(m_{3c}(0))}{g_z'(m_{3c}(0))} -\frac{g_z(m_{3c}(0)+ \e^{(k)}) -  g_z(m_{3c}(0))-g_z'(m_{3c}(0))\e^{(k)} }{g_z'(m_{3c}(0))} .
 \ee
% \begin{equation}\nonumber%\label{selfomega4}
%\begin{split}
%& \left\{\frac{r_1}{m_{2c}^2  } - \frac1n\sum_{i=1}^p \frac{\sigma_i^4   }{(\lambda_i^2m_{2c}  + m_{3c})(z+\lambda_i^2 m_{2c}  + m_{3c} ) } \right\} \e^{(k+1)}_{2}  -  \frac1n\sum_{i=1}^p \frac{\lambda_i^2   }{(\lambda_i^2m_{2c}  + m_{3c})(z+\lambda_i^2 m_{2c}  + m_{3c} ) } \e_3^{(k+1)} \\
%& = \frac1n\sum_{i=1}^p \frac{\lambda_i^2 z}{(\lambda_i^2m_{2c}  + m_{3c})(z+\lambda_i^2m_{2c}  + m_{3c} ) } + \frac{r_1 [\e_2^{(k)}]^2}{m_{2c}^2 (m_{2c}+\e_2^{(k)})} \\
%&+   \frac1n\sum_{i=1}^p \left( \frac{\lambda_i^2 (z+\lambda_i^2 \e_2^{(k)} +\e_3^{(k)})}{(\lambda_i^2m_{2c}  + m_{3c})(z+\lambda_i^2(m_{2c}+\e_2^{(k)})  + (m_{3c}+\e_3^{(k)}) ) } - \frac{\lambda_i^2 (z+\lambda_i^2 \e_2^{(k)} +\e_3^{(k)})}{(\lambda_i^2m_{2c}  + m_{3c})(z+\lambda_i^2m_{2c}  + m_{3c}) }\right),\\
%& \left\{\frac{r_2}{m_{3c}^2  } - \frac1n\sum_{i=1}^p \frac{1 }{(\lambda_i^2m_{2c}  + m_{3c})(z+\lambda_i^2 m_{2c}  + m_{3c} ) } \right\} \e^{(k+1)}_{3}  -  \frac1n\sum_{i=1}^p \frac{1 }{(\lambda_i^2m_{2c}  + m_{3c})(z+\lambda_i^2 m_{2c}  + m_{3c} ) } \e_2^{(k+1)} \\
%& = \frac1n\sum_{i=1}^p \frac{ z}{(\lambda_i^2m_{2c}  + m_{3c})(z+\lambda_i^2m_{2c}  + m_{3c} ) } + \frac{r_2 [\e_3^{(k)}]^2}{m_{3c}^2 (m_{3c}+\e_3^{(k)})} \\
%&+   \frac1n\sum_{i=1}^p \left( \frac{ z+\lambda_i^2 \e_2^{(k)} +\e_3^{(k)}}{(\lambda_i^2m_{2c}  + m_{3c})(z+\lambda_i^2(m_{2c}+\e_2^{(k)})  + (m_{3c}+\e_3^{(k)}) ) } - \frac{ z+\lambda_i^2 \e_2^{(k)} +\e_3^{(k)}}{(\lambda_i^2m_{2c}  + m_{3c})(z+\lambda_i^2m_{2c}  + m_{3c}) }\right).
%\end{split}
%\ee
Then \eqref{selfomega2} defines a mapping $h:\C\to \C$, which maps $\e^{(k)}$ to $\e^{(k+1)}=h(\e^{(k)})$.
 
%\be\label{iteration} 
%{\e}^{(k+1)}= \mathbf f({ \e}^{(k)}), \quad \mathbf f(\mathbf x):=S^{-1}\mathbf x_0+ S^{-1} \mathbf e(\mathbf x),
%\ee
%where $S$ is a $2\times 2$ matrix with
%\begin{align*}
%S_{11}=\frac{r_1}{m_{2c}^2  } - \frac1n\sum_{i=1}^p \frac{\sigma_i^4   }{(\lambda_i^2m_{2c}  + m_{3c})(z+\lambda_i^2 m_{2c}  + m_{3c} ) }, \quad S_{12}=-  \frac1n\sum_{i=1}^p \frac{\lambda_i^2   }{(\lambda_i^2m_{2c}  + m_{3c})(z+\lambda_i^2 m_{2c}  + m_{3c} ) },\\
%S_{21}=-  \frac1n\sum_{i=1}^p \frac{1 }{(\lambda_i^2m_{2c}  + m_{3c})(z+\lambda_i^2 m_{2c}  + m_{3c} ) } , \quad S_{22}=\frac{r_2}{m_{3c}^2  } - \frac1n\sum_{i=1}^p \frac{1 }{(\lambda_i^2m_{2c}  + m_{3c})(z+\lambda_i^2 m_{2c}  + m_{3c} ) } ,
%\end{align*}
%%$$
%%S:=\begin{pmatrix} \frac{c_1}{m_{1c}^2  } -  \frac{ \theta_l^2 (1-\theta_l )^2 }{ (1-t_l)^2}g(m_{2c})^2  & -  \frac{ (1-\theta_l )^2\theta_l }{ (1-t_l)^2} \\ -   \frac{ (1-\theta_l )^2\theta_l }{ (1-t_l)^2}  &\frac{c_2}{m_{2c}^2  } -  \frac{ \theta_l^2(1-\theta_l )^2 }{ (1-t_l)^2}g(m_{1c})^2  \end{pmatrix},  
%%$$
%$\mathbf x_0$ is a vector with
%$$\mathbf x_0=\begin{pmatrix} \frac1n\sum_{i=1}^p \frac{\lambda_i^2 z}{(\lambda_i^2m_{2c}  + m_{3c})(z+\lambda_i^2m_{2c}  + m_{3c} ) } \\ \frac1n\sum_{i=1}^p \frac{ z}{(\lambda_i^2m_{2c}  + m_{3c})(z+\lambda_i^2m_{2c}  + m_{3c} ) }  \end{pmatrix},$$
%and $\mathbf e(\mathbf x)$ is
%$$\mathbf e(\mathbf x):= \begin{pmatrix}\frac{r_1 [\e_2^{(k)}]^2}{m_{2c}^2 (m_{2c}+\e_2^{(k)})} +   \frac1n\sum_{i=1}^p \left( \frac{\lambda_i^2 (z+\lambda_i^2 \e_2^{(k)} +\e_3^{(k)})}{(\lambda_i^2m_{2c}  + m_{3c})(z+\lambda_i^2(m_{2c}+\e_2^{(k)})  + (m_{3c}+\e_3^{(k)}) ) } - \frac{\lambda_i^2 (z+\lambda_i^2 \e_2^{(k)} +\e_3^{(k)})}{(\lambda_i^2m_{2c}  + m_{3c})(z+\lambda_i^2m_{2c}  + m_{3c}) }\right) \\ 
% \frac{r_2 [\e_3^{(k)}]^2}{m_{3c}^2 (m_{3c}+\e_3^{(k)})} +   \frac1n\sum_{i=1}^p \left( \frac{ z+\lambda_i^2 \e_2^{(k)} +\e_3^{(k)}}{(\lambda_i^2m_{2c}  + m_{3c})(z+\lambda_i^2(m_{2c}+\e_2^{(k)})  + (m_{3c}+\e_3^{(k)}) ) } - \frac{ z+\lambda_i^2 \e_2^{(k)} +\e_3^{(k)}}{(\lambda_i^2m_{2c}  + m_{3c})(z+\lambda_i^2m_{2c}  + m_{3c}) }\right) \end{pmatrix}.$$
%%Here we have used $\theta_lg(m_{1c})g( m_{2c} ) = f_c(\theta_l) = t_l$ (recall \eqref{fcz}) to simplify the expressions a little bit.

With direct calculation, one can get the derivative
$$g_z'(m_{3c}(0)) = -1 - \frac1n\sum_{i=1}^p \frac{ \lambda_i^2(1-\gamma_n) - z\left[1- \lambda_i^2 \left(  2m_{3c}^{-1}(0)+1\right)\right]  }{  \left[z -\lambda_i^2(1-\gamma_n)+ (1 - \lambda_i^2) r_2m_{3c}(0) - \lambda_i^2 z\left( m_{3c}^{-1}(0)+1\right)\right]^2 }.$$
%Using \eqref{a23}, 
Then it is easy to check that there exist constants $\wt c, \wt C>0$ depending only on $\tau$ in \eqref{assm32} and \eqref{a23} such that
\be\label{dust}
\left|[g_z'(m_{3c}(0))]^{-1}\right|\le \wt C, \quad   \left|\frac{g_z(m_{3c}(0)) - g_0(m_{3c}(0))}{g_z'(m_{3c}(0))}\right|  \le \wt C|z| ,
\ee
and 
\be\label{dust222}
\left|\frac{g_z(m_{3c}(0)+ \e_1) -  g_z(m_{3c}(0)+\e_2)-g_z'(m_{3c}(0))(\e_1-\e_2) }{g_z'(m_{3c}(0))}\right|  \le \wt C|\e_1-\e_2|^2 ,
\ee
for all $|z|\le \wt c$ and $|\e_1|  \le \wt c$, $|\e_2|  \le \wt c$. Then with \eqref{dust} and \eqref{dust222}, it is easy to see that there exists a sufficiently small constant $\delta>0$ depending only on $\wt C$, such that $h$ is a self-mapping 
$$  h: B_r  \to B_r , \quad B_r:=\{\e \in \C: |\e| \le r \},$$
as long as $r\le \delta$ and $|z| \le c_\delta$ for some constant $c_\delta>0$ depending only on $\wt C$ and $\delta$. 
%\be\label{priori_cond}
%\zeta+\| \b g\|_\infty+ |z-\wt z|\le c_r
%\ee
%for some constant $ c_r>0$ depending on $r$. 
Now it suffices to prove that $h$ restricted to $B_r $ is a contraction, which then implies that ${\e}:=\lim_{k\to\infty} { \e}^{(k)}$ exists and $m_{3c}(0)+\e$ is a unique solution to the second equation of \eqref{selfalter} subject to the condition $\|{\e}\|_\infty \le r$. 


From the iteration relation \eqref{selfomega2}, using \eqref{dust} one can readily check that
\be\label{k1k}
{ \e}^{(k+1)} - { \e}^{(k)}= h({\e}^{(k)}) - h({\e}^{(k-1)}) \le \wt C | { \e}^{(k)}-{ \e}^{(k-1)}|^2.
\ee
Hence as long as $r$ is chosen to be sufficiently small such that $2r\wt C\le 1/2$, then 
%compared to $\theta_l^{-1}-g(m_{1c})g( m_{2c} )= (1-t_l)\theta_l^{-1}$, we have
%%where $q(\bx)$ denotes a vector with components $q(x_i)$. 
%%Using $|q'(0)| = 0$ and \eqref{dust}, we get from \eqref{k1k} that
%$$
% \left\|\mathbf e({{} \e}^{(k)}) -\mathbf e({{} \e}^{(k-1)})\right\|_\infty \le C (\|\bx^{k }\|_\infty +\|\bx^{k-1 }\|_\infty)\|{{} \e}^{(k)} - {{} \e}^{(k-1)}\|_\infty
% %\|\bx^{k+1}-\bx^k\|_\infty\le  C_\kappa \left(\zeta+\|\bx^{k }\|_\infty +\|\bx^{k-1 }\|_\infty  \right)\cdot \|\bx^{k }-\bx^{k-1}\|_\infty
%$$
%for some constant $C >0$ depending only on $c_1, c_2$ and $\delta_l$. Thus we can choose a sufficiently small constant $0<r \le \min\{\tau, (2C)^{-1}\}$ such that $Cr \le 1/2$, i.e. 
$h$ is indeed a contraction mapping on $ B_r$, which proves both the existence and uniqueness of the solution $m_{3c}(z)=m_{3c}(0)+\e$, if we choose $c_0$ in \eqref{prior1} as $c_0=\min\{c_\delta, r\}$. After obtaining $m_{3c}(z)$, we can then find $m_{2c}(z)$ using the first equation in \eqref{selfalter}. 

Note that with \eqref{dust222} and ${\e}^{(0)}= 0$, we get from \eqref{selfomega2} that $ |{ \e}^{(1)}| \le \wt C|z| .$
With the contraction mapping, we have the bound 
\be\label{endalter}|{ \e}| \le \sum_{k=0}^\infty |{  \e}^{(k+1)}-{ \e}^{(k)}| \le 2\wt C|z|.\ee
This gives the bound \eqref{Lipomega} for $m_{3c}(z)$. Using the first equation in \eqref{selfalter}, we immediately obtain the bound  $ r_1|m_{2c}(z)-m_{2c}(0)| \le C|z|.$ This gives \eqref{Lipomega} for $m_{2c}(z)$ as long as if $r_1\gtrsim 1$. To deal with the small $r_1$ case, we go back to the first equation in \eqref{selfomega} and treat $m_{2c}(z)$ as the solution to the following equation:
$$\wt g_z(m_{2c}(z))=1,\quad \wt g_z(x):=- x + \frac{\gamma_n}p\sum_{i=1}^p \frac{\lambda_i^2 x}{  z+\lambda_i^2 r_1 x +r_2 m_{3c}(z) }. $$
%We can calculate that 
%$$g_z'(m_{2c}(0))= -1 +  \frac{\gamma_n}p\sum_{i=1}^p \frac{\lambda_i^2 (z + r_2m_{3c}(z))}{  (z+\lambda_i^2 r_1 m_{2c}(0) +r_2 m_{3c}(z))^2 }.$$
%At $z=0$, we have 
%$$ |g_0'(m_{2c}(0))|= \left|1+\frac{\gamma_n}p\sum_{i=1}^p \frac{\lambda_i^2  r_2 x_3}{  (\lambda_i^2 r_1 x_2 + r_2 x_3)^2 }\right|\ge 1,$$
%where $x_2$ and $x_3$ satisfy \eqref{a23}. Thus under \eqref{prior1} we have $|g_z'(m_{2c}(0))|\sim 1$ as long as $c_0$ is taken sufficiently small. 
Then with similar arguments as above between \eqref{selfalter} and \eqref{endalter}, we can conclude \eqref{Lipomega} for $m_{2c}(z)$. 
%$c_0$ is taken sufficiently small. 
This concludes the proof of Lemma \ref{lem_mbehaviorw}.
\end{proof}

\begin{proof}[Proof of Lemma \ref{lem_stabw}]
Under \eqref{prior1}, we can obtain equation \eqref{selfalter} approximately up to some small error
\be\label{selfalter2}r_1 m_{2c}=-(1-\gamma_n) - r_2m_{3c} - z\left(  {m_{3c}^{-1}}+1\right) + \cal E'_2(z),\quad g_z(m_{3c}(z))=1+ \cal E'_3(z),\ee
 with $|\cal E'_2(z)|+ |\cal E'_3(z)|=\OO(\delta(z))$. Then we subtract the equations \eqref{selfalter} from \eqref{selfalter2}, and consider the contraction principle for the functions $\e (z):= m_{3}(z) - m_{3 c}(z)$.  The rest of the proof is exactly the same as the one for Lemma \ref{lem_mbehaviorw}, so we omit the details.
\end{proof}

