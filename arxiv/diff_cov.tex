\section{The Different Covariates Setting}

When the covariates across tasks are different, we can no longer obtain an optimal solution for minimizing equation \eqref{eq_mtl} as in Claim \ref{lem_exp_opt}.
This is because the minimization problem is non-convex with respect to $A$ and $B$.
Therefore, we consider an arbitrary local minimum of equation \eqref{eq_mtl} and provide generalization bounds that scale with key properties of task data such as sample sizes and covariate shifts.
We focus on the two-task case to better understand the impact of having different sample sizes and different covariates in hard parameter sharing.
We will develop the key technical results in this section, and describe their implications in the next section.

We first provide a key technical tool that we derived using recent developments in random matrix theory \cite{??}.
To motivate our study, we consider a specialized setting where the output layer weights $A_1, A_2$ are both equal to one.
Without loss of generality, we consider the prediction loss of hard parameter sharing for task two and the same analysis applies to task one.
By solving $B$ in equation \eqref{eq_mtl}, we obtain the hard parameter sharing estimator $\hat{\beta}_2^{\MTL} = B A_2 = B$ as follows:
\begin{align*}
	\hat{\beta}_2^{\MTL} = {\hat{\Sigma}}^{-1} ({X^{(1)}}^{\top} Y^{(1)} + {X^{(2)}}^{\top} Y^{(2)}), \text{ where } \hat{\Sigma} = {X^{(1)}}^{\top} X^{(1)} + {X^{(2)}}^{\top} X^{(2)}.
\end{align*}
For the prediction loss, the empirical loss is close to the expected loss similar to what we have shown in Claim \ref{claim_pred_err}.
Hence, we consider the expected loss conditional on the covariates as follows:
\begin{align*}
	\exarg{\cE}{L(\hat{\beta}_2^{\MTL}) \mid X^{(1)}, X^{(2)}} =
	\bignorm{(\Sigma^{(2)})^{1/2} \hat{\Sigma}^{-1} {X^{(1)}}^{\top} X^{(1)} (\beta^{(1)} - \beta^{(2)})}^2
	+ \sigma^2 \bigtr{\Sigma^{(2)}\hat{\Sigma}^{-1}}.
\end{align*}

Interestingly, the prediction loss admits a bias-variance decomposition similar to the same covariates setting.
In both the bias and variance equations, the matrix inverse $\hat{\Sigma}^{-1}$ is a crucial quantity.
Intuitively, this matrix inverse combines the sample covariance matrix of both tasks, and it is not hard to see that the expectation of $\hat{\Sigma}$ is equal to a mixture of their population covariance matrices, with mixing proportions determined by the sample sizes.
However, since the two tasks have different covariates, the scaling of $\hat{\Sigma}^{-1}$ not only depends on the tasks' sample sizes, but also depends on the covariate shift between the two tasks.
Building on the intuition, We introduce a key result that shows tight asymptotic convergence rate of the bias and variance as $p$ goes to infinity.
We define a key quantity $M \define \Sigma_1^{1/2}\Sigma_2^{-1/2}$ that captures the covariate shift between task $1$ and task $2$.
Let $U\Lambda V^\top$ denote the singular value decomposition of $M$, where $\Lambda\in\real^{p\times p}$ consists of $M$'s singular values $\lambda_1, \lambda_2, \dots, \lambda_p$ as its diagonal entries in descending order.

\begin{theorem}\label{thm main RMT}
	Suppose that $X_1$ and $X_2$ are both random matrices that satisfy Assumption \ref{assume_rm}.
	With high probability over the randomness of $X_1$ and $X_2$, we have the following limits:
	\begin{enumerate}
		\item[i)]  The variance equation converges to the following asymptotic limit
			\begin{align}\label{lem_cov_shift_eq}
				\bigabs{\bigtr{\Sigma^{(2)} \bigbrace{(n_1 + n_2){\hat{\Sigma}^{-1}} - (a_1 \Sigma^{(1)} + a_2 \Sigma^{(2)})^{-1}}}} \lesssim p^{-c_{\varphi}}\norm{\Sigma^{(2)}},
			\end{align}
			where $a_1$ and $a_2$ are the solutions of the following two equations:
			\begin{align}
				a_1 + a_2 = 1- \frac{1}{\rho_1 + \rho_2},\quad a_1 + \frac1{\rho_1 + \rho_2}\cdot \bigbrace{\frac{1}{p}\sum_{i=1}^p \frac{\lambda_i^2 a_1}{\lambda_i^2 a_1 + a_2}} = \frac{\rho_1}{\rho_1 + \rho_2}. \label{eq_a12extra}
			\end{align}
		\item[ii)] For any fixed vector $w\in\real^p$ with unit norm, the bias equation converges to the following asymptotic limit
			\begin{align}\label{def Pibias}
				\bigabs{w^{\top} \Sigma^{(1)} \bigbrace{(n_1+n_2)^2\hat{\Sigma}^{-1} \Sigma^{(2)} \hat{\Sigma}^{-1} - {\Sigma^{(2)}}^{-1/2} V {\frac{a_3 \Lambda^2 + (a_4 + 1)\id}{(a_1 \Lambda + a_2\id)^2}} V^{\top} {\Sigma^{(2)}}^{-1/2}} \Sigma^{(1)} w} \lesssim p^{-c_{\varphi}},
			\end{align}
				where $a_{3}$ and $a_4$ are the solutions of the following two equations with $b_k = \frac1{p}\sum_{i=1}^p \frac{\lambda_i^{2k}} {(\lambda_i^2 a_1 + a_2)^2}$, for $k = 0, 1, 2$:
			\begin{align}\label{eq_a34extra}
				\left(\frac{\rho_1}{a_1^{2}} -  b_2  \right)\cdot  a_3 -  b_1 \cdot  a_4 = b_1,\quad \left(\frac{\rho_2}{a_2^{2}}-  b_0\right)\cdot  a_4 - b_1 \cdot  a_3
				= b_0.
			\end{align}
	\end{enumerate}
\end{theorem}

Lemma \ref{lem_cov_shift} also implies the asymptotic limit of the prediction loss of single-task learning.
%The next lemma, which is , helps to determine the asymptotic limit of $\te(\hat{\beta}_t^{\STL})=\sigma^2   \bigtr{(X_2^{\top}X_2)^{-1}\Sigma_2}$ as $p\to \infty$.
In particular, by setting $X_1 = 0$, we obtain Lemma \ref{lem_minv}.
%\begin{corollary}[See e.g. Theorem 2.4 in \citet{isotropic}]\label{lem_minv}
%	In the setting of Lemma \ref{lem_cov_shift}, with high probability we have that
%	\be\label{XXA}  \bigtr{(X_2^{\top}X_2)^{-1}\Sigma} = \frac{1}{n_2 - p} \cdot \bigtr{\Sigma_2^{-1}\Sigma} + o(\norm{\Sigma}). \ee
%\end{corollary} %\bigo{ \|A\|p^{-1/2+\epsilon}}
To see this, note that when $n_1=0$, $a_1 = 0$ and $a_2 = (n_2-p) / n_2$ is the solution of \eqref{eq_a12extra}, and one can see that equation \eqref{lem_cov_shift_eq} reduces to equation \eqref{XXA}.
We briefly describe the history behind the above well-known result.
When the entries of $X_2$ are multivariate Gaussian, this result recovers the classical result for the mean of inverse Wishart distribution \cite{anderson1958introduction}.
For general non-Gaussian random matrices, it can be obtained using Stieltjes transform method; see e.g., Lemma 3.11 of \cite{bai2009spectral}.
Here we have stated a result from Theorem 2.4 in \cite{isotropic}, which gives an almost sharp concentration error bound.
One can see that our result extends Lemma \ref{lem_minv} from a single sample covariance matrix to the sum of two independent sample covariance matrices.
%We now state our main result for two tasks with both covariate and model shift. It shows that the information transfer is determined by two deterministic quantities $\Delta_{\bias}$ and $\Delta_{\vari}$, which give the change of model shift bias and the change of variance, respectively. The exact forms of $\Delta_{\bias}$ and $\Delta_{\vari}$ will be given in Lemma \ref{thm_model_shift}.

Interestingly, the asymptotic limit of the variance only depends on the sample sizes and the covariate shift matrix.
Next, we derive the asymptotic limit of the bias equation \eqref{eq_bias_2task}, which depends on the scaled model distance in addition to sample size and covariate shift. %(cf. Lemma \ref{lem_cov_derivative} in Appendix \ref{app_proof_main_thm}).


In the rest of section, we describe briefly the main ideas in the proof of Lemma \ref{lem_cov_shift} and Lemma \ref{lem_cov_derivative}. In particular, we will give a simple (although not totally rigorous) derivation of the equations \eqref{eq_a12extra} and \eqref{eq_a34extra}.
The full proof of Lemma \ref{lem_cov_shift} and Lemma \ref{lem_cov_derivative} will be given in Section \ref{appendix RMT}.
%\subsection{The Bias and Variance Asymptotics}

%\todo{merge} Finally, in this subsection, we describe briefly the main ideas in the proofs of Lemmas \ref{lem_cov_shift} and \ref{lem_cov_derivative}.
%We first describe the proof of Theorem \ref{lem_cov_shift_informal}.
%We briefly describe the key ideas for proving Lemma \ref{lem_cov_shift} and Lemma \ref{lem_cov_derivative}.

\paragraph{Preliminaries.} Both Lemma \ref{lem_cov_shift} and Lemma \ref{lem_cov_derivative} require a study of the matrix inverse $[(X^{(1)})^\top X^{(1)}+(X^{(2)})^\top X^{(2)}]^{-1}$ for $X^{(1)}= Z^{(1)}\Sigma_1^{1/2}$ and $X^{(2)}= Z^{(2)}\Sigma_2^{1/2}$.
Then using equation \eqref{eigen2}, we can write
\be\label{eigen2extra}[(X^{(1)})^\top (X^{(1)})+(X^{(2)})^\top (X^{(2)})]^{-1}= \Sigma_2^{-1/2}V\left(   \Lambda U^\top (Z^{(1)})^\top Z^{(1)} U\Lambda  + V^\top (Z^{(2)})^\top Z^{(2)}V\right)^{-1}V^\top\Sigma_2^{-1/2}.\ee
%We divide the discussion into the following parts.
To study this expression, we shall use the standard Stieltjes transform method (or the resolvent method) in random matrix theory \cite{bai2009spectral,tao2012topics,erdos2017dynamical}. We now give some preliminaries about this method before discussing the proof strategy.

For any measure $\mu$ supported on $[0,\infty)$, it defines a Stieltjes transform through
$$m_\mu(z):= \int_0^\infty \frac{\dd\mu(x)}{x-z},\quad z\in \C\setminus [0,\infty).$$
Conversely, given a Stieltjes transform $m_{\mu}(z)$, we can recover the measure $\mu$ through the inverse formula
$$ \rho(x)=\frac{1}{\pi}\lim_{\eta \downarrow 0}\im m_{\mu}(x+\ii \eta),$$
where $\rho(x)$ is the density for $\mu$, i.e. $\dd \mu(x)=\rho(x)\dd x$. Hence the Stieltjes transform method reduces the study of a measure $\mu$ to the study of an analytic function $m_\mu(z)$. Let $\mu=p^{-1}\sum_{i=1}^p \delta_{\sigma_i}$ be the empirical spectral distribution of the matrix $Y_p:=(n_1+n_2)^{-1}(\Lambda U^\top (Z^{(1)})^\top Z^{(1)} U\Lambda  + V^\top (Z^{(2)})^\top Z^{(2)}V)$, where $\sigma_i$ denote the eigenvalues of $Y_p$, and the scaling $(n_1+n_2)^{-1}$ is chosen such that $\pi$ is supported on a bounded interval. Then with the above notations, the Stieltjes transform of the empirical spectral distribution $\mu$ is
$$ m_{\mu}=\frac{1}{p}\sum_{i=1}^p \frac{1}{\sigma_i - z}= p^{-1}\tr\left[ (Y_p-z\id_p)^{-1}\right].$$
The Stieltjes transform method in random matrix theory aims to show that $m_{\mu}$ satisfies a fixed-point equation asymptotically, which allows us to derive the asymptotic limit of $\mu$.


However, for our purpose, we not only need to know the limit of the trace $\tr\left[ (Y_p-z\id_p)^{-1}\right]$, but also need to obtain the limit of $\tr\left[ (Y_p-z\id_p)^{-1}B\right]$ for any given deterministic matrix $B$. This  requires us to study the matrix $(Y_p-z\id_p)^{-1}$ directly, which is called resolvent or Green's function. Roughly speaking, the resolvent can be regarded as a matrix version of the Stieltjes transform. The convergence of resolvent can be formulated formally in a slightly more modern language in terms of local laws, see e.g. \cite{isotropic,erdos2017dynamical,Anisotropic}, or deterministic equivalents, see e.g., \cite{Hachem2007deterministic,DS18}. More precisely, given a deterministic $p\times p$ matrix, say $R_p(z)$, of growing dimensions, we say that it is a deterministic equivalent of $(Y_p-z\id_p)^{-1}$ if for any sequence of deterministic matrices $B_p$ of uniformly bounded operator norms,
$$\tr \left\{\left[ (Y_p-z\id_p)^{-1}-R_p(z)\right]B_p\right\}\to 0.$$
If this holds, then we shall say that $(Y_p-z\id_p)^{-1}$ has a matrix limit $R_p$. In this paper, we shall prove a stronger convergence of the resolvent $(Y_p-z\id_p)^{-1}$ in the sense of local laws, which also gives an (almost) sharp convergence rate. Our proof will be centralized around the resolvent and its local law.


\paragraph{Proof overview.} We divide the discussion into the following parts.

\medskip
\noindent\textbf{Defining the resolvents.} For our purpose, we use a convenient linearization trick in linear algebra, that is, the SVD of a rectangular matrix $A$ is equivalent to the study of the eigendecomposition of the symmetric block matrix
$$H(A):=\begin{pmatrix}0 & A \\ A^\top & 0\end{pmatrix},$$
which is linear in $A$. This trick has been used in many random matrix literatures, such as \cite{Anisotropic, AEK_Gram, XYY_circular,DY20201}. Our goal is to study the SVD of the rectangular matrix $(\Lambda U^\top (Z^{(1)})^\top,V^\top (Z^{(2)})^\top)$. %For this purpose,
%}\HZ{what does this trick mean? use less technical words},
%This idea dates back at least to Girko, see e.g., the works \cite{girko1975random,girko1985spectral} and references therein.
Then we define the following $(p+n)\times (p+n)$ symmetric block matrix
%which is a linear function of $(Z^{(1)})$ and $(Z^{(2)})$:
%\begin{definition}[Linearizing block matrix]\label{def_linearHG}%definiton of the Green function
%We define the $(n+N)\times (n+N)$ block matrix
 \begin{equation}\label{linearize_block}
    H\left(\Lambda U^\top (Z^{(1)})^\top,V^\top (Z^{(2)})^\top\right): = n^{-1/2}\left( {\begin{array}{*{20}c}
   { 0 } & \Lambda U^{\top}(Z^{(1)})^\top & V^\top (Z^{(2)})^\top  \\
   {Z^{(1)} U\Lambda  } & {0} & 0 \\
   {Z^{(2)}V} & 0 & 0
   \end{array}} \right).
 \end{equation}
For simplicity of notations, we abbreviate it as $H$ in the following discussion, and define the index sets
$$\cal I_0:=\llbracket 1,p\rrbracket, \quad  \cal I_1:=\llbracket p+1,p+n_1\rrbracket, \quad \cal I_2:=\llbracket p+n_1+1,p+n_1+n_2\rrbracket ,\quad \cal I:=\cal I_0\cup \cal I_1\cup \cal I_2  .$$
 We will consistently use the latin letters $i,j\in\sI_{0}$ and greek letters $\mu,\nu\in\sI_{1}\cup \sI_{2}$. Correspondingly, the indices of the matrices $Z^{(1)}$ and $Z^{(2)}$ are labelled as
 \be\label{labelZ}
 Z^{(1)}= (Z^{(1)}_{\mu i}:i\in \mathcal I_0, \mu \in \mathcal I_1), \quad Z^{(2)}= (Z^{(2)}_{\nu i}:i\in \mathcal I_0, \nu \in \mathcal I_2).\ee
Now we define the resolvents as follows.
\begin{definition}[Resolvents]
We define the resolvent (or Green's function) of $H$ as
 \begin{equation}\label{eqn_defG}
 G (z):= \left[H -\left( {\begin{array}{*{20}c}
   { z\id_{p}} & 0 & 0 \\
   0 & { \id_{n_1}}  & 0\\
      0 & 0  & { \id_{n_2}}\\
\end{array}} \right)\right]^{-1} , \quad z\in \mathbb C ,
 \end{equation}
%It is easy to verify that the eigenvalues $\lambda_1(H)\ge \ldots \ge \lambda_{n+N}(H)$ of $H$ are related to the ones of $\mathcal Q_1$ through
%\begin{equation}\label{Heigen}
%\lambda_i(H)=-\lambda_{n+N-i+1}(H)=\sqrt{\lambda_i\left(\mathcal Q_2\right)}, \ \ 1\le i \le n\wedge N, \quad \text{and}\quad \lambda_i(H)=0, \ \ n\wedge N + 1 \le i \le n\vee N.
%\end{equation}
%and
%$$\lambda_i(H)=0, \ \ n\wedge N + 1 \le i \le n\vee N.$$
%where we used the notations $n\wedge N:=\min\{N,M\}$ and $n\vee N:=\max\{N,M\}$.
%\begin{definition}[Index sets]\label{def_index}
and the resolvent of $  n^{-1}\left(\Lambda U^\top (Z^{(1)})^\top Z^{(1)} U\Lambda  + V^\top (Z^{(2)})^\top Z^{(2)}V\right)$ as
\be\label{mainG}
\cal G(z):=\left( n^{-1}  \Lambda U^\top (Z^{(1)})^\top Z^{(1)} U\Lambda  + n^{-1}V^\top (Z^{(2)})^\top Z^{(2)}V -z\right)^{-1},\quad z\in \C.
\ee
Moreover, we also define the following averaged resolvents, which are the (weighted) partial traces of $G$:
\be\label{defm}
\begin{split}
m(z) :=\frac1p\sum_{i\in \cal I_0} G_{ii}(z) ,\quad & m_0(z):=\frac1p\sum_{i\in \cal I_0} \lambda_i^2 G_{ii}(z),\\
 m_1(z):= \frac{1}{n_1}\sum_{\mu \in \cal I_1}G_{\mu\mu}(z) ,\quad & m_2(z):= \frac{1}{n_2}\sum_{\nu\in \cal I_2}G_{\nu\nu}(z) .
\end{split}
\ee
We will show that these averaged resolvents satisfy some deterministic self-consistent equations asymptotically, which will be the core part of the proof. We refer the reader to the discussions below \eqref{0self_Gii} and \eqref{0self_Gmu1}.
\end{definition}
With Schur complement formula, we can check that %\HZ{Has $G(z)$ been defined before?}
 \begin{equation} \label{green2}
 G (z):=  \left( {\begin{array}{*{20}c}
   { \cal G (z)} & \cal G(z)A  \\
   A^\top \cal G(z) & z (A^\top A - z)^{-1}
\end{array}} \right)^{-1}, %\quad \cal G_R:=(W^\top W - z)^{-1} ,
 \end{equation}
 where we abbreviated
 \be\label{abbr_A}
 A:= n^{-1/2}(\Lambda U^\top (Z^{(1)})^\top, V^\top (Z^{(2)})^\top).
 \ee
 %\HZ{$W$ is overloaded with the output layers}
 This shows that a control of $G(z)$ yields directly a control of $\mathcal G(z)$ as the upper-left block. On the other hand, $G(z)$ is a little  easier to use than $\cal G(z)$, and obviously contains more information.
%one can find that %(recall equation \eqref{mainG})
%\be \label{green2}
%\begin{split}
%& \cal G_{L}=\left(WW^\top -z \right)^{-1} =\cal G ,\quad \cal G_{LR}=\cal G_{RL}^\top  = \cal G W , \quad \cal G_R= z\left(W^\top W - z\right)^{-1}.
%\end{split}
%\ee
%, and the subindex of $\cal G_R$ means the lower-right block.


%Now using Schur complement formula, we can verify that the (recall equation \eqref{def_green})
%\begin{align}
%G = \left( {\begin{array}{*{20}c}
%   { z\mathcal G_1} & \mathcal G_1 \Sig^{1/2} U^{*}X V\tilde \Sig^{1/2}  \\
%   {\tilde\Sig^{1/2}V^\topX^\top U\Sig^{1/2} \mathcal G_1} & { \mathcal G_2 }  \\
%\end{array}} \right) = \left( {\begin{array}{*{20}c}
%   { z\mathcal G_1} & \Sig^{1/2} U^{*}X V\tilde \Sig^{1/2} \mathcal G_2   \\
%   {\mathcal G_2}\tilde\Sig^{1/2}V^\topX^\top U\Sig^{1/2} & { \mathcal G_2 }  \\
%\end{array}} \right). \label{green2}
%\end{align}
%where $\mathcal G_{1,2}$ are defined in (\ref{def_green}).

\medskip
\noindent\textbf{Defining the asymptotic limits of the resolvents.}
%\label{sec aymp_limit_G}
%\subsection{Resolvents and limiting law}
We now describe the asymptotic limit of $G(z)$ as $n\to \infty$. First we define the deterministic limits of $(m_1(z), m_{2}(z))$ by $\left(-\frac{\rho_1+\rho_2}{\rho_1}a_{1}(z),-\frac{\rho_1+\rho_2}{\rho_2}a_{2}(z)\right)$, where $(a_1(z), a_2(z))$ is
%\HZ{This $M_{1}, M_{2}$ notation is awkward; consider changing it to ${M}_1(z), {M}_2(z)$ (or  better, say $a_1(z), a_2(z)$? if they correspond to $a_1, a_2$ when $z=0$)},
the unique solution to the following system of equations
\begin{equation}\label{selfomega_a}
\begin{split}
& \frac{\rho_1}{a_{1}(z)} = \frac{1}{p}\sum_{i=1}^p \frac{\lambda_i^2}{ - z+\lambda_i^2 a_{1}(z) +a_{2} (z) } + (\rho_1+\rho_2),\  \frac{\rho_2}{a_{2}(z)} = \frac{1}{p}\sum_{i=1}^p \frac{1 }{  -z+\lambda_i^2 a_{1}(z) +  a_{2}(z)  }+ (\rho_1+\rho_2) .
\end{split}
\ee
%satisfying that $\im a_{1}(z)< 0$ and $\im a_{2}(z)<0$ for $z\in \C_+$ with $\im z$.
The existence and uniqueness of the solution $(a_1(z), a_2(z))$ for $z$ around 0 will be proved in Section \ref{sec contract}.
%Here, for simplicity of notations, we introduced the following ratios
%\be\label{ratios}
% \gamma_n :=\frac{p}{n}=\frac{1}{\rho_1+\rho_2},\quad r_1 :=\frac{n_1}{n}=\frac{\rho_1}{\rho_1+\rho_2},\quad r_2 :=\frac{n_2}{n}=\frac{\rho_2}{\rho_1+\rho_2}.
%\ee
We define the matrix limit of $G(z)$ as
\be \label{defn_piw}
\Gi(z) := \begin{pmatrix} (-z\id_p+a_{1}(z)\Lambda^2  +  a_{2}(z)\id_p)^{-1} & 0 & 0 \\ 0 & - \frac{\rho_1+\rho_2}{\rho_1} a_{1}(z)\id_{n_1} & 0 \\ 0 & 0 & -\frac{\rho_1+\rho_2}{\rho_2}a_{2}(z)\id_{n_2}  \end{pmatrix}.\ee
In particular, the matrix limit of $\cal G(z)$ is given by
\be\label{matrix limit}(-z\id_p+a_{1}(z)\Lambda^2 + a_{2}(z)\id_p)^{-1}.\ee

\medskip
\noindent{\bf Deriving the variance asymptotics.} Using definition \eqref{mainG}, we can write equation \eqref{eigen2extra} as
\be\label{rewrite X as R} [(X^{(1)})^\top X^{(1)}+(X^{(2)})^\top X^{(2)}]^{-1}=n^{-1}\Sigma_2^{-1/2}V\cal G(0)V^\top\Sigma_2^{-1/2}.\ee
When $z=0$, it is easy to check that equation \eqref{selfomega_a} can be reduced to equation \eqref{eq_a12extra}, which means that we actually have $a_1(0)=a_1$ and $a_2(0)=a_2$. Hence the matrix limit of $\cal G(0)$ is given by $(a_{1}\Lambda^2 + a_{2}\id_p)^{-1}$. Then inserting this limit into equation \eqref{rewrite X as R}, we can write the left-hand side of equation \eqref{lem_cov_shift_eq} as
\begin{align}
&\bigtr{\left( (X^{(1)})^{\top}X^{(1)} + (X^{(2)})^{\top}X^{(2)}\right)^{-1} \Sigma}\approx n^{-1}\bigtr{\Sigma_2^{-1/2}V\cal (a_{1}\Lambda^2 + a_{2}\id_p)^{-1}V^\top\Sigma_2^{-1/2}\Sigma}  \nonumber\\
&=n^{-1}\bigtr{\Sigma_2^{-1/2}\cal (a_{1}\Sigma_2^{-1/2}\Sigma_1\Sigma_2^{-1/2} + a_{2}\id_p)^{-1}\Sigma_2^{-1/2}\Sigma}  = n^{-1}\bigtr{\cal (a_{1} \Sigma_1  + a_{2}\Sigma_2)^{-1}\Sigma}  ,\label{Gi00}
\end{align}
where in the second step we used $V \Lambda^2 V^\top=M^\top M=\Sigma_2^{-1/2}\Sigma_1\Sigma_2^{-1/2}$ by equation \eqref{eigen2}.  This concludes equation \eqref{lem_cov_shift_eq} up to a small error. The rigorous proof of equation \eqref{lem_cov_shift_eq} will be given in Section \ref{sec pf RMTlemma}.

%\alert{Together with equation \eqref{rewrite X as R} and $V \Lambda^2 V^\top=M^\top M=\Sigma_2^{-1/2}\Sigma_1\Sigma_2^{-1/2}$ by equation \eqref{eigen2}, we can obtain equation \eqref{lem_cov_shift_eq} in the large $n$ limit.}\HZ{I do not understand this sentence.}







