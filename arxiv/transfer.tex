\section{Bias-variance Decomposition for the General Case}\label{sec_general}

%\begin{proposition}[Bias-variance tradeoff]
%	Variance always reduces and bias always increases.
%\end{proposition}
In this section, we compare the prediction loss of the multi-task learning estimator to that of the single-task learning estimator.
First, we consider the two-task case.
We provide precise asymptotics of the bias and variance of the multi-task learning estimator.
We apply recent developments from the random matrix theory literature to characterize the bias and variance.
The results scale with key properties of task data such as sample size and covariance shift, and allow us to study the performance of multi-task learning by varying these properties (Section \ref{sec_special}).
Second, we consider the multiple-task case where all tasks have the same features but different labels.
We extend the bias-variance decomposition of the two-task case to this setting and show qualitatively similar results.
%For the single-task learning estimator, there are well-known results that relate its prediction loss to sample size and noise variance.

%We focus on the high-dimensional linear regression setting \cite{}


\iffalse
First, we give the basic assumption for our main objects---the random matrices $X_i$, $i=1,2$.

\begin{assumption}[Moment assumptions]%\label{assm_secA1}
We will consider $n\times p$ random matrices of the form $X=Z\Sigma^{1/2}$, where $\Sigma$  is a $p\times p$ deterministic positive definite symmetric matrix, and $Z=(z_{ij})$ is an $n\times p$ random matrix with real i.i.d. entries with mean zero and variance one. Note that the rows of $X$ are i.i.d. centered random vectors with covariance matrix $\Sigma$. For simplicity, we assume that all the moments of $z_{ij}$ exists, that is, for any fixed $k\in \N$, there exists a constant $C_k>0$ such that
\begin{equation}\label{assmAhigh}
\mathbb{E} |z_{ij}|^k \le C_k ,\quad 1\le i \le n, \ \ 1\le j \le p.
\end{equation}
 We assume that $n=\rho p$ for some fixed constant $\rho>1$. Without loss of generality, after a rescaling we can assume that the norm of $\Sigma$ is bounded by a constant $C>0$. Moreover, we assume that $\Sigma$ is well-conditioned: $\kappa(\Sigma)\le C$, where $\kappa(\cdot)$ denotes the condition number.
\end{assumption}
Here we have assumed \eqref{assmAhigh} solely for simplicity of representation. If the entries of $Z$ only have finite $a$-th moment for some $a>4$, then all the results below still hold except that we need to replace $\OO(p^{-\frac12+\e})$ with $\OO( p^{-\frac12+\frac2a +\epsilon})$ in some error bounds.
We will not get deeper into this issue in this section, but refer the reader to Corollary \ref{main_cor} in Section \ref{sec locallaw1}.
\fi


\subsection{Two-task Case}\label{sec_two}



As an example, for the setting of two tasks, we can decompose $L(\hat{\beta}_t^{\MTL}) - L(\hat{\beta}_t^{\STL})$ into a bias term and a variance term as follows (derived in Appendix \ref{app_proof_sec3}).
%Recall that $\hat{\beta}_t^{\MTL}$ is defined as $BW_t$ after solving equation \eqref{eq_mtl}.
%We can  the test error of $\hat{\beta}_{t}^{\MTL}$ on the target task into two parts as follows.
{\small\begin{align}
	\te(\hat{\beta}_t^{\MTL}) - L(\hat{\beta}_t^{\STL}) =& ~ \hat{v}^2 \bignorm{\Sigma_2^{1/2} (\hat{v}^2 X_1^{\top}X_1 + X_2^{\top}X_2)^{-1}X_1^{\top}X_1 (\beta_1 - \hat{v}\beta_2)}^2 \label{eq_te_model_shift} \\
	+&~ \sigma^2 \bigbrace{\bigtr{(\hat{v}^2 X_1^{\top}X_1 + X_2^{\top}X_2)^{-1}\Sigma_2} - \bigtr{(X_2^{\top}X_2)^{-1}\Sigma_2}}. \label{eq_te_var}
\end{align}}%
In the above, $\hat{v} = W_1 / W_2$ where $W_1, W_2$ are obtained from solving equation \eqref{eq_mtl_eval} (recalling that $W_1, W_2$ are scalars for two tasks).
The role of $\hat{v}$ is to scale the shared subspace $B$ to fit each task.

Equation \eqref{eq_te_model_shift} corresponds to the bias of $\hat{\beta}_t^{\MTL}$.
Hence, the bias term introduces a negative effect that depends on the \textit{similarity} between $\beta_1$ and $\beta_2$.
Equation \eqref{eq_te_var} corresponds to the variance of $\hat{\beta}_t^{\MTL}$ minus the variance of $\hat{\beta}_t^{\STL}$, which is always negative.
Intuitively, the more \textit{samples} we have, the smaller the variance is.
Meanwhile, \textit{covariate shift} also affects how small the variance can be.

%Then we make the following assumptions on the data models.
%\begin{assumption}[Linear regression model]\label{assm_secA2}
%For some fixed $t\in \N$, let $Y_i = X_i\beta_i + \varepsilon_i$, $1\le i \le t$, be independent data models, where $X_i$, $\beta_i$ and $\varepsilon_i$ are also independent of each other. Suppose that $X_i=Z_i\Sigma_i^{1/2}\in \R^{n_i\times p}$ satisfy Assumption \ref{assm_secA1} with $\rho_i:=n_i/p>1$ being fixed constants.
%$\e_i\in \R^{n_i}$ are random vectors with i.i.d. entries with mean zero, variance $\sigma_i^2$ and all moments as in \eqref{assmAhigh}.
%\end{assumption}
 

We now state our main result for two tasks with both covariate and model shift. It shows that the information transfer is determined by two deterministic quantities $\Delta_{\bias}$ and $\Delta_{\vari}$, which give the change of model shift bias and the change of variance, respectively. The exact forms of $\Delta_{\bias}$ and $\Delta_{\vari}$ will be given in Lemma \ref{thm_model_shift}.

\begin{theorem}[Two tasks]\label{thm_main_informal}
	For the setting of two tasks, let $\delta > 0$ be a fixed error margin, $\rho_2 > 1$ and $\rho_1 \gtrsim \delta^{-2}\cdot \lambda_{\min}(\Sigma_1^{1/2}\Sigma_2^{-1/2})^{-4} \norm{\Sigma_1} \max(\norm{\beta_1}^2, \norm{\beta_2}^2)$.
 	There exist two deterministic functions $\Delta_{\bias}$ and $\Delta_{\vari}$ that only depend on $\set{\hat{v}, \Sigma_1, \Sigma_2, \rho_1, \rho_2, \beta_1, \beta_2}$ such that
	\squishlist
		\item If $\Delta_{\bias} - \Delta_{\vari} < -\delta$, then w.h.p. over the randomness of $X_1, X_2$, we have $\te(\hat{\beta}_t^{\MTL}) < \te(\hat{\beta}_t^{\STL})$.
		\item If $\Delta_{\bias} - \Delta_{\vari} > \delta$, then w.h.p. over the randomness of $X_1, X_2$, we have $\te(\hat{\beta}_t^{\MTL}) > \te(\hat{\beta}_t^{\STL})$.
	\squishend
\end{theorem}

Theorem \ref{thm_main_informal} applies to settings where large amounts of source task data are available but the target sample size is small.
For such settings, we obtain a sharp transition from positive transfer to negative transfer determined by $\Delta_{\bias} - \Delta_{\vari}$.
%determined by the covariate shift matrix and the model shift.
%The bounds get tighter and tighter as $\rho_1$ increases.
While the general form of these functions can be complex (as are previous generalization bounds for MTL), they admit interpretable forms for simplified settings.

%\textbf{Proof overview.}\todo{}
%Theorem \ref{lem_cov_shift_informal} extends a well-known result for the single-task setting when $X_1, \rho_1, a_1$ are all equal to zero \cite{S07}.
%Applying Theorem \ref{lem_cov_shift_informal} to \eqref{eq_te_var}, we can calculate the amount of reduced variance compared to STL, which is given asymptotically by $\Delta_{\vari}$.
%For the bias term in equation \eqref{eq_te_model_shift}, we apply the approximate isometry property to $X_1^{\top}X_1$, which is close to $n_1^2\Sigma_1$. This results in the error term $\delta$, which scales as $(1 + 1/\sqrt{\rho_1})^4-1$.
%Then, we apply a similar identity to Theorem \ref{lem_cov_shift_informal} for bounding the bias term, noting that the derivative of $R(z)$ with respect to $z$ can be approximated by $R_\infty'(z)$.
%This estimates the negative effect given by $\Delta_{\bias}$. %, which will be used to estimate the first term on the right hand side of \eqref{eq_te_model_shift}.
%During this process, we will get the $\Delta_{\bias}$ term up to an error $\delta$ depending on $\rho_1$.
The proof of Theorem \ref{thm_main_informal} is presented in Appendix \ref{app_proof_main_thm} and the proof of Lemma \ref{lem_cov_shift_informal} is in Appendix \ref{sec_maintools}.



%The formal statement is stated in Theorem \ref{thm_many_tasks} and its proof can be found in Appendix \ref{app_proof_many_tasks}.
%The technical crux of our approach is to derive the asymptotic limit of $\te(\hat{\beta}_t^{\MTL})$ in the high-dimensional setting, when $p$ approaches infinity.
%We derive a precise limit of $\bigtr{(X_1^{\top}X_1 + X_2^{\top}X_2)^{-1}\Sigma_2}$, which is a deterministic function that only depends on $\Sigma_1, \Sigma_2$ and $n_1/p, n_2/p$ (see Lemma \ref{lem_cov_shift} in Appendix \ref{app_proof_main} for the result).
%Based on the result, we show how to determine positive versus negative transfer as follows.
%, where $\lambda_{\min}(M)$ is the smallest singular value of $M_1$

Next, we derive a closed-form solution of the multi-task learning estimator for the case of two tasks.
From \cite{WZR20}, we know that we need to explicitly restrict the output dimension $r$ of $B$ so that there is transfer between the two tasks.
Hence for the case of two tasks, we consider the setting where $r=1$.
For simplicity of notations, we shall denote $(X_i^{tr},Y_i^{tr})$ and $(X_i^{val},Y_i^{val})$ as $(X_i,Y_i)$ and  $(\wt X_i,\wt Y_i)$, respectively. Then equation \eqref{eq_mtl} simplifies to
\begin{align}\label{eq_mtl_2task}
	f(B; w_1, w_2) = \bignorm{X_1 B w_1 - Y_1}^2 + \bignorm{X_2 B w_2 - Y_2}^2,
\end{align}
where $B\in\real^p$ and $w_1, w_2$ are both real numbers. To solve the above problem, suppose that $w_1, w_2$ are fixed, by local optimality, we find the optimal $B$ as
\begin{align}
	& \hat{B}(w_1, w_2) = (w_1^2 X_1^{\top}X_1 + w_2^2 X_2^{\top}X_2)^{-1} (w_1 X_1^{\top}Y_1 + w_2 X_2^{\top}Y_2) \label{hatB}\\
	&= \frac{1}{w_2} \left( \frac{w_1^2}{w_2^2}  X_1^{\top}X_1 + X_2^{\top}X_2\right)^{-1} \left(\frac{w_1}{w_2} X_1^{\top}Y_1 + X_2^{\top}Y_2\right) \nonumber\\
	&= \frac{1}{w_2}\left[\beta_2 + \left(\frac{w_1^2}{w_2^2} X_1^{\top}X_1 + X_2^{\top}X_2\right)^{-1}\bigbrace{X_1^{\top}X_1\left(\frac{w_1}{w_2}\beta_1 - \frac{w_1^2}{w_2^2} \beta_2\right) + \left(\frac{w_1}{w_2} X_1^{\top}\varepsilon_1 + X_2^{\top}\varepsilon_2\right)}\right]. \nonumber
\end{align}
As a remark, when $w_1 = w_2 = 1$, we obtain linear regression.
If $\beta_1$ is a scaling of $\beta_2$, then  $w_1, w_2$ can be scaled accordingly to fix both tasks more accurately than linear regression.

%For the discussions below, we assume that the entries of $\e_1$ and $\e_2$ all have the same variance $\sigma^2$. This holds for most parts of our discussion, except in Proposition \ref{prop_var_transition}. We will derive different expressions for the validation loss and the test error

Next we consider $N_i$ independent samples of the training set $\{(\wt x_k^{(i)},\wt y_k^{(i)}): 1\le k \le N_i\}$ from task-$i$, $i=1,2$. With these sample, we form the random matrices $\wt X_i \in \R^{N_i\times p}$ and $\wt Y_i\in \R^{N_i}$, $i=1,2,$ whose row vectors are given by $\wt x_k^{(i)}$ and $\wt y_k^{(i)}$. We assume that $N_1$ and $N_2$ satisfy $N_1/N_2=n_1/n_2$ and $N_i \ge n_i^{1-\e_0}$ for some constant $\e_0>0$. Then we write the validation loss in \eqref{eq_mtl_eval} as
\begin{align}\label{eq_mtl_2tasktilde}
	g(w_1,w_2) = \bignorm{\wt X_1 \hat B w_1 - \wt Y_1}^2 + \bignorm{\wt X_2 \hat B w_2 - \wt Y_2}^2.
\end{align}
Inserting \eqref{hatB} into \eqref{eq_mtl_2tasktilde}, one can see that the optimal solution of $g$ only depends on the ratio $v:=w_1/w_2$.
Hence we overload the notation by writing $g(v)$ in the following discussion.
The expectation of $g(v)$ can be written as follows.
\begin{align}
		\val(v) \define& \exarg{\varepsilon_1,\e_2} {\sum_{i=1}^2 \left\|\Sigma_i^{1/2}( \hat B w_i - \beta_i) \right\|^2} \nonumber\\
	=&  N_1 \cdot \bignorm{\Sigma_1^{1/2}\left(v^2 X_1^{\top}X_1 + X_2^{\top}X_2\right)^{-1}X_2^{\top}X_2\left (\beta_1 - v\beta_2\right)}^2 \nonumber \\
	&+ N_2 \cdot v^2\bignorm{\Sigma_2^{1/2}\left(v^2 X_1^{\top}X_1 + X_2^{\top}X_2\right)^{-1}X_1^{\top}X_1\left(\beta_1 - v\beta_2\right)}^2 \nonumber \\
		&+ N_1   \cdot v^2 \bigtr{\Sigma_1\left(v^2 X_1^{\top}X_1 + X_2^{\top}X_2\right)^{-2} \left(\sigma_1^2 \cdot v^2X_1^{\top}X_1 + \sigma_2^2 \cdot X_2^{\top}X_2\right)} \nonumber \\
		&+ N_2  \cdot \bigtr{\Sigma_2\left(v^2 X_1^{\top}X_1 + X_2^{\top}X_2\right)^{-2} \left(\sigma_1^2 \cdot v^2  X_1^{\top}X_1 + \sigma_2^2  \cdot X_2^{\top}X_2\right)}. \label{revise_eq_val_mtl}
\end{align}

\iffalse
{\color{red}\begin{align*} 
			& f(W_1, W_2) = \bignorm{X_1 \hat B w_1 - Y_1}^2 + \bignorm{X_2 \hat B w_2 - Y_2}^2\\
			& =\bignorm{X_1\left( v^2 X_1^{\top}X_1 + X_2^{\top}X_2\right)^{-1} \left(v^2 X_1^{\top}Y_1 + vX_2^{\top}Y_2\right) - Y_1}^2 \\
			&+ \bignorm{X_2 \left( v^2 X_1^{\top}X_1 + X_2^{\top}X_2\right)^{-1} \left(vX_1^{\top}Y_1 + X_2^{\top}Y_2\right) - Y_2}^2 \\
			& =\bignorm{X_1\left( v^2 X_1^{\top}X_1 + X_2^{\top}X_2\right)^{-1} \left(v^2 X_1^{\top}\e_1 + vX_2^{\top}\e_2\right) - \e_1 + X_1\left( v^2 X_1^{\top}X_1 + X_2^{\top}X_2\right)^{-1}  X_2^{\top}X_2(v\beta_2-\beta_1) }^2 \\
			&+ \bignorm{X_2 \left( v^2 X_1^{\top}X_1 + X_2^{\top}X_2\right)^{-1} \left(vX_1^{\top}\e_1 + X_2^{\top}\e_2\right) - \e_2 + vX_2 \left( v^2 X_1^{\top}X_1 + X_2^{\top}X_2\right)^{-1}  X_1^{\top}X_1(\beta_1-v\beta_2) }^2 \\
			&=\val(v)\cdot \left( 1+\OO(p^{-1/2\e})\right) \quad \text{w.h.p.},
		\end{align*}
		where
		\begin{align*}
		\val(v)&=\bignorm{X_1\left( v^2 X_1^{\top}X_1 + X_2^{\top}X_2\right)^{-1}  X_2^{\top}X_2(v\beta_2-\beta_1) }^2 \\
			&+ v^2\bignorm{X_2 \left( v^2 X_1^{\top}X_1 + X_2^{\top}X_2\right)^{-1}  X_1^{\top}X_1(\beta_1-v\beta_2) }^2 \\
			&+\sigma^2 \tr\left(v^2 X_1\left( v^2 X_1^{\top}X_1 + X_2^{\top}X_2\right)^{-1}   X_1^{\top} -\id\right)^2 \\
			&+\sigma^2 \tr\left(v^2 X_1\left( v^2 X_1^{\top}X_1 + X_2^{\top}X_2\right)^{-1}X_2^{\top}X_2 \left( v^2 X_1^{\top}X_1 + X_2^{\top}X_2\right)^{-1} X_1^\top\right)\\
			&+\sigma^2 \tr\left(X_2 \left( v^2 X_1^{\top}X_1 + X_2^{\top}X_2\right)^{-1} X_2^{\top} -\id\right)^2 \\
			&+\sigma^2 \tr\left(v^2 X_2\left( v^2 X_1^{\top}X_1 + X_2^{\top}X_2\right)^{-1}X_1^{\top}X_1 \left( v^2 X_1^{\top}X_1 + X_2^{\top}X_2\right)^{-1} X_2^\top\right)\\
			&=\bignorm{X_1\left( v^2 X_1^{\top}X_1 + X_2^{\top}X_2\right)^{-1}  X_2^{\top}X_2(v\beta_2-\beta_1) }^2 \\
			&+ v^2\bignorm{X_2 \left( v^2 X_1^{\top}X_1 + X_2^{\top}X_2\right)^{-1}  X_1^{\top}X_1(\beta_1-v\beta_2) }^2 +(n_1+n_2-p)\sigma^2.
		\end{align*}
	}
\fi

\begin{claim}
	In the setting described above, we have that
	\be\label{approxvalid}
		g(v)=  \left[\val(v) + (N_1\sigma^2_1+N_2\sigma^2_2)\right]\cdot \left( 1+\OO(p^{-(1-\e_0)/2+\e})\right).
	\ee
\end{claim}
\begin{proof}
	We use the fact that our random vectors have i.i.d. entries.
%Before doing that, we first need to fix the setting for the following discussions, because we want to keep track of the error rate carefully instead of obtaining an asymptotic result only.
	Recall that $Y_i = X_i\beta_i + \varepsilon_i$ and $\wt Y_i = \wt X_i\beta_i + \wt\varepsilon_i$, $i=1,2$, all satisfy Assumption \ref{assm_secA2}. Then we rewrite \eqref{eq_mtl_2tasktilde} as
$$	g( v) = \sum_{i=1}^2\left\| \wt X_i\wt\beta_i  - \wt \e_i\right\|^2 , \quad \wt\beta:=\hat B w_i-\beta_i.$$
Since $ \wt X_i\wt\beta$ and $ \wt \e_i$ are independent random vectors with i.i.d. centered entries, we can use the concentration result, Lemma \ref{largedeviation}, to get that for any constant $\e>0$,
\begin{align*}
\left|\left\| \wt X_i\wt\beta_i  - \wt \e_i\right\|^2 -  \exarg{\wt X_i,\wt{\e}_i} {\left\| \wt X_i\wt\beta_i  - \wt \e_i\right\|^2} \right| & =\left|\left\| \wt X_i\wt\beta_i  - \wt \e_i\right\|^2 - N_i (\wt\beta_i^\top \Sigma_i \wt\beta_i + \sigma_i^2) \right| \\
&\le N_i^{1/2+\e} (\wt\beta_i^\top \Sigma_i \wt\beta_i + \sigma_i^2),
\end{align*}
with high probability. Thus we obtain that
$$g(v)= \left[\sum_{i=1}^2 N_i\left\|\Sigma_i^{1/2}( \hat B w_i - \beta_i) \right\|^2 + (N_1\sigma^2_1+N_2\sigma^2_2)\right]\cdot \left( 1+\OO(p^{-(1-\e_0)/2+\e})\right),$$
where we also used $N_i\ge p^{-1+\e_0}$. Inserting \eqref{hatB} into the above expression and using
 again the concentration result, Lemma \ref{largedeviation}, we get that
$$ \sum_{i=1}^2 N_i\left\|\Sigma_i^{1/2}( \hat B w_1 - \beta_i) \right\|^2 = \val(v)\cdot \left( 1+\OO(p^{-1/2+\e})\right)$$
with high probability.
%-----old-------
%Suppose that the entries of $\e_1$ and $\e_2$ have variance $\sigma^2$.  Using a validation set that is sub-sampled from the original training dataset, we get a validation loss as follows
%\begin{align}
%		&\val(\hat{B}; w_1, w_2):= \exarg{\varepsilon_1,\e_2} \sum_{i=1}^2 \left\|\Sigma_i^{1/2}( \hat B w_1 - \beta_i) \right\|^2 \\
%	&=  n_1 \cdot \bignorm{\Sigma_1^{1/2}\left(\frac{w_1^2}{w_2^2} X_1^{\top}X_1 + X_2^{\top}X_2\right)^{-1}X_2^{\top}X_2\left (\beta_s - \frac{w_1}{w_2}\beta_t\right)}^2 \nonumber \\
%		&+ n_1 \sigma^2 \cdot \frac{w_1^2}{w_2^2} \bigtr{\left(\frac{w_1^2}{w_2^2}  X_1^{\top}X_1 + X_2^{\top}X_2\right)^{-1}\Sigma_1} \nonumber \\
%		&+ n_2 \cdot \frac{w_1^2}{w_2^2}\bignorm{\Sigma_2^{1/2}\left(\frac{w_1^2}{w_2^2} X_1^{\top}X_1 + X_2^{\top}X_2\right)^{-1}X_1^{\top}X_1\left(\beta_s - \frac{w_1}{w_2}\beta_t\right)}^2 \nonumber \\
%		&+ n_2 \sigma^2 \cdot \bigtr{\left(\frac{w_1^2}{w_2^2} X_1^{\top}X_1 + X_2^{\top}X_2\right)^{-1}\Sigma_2}. \label{eq_val_mtl}
%\end{align}
%\nc
%------------------
Thus we conclude the proof.
\end{proof}

Hence to minimize $g(v)$, it suffices to minimize $\val(v)$ over $v$.
Let $\hat v=\hat{w_1}/\hat{w_2}$ be the global minimizer of $g(v)$.
Now we can define the multi-task learning estimator for the target task as
	\[ \hat{\beta}_2^{\MTL} = \hat{w}_{2}\hat{B}(\hat{w}_1, \hat{w}_2) .\]
%	where $t=2$ since we are considering the two task case, and it also stands for the ``target task".
%The intuition for deriving $\hat{\beta}_2^{\MTL}$ is akin to performing multi-task training in practice.
%Let $\hat{v} = \hat{w_1} / \hat{w_2}$ for the simplicity of notation.
The prediction loss of using $\hat{\beta}_2^{\MTL}$ for the target task is
\begin{align}
	\te(\hat{\beta}_2^{\MTL}) =&~ \hat{v}^2 \bignorm{\Sigma_2^{1/2}(\hat{v}^2 X_1^{\top}X_1 + X_2^{\top}X_2)^{-1} X_1^{\top}X_1 (\beta_1 - \hat{v} \beta_2)}^2 \nonumber \\
			&+~  \bigtr{\Sigma_2(\hat{v}^2 X_1^{\top}X_1 + X_2^{\top}X_2)^{-2}\left(\sigma_1^2 \cdot \hat v^2  X_1^{\top}X_1 + \sigma_2^2  \cdot X_2^{\top}X_2\right) }, \label{eq_te_mtl_2task}
\end{align}
which only depends on $\hat v$, the sample covariance matrices, and $\beta_1,\beta_2$.

%\begin{lemma}[Variance bound]\label{lem_cov_shift_informal}
%	In the setting of two tasks,
%	let $n_1 = \rho_1 \cdot p$ and $n_2 = \rho_2 \cdot$ be the sample size of the two tasks.
%	Let $\lambda_1, \dots, \lambda_p$ be the singular values of the covariate shift matrix $\Sigma_1^{1/2}\Sigma_2^{-1/2}$ in decreasing order.
%	%let $n_1 = \rho_1 \cdot p$ and $n_2 = \rho_2 \cdot p$ denote the sample sizes of each task.
%	%Let $\Sigma_1$ and $\Sigma_2$ denote the covariance matrix of each task.
%	With high probability, the variance of the multi-task estimator $\hat{\beta}_t^{\MTL}$ equals
%	%let $M = \Sigma_1^{1/2}\Sigma_2^{-1/2}$ and $\lambda_1, \lambda_2, \dots, \lambda_p$ be the singular values of $M^{\top}M$ in descending order.
%%	For any constant $\e>0$, w.h.p. over the randomness of $X_1, X_2$, we have that
%	{\small\begin{align*}%\label{eq_introX1X2}
%		%\bigtr{(X_1^{\top}X_1 + X_2^{\top}X_2)^{-1}\Sigma_2} =
%		\frac{\sigma^2}{n_1+n_2}\cdot \bigtr{ (\hat{v}^2 a_1 \Sigma_2^{-1/2}\Sigma_1\Sigma_2^{-1/2} + a_2\id)^{-1}} +\bigo{{p^{-1/2+o(1)}}},
%	\end{align*}}%
%	where $a_1, a_2$ are solutions of the following equations:
%	{\small\begin{align*}
%		a_1 + a_2 = 1- \frac{1}{\rho_1 + \rho_2},\quad a_1 + \frac1{\rho_1 + \rho_2}\cdot \frac{1}{p}\sum_{i=1}^p \frac{\hat{v}^2\lambda_i^2 a_1}{\hat{v}^2\lambda_i^2 a_1 + a_2} = \frac{\rho_1}{\rho_1 + \rho_2}.
%	\end{align*}}
%%are both fixed values that roughly scales with the sample sizes $\rho_1, \rho_2$, and satisfy $a_1 + a_2 = 1 - (\rho_1 + \rho_2)^{-1}$ plus another deterministic equation.
%\end{lemma}

We shall refer to random matrices of the form $X^\top X$ as sample covariance matrices following the standard notations in high-dimensional statistics. The following two lemmas  are the main random matrix theoretical results of this paper, which will be used to estimate the two terms on the right-hand side of \eqref{eq_te_mtl_2task}.
%which deals with the inverse of the sum of two random matrices, which
%any is can be viewed as a special case of Theorem \ref{thm_model_shift}.

\begin{lemma}[Variance bound]\label{lem_cov_shift}
	%Let $X_i\in\real^{n_i\times p}$ be a random matrix that contains i.i.d. row vectors with mean $0$ and variance $\Sigma_i\in\real^{p\times p}$, for $i = 1, 2$.
	Suppose $X_1=Z_1\Sigma_1^{1/2}\in \R^{n_1\times p}$ and $X_2=Z_2\Sigma_2^{1/2}\in \R^{n_2\times p}$ satisfy Assumption \ref{assm_secA1} with $\rho_1:=n_1/p>1$ and $\rho_2:=n_2/p>1$ being fixed constants.
	Denote by $M = \Sigma_1^{1/2}\Sigma_2^{-1/2}$ and let $\lambda_1, \lambda_2, \dots, \lambda_p$ be the singular values of $M$ in descending order. Let $A$ be any $p\times p$ matrix that is independent of $X_1$ and $X_2$. We have that for any constant $\e>0$,
	%When $n_1 = c_1 p$ and $n_2 = c_2 p$, we have that with high probability over the randomness of $X_1$ and $X_2$, the following equation holds
	\begin{align}\label{lem_cov_shift_eq}
		\bigtr{(\hat v^2 X_1^{\top}X_1 + X_2^{\top}X_2)^{-1}A} = \frac{1}{\rho_1+\rho_2}\frac1p\bigtr{ (\hat v^2 a_1 \Sigma_1 + a_2\Sigma_2)^{-1} A} +\bigo{\|A\| p^{-1/2+\epsilon}}
	\end{align}
with high probability, where $(a_1, a_2)$ is the solution to the following deterministic equations:
	\begin{align}
		a_1 + a_2 = 1- \frac{1}{\rho_1 + \rho_2},\quad a_1 + \frac1{\rho_1 + \rho_2}\cdot \frac{1}{p}\sum_{i=1}^p \frac{\hat v^2 \lambda_i^2 a_1}{\hat v^2 \lambda_i^2 a_1 + a_2} = \frac{\rho_1}{\rho_1 + \rho_2}. \label{eq_a12extra}
	\end{align}
\end{lemma}


Lemma \ref{lem_cov_shift} allows us to get a tight bound on equation \eqref{eq_te_var}, that only depends on \textit{sample size}, \textit{covariate shift} and the scalar $\hat{v}$.
As a remark, the concentration error $\bigo{\|A\| p^{-1/2+\epsilon}}$ of our result is nearly optimal.
For the bias term of equation \eqref{eq_te_model_shift}, a similar result that scales with task model distance in addition to sample size and covariate shift holds (cf. Lemma \ref{lem_cov_derivative} in Appendix \ref{app_proof_main_thm}).

\begin{lemma}[Bias bound]\label{lem_cov_derivative}
In the setting of Lemma \ref{lem_cov_shift}, let $\beta \in \R^p$ be any vector that is independent of $X_1$ and $X_2$. We have that for any constant $\e>0$,
\begin{equation}\label{lem_cov_derv_eq}
\begin{split}
&(n_1+n_2)^2\bignorm{\Sigma_2^{1/2} (\hat v^2 X_1^{\top}X_1 + X_2^{\top}X_2)^{-1}\beta}^2 \\
&= \beta^{\top} \Sigma_2^{-1/2}  \frac{(1 + a_3)\id + \hat v^2a_4 {M}^{\top}{M}}{( \hat v^2a_1 {M}^{\top}{M}+a_2)^2} \Sigma_2^{-1/2} \beta +\OO(p^{-1/2+\e}\|\beta\|^2),
\end{split}
\end{equation}
with high probability, where $a_{3}$ and $a_4$ satisfy the following system of linear equations:
\begin{gather}\label{eq_a34extra}
		\left(\rho_2  a_2^{-2}-  b_0\right)\cdot  a_3 - b_1 \cdot  a_4
		= b_0, \quad \left(\rho_1 a_1^{-2} -  b_2  \right)\cdot  a_4 -  b_1 \cdot  a_3 = b_1 .
%		\left(\frac{n_1}{\hat a_1^2} -  \sum_{i=1}^p \frac{\hat \lambda_i^4   }{  (\hat a_2 + \hat \lambda_i^2\hat a_1)^2  }\right)\hat a_4 -\left(\sum_{i=1}^p \frac{\hat \lambda_i^2  }{  (\hat a_2 + \hat \lambda_i^2\hat a_1)^2  }\right)\hat a_3
%		= \sum_{i=1}^p \frac{\hat \lambda_i^2 }{  (\hat a_2 + \hat \lambda_i^2\hat a_1)^2  }. \label{eq_a4}
	\end{gather}
Here $b_0$, $b_1$ and $b_2$ are defined as
$$ b_k:= \frac1{p}\sum_{i=1}^p \frac{ \hat v^{2k}\lambda_i^{2k}}{ ( a_2 + \hat v^2\lambda_i^2 a_1)^2  },\quad k=0,1,2.$$
\end{lemma}



\textbf{Proof overview.}
%We first describe the proof of Theorem \ref{lem_cov_shift_informal}.
The proofs of Lemma \ref{lem_cov_shift} and Lemma \ref{lem_cov_derivative} are based on the Stieltjes transform method (or the resolvent method) in random matrix theory \cite{bai2009spectral,tao2012topics,erdos2017dynamical}. Roughly speaking, we study the resolvent $R(z):=[\Sigma_2^{-1/2}( X_1^{\top}X_1 + X_2^{\top}X_2)\Sigma_2^{-1/2}-z]^{-1}$ for $z\in \C$ around $z=0$.
Using the methods in \cite{Anisotropic,yang2019spiked}, we find the asymptotic limit, say $R_\infty(z)$, of $R(z)$ for any $z$ as $p\to \infty$ with an almost optimal convergence rate. In particular, when $z=0$, $\tr[\Sigma_2^{-1/2}A\Sigma_2^{-1/2}R_\infty(0)]$ gives the limit in \eqref{lem_cov_shift_eq}. 
On the other hand, we can write 
$$\bignorm{\Sigma_2^{1/2} (X_1^{\top}X_1 + X_2^{\top}X_2)^{-1}\beta}^2= \beta^\top \Sigma_2^{-1/2}R'(0)\Sigma_2^{-1/2} \beta.$$ 
Hence its limit can be calculated through $R_\infty'(z)$, which gives the expression in \eqref{lem_cov_derv_eq}. The details can be found in Appendix \ref{sec_maintools}. 
%The proof of Lemma \ref{lem_cov_shift} and Lemma \ref{lem_cov_derivative} is a main focus of Section \ref{sec_maintools}. 
We remark that one can probably derive the same asymptotic result using free probability theory (see e.g. \cite{nica2006lectures}), but our results \eqref{lem_cov_shift_eq} and \eqref{lem_cov_derv_eq} also give an almost sharp error bound $\bigo{ p^{-1/2+\epsilon}}$.


%We now state several helper lemmas to get estimates on $L(\hat{\beta}_t^{\STL})$ and $L(\hat{\beta}_t^{\MTL})$ for $t=2$. 
The next lemma, which is a folklore result in random matrix theory, helps to determine the asymptotic limit of $\te(\hat{\beta}_t^{\STL})=\sigma^2   \bigtr{(X_2^{\top}X_2)^{-1}\Sigma_2}$ as $p\to \infty$. When the entries of $X_2$ are multivariate Gaussian, this lemma recovers the classical result for the mean of inverse Wishart distribution \cite{anderson1958introduction}. For general non-Gaussian random matrices, it can be obtained using Stieltjes transform method; see e.g., Lemma 3.11 of \cite{bai2009spectral}. Here we shall state a result obtained from Theorem 2.4 in \cite{isotropic}, which gives an almost sharp error bound.
\begin{lemma}\label{lem_minv}
Suppose $X$ satisfies assumption \ref{assm_secA1}. Let $A$ be any $p\times p$ matrix that is independent of $X$. We have that for any constant $\e>0$,
	\be\label{XXA}  \bigtr{(X_2^{\top}X_2)^{-1}A} = \frac{1}{\rho_2 - 1} \frac1p\tr(\Sigma^{-1}A) +\bigo{ \|A\|p^{-1/2+\epsilon}} \ee
with high probability.
\end{lemma}
Note that Lemma \ref{lem_cov_shift} extends Lemma \ref{lem_minv} for a single sample covariance matrix to the sum of two independent sample covariance matrices. Note that if we set $n_1=0$ and $n_2=n$, then $a_1 = 0$ and $a_2 = (n_2-p) / n_2$ is the solution to \eqref{eq_a12extra}, and one can see that \eqref{lem_cov_shift_eq} reduces to \eqref{XXA}. 


Combining the above three lemmas, we can provide a sharp analysis of the bias-variance tradeoff of the multi-task estimator.


\subsection{Multiple-task Case}\label{sec_multiple}

A well-known result in the high-dimensional linear regression setting states that $\tr[(X_2^{\top}X_2)^{-1}\Sigma_2]$ is concentrated around $1 / (\rho_2 - 1)$ (e.g. Chapter 6 of \cite{S07}), which scales with the sample size of the target task.
Our main technical contribution is to extend this result to two tasks.
We show how the variance of the multi-task estimator scales with sample size and covariate shift in the following result.






Next, we describe our result for more than two tasks with same features, i.e. $X_i = X$ for any $i$.
This setting is prevalent in applications of multi-task learning to image classification, where there are multiple prediction labels/tasks for every image \cite{chexnet17,EA20}.
\begin{theorem}[Many tasks]\label{thm_many_tasks}
%Suppose $X=Z\Sigma^{1/2}\in \R^{n\times p}$ satisfy Assumption \ref{assm_secA1} with $\rho:=n/p>1$ being some fixed constant. Consider data models  $Y_i = X\beta_i + \varepsilon_i$, $i=1,2,\cdots, t$, where $\e_i\in \R^{n}$ are random vectors with i.i.d. entries with mean zero, variance $\sigma^2$ and all moments as in \eqref{assmAhigh}. Moreover, assume that $X$, $\beta_i$ and $\e_i$ are all independent of each other.
	%Let $n = c \cdot p$.
	%Let $X\in\real^{n\times p}$ and $Y_i = X\beta_i + \varepsilon_i$, for $i = 1,\dots,k$.
%	Consider $t$ data models $Y_i = X\beta_i + \varepsilon_i$, $i=1,2,\cdots, t$, where $X$ has covariance matrix $\Sigma$, and the entries of $\e_i$ are i.i.d. with mean zero and variance $\sigma^2$.
	%that satisfy Assumption \ref{assm_secA2} in the appendix.
	For the setting of $t$ tasks where $X_i = X$, for all $1\le i\le t$,
	let $B^\star := [{\beta}_1,{\beta}_2,\dots,{\beta}_{t}]$ and $U_r\in\real^{t\times r}$ denote the linear model parameters.
	Let $U_r U_r^{\top}$ denote the best rank-$r$ subspace approximation of $(B^{\star})^\top\Sigma B^{\star}$.
	Assume that $\lambda_{\min}({B^{\star}}^\top\Sigma B^{\star})\gtrsim \sigma^2$.
	Let $v_i$ denote the $i$-th row vector of $U_r$.
	There exists a value $\delta = \oo \left( \|B^\star\|^2 + \sigma^2\right)$ such that
	\squishlist
		\item  If	$\left(1 - \norm{v_t}^2 \right)\frac{\sigma^2}{\rho - 1} - \norm{\Sigma^{1/2} (B^{\star} U_r v_t - \beta_t)}^2 > \delta$, then w.h.p $\te(\hat{\beta}_t^{\MTL}) < \te(\hat{\beta}_t^{\STL})$.
		\item If $\left(1 - \norm{v_t}^2\right)\frac{\sigma^2}{\rho - 1} - \norm{\Sigma^{1/2}(B^{\star} U_r v_t - \beta_t)}^2 < -\delta$, then w.h.p. $\te(\hat{\beta}_t^{\MTL}) > \te(\hat{\beta}_t^{\STL})$.
	\squishend
\end{theorem}
Theorem \ref{thm_many_tasks} provides a sharp analysis of the bias-variance tradeoff beyond two tasks.
Specifically, $(1 - \norm{v_t}^2)\sigma^2/(\rho - 1)$ shows the amount of reduced variance and $\norm{\Sigma (B^{\star} U_r v_t - \beta_t)}$ shows the bias of the multi-task estimator.
The proof of \ref{thm_many_tasks} can be found in Appendix \ref{app_proof_many_tasks}.










\section{Special Case: Isotropic Covariance and Covariate Shift}\label{sec_special}

%We provide precise explanations to the phenomena of negative transfer in multi-task learning.
We provide tight bounds on the bias and variance of the multi-task estimator for two tasks.
We show theoretical implications for understanding the performance of multi-task learning.
(a) \textit{Task similarity}: we explain the phenomenon of negative transfer precisely as tasks become more different.
(b) \textit{Sample ratio}: we further explain a curious phenomenon where increasing the sample ratio helps initially, but hurts eventually.
(c) \textit{Covariate shift}: as the source sample size increases, we show that the covariate shift worsens the performance of the multi-task estimator.
Finally, we extend our results from two tasks to many tasks with the same features.
%We explain from three perspectives, including \textit{task similarity}, \textit{sample size} and \textit{covariate shift}.
%We show how negative transfer occurs by varying task similarity or sample size.
%Then we show that when source task sample size becomes large, covariate shift causes more negative effects.




\subsection{Task Similarity}\label{sec_similarity}

It is well-known since the seminal work of Caruana \cite{C97} that how well multi-task learning performs depends on task relatedness.
We formalize this connection in the following simplified setting, where we can perform explicit calculations.
We show that as we increase the distance between $\beta_1$ and $\beta_2$, there is a transition from positive transfer to negative transfer in MTL.

\textit{The isotropic model.}
	Consider two tasks with isotropic covariances $\Sigma_1 = \Sigma_2 = \id$.
	Each task has sample size $n_1 = \rho_1 \cdot p$ and $n_2 = \rho_2 \cdot p$.
	%And $X_1\in\real^{n_1\times p}, X_2\in\real^{n_2\times p}$ denote the covariates of the two tasks, respectively.
	Assume that for  task two, $\beta_2$ has i.i.d. entries with mean zero and variance $\kappa^2$.
	For the source task, $\beta_1 $ equals $\beta_2$ plus i.i.d. entries with mean $0$ and variance $d^2$.
	The labels are $Y_i = X_i\beta_i + \varepsilon_i$, where $\e_i$ consists of i.i.d. entries with mean zero and variance $\sigma^2$.
	For our purpose, it is enough to think of the order of $d$ being $1/\sqrt{p}$ and $pd^2/\sigma^2$ being constant. The more precise conditions on the relations between $d^2$, $\sigma^2$ and $\kappa^2$ are given in \eqref{choiceofpara}.
%	We assume that all the random variables have subexponential decay, while keeping in mind that our results can be applied under weaker moments assumptions as shown in Appendix \ref{sec_maintools}.

%In the isotropic model, we show that as we increase the distance between $\beta_1$ and $\beta_2$ (or $d$), there is a transition from positive transfer to negative transfer in MTL.
%We measure model dissimilarity as $\norm{\beta_1 - \beta_2}^2$, which is the distance between source and target in the isotropic model.
%Figure \ref{fig_model_shift} provides a simulation when $p = 200$.
%{The rest of parameter settings can be found in Appendix \ref{app_synthetic}.}
%Our result below will provide an explanation to this phenomenon.
%Based on Theorem \ref{thm_model_shift}, we derive the transition threshold in the following proposition.
We introduce the following notations.
{\small\begin{align*}
	\Psi(\beta_1, \beta_2) = {\ex{\bignorm{\beta_1 - \beta_2}^2}} / {\sigma^2},  \quad \Phi(\rho_1, \rho_2) = \frac{(\rho_1 + \rho_2 - 1)^2}{\rho_1 (\rho_1 + \rho_2) (\rho_2 - 1)}.
\end{align*}}

\begin{proposition}[Task model distance]\label{prop_dist_transition}
	In the isotropic model, suppose that $\rho_1$ and $\rho_2 > 1$.
	Then
	%Whether $\te(\hat{\beta}_t^{\MTL})$ is lower than $\te(\hat{\beta}_t^{\STL})$ is determined by the ratio between $\Psi(\beta_1, \beta_2)$ and $\Phi(\rho_1, \rho_2)$:
	\squishlist
		\item If $\Psi(\beta_1, \beta_2) < \frac{1}{\nu} \cdot  \Phi(\rho_1, \rho_2)$, then w.h.p. over the randomness of $X_1,X_2$, $\te(\hat{\beta}_t^{\MTL}) < \te(\hat{\beta}_t^{\STL})$.
		\item If $\Psi(\beta_1, \beta_2) > {\nu} \cdot  \Phi(\rho_1, \rho_2)$, then w.h.p. over the randomness of $X_1,X_2$, $\te(\hat{\beta}_t^{\MTL}) > \te(\hat{\beta}_t^{\STL})$.
	\squishend
	Here {\small$\nu = (1+\oo(1)) \cdot (1 - 1/\sqrt{\rho_1})^{-4}$}.
	Concretely, if $\rho_1 > 40$, then $\nu\in (1,2)$.
\end{proposition}


Proposition \ref{prop_dist_transition} simplifies Theorem \ref{thm_main_informal} in the isotropic model, allowing for a more explicit statement of the bias-variance tradeoff.
Concretely, $\Psi(\beta_1, \beta)$ and $\Phi(\rho_1, \rho_2)$ corresponds to $\Delta_{\bias}$ and $\Delta_{\vari}$, respectively.
Roughly speaking, the transition threshold scales as $\frac{pd^2}{\sigma^2} - \frac{1}{\rho_1} - \frac{1}{\rho_2}$.
We apply Proposition \ref{prop_dist_transition} to the parameter setting of Figure \ref{fig_model_shift} (the details are left to Appendix \ref{app_synthetic}).
We can see that our result is able to predict positive or negative transfer  accurately and matches the empirical curve.
There are several unexplained observations near the transition threshold $0$, which are caused by the concentration error $\nu$.
%We fix the target task and vary the source task, in particular the parameter $d$ which determines $\norm{\beta_1 - \beta_2}$.
%Figure \ref{fig_model_shift} shows the result.
%We observe that Proposition \ref{prop_dist_transition} explains most of the observations in Figure \ref{fig_model_shift}.
%The proof of Proposition \ref{prop_dist_transition} involves two parts.
%First, in equation \eqref{eq_te_var}, the positive variance reduction effect scales with $n_1 = \rho_1 p$, the number of source task data points.
%Second, we show that the negative effect of model-shift bias scales with $pd^2$, which is the expectation of $\norm{\beta_1 - \beta_2}^2$.
The proof of Proposition \ref{prop_dist_transition} can be found in Appendix \ref{app_proof_31}.
A key part of the analysis shows that $\hat{v}\approx 1$ in the isotropic model,
thus simplifying the result of Theorem \ref{thm_main_informal}.

\subsection{Sample Ratio}\label{sec_data_size}

In classical Rademacher or VC based theory of multi-task learning, the generalization bounds are usually presented for settings where the sample sizes are equal for all tasks \cite{B00,M06,MPR16}.
%More generally, such results are still applicable when all task data are being added simultaneously.
On the other hand, uneven sample sizes between different tasks (or even dominating tasks) have been empirically observed as a cause of negative transfer \cite{YKGLHF20}.
For such settings, we have also observed that adding more labeled data from one task does not always help.
%On the other hand, we have observed that adding more labeled data does not always improve performance in multi-task learning.
In the isotropic model, we consider what happens if we vary the source task sample size.
Our theory accurately predicts a curious phenomenon, where increasing the sample size of the source task results in negative transfer!
%Figure \ref{fig_size} provides a simulation result for such a setting.
%We observe that as $n_1 / n_2$ increases, there is a transition from positive to negative transfer.

\begin{proposition}[Source/target sample ratio]\label{prop_data_size}
	In the isotropic model, suppose that $\rho_1 > 40$ and $\rho_2 > 110$ are fixed constants, and $\Psi(\beta_1, \beta_2) > 2/(\rho_2 - 1)$.
	Then we have that
	\squishlist
		\item If $\frac{n_1}{n_2} = \frac{\rho_1}{\rho_2} < \frac{1}{\nu} \cdot \frac{1 - 2\rho_2^{-1}}{\Psi(\beta_1, \beta_2) (\rho_2 - 1) - \nu^{-1}}$, then w.h.p. $\te(\hat{\beta}_t^{\MTL}) < \te(\hat{\beta}_t^{\STL})$.
		\item If $\frac{n_1}{n_2} = \frac{\rho_1}{\rho_2} > {\nu} \cdot \frac{1 - 2\rho_2^{-1}}{\Psi(\beta_1, \beta_2) (\rho_2 - 1.5) - \nu}$, then w.h.p. $\te(\hat{\beta}_t^{\MTL}) > \te(\hat{\beta}_t^{\STL})$.
	\squishend
\end{proposition}
Proposition \ref{prop_data_size} describes the bias-variance tradeoff in terms of the sample ratio $\rho_1 / \rho_2$.
We apply the result to the setting of Figure \ref{fig_size} (described in Appendix \ref{app_synthetic}).
There are several unexplained observations near $y = 0$ caused by $\nu$.
The proof of Proposition \ref{prop_data_size} can be found in Appendix \ref{app_proof_32}.

\subsection{Covariate Shift}\label{sec_covshift}

So far we have considered the isotropic model where $\Sigma_1 = \Sigma_2$.
This setting is relevant for settings where different tasks share the same input features such as multi-class image classification.
In general, the covariance matrices of the two tasks may be different such as in text classification.
In this part, we consider what happens when $\Sigma_1 \neq \Sigma_2$.
We show that when $n_1 / n_2$ is large, MTL with covariate shift can be suboptimal compared to MTL without covariate shift.

\textit{Example.}
	We measure covariate shift by $M = \Sigma_1^{1/2} \Sigma_2^{-1/2}$.
	Assume that $\Psi(\beta_1, \beta_2) = 0$ for simplicity.
	We compare two cases: (i) when $M = \id$; (ii) when $M$ has $p/2$ singular values that are equal to $\lambda$ and $p/2$ singular values that are equal to $1 / \lambda$.
	Hence, $\lambda$ measures the severity of the covariate shift.
	Figure \ref{fig_covariate} shows a simulation of this setting by varying $\lambda$.
	We observe that as source/target sample ratio increases, the performance gap between the two cases increases.

%By applying Lemma \ref{lem_cov_shift_informal}, we find that when $n_1 / n_2$ is large, having no covariate shift is the optimal choice provided that the determinant of $M^{\top}M$ is bounded.
We compare different choices of $M$ that belong to the following bounded set.
Let $\lambda_i$ be the $i$-th singular value of $M$.
Let $\mu_{\min} < \mu < \mu_{\max}$ be fixed values that do not grow with $p$.
\vspace{-0.025in}
{\small\begin{align*}
		\cS_{\mu}\define\bigset{M \left| \prod_{i=1}^p \lambda_i \le \mu^p, \mu_{\min} \le \lambda_i\le \mu_{\max}, \text{ for all } 1\le i\le p\right.},
\end{align*}}
%	We assume that $\beta_1$ and $\beta_2$ are generated following the isotropic model with $d = 0$.
\begin{proposition}[Covariate shift]\label{prop_covariate}
	Assume that $\Psi(\beta_1, \beta_2) = 0$ and $\rho_1, \rho_2>1$.
	Let $g(M)$ denote the prediction loss of $\hat{\beta}_t^{\MTL}$ when $M = \Sigma_1^{1/2}\Sigma_2^{-1/2} \in\cS_{\mu}$.
	We have that
	{\small\[ g(\mu\id) \le \bigbrace{1+ \bigo{{\rho_2}/{\rho_1}  }} \min_{M\in\cS_{\mu}} g(M). \]}
\end{proposition}
This proposition shows that when source/target sample ratio is large, then having no covariate shift is optimal.
The proof of Proposition \ref{prop_covariate} is left to Appendix \ref{app_proof_33}.
%Proposition \ref{prop_covariate} implies that when $\rho_1\gg \rho_2$, having no covariate shift is the optimal choice for choosing the source task.
%This provides evidence that covariate shift is unfavorable when there are many source task datapoints,

%\todo{} To complement the result, we show an example when the statement is not true if $n_1 \le n_2$.

%We ask: is it better to have $M$ as being close to identity, or should $M$ involve varying levels of singular values?
%Understanding this question has implications for applying normalization methods in multi-task learning \cite{LV19,CBLR18,YKGLHF20}.
%We show that if $n_1$ is much larger than $n_2$, then the optimal $M$ matrix should be proportional to identity, under certain assumptions on its range of singular values (to be formulated in Proposition \ref{prop_covariate}).
%On the other hand, if $n_1$ is comparable or even smaller than $n_2$, we show an example where having ``complementary'' covariance matrices is better performing than having the same covariance matrices.


\subsection{Extension to Multiple Tasks}




\section{Theoretical Implication}


Next we consider the case where the two tasks have different noise variances $\sigma_1^2\ne \sigma_2^2$. In particular, we show Proposition \ref{prop_var_transition}, which gives a transition threshold with respect to the difference between the noise levels of the two tasks.

\begin{proposition}%[Labeled data de-noising]
\label{prop_var_transition}
	In the isotropic model, assume that $\rho_1 > 40$ and $\ex{\norm{\beta_1 - \beta_2}^2}< \frac{1}{2} {\sigma_2^2}  \cdot \Phi(\rho_1, \rho_2)$.
	Then we have the following transition with respect to $\sigma_1^2$:
	\begin{itemize}
		\item If $\sigma_1^2 < - \nu^{1/2} \rho_1 \cdot p d^2+\left(1+ \nu^{-1/2}\rho_1 \Phi(\rho_1, \rho_2)\right)\cdot\sigma_2^2$, then w.h.p. $\te(\hat{\beta}_2^{\MTL}) < \te(\hat{\beta}_2^{\STL})$.
		\item If $\sigma_1^2 > -\nu^{-1/2}\rho_1\cdot p d^2   +\left(1+ \nu^{1/2}\rho_1\Phi(\rho_1, \rho_2)\right) \cdot \sigma_2^2$, then w.h.p. $\te(\hat{\beta}_2^{\MTL}) > \te(\hat{\beta}_2^{\STL})$.
	\end{itemize}
\end{proposition}
As a corollary, if $\sigma_1^2 \le \sigma_2^2$, then we always get positive transfer.




\textbf{Algorithmic consequence.}
We can in fact extend the result to the cases where the noise variances are different.
In this case, we will see that MTL is particularly effective.
%As an extension of Proposition \ref{prop_dist_transition}, we observe that MTL is particular powerful when the labeled data of the source task is less noisy compared to the target task.
Concretely, suppose the noise variance $\sigma_1^2$ of task $1$ differs from the noise variance $\sigma_2^2$ of task $2$.
If $\sigma_1^2$ is too large, the source task provides a negative transfer to the target.
If $\sigma_1^2$ is small, the source task is more helpful.
We leave the result to Proposition \ref{prop_var_transition} in Appendix \ref{app_proof_31}.
Inspired by the observation, we propose a single-task based metric to help understand MTL results using STL results.
\squishlist
	\item For each task, we train a single-task model.
	Let $z_s$ and $z_t$ be the prediction accuracy of each task, respectively.
	Let $\tau\in(0, 1)$ be a fixed threshold.
	\item If $z_s - z_t > \tau$, then we predict that there will be positive transfer when combining the two tasks using MTL.
	If $z_s - z_t < -\tau$, then we predict negative transfer.
\squishend


\textbf{Connection to Taskonomy.} We use our tools to explain a key result of Taskonomy by Zamir et al. \cite{ZSSGM18}, which shows that MTL can reduce the amount of labeled data needed to achieve comparable performance to STL.
%More precisely, suppose we have $n_i$ datapoints for each task, for $i= 1, 2$.
For $i = 1, 2$, let $\hat{\beta}_i^{\MTL}(x)$ denote the estimator trained using $x \cdot n_i$ datapoints from every task. The data efficiency ratio is defined as
{\small\[ \argmin_{x\in(0, 1)} ~~
		\te_1(\hat{\beta}_1^{\MTL}(x)) + \te_2(\hat{\beta}_2^{\MTL}(x))
		\le \te_1(\hat{\beta}_1^{\STL}) + \te_2(\hat{\beta}_2^{\STL}). \]}
For example, the data efficiency ratio is $1$ if there is negative transfer.
Using our tools, we show that in the isotropic model, the data efficiency ratio is
roughly
{\small\[ \frac{1}{\rho_1 + \rho_2} {+ \frac{2}{(\rho_1 +\rho_2)(\rho_1^{-1} + \rho_2^{-1} - \Theta(\Psi(\beta_1, \beta_2)))}}. \]}%
Compared with Proposition \ref{prop_dist_transition}, we see that when $\Psi(\beta_1, \beta_2)$ is smaller than $\rho_1^{-1} + \rho_2^{-1}$ (up to a constant multiple), the transfer is positive.
Moreover, the data efficiency ratio quantifies how effective the positive transfer is using MTL.
%In addition to $\rho_1,\rho_2$, the data efficiency ratio also scales with $\Psi(\beta_1, \beta_2)$, the model distance between the tasks.


Next we state Proposition \ref{prop_data_efficiency}, which gives precise upper and lower bounds on the data efficiency ratio for Taskonomy. Its proof can be found in Appendix \ref{app_proof_32}. %In the statement, we shall denote the data efficiency ratio as $\al^\star$.

\begin{proposition}%[Labeled data efficiency]
\label{prop_data_efficiency}
	In the isotropic model, assume that $\rho_1,\rho_2 \ge 9$ and $\Psi(\beta_1, \beta_2) < (5(\rho_1-1))^{-1} + (5(\rho_2-1))^{-1}$.
	Then the data efficiency ratio $x^\star$ satisfies
	\be\label{eq_uplowx} x_l  \le x^\star\le \frac{1}{\rho_1 + \rho_2} \bigbrace{  \frac{2}{(\rho_1-1)^{-1} + (\rho_2-1)^{-1} - 5\Psi(\beta_1, \beta_2)}+1}, \ee
	where we denoted
	$$x_l:= \frac1{\rho_1+\rho_2}\left(\frac{2}{(\rho_1-1)^{-1}+(\rho_2 -1)^{-1}}+1\right).$$
\end{proposition}



\textbf{Algorithmic consequence.} An interesting consequence of Proposition \ref{prop_data_size} is that $L(\hat{\beta}_t^{\MTL})$ is not monotone in $\rho_1$.
In particular, Figure \ref{fig_size} (and our analysis) shows that $L(\hat{\beta}_t^{\MTL})$ behaves as a quadratic function over $\rho_1$.
More generally, depending on how large $\Psi(\beta_1, \beta_2)$ is, $L(\hat{\beta}_t^{\MTL})$ may also be monotonically increasing or decreasing.
Based on this insight, we propose an incremental optimization schedule to improve MTL training efficiency.
\squishlist
	\item We divide the source task data into $S$ batches.
	For $S$ rounds, we incrementally add the source task data by adding one batch at a time.
	\item After training $T$ epochs, if the validation accuracy becomes worse than the previous round's result, we terminate.
	Algorithm \ref{alg_inc_train} in Appendix \ref{app_experiments} describes the procedure in detail.
\squishend

\begin{algorithm}[!t]
	\caption{An incremental training schedule for efficient multi-task learning with two tasks}
	\label{alg_inc_train}
	\begin{algorithmic}[1]
		\Input Two tasks $(X_1, Y_1)$ and $(X_2, Y_2)$.
		\Param A shared module $B$, output layers $W_1, W_2$ as in the hard parameter sharing architecture.
		\Req \# batches $S$, epochs $T$, task $2$'s validation accuracy $\hat{g}(B; W_2)$, a threshold $\tau\in(0,1)$.
		\Output The trained modules $B, W_2$ optimized for task $2$.
		\State Divide $(X_1, Y_1)$ randomly into $S$ batches: $(x^{(1)}, y^{(1)}), \dots, (x^{(S)}, y^{(S)})$.
		\For{$i = 1,\dots, S$}
			\For{$j = 1,\dots, T$}
				\State Update $B, W_1, W_2$ using the training data $\set{x^{(k)}, x^{(k)}}_{k=1}^i$ and  $(X_2, Y_2)$.
			\EndFor
			\State Let $a_i = \hat{g}(B; W_2)$ be the validation accuracy.
			\If{$a_i < a_{i-1}$ or $a_i > \tau$}
				\State \textbf{break}
			\EndIf
		\EndFor
	\end{algorithmic}
\end{algorithm}


\textbf{Algorithmic consequence.}
Our observation highlights the need to correct covariate shift when $n_1 / n_2$ is large.
Hence for such settings, we expect procedures that aim at correcting covariate shift to provide more significant gains.
We consider a covariance alignment procedure proposed in \cite{WZR20}, which is designed for the purpose of correcting covariate shift.
The idea is to add an alignment module between the input and the shared module $B$.
This new module is then trained together with $B$ and the output layers.
We validate our insight on this procedure in the experiments.

