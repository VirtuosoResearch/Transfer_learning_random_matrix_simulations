\section{Proof Overview of Main Results}\label{sec_general}

%\begin{proposition}[Bias-variance tradeoff]
%	Variance always reduces and bias always increases.
%\end{proposition}
We describe the main ideas for showing the generalization error of hard parameter sharing estimators.
There are two key ideas in the proof.
First, we study a bias-variance decomposition of hard parameter sharing, and show that the variance always reduces compared to single-task learning, while the bias always increases.
Second, we study the limiting bias and variance of hard parameter sharing, and derive their asymptotics using recent development from the random matrix theory literature.
The asymptotic results scale with key properties of task data such as sample size and covariance shift, and allow us to study the performance of multi-task learning by varying these properties, which will be the focus of Section \ref{sec_special}.
%In this section, we compare the prediction loss of the multi-task learning estimator to that of the single-task learning estimator.
%First, we consider the two-task case.
%We provide precise asymptotics of the bias and variance of the multi-task learning estimator.
%We apply recent developments from the random matrix theory literature to characterize the bias and variance.
%The results scale with key properties of task data such as sample size and covariance shift, and allow us to study the performance of multi-task learning by varying these properties (Section \ref{sec_special}).
%Second, we consider the multiple-task case where all tasks have the same features but different labels.
%We extend the bias-variance decomposition of the two-task case to this setting and show qualitatively similar results.
%For the single-task learning estimator, there are well-known results that relate its prediction loss to sample size and noise variance.

%We focus on the high-dimensional linear regression setting \cite{}






\subsection{Same Covariates}
%A well-known result in the high-dimensional linear regression setting states that $\tr[(X_2^{\top}X_2)^{-1}\Sigma_2]$ is concentrated around $1 / (\rho_2 - 1)$ (e.g. Chapter 6 of \cite{S07}), which scales with the sample size of the target task.
%Our main technical contribution is to extend this result to two tasks.
%We show how the variance of the multi-task estimator scales with sample size and covariate shift in the following result.
Next, we describe our result for the multiple-task case where all tasks have the same covariates, that is, $X_i = X$, $n_i = n$, and $\Sigma_i = \Sigma$ for all $i = 1, \dots, t$.
This setting is prevalent in applications of multi-task learning to image classification, where there are multiple prediction labels/tasks for every image \cite{chexnet17,EA20}.
Similar to the two-task case, we consider an arbitrary local minimum $B, W_1, \dots, W_2$ of the optimization objective.
We extend the bias-variance decomposition from the two-task case to the multiple-task case.
Let $\cal W = [W_1, W_2, \dots, W_t] \in \real^{r\times t}$ be a matrix that contains all the parameters of the output layers.
We observe that the expected prediction loss of $\hat{\beta}_t^{\MTL}$ conditional on $X$ consists of a bias and a variance equation as follows
\begin{align}
	\exarg{\varepsilon_1, \dots, \varepsilon_t}{L(\hat{\beta}_t^{\MTL}) \mid X}
	=& \bignorm{\Sigma^{1/2} \bigbrace{B^{\star} \cW^{\top} (\cW \cW^{\top})^{-1} W_t - \beta_t}}^2 \label{eq_bias_multiple} \\
	&+ \sigma^2 \cdot (W_t^{\top} (\cW \cW^{\top})^{-1} W_t) \cdot \bigtr{\Sigma (X^{\top} X)^{-1}} \label{eq_var_multiple}
\end{align}
One can see that equation \eqref{eq_bias_multiple} is the bias of the multi-task learning estimator and equation \eqref{eq_var_multiple} is its variance.
Compared to the prediction loss of single-task learning (cf. equation \eqref{eq_var_stl}), we observe that the variance equation \eqref{eq_var_multiple} is always smaller because $W_t^{\top} (\cW \cW^{\top})^{-1} W_t \le 1$.
On the other hand, the bias equation \eqref{eq_bias_multiple} is always larger because of the difference between the task models.

Our main result for the multiple-task case provides generalization bounds that relate the empirical bias and variance of multi-task learning to the sample size $n$ and the distance between task models.
Before stating the result, we define the following notations.
Let $B^\star := [{\beta}_1,{\beta}_2,\dots,{\beta}_{t}] \in \real^{p\times t}$ denote the ground truth model parameters.
Let $U_r U_r^{\top}$ denote the best rank-$r$ subspace approximation of $(B^{\star})^\top\Sigma B^{\star}$, that is,
\[ U_r \define \argmin_{X\in\real^{t\times r} : X^{\top} X = \id_{r\times r}} \inner{X X^{\top}} {B^{\star} \Sigma B^{\star}}. \]
For $i = 1,\dots, t$, let $v_i \in\real^r$ denote the $i$-th row of $U_r$.

\begin{theorem}[Multiple-task case with the same covariates]\label{thm_many_tasks}
%Suppose $X=Z\Sigma^{1/2}\in \R^{n\times p}$ satisfy Assumption \ref{assm_secA1} with $\rho:=n/p>1$ being some fixed constant. Consider data models  $Y_i = X\beta_i + \varepsilon_i$, $i=1,2,\cdots, t$, where $\e_i\in \R^{n}$ are random vectors with i.i.d. entries with mean zero, variance $\sigma^2$ and all moments as in \eqref{assmAhigh}. Moreover, assume that $X$, $\beta_i$ and $\e_i$ are all independent of each other.
	%Let $n = c \cdot p$.
	%Let $X\in\real^{n\times p}$ and $Y_i = X\beta_i + \varepsilon_i$, for $i = 1,\dots,k$.
%	Consider $t$ data models $Y_i = X\beta_i + \varepsilon_i$, $i=1,2,\cdots, t$, where $X$ has covariance matrix $\Sigma$, and the entries of $\e_i$ are i.i.d. with mean zero and variance $\sigma^2$.
	%that satisfy Assumption \ref{assm_secA2} in the appendix.
	Consider the multiple-task case where all tasks have the same covariates.
	Assume that $\lambda_{\min}({B^{\star}}^\top\Sigma B^{\star})\gtrsim \sigma^2$.
	Let $\delta$ be any fixed value such that $\delta \le \oo \left( \|B^\star\|^2 + \sigma^2\right)$.
	Then, the following is true:
	\begin{itemize}
		\item \textbf{Positive transfer:} If $\left(1 - \norm{v_t}^2 \right)\frac{\sigma^2}{\rho - 1} - \norm{\Sigma^{1/2} (B^{\star} U_r v_t - \beta_t)}^2 > \delta$, then w.h.p over the randomness of $X, \varepsilon_1, \dots, \varepsilon_t$, we have that
		 \[ \te(\hat{\beta}_t^{\MTL}) < \te(\hat{\beta}_t^{\STL}). \]
		\item \textbf{Negative transfer:} If $\left(1 - \norm{v_t}^2\right)\frac{\sigma^2}{\rho - 1} - \norm{\Sigma^{1/2}(B^{\star} U_r v_t - \beta_t)}^2 < -\delta$, then w.h.p. over the randomness of $X, \varepsilon_1, \dots, \varepsilon_t$, we have that
		\[ \te(\hat{\beta}_t^{\MTL}) > \te(\hat{\beta}_t^{\STL}). \]
	\end{itemize}
\end{theorem}
Theorem \ref{thm_many_tasks} provides a sharp analysis of the bias-variance tradeoff in the multiple-task case.
As shown in the proof of Theorem \ref{thm_many_tasks}, the equation $(1 - \norm{v_t}^2)\frac {\sigma^2}{\rho - 1}$ is the amount of variance reduced using multi-task learning and $\norm{\Sigma (B^{\star} U_r v_t - \beta_t)}$ is the bias.
The proof of \ref{thm_many_tasks} can be found in Section \ref{app_proof_many_tasks}.





\subsection{Different Covariates: The Two-task Case}
As mentioned in Section \ref{sec_prelim}, the optimization objective \eqref{eq_mtl} is non-convex in general.
Therefore, we consider an arbitrary local minimum $B, W_1, W_2$, in particular when $B = \hat{B}(W_1, W_2)$ satisfies the local optimality condition.
We derive deterministic conditions to compare the prediction loss of the local minimum to single-task learning.
Our key insight is a bias-variance decomposition of the expected prediction loss of $\hat{\beta}_2^{\MTL} = \hat{B} W_2$ as follows
\begin{align}
	\exarg{\epsilon_1, \epsilon_2}{L(\hat{\beta}_2^{\MTL}) \mid X_1, X_2}
	=&~ \frac{W_1^2}{W_2^2} \bignorm{\Sigma_2^{1/2}(\frac{W_1^2}{W_2^2} X_1^{\top}X_1 + X_2^{\top}X_2)^{-1} X_1^{\top}X_1 (\beta_1 - \frac{W_1}{W_2} \beta_2)}^2 \label{eq_bias_2task} \\
			&+~  \sigma^2\cdot \bigtr{\Sigma_2(\frac{W_1^2}{W_2^2} X_1^{\top}X_1 + X_2^{\top}X_2)^{-1} }. \label{eq_var_2task}
\end{align}
See Section \ref{app_proof_main_thm} for its derivation.
Equation \eqref{eq_bias_2task} is the bias of $\hat{\beta}_t^{\MTL}$ and
equation \eqref{eq_var_2task} is the variance of $\hat{\beta}_t^{\MTL}$.
%minus the variance of $\hat{\beta}_t^{\STL}$, which is always negative.
Comparing the above to single-task learning, that is,
\begin{align}
	\exarg{\epsilon_2}{L(\hat{\beta}_2^{\STL}) \mid X_2} = \sigma^2 \cdot \bigtr{\Sigma_2 (X_2^{\top} X_2)^{-1}}, \label{eq_var_stl}
\end{align}
we observe that while the bias of $\hat{\beta}_2^{\MTL}$ is always larger than that of $\hat{\beta}_2^{\STL}$, which is zero, the variance of $\hat{\beta}_2^{\MTL}$ is always lower than that of $\hat{\beta}_2^{\STL}$.\footnote{To see why this is true, we apply the Woodbury matrix identity over equation \eqref{eq_var_2task} and use the fact that for the product of two PSD matrices, its trace is always nonnegative.}
In other words, training both tasks together helps predict the target task by reducing variance while incurring a bias.
Therefore, whether multi-task learning outperforms single-task learning is determined by the bias-variance decomposition!


Based on the above intuition, in the following we develop generalization bounds that relate the bias and variance of multi-task learning to the sample sizes and the covariance matrices of both tasks.
We introduce a key quantity $M = \Sigma_1^{1/2}\Sigma_2^{-1/2}$ that captures the covariate shift between task $1$ and task $2$.

%We now state our main result for two tasks with both covariate and model shift. It shows that the information transfer is determined by two deterministic quantities $\Delta_{\bias}$ and $\Delta_{\vari}$, which give the change of model shift bias and the change of variance, respectively. The exact forms of $\Delta_{\bias}$ and $\Delta_{\vari}$ will be given in Lemma \ref{thm_model_shift}.

\begin{theorem}[Two-task case]\label{thm_main_informal}
	%For the setting of two tasks, let $\delta > 0$ be a fixed error margin, $\rho_2 > 1$ and $\rho_1 \gtrsim \delta^{-2}\cdot \lambda_{\min}(M)^{-4} \norm{\Sigma_1} \max(\norm{\beta_1}^2, \norm{\beta_2}^2)$.
	For the setting of two tasks, let $C$ be a fixed constant and let $\delta = \frac{ C\cdot \max(\norm{\beta_1}, \norm{\beta_2}) \cdot \sqrt{\norm{\Sigma_1}} } {\lambda_{\min}^2(M)) }$.
	Let $B, W_1, W_2$ be any local minimum of equation \eqref{eq_mtl}.
 	There exist two deterministic functions $\Delta_{\bias}$ and $\Delta_{\vari}$ that only depend on scaled model distance $\beta_1 - \frac{W_1}{W_2} \beta_2$, sample sizes $n_1 = \rho_1 \cdot p, n_2 = \rho_2 \cdot p$, and covariate shift matrix $M$ such that
	\begin{enumerate}
		\item[a)] \textbf{Positive transfer:} If $\Delta_{\bias} < \Delta_{\vari} -  \frac{\delta}{\sqrt{\rho_1}} $, then w.h.p. over the randomness of $X_1, X_2, \varepsilon_1, \varepsilon_2$, we have
			\[ \te(\hat{\beta}_2^{\MTL}) < \te(\hat{\beta}_2^{\STL}).  \]
		\item[b)] \textbf{Negative transfer:} If $\Delta_{\bias} > \Delta_{\vari} + \frac{\delta}{\sqrt{\rho_1}}$, then w.h.p. over the randomness of $X_1, X_2, \varepsilon_1, \varepsilon_2$, we have
			\[ \te(\hat{\beta}_2^{\MTL}) > \te(\hat{\beta}_2^{\STL}). \]
	\end{enumerate}
\end{theorem}
In words, the above result shows that a deterministic function $\Delta_{\bias} - \Delta_{\vari}$ determines whether the prediction loss of the empirical multi-task learning estimator is lower than that of single-task learning, up to an error that scales as $\delta / \sqrt{\rho_1}$.
We make several remarks about Theorem \ref{thm_main_informal}.
First, as the amount of data from the source task increases, we get more accurate predictions  since the error scales down.
This applies to many practical settings where collecting labeled data for the target task is expensive and auxillary (source) task data is easier to obtain.
%Theorem \ref{thm_main_informal} applies to settings where large amounts of source task data are available but the target sample size is small.
%For such settings, we obtain a sharp transition from positive transfer to negative transfer determined by $\Delta_{\bias} - \Delta_{\vari}$.
%determined by the covariate shift matrix and the model shift.
%The bounds get tighter and tighter as $\rho_1$ increases.
Second, the deterministic function $\Delta_{\bias} - \Delta_{\vari}$ depends on the three task properties that we care about and the precise form can be found in Section \ref{sec_proof_general}.
Finally, later on in Section \ref{sec_special}, we will study how varying each task property affects the performance of multi-task learning in depth.
%While the general form of these functions can be complex (as are previous generalization bounds for MTL), they admit interpretable forms for simplified settings.

%\textbf{Proof overview.}\todo{}
%Theorem \ref{lem_cov_shift_informal} extends a well-known result for the single-task setting when $X_1, \rho_1, a_1$ are all equal to zero \cite{S07}.
%Applying Theorem \ref{lem_cov_shift_informal} to \eqref{eq_te_var}, we can calculate the amount of reduced variance compared to STL, which is given asymptotically by $\Delta_{\vari}$.
%For the bias term in equation \eqref{eq_te_model_shift}, we apply the approximate isometry property to $X_1^{\top}X_1$, which is close to $n_1^2\Sigma_1$. This results in the error term $\delta$, which scales as $(1 + 1/\sqrt{\rho_1})^4-1$.
%Then, we apply a similar identity to Theorem \ref{lem_cov_shift_informal} for bounding the bias term, noting that the derivative of $R(z)$ with respect to $z$ can be approximated by $R_\infty'(z)$.
%This estimates the negative effect given by $\Delta_{\bias}$. %, which will be used to estimate the first term on the right hand side of \eqref{eq_te_model_shift}.
%During this process, we will get the $\Delta_{\bias}$ term up to an error $\delta$ depending on $\rho_1$.
%The proof of Theorem \ref{thm_main_informal} is presented in Appendix \ref{app_proof_main_thm} and the proof of Lemma \ref{lem_cov_shift_informal} is in Appendix \ref{sec_maintools}.
%
%
%
%The formal statement is stated in Theorem \ref{thm_many_tasks} and its proof can be found in Appendix \ref{app_proof_many_tasks}.
%The technical crux of our approach is to derive the asymptotic limit of $\te(\hat{\beta}_t^{\MTL})$ in the high-dimensional setting, when $p$ approaches infinity.
%We derive a precise limit of $\bigtr{(X_1^{\top}X_1 + X_2^{\top}X_2)^{-1}\Sigma_2}$, which is a deterministic function that only depends on $\Sigma_1, \Sigma_2$ and $n_1/p, n_2/p$ (see Lemma \ref{lem_cov_shift} in Appendix \ref{app_proof_main} for the result).
%Based on the result, we show how to determine positive versus negative transfer as follows.
%, where $\lambda_{\min}(M)$ is the smallest singular value of $M_1$

\iffalse
Next, we derive a closed-form solution of the multi-task learning estimator for the case of two tasks.
From \cite{WZR20}, we know that we need to explicitly restrict the output dimension $r$ of $B$ so that there is transfer between the two tasks.
Hence for the case of two tasks, we consider the setting where $r=1$.
For simplicity of notations, we shall denote $(X_i^{tr},Y_i^{tr})$ and $(X_i^{val},Y_i^{val})$ as $(X_i,Y_i)$ and  $(\wt X_i,\wt Y_i)$, respectively. Then equation \eqref{eq_mtl} simplifies to
\begin{align}\label{eq_mtl_2task}
	f(B; w_1, w_2) = \bignorm{X_1 B w_1 - Y_1}^2 + \bignorm{X_2 B w_2 - Y_2}^2,
\end{align}
where $B\in\real^p$ and $w_1, w_2$ are both real numbers. To solve the above problem, suppose that $w_1, w_2$ are fixed, by local optimality, we find the optimal $B$ as
\begin{align}
	& \hat{B}(w_1, w_2) = (w_1^2 X_1^{\top}X_1 + w_2^2 X_2^{\top}X_2)^{-1} (w_1 X_1^{\top}Y_1 + w_2 X_2^{\top}Y_2) \label{hatB}\\
	&= \frac{1}{w_2} \left( \frac{w_1^2}{w_2^2}  X_1^{\top}X_1 + X_2^{\top}X_2\right)^{-1} \left(\frac{w_1}{w_2} X_1^{\top}Y_1 + X_2^{\top}Y_2\right) \nonumber\\
	&= \frac{1}{w_2}\left[\beta_2 + \left(\frac{w_1^2}{w_2^2} X_1^{\top}X_1 + X_2^{\top}X_2\right)^{-1}\bigbrace{X_1^{\top}X_1\left(\frac{w_1}{w_2}\beta_1 - \frac{w_1^2}{w_2^2} \beta_2\right) + \left(\frac{w_1}{w_2} X_1^{\top}\varepsilon_1 + X_2^{\top}\varepsilon_2\right)}\right]. \nonumber
\end{align}
As a remark, when $w_1 = w_2 = 1$, we obtain linear regression.
If $\beta_1$ is a scaling of $\beta_2$, then  $w_1, w_2$ can be scaled accordingly to fix both tasks more accurately than linear regression.

%For the discussions below, we assume that the entries of $\e_1$ and $\e_2$ all have the same variance $\sigma^2$. This holds for most parts of our discussion, except in Proposition \ref{prop_var_transition}. We will derive different expressions for the validation loss and the test error

Next we consider $N_i$ independent samples of the training set $\{(\wt x_k^{(i)},\wt y_k^{(i)}): 1\le k \le N_i\}$ from task-$i$, $i=1,2$. With these sample, we form the random matrices $\wt X_i \in \R^{N_i\times p}$ and $\wt Y_i\in \R^{N_i}$, $i=1,2,$ whose row vectors are given by $\wt x_k^{(i)}$ and $\wt y_k^{(i)}$. We assume that $N_1$ and $N_2$ satisfy $N_1/N_2=n_1/n_2$ and $N_i \ge n_i^{1-\e_0}$ for some constant $\e_0>0$. Then we write the validation loss in \eqref{eq_mtl_eval} as
\begin{align}\label{eq_mtl_2tasktilde}
	g(w_1,w_2) = \bignorm{\wt X_1 \hat B w_1 - \wt Y_1}^2 + \bignorm{\wt X_2 \hat B w_2 - \wt Y_2}^2.
\end{align}
Inserting \eqref{hatB} into \eqref{eq_mtl_2tasktilde}, one can see that the optimal solution of $g$ only depends on the ratio $v:=w_1/w_2$.
Hence we overload the notation by writing $g(v)$ in the following discussion.
The expectation of $g(v)$ can be written as follows.
\begin{align}
		\val(v) \define& \exarg{\varepsilon_1,\e_2} {\sum_{i=1}^2 \left\|\Sigma_i^{1/2}( \hat B w_i - \beta_i) \right\|^2} \nonumber\\
	=&  N_1 \cdot \bignorm{\Sigma_1^{1/2}\left(v^2 X_1^{\top}X_1 + X_2^{\top}X_2\right)^{-1}X_2^{\top}X_2\left (\beta_1 - v\beta_2\right)}^2 \nonumber \\
	&+ N_2 \cdot v^2\bignorm{\Sigma_2^{1/2}\left(v^2 X_1^{\top}X_1 + X_2^{\top}X_2\right)^{-1}X_1^{\top}X_1\left(\beta_1 - v\beta_2\right)}^2 \nonumber \\
		&+ N_1   \cdot v^2 \bigtr{\Sigma_1\left(v^2 X_1^{\top}X_1 + X_2^{\top}X_2\right)^{-2} \left(\sigma_1^2 \cdot v^2X_1^{\top}X_1 + \sigma_2^2 \cdot X_2^{\top}X_2\right)} \nonumber \\
		&+ N_2  \cdot \bigtr{\Sigma_2\left(v^2 X_1^{\top}X_1 + X_2^{\top}X_2\right)^{-2} \left(\sigma_1^2 \cdot v^2  X_1^{\top}X_1 + \sigma_2^2  \cdot X_2^{\top}X_2\right)}. \label{revise_eq_val_mtl}
\end{align}

{\color{red}\begin{align*}
			& f(W_1, W_2) = \bignorm{X_1 \hat B w_1 - Y_1}^2 + \bignorm{X_2 \hat B w_2 - Y_2}^2\\
			& =\bignorm{X_1\left( v^2 X_1^{\top}X_1 + X_2^{\top}X_2\right)^{-1} \left(v^2 X_1^{\top}Y_1 + vX_2^{\top}Y_2\right) - Y_1}^2 \\
			&+ \bignorm{X_2 \left( v^2 X_1^{\top}X_1 + X_2^{\top}X_2\right)^{-1} \left(vX_1^{\top}Y_1 + X_2^{\top}Y_2\right) - Y_2}^2 \\
			& =\bignorm{X_1\left( v^2 X_1^{\top}X_1 + X_2^{\top}X_2\right)^{-1} \left(v^2 X_1^{\top}\e_1 + vX_2^{\top}\e_2\right) - \e_1 + X_1\left( v^2 X_1^{\top}X_1 + X_2^{\top}X_2\right)^{-1}  X_2^{\top}X_2(v\beta_2-\beta_1) }^2 \\
			&+ \bignorm{X_2 \left( v^2 X_1^{\top}X_1 + X_2^{\top}X_2\right)^{-1} \left(vX_1^{\top}\e_1 + X_2^{\top}\e_2\right) - \e_2 + vX_2 \left( v^2 X_1^{\top}X_1 + X_2^{\top}X_2\right)^{-1}  X_1^{\top}X_1(\beta_1-v\beta_2) }^2 \\
			&=\val(v)\cdot \left( 1+\OO(p^{-1/2\e})\right) \quad \text{w.h.p.},
		\end{align*}
		where
		\begin{align*}
		\val(v)&=\bignorm{X_1\left( v^2 X_1^{\top}X_1 + X_2^{\top}X_2\right)^{-1}  X_2^{\top}X_2(v\beta_2-\beta_1) }^2 \\
			&+ v^2\bignorm{X_2 \left( v^2 X_1^{\top}X_1 + X_2^{\top}X_2\right)^{-1}  X_1^{\top}X_1(\beta_1-v\beta_2) }^2 \\
			&+\sigma^2 \tr\left(v^2 X_1\left( v^2 X_1^{\top}X_1 + X_2^{\top}X_2\right)^{-1}   X_1^{\top} -\id\right)^2 \\
			&+\sigma^2 \tr\left(v^2 X_1\left( v^2 X_1^{\top}X_1 + X_2^{\top}X_2\right)^{-1}X_2^{\top}X_2 \left( v^2 X_1^{\top}X_1 + X_2^{\top}X_2\right)^{-1} X_1^\top\right)\\
			&+\sigma^2 \tr\left(X_2 \left( v^2 X_1^{\top}X_1 + X_2^{\top}X_2\right)^{-1} X_2^{\top} -\id\right)^2 \\
			&+\sigma^2 \tr\left(v^2 X_2\left( v^2 X_1^{\top}X_1 + X_2^{\top}X_2\right)^{-1}X_1^{\top}X_1 \left( v^2 X_1^{\top}X_1 + X_2^{\top}X_2\right)^{-1} X_2^\top\right)\\
			&=\bignorm{X_1\left( v^2 X_1^{\top}X_1 + X_2^{\top}X_2\right)^{-1}  X_2^{\top}X_2(v\beta_2-\beta_1) }^2 \\
			&+ v^2\bignorm{X_2 \left( v^2 X_1^{\top}X_1 + X_2^{\top}X_2\right)^{-1}  X_1^{\top}X_1(\beta_1-v\beta_2) }^2 +(n_1+n_2-p)\sigma^2.
		\end{align*}
	}





Hence to minimize $g(v)$, it suffices to minimize $\val(v)$ over $v$.
Let $\hat v=\hat{w_1}/\hat{w_2}$ be the global minimizer of $g(v)$.
Now we can define the multi-task learning estimator for the target task as
	\[ \hat{\beta}_2^{\MTL} = \hat{w}_{2}\hat{B}(\hat{w}_1, \hat{w}_2) .\]
%	where $t=2$ since we are considering the two task case, and it also stands for the ``target task".
%The intuition for deriving $\hat{\beta}_2^{\MTL}$ is akin to performing multi-task training in practice.
%Let $\hat{v} = \hat{w_1} / \hat{w_2}$ for the simplicity of notation.
The prediction loss of using $\hat{\beta}_2^{\MTL}$ for the target task is
\begin{align}
	\te(\hat{\beta}_2^{\MTL}) =&~ \hat{v}^2 \bignorm{\Sigma_2^{1/2}(\hat{v}^2 X_1^{\top}X_1 + X_2^{\top}X_2)^{-1} X_1^{\top}X_1 (\beta_1 - \hat{v} \beta_2)}^2 \nonumber \\
			&+~  \bigtr{\Sigma_2(\hat{v}^2 X_1^{\top}X_1 + X_2^{\top}X_2)^{-2}\left(\sigma_1^2 \cdot \hat v^2  X_1^{\top}X_1 + \sigma_2^2  \cdot X_2^{\top}X_2\right) }, \label{eq_te_mtl_2task}
\end{align}
which only depends on $\hat v$, the sample covariance matrices, and $\beta_1,\beta_2$.
\fi

%\begin{lemma}[Variance bound]\label{lem_cov_shift_informal}
%	In the setting of two tasks,
%	let $n_1 = \rho_1 \cdot p$ and $n_2 = \rho_2 \cdot$ be the sample size of the two tasks.
%	Let $\lambda_1, \dots, \lambda_p$ be the singular values of the covariate shift matrix $\Sigma_1^{1/2}\Sigma_2^{-1/2}$ in decreasing order.
%	%let $n_1 = \rho_1 \cdot p$ and $n_2 = \rho_2 \cdot p$ denote the sample sizes of each task.
%	%Let $\Sigma_1$ and $\Sigma_2$ denote the covariance matrix of each task.
%	With high probability, the variance of the multi-task estimator $\hat{\beta}_t^{\MTL}$ equals
%	%let $M = \Sigma_1^{1/2}\Sigma_2^{-1/2}$ and $\lambda_1, \lambda_2, \dots, \lambda_p$ be the singular values of $M^{\top}M$ in descending order.
%%	For any constant $\e>0$, w.h.p. over the randomness of $X_1, X_2$, we have that
%	{\small\begin{align*}%\label{eq_introX1X2}
%		%\bigtr{(X_1^{\top}X_1 + X_2^{\top}X_2)^{-1}\Sigma_2} =
%		\frac{\sigma^2}{n_1+n_2}\cdot \bigtr{ (\hat{v}^2 a_1 \Sigma_2^{-1/2}\Sigma_1\Sigma_2^{-1/2} + a_2\id)^{-1}} +\bigo{{p^{-1/2+o(1)}}},
%	\end{align*}}%
%	where $a_1, a_2$ are solutions of the following equations:
%	{\small\begin{align*}
%		a_1 + a_2 = 1- \frac{1}{\rho_1 + \rho_2},\quad a_1 + \frac1{\rho_1 + \rho_2}\cdot \frac{1}{p}\sum_{i=1}^p \frac{\hat{v}^2\lambda_i^2 a_1}{\hat{v}^2\lambda_i^2 a_1 + a_2} = \frac{\rho_1}{\rho_1 + \rho_2}.
%	\end{align*}}
%%are both fixed values that roughly scales with the sample sizes $\rho_1, \rho_2$, and satisfy $a_1 + a_2 = 1 - (\rho_1 + \rho_2)^{-1}$ plus another deterministic equation.
%\end{lemma}

\paragraph{Key ingredients.}
We introduce two key lemmas that show tight asymptotic convergence rate of the bias equation \eqref{eq_bias_2task} and the variance equation  as sample sizes increase to infinity.
%The following two lemmas  are the main random matrix theoretical results of this paper, which will be used to estimate the two terms on the right-hand side of \eqref{eq_te_mtl_2task}.
For the covariate shift matrix $M$, let $\lambda_1, \lambda_2, \dots, \lambda_p$ be its singular values in descending order.
%which deals with the inverse of the sum of two random matrices, which
%any is can be viewed as a special case of Theorem \ref{thm_model_shift}.

\begin{lemma}[Variance asymptotics]\label{lem_cov_shift}
	%Let $X_i\in\real^{n_i\times p}$ be a random matrix that contains i.i.d. row vectors with mean $0$ and variance $\Sigma_i\in\real^{p\times p}$, for $i = 1, 2$.
	%Suppose $X_1=Z_1\Sigma_1^{1/2}\in \R^{n_1\times p}$ and $X_2=Z_2\Sigma_2^{1/2}\in \R^{n_2\times p}$ satisfy Assumption \ref{assm_secA1} with $\rho_1:=n_1/p>1$ and $\rho_2:=n_2/p>1$ being fixed constants.
	%Denote by $M = \Sigma_1^{1/2}\Sigma_2^{-1/2}$ and
	Let $\Sigma \in \real^{p \times p}$ be any fixed and deterministic matrix.
	In the setting of Theorem \ref{thm_main_informal},
	with probability $1-\oo(1)$ over the randomness of $X_1$ and $X_2$, we have that %for any constant $\e>0$,
	%When $n_1 = c_1 p$ and $n_2 = c_2 p$, we have that with high probability over the randomness of $X_1$ and $X_2$, the following equation holds
	\begin{align}\label{lem_cov_shift_eq}
		\bigtr{(X_1^{\top}X_1 + X_2^{\top}X_2)^{-1} \Sigma} = \frac{1}{n_1 + n_2}\cdot \bigtr{ (a_1 \Sigma_1 + a_2\Sigma_2)^{-1} \Sigma} + o(\norm{\Sigma}). %\bigo{\|\| p^{-1/2+\epsilon}}
	\end{align}
	In the above equation, $a_1$ and $a_2$ are the solutions of the following two equations that only depend on the sample sizes $\rho_1, \rho_2$, and the singluar values of the covariate shift matrix $M$:
	\begin{align}
		a_1 + a_2 = 1- \frac{1}{\rho_1 + \rho_2},\quad a_1 + \frac1{\rho_1 + \rho_2}\cdot \bigbrace{\frac{1}{p}\sum_{i=1}^p \frac{\lambda_i^2 a_1}{\lambda_i^2 a_1 + a_2}} = \frac{\rho_1}{\rho_1 + \rho_2}. \label{eq_a12extra}
	\end{align}
\end{lemma}

Lemma \ref{lem_cov_shift} derives the asymptotic limit of the variance equation \eqref{eq_var_2task}.
To see this, we rescale $X_1$ with $W_1 / W_2$ and set the matrix $\Sigma$ as $\sigma^2 \Sigma_2$.
%allows us to get a tight bound on equation \eqref{eq_te_var}, that only depends on \textit{sample size}, \textit{covariate shift} and the scalar $\hat{v}$.
As a remark, the concentration error $o(\|A\|)$ on the right hand side of equation \eqref{lem_cov_shift_eq} reduces provided with stronger moment assumptions on $X_1$ and $X_2$. % p^{-1/2+\epsilon}
Recall from Section \ref{sec_prelim} that the feature vectors are generated by $\Sigma_i^{1/2} z$, where $z\in\real^p$ consists of i.i.d. entries with mean zero and unit variance.
Provided that for every entry of $z$, its $k$-th moment exists, then we can obtain a concentration error at most $\bigo{\norm{A} p^{-1/2 + 2/k + o(1)}}$.
\todo{We remark that one can probably derive the same asymptotic result using free probability theory (see e.g. \cite{nica2006lectures}), but our results \eqref{lem_cov_shift_eq} and \eqref{lem_cov_derv_eq} also give an almost sharp error bound $\bigo{ p^{-1/2+\epsilon}}$.}

Lemma \ref{lem_cov_shift} also implies the asymptotic limit of the prediction loss of single-task learning.
%The next lemma, which is , helps to determine the asymptotic limit of $\te(\hat{\beta}_t^{\STL})=\sigma^2   \bigtr{(X_2^{\top}X_2)^{-1}\Sigma_2}$ as $p\to \infty$.
In particular, by setting $X_1 = 0$, we obtain the following corollary, which is a well-known result in random matrix theory.
\begin{corollary}[See e.g. Theorem 2.4 in \citet{isotropic}]\label{lem_minv}
	In the setting of Lemma \ref{lem_cov_shift}, with high probability we have that
	\be\label{XXA}  \bigtr{(X_2^{\top}X_2)^{-1}A} = \frac{1}{n_2 - p} \cdot \bigtr{\Sigma_2^{-1}A} + o(\norm{A}). \ee
\end{corollary} %\bigo{ \|A\|p^{-1/2+\epsilon}}
To see this, note that when $n_1=0$, $a_1 = 0$ and $a_2 = (n_2-p) / n_2$ is the solution of \eqref{eq_a12extra}, and one can see that equation \eqref{lem_cov_shift_eq} reduces to equation \eqref{XXA}.
We briefly describe the history behind the above well-known result.
When the entries of $X_2$ are multivariate Gaussian, this result recovers the classical result for the mean of inverse Wishart distribution \cite{anderson1958introduction}.
For general non-Gaussian random matrices, it can be obtained using Stieltjes transform method; see e.g., Lemma 3.11 of \cite{bai2009spectral}.
Here we have stated a result from Theorem 2.4 in \cite{isotropic}, which gives an almost sharp concentration error bound.
One can see that our result extends Lemma \ref{lem_minv} from a single sample covariance matrix to the sum of two independent sample covariance matrices.

Interestingly, the asymptotic limit of the variance only depends on the sample sizes and the covariate shift matrix.
Next, we derive the asymptotic limit of the bias equation \eqref{eq_bias_2task}, which depends on the scaled model distance in addition to sample size and covariate shift. %(cf. Lemma \ref{lem_cov_derivative} in Appendix \ref{app_proof_main_thm}).
We will use the matrix fractional notation $\frac{X}{Y}$ for two PSD matrices that share the same eigenspace.
That is, suppose that the SVD of $X, Y$ is $X = U D_1 U^{\top}$ and $Y = U D_2 U^{\top}$.
We denote $\frac{X}{Y} = U D_{1} D_2^{-1} U^{\top}$.
Using the notation, we introduce an important quantity that describes the limit of the bias equation as $p$ goes to infinity.

\begin{definition}[Limiting bias]
Recall that $a_1$ and $a_2$ are the solutions of the equation system \eqref{eq_a12extra}. Let
	\[ \Pi \define  M \frac{a_3 M^{\top}M + (1 + a_4) \id_{p\times p}}{(a_1 M^{\top}M + a_2\id_{p\times p})^2} M^{\top}, \]
where $a_{3}$ and $a_4$ are the solutions of the following two equations:
\begin{gather}\label{eq_a34extra}
		\left(\frac{\rho_2}{a_2^{2}}-  b_0\right)\cdot  a_4 - b_1 \cdot  a_3
		= b_0, \quad \left(\frac{\rho_1}{a_1^{2}} -  b_2  \right)\cdot  a_3 -  b_1 \cdot  a_4 = b_1 .
%		\left(\frac{n_1}{\hat a_1^2} -  \sum_{i=1}^p \frac{\hat \lambda_i^4   }{  (\hat a_2 + \hat \lambda_i^2\hat a_1)^2  }\right)\hat a_4 -\left(\sum_{i=1}^p \frac{\hat \lambda_i^2  }{  (\hat a_2 + \hat \lambda_i^2\hat a_1)^2  }\right)\hat a_3
%		= \sum_{i=1}^p \frac{\hat \lambda_i^2 }{  (\hat a_2 + \hat \lambda_i^2\hat a_1)^2  }. \label{eq_a4}
	\end{gather}
In the above equation, we denote $b_k$ as $\frac1{p}\sum_{i=1}^p {\lambda_i^{2k}} / { (\lambda_i^2 a_1 + a_2)^2  }$, for $k = 0, 1, 2$.
\end{definition}

Our next result shows that the bias equation \eqref{eq_bias_2task} is described by the above limiting bias asymptotically.
\begin{lemma}[Bias asymptotics]\label{lem_cov_derivative}
In the setting of Theorem \ref{thm_main_informal}, let $w \in \R^p$ be any vector that is independent of $X_1$ and $X_2$.
With high probability, we have that
\begin{equation}\label{lem_cov_derv_eq}
\begin{split}
\bignorm{\Sigma_2^{1/2} \bigbrace{\frac{X_1^{\top}X_1 + X_2^{\top}X_2}{n_1 + n_2}}^{-1} \Sigma_1^{1/2} w}^2
= w^{\top} \Pi w + o(\|w\|^2),
\end{split}
\end{equation}

\end{lemma}
To apply the above result to equation \eqref{eq_bias_2task}, we first approximate $X_1^{\top}X_1$ by $\Sigma_1$, and then use Lemma \ref{lem_cov_derivative} with $w = \Sigma_1 (\beta_1 - \frac{W_1}{W_2} \beta_2)$.
Note that we can rescale $X_1$ by $W_1 / W_2$ and $\Sigma_1$ by $(W_1 / W_2)^2$, and the rest of the steps follows.
As a remark, the approximation of $X_1^{\top}X_1$ by $\Sigma_1$ incurs a generalization error that scales down with $\rho_1$, which is the $\delta / \sqrt{\rho_1}$ term in Theorem \ref{thm_main_informal} more precisely.
Finally, while the limiting bias $\Pi$ provides a precise connection with general covariate shift matrices $M$, it can be difficult to interpret.
In Section \ref{sec_special}, we show that $\Pi$ can be significantly simplified in an isotropic covariance setting.

In Section \ref{app_proof_main_thm}, we will describe briefly the main ideas in the proof of Lemma \ref{lem_cov_shift} and Lemma \ref{lem_cov_derivative}. In particular, we will give a simple (although not totally rigorous) derivation of the equations \eqref{eq_a12extra} and \eqref{eq_a34extra}.











