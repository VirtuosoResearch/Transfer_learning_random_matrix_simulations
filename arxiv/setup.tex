\subsection{Setup and Main Results}
\label{sec_prelim}

We begin by defining our problem setup including the multi-task estimator we study.
Then, we describe the bias-variance tradeoff of the multi-task estimator and connect the bias and variance of the estimator to \textit{task similarity}, \textit{sample size}, and \textit{covariate shift}.
%Finally, we show a tight concentration bound for the bias and variance quantities using random matrix theory.

%\subsection{Problem Formulation}

\todo{Consider adding some motivation later.}

\noindent\textbf{Data model.}
Suppose we have $t$ labeled regression tasks. %where $t$ is a fixed value that does not grow with the feature dimension $p$.
For each task $k$ from $1$ to $t$, we have $n_k$ samples $x_1, x_2, \dots, x_{n_k}$ that are all $p$-dimensional feature vectors with real-valued labels $y_1, y_2, \dots, y_{n_k}$.
Without loss of generality, we focus on predicting the $t$-th task and refer the $t$-th task as the target.
We refer to task $1$ to task $t-1$ as source tasks.
We assume that the labels of each task $k$ satisfy the linear model with unknown parameters $\beta_k\in\real^p$, that is,
	\[ y_i = x_i^{\top}\beta_k + e_i, \text{ for all } 1 \le i \le n_k, \]
where $e_i$ denotes random noise with mean zero and variance $\sigma^2$.
To write the above notations more succiently, let $X_k \in \real^{n_k \times p}$ denote the covariates that consists of a feature vector in every row.
Let
	\[ Y_k = X_k \beta_k + \varepsilon_k, \text{ for all } 1\le k \le t \] denote the label vector of task $k$ and $\varepsilon_k$ denote an i.i.d. random vector with mean zero and variance $\sigma^2$.

Following \citet{HMRT19} and \citet{BLLT20},
%In the high-dimensional linear regression setting (e.g. ), the features of the $k$-th task, denoted by $X_k\in\real^{n_i\times p}$, consist of $n_k$ feature vectors given by $x_1, x_2, \dots, x_{n_k}$.
we assume that for each task $k$ from $1$ to $t$, each feature vector $x_i = \Sigma^{1/2}_k z_i$ for $i = 1, \dots, n_k$, where $z_i\in\real^p$ is an i.i.d. random vector with  mean zero and unit variance.
Recall that $\Sigma_k$ is the population covariance matrix task $k$'s feature vectors.
Let $\Sigma_k = U_k D_k U_k^{\top}$ denote the singular value decomposition of $\Sigma_k$.
We denote $\Sigma_k = U_k D_k^{1/2}$ as the square root of $\Sigma_k$.
The sample size of task $k$, given by $n_k$, is equal to $\rho_k\cdot p$ for a fixed value $\rho_k$ that does not grow with feature dimension $p$.
This setting is also known as the high-dimensional linear regression setting in the literature.
There are two motivations for studying the high-dimensional linear regression setting for multi-task learning.
First, this setting captures salient properties of modern large-scale datasets, where the sample sizes are usually on the order of tens to hundreds of the number of features \cite{sur2019modern}.
Second, as we will see soon in Section \ref{sec_general}, we can derive precise asymptotics of the generalization error that scale with properties of the task data such as sample sizes.
%The labels $Y_k = X_k \beta_k + \varepsilon_k$, where $\beta_k$ denotes the linear model parameters and $\varepsilon_k$ denotes i.i.d. noise with mean zero and variance $\sigma^2$.
%Recall that we have $t$ labeled training datasets, denoted by $(X_1, Y_1), (X_2, Y_2), \dots, (X_t, Y_t)$, where $X_i\in\real^{n_i\times p}$ and $Y_i\in\real^{n_i}$ for $1\le i\le t$.
%Following \cite{HMRT19,BLLT20}, we assume that for each task $i = 1,2,\dots,t$,  every feature vector is generated as $x = \Sigma_i^{1/2} z$, where $z\in\real^p$ is a random vector with i.i.d. entries of mean zero and unit variance and $\Sigma_i\in\real^{p\times p}$ is a positive semidefinite matrix.
%Without loss of generality, let the $t$-th task be the target task.

We make several standard assumptions on the covariance matrix $\Sigma_k$ and the random vector $z_i$, for $k = 1,\dots, t$ and $i = 1,\dots, n_k$.
First, for every entry of $z_i$, we assume that its $(4+\e)$-th moment exists, that is, there exist constants $\e, C > 0$ such that
\begin{equation}\label{assmAhigh}
	\ex{\abs{z_{i,j}}^{4+\e}} \le C,\ 1\le j \le p.
\end{equation}
Second, we assume that the spectral norm of $\Sigma_k$ is bounded by a fixed constant, for every $k = 1,\dots, t$.
This is without loss of generality since we can always rescale $\Sigma_k$ so that its spectral norm satisfies the condition.
Finally, for the noise vector $\varepsilon_k$, we assume that there exists an absolute constant such that for all $l$ \todo{check}, the $l$-th moment of the noise entries exists:
\begin{align*}
	\ex{\abs{\varepsilon_{k, j}}^l \le C}, \text{ for all } j = 1, \dots, n_k.
\end{align*}

\medskip
\noindent\textbf{Hard parameter sharing.}
We study multi-task learning models that use hard parameter sharing, which is the most commonly used architecture in practice \cite{R17}.
Specifically, we consider a two-layer neural network with a shared layer $B\in\real^{p\times r}$ for all tasks and $t$ output layers $W_1, \dots, W_t$ for every task from $1$ to $t$, where each output layer is a $r$-dimensional vector.
The width of $B$, denoted by $r$, plays an important role in regularization.
As observed in Proposition 1 of \citet{WZR20}, if $r \ge t$, there is no regularization effect.
Hence, we assume that $r < t$ in our study.
%For example, when there are only two tasks, $r = 1$ and $B$ reduces to a vector whereas $W_1, W_2$ become scalars.
We study the following procedure inspired by how hard parameter sharing models are trained in practice.
\begin{enumerate}
%	\item Separate each dataset $(X_i, Y_i)$ randomly into a training set $(X_i^{tr}, Y_i^{tr})$ and a validation set $(X_i^{val}, Y_i^{val})$.
%	The size of each set is described below.
	\item Learn the shared layer $B$: minimize the training loss over $B$ and $W_1, \dots, W_t$, leading to a local minimum of $B$ that depends on $W_1, \dots, W_t$, denoted by $\hat{B} = \hat{B}(W_1, \dots, W_t)$.
		{\begin{align}\label{eq_mtl}
			f(B; W_1, \dots, W_t) = \sum_{k=1}^t \norm{X_k B W_k - Y_k}^2.
		\end{align}}
	\item Learn the output layers $W_1, W_2, \dots, W_t$: set $B = \hat{B}$ and minimize the training loss over $W_1, W_2, \dots, W_t$.
		{\begin{align}\label{eq_mtl_eval}
			g(W_1, \dots, W_t) = \sum_{k=1}^t \norm{X_k \hat{B} W_k - Y_k}^2.
		\end{align}}
\end{enumerate}
Let $\hat{\beta}_t^{\MTL}$ denote the multi-task learning estimator obtained from the procedure above.
By contrast, let $\hat{\beta}_t^{\STL} = (X_t^{\top}X_t)^{-1}X_t^{\top}{Y_t}$ denote the single-task learning estimator. % denoted by $L(\hat{\beta}_t^{\STL})$.
%where $B\in\real^{p\times r}$ and $W_k\in\real^r$ for every $1\le k\le t$.
%Following , we assume that $r < t$, because otherwise minimizing $f(\cdot)$ could result in $BW_i$ being the single-task optimum.

\smallskip
\noindent\textit{Remark.}
In general, the multi-task learning objective $f(\cdot)$ is non-convex with respect to $B$ and $W_1, \dots, W_t$.
Therefore, we first minimize $B$ in equation \eqref{eq_mtl} and then minimize $W_k$ given $B$ in equation \eqref{eq_mtl_eval}.
For our results later in Section \ref{sec_general} and \ref{sec_special}, we will identify tractable cases and provide guarantees to the above procedure.


\medskip
\noindent\textbf{Problem statement.}
%We focus on predicting a particular task, say the $t$-th task, without loss of generality.
For an estimator $\hat{\beta}$ of the target task model $\beta_t$, we define the prediction loss of the estimator $\hat{\beta}$ as
	{\begin{align*}
		\te(\hat{\beta}) = \exarg{x = \Sigma_t^{1/2} z}{({x}^{\top}\hat{\beta} - {x}^{\top}\beta_t)^2}
		= \bignorm{\Sigma_t^{1/2} (\hat{\beta} - \beta_t)}^2,
	\end{align*}}%
where $x = \Sigma_t^{1/2} z$ denotes a random feature vector with covariance $\Sigma_t$.
In the above equation, $x^{\top}\beta_t$ is the true label of $x$.
%= (\hat{\beta} - \beta_t)^{\top}\Sigma_t(\hat{\beta} - \beta_t)
We say that the source tasks provide a \textit{positive transfer} to the target task if the prediction loss of the MTL estimator is lower than that of the STL estimator, that is, if
	\[ L(\hat{\beta}_t^{\MTL}) > L(\hat{\beta}_t^{\STL}). \]
On the other hand, we say that the source tasks provide a \textit{negative transfer} to the target task if $L(\hat{\beta}_t^{\MTL}) < L(\hat{\beta}_t^{\STL})$.
Our goal is to study when the source tasks provide a positive transfer to the target task.
More specifically, we study how varying properties of task data including task similarity, sample ratio, and covariate shift affects information transfer in multi-task learning.


%This part introduces a positive variance reduction effect from adding the source labels.
%Hence, whether $\te(\hat{\beta}_t^{\MTL}) < \te(\hat{\beta}_t^{\STL})$ is determined precisely by the tradeoff between the negative effect of the bias term and the positive effect of the variance term!
%(i) the negative effect from model shift bias.
%(ii) the positive effect from variance reduction;


\subsection{Related Work}

We refer the interested readers to several excellent surveys on multi-task  learning for a comprehensive survey \cite{PY09,R17,ZY17,V20}.
Our setting is closely related to domain adaptation \cite{DM06,BB07,BC08,DH09,MMR09,CWB11,ZS13,NB17,ZD19}.
The important distinction is that we focus on predicting the target task using a hard parameter sharing model.
For such models, their output dimension plays an important role of regularization \cite{KD12}.
Linear models in multi-task learning have been studied in various settings, including representation learning \cite{BHKL19}, online learning \cite{CCG10,DCSP18}, and sparse regression \cite{LPVT11}.
Below, we describe several lines of work that are most related to this work.

\medskip
\noindent\textbf{Theoretical works.}
Some of the earliest works on multi-task learning are Baxter \cite{B00}, Ben-David and Schuller \cite{BS03}.
Mauer \cite{M06} studies generalization bounds for linear separation settings of MTL.
Ben-David et al. \cite{BBCK10} provides uniform convergence bounds that combines source and target errors optimally.
The benefit of learning multi-task representations has been studied for learning certain half-spaces \cite{MPR16} and sparse regression \cite{LPTV09,LPVT11}.
Our work is closely related to Wu et al. \cite{WZR20}.
While Wu et al. provide generalization bounds to show that adding more labeled helps learn the target task more accurately, their techniques cannot be used to explain when MTL outperforms STL.
\todo{spell out the challenge more explicitly}

% Adding a regularization over $B$, e.g. .
\medskip
\noindent\textbf{Methodological works.}
Ando and Zhang \cite{AZ05} introduces an alternating minimization framework for learning multiple tasks.
Argyriou et al. \cite{AEP08} present a convex algorithm which learns common sparse representations across a pool of related tasks.
Evgeniou et al. \cite{EMP05} develop a framework for multi-task learning in the context of kernel methods.
%\cite{KD12} observed that controlling the capacity can outperform the implicit capacity control of adding regularization over $B$.
The multi-task learning model that we have focused on uses the idea of hard parameter sharing \cite{C93,KD12,R17}.
We believe that our theoretical framework can apply to other approaches to multi-task learning.

\medskip
\noindent\textbf{Random matrix theory.}
The random matrix theory tool and related proof of our work fall into a paradigm of the so-called local law of random matrices \cite{erdos2017dynamical}.
For a sample covariance matrix $X^\top X$ with $\Sigma=\id$, such a local law was proved in \cite{isotropic}.
It was later extended to sample covariance matrices with non-identity $\Sigma$ \cite{Anisotropic}, and separable covariance matrices \cite{yang2019spiked}. On the other hand, one may derive the asymptotic result in Theorem \ref{lem_cov_shift_informal} with error $\oo(1)$ using the free addition of two independent random matrices in free probability theory \cite{nica2006lectures}. To the best of my knowledge, we do not find an {\it explicit result} for the sum of two sample covariance matrices with general covariates in the literature.



\paragraph{Organizations.}
\todo{revise} The appendix is organized as follows.
In Section \ref{app_proof_sec3}, we describe an extended background and a further reivew of related work.
In Section \ref{app_proof_main_thm}, we present the analysis of the bias-variance tradeoff for the case of two tasks.
In Section \ref{app_iso_cov}, we illustrate the above analysis in simplified settings including the isotropic model and covariate shifted settings without model distance.
In Section \ref{app_proof_many_tasks}, we extend our analysis to many tasks with the same features.
In Section \ref{sec_maintools}, we prove the bias and variance bounds used in the analysis.
Finally in Section \ref{app_experiments}, we fill in the missing details of our experiments.


\section{Preliminaries}



\subsection{Notations}
Throughout the appendix, we shall say an event $\Xi$ holds with high probability (w.h.p.) if for any fixed $D>0$, $\P(\Xi)\ge 1- p^{-D}$ for large enough $p$. Moreover, we shall use $\oo(1)$ to mean a small positive quantity that converges to 0 as $p\to \infty$.
For a matrix $X$, let $\lambda_{\min}(X)$ denote its smallest singular value and $\norm{X}$ denote its spectral norm.
We shall refer to random matrices of the form $X^\top X$ as sample covariance matrices following the standard notations in high-dimensional statistics.
%\subsection{Bias and Variance}

%\noindent\textbf{Positive/Negative transfer.}

\subsection{Hard Parameter Sharing}

\subsection{Bias and Variance}

\subsection{Random Matrices}