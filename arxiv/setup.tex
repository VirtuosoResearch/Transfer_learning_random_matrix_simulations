
\paragraph{Linear models.}
Suppose we have $t$ labeled regression tasks. %where $t$ is a fixed value that does not grow with the feature dimension $p$.
For each task $k$ from $1$ to $t$, we have $n_k$ samples $x_1, x_2, \dots, x_{n_k}$ that are all $p$-dimensional feature vectors with real-valued labels $y_1, y_2, \dots, y_{n_k}$.
Without loss of generality, we focus on predicting the $t$-th task and refer the $t$-th task as the target.
We refer to task $1$ to task $t-1$ as source tasks.
We assume that the labels of each task $k$ satisfy the linear model with unknown parameters $\beta_k\in\real^p$, that is,
	\[ y_i = x_i^{\top}\beta_k + e_i, \text{ for all } 1 \le i \le n_k, \]
where $e_i$ denotes random noise with mean zero and variance $\sigma^2$.
To write the above notations more succiently, let $X_k \in \real^{n_k \times p}$ denote the covariates that consists of a feature vector in every row.
Let
	\[ Y_k = X_k \beta_k + \varepsilon_k, \text{ for all } 1\le k \le t \] denote the label vector of task $k$ and $\varepsilon_k$ denote an i.i.d. random vector with mean zero and variance $\sigma^2$.

Following \citet{HMRT19} and \citet{BLLT20},
%In the high-dimensional linear regression setting (e.g. ), the features of the $k$-th task, denoted by $X_k\in\real^{n_i\times p}$, consist of $n_k$ feature vectors given by $x_1, x_2, \dots, x_{n_k}$.
we assume that for each task $k$ from $1$ to $t$, each feature vector $x_i = \Sigma^{1/2}_k z_i$ for $i = 1, \dots, n_k$, where $z_i\in\real^p$ is an i.i.d. random vector with  mean zero and unit variance.
Recall that $\Sigma_k$ is the population covariance matrix task $k$'s feature vectors.
Let $\Sigma_k = U_k D_k U_k^{\top}$ denote the singular value decomposition of $\Sigma_k$.
We denote $\Sigma_k = U_k D_k^{1/2}$ as the square root of $\Sigma_k$.
The sample size of task $k$, given by $n_k$, is equal to $\rho_k\cdot p$ for a fixed value $\rho_k$ that does not grow with feature dimension $p$.
This setting is also known as the high-dimensional linear regression setting in the literature.
There are two motivations for studying the high-dimensional linear regression setting for multi-task learning.
First, this setting captures salient properties of modern large-scale datasets, where the sample sizes are usually on the order of tens to hundreds of the number of features \cite{sur2019modern}.
Second, as we will see soon in Section \ref{sec_general}, we can derive precise asymptotics of the generalization error that scale with properties of the task data such as sample sizes.
%The labels $Y_k = X_k \beta_k + \varepsilon_k$, where $\beta_k$ denotes the linear model parameters and $\varepsilon_k$ denotes i.i.d. noise with mean zero and variance $\sigma^2$.
%Recall that we have $t$ labeled training datasets, denoted by $(X_1, Y_1), (X_2, Y_2), \dots, (X_t, Y_t)$, where $X_i\in\real^{n_i\times p}$ and $Y_i\in\real^{n_i}$ for $1\le i\le t$.
%Following \cite{HMRT19,BLLT20}, we assume that for each task $i = 1,2,\dots,t$,  every feature vector is generated as $x = \Sigma_i^{1/2} z$, where $z\in\real^p$ is a random vector with i.i.d. entries of mean zero and unit variance and $\Sigma_i\in\real^{p\times p}$ is a positive semidefinite matrix.
%Without loss of generality, let the $t$-th task be the target task.

\paragraph{Hard parameter sharing estimators.}
We study multi-task learning models that use hard parameter sharing, which is the most commonly used architecture in practice \cite{R17}.
Specifically, we consider a two-layer neural network with a shared layer $B\in\real^{p\times r}$ for all tasks and $t$ output layers $W_1, \dots, W_t$ for every task from $1$ to $t$, where each output layer is a $r$-dimensional vector.
The width of $B$, denoted by $r$, plays an important role in regularization.
As observed in Proposition 1 of \citet{WZR20}, if $r \ge t$, there is no regularization effect.
Hence, we assume that $r < t$ in our study.
%For example, when there are only two tasks, $r = 1$ and $B$ reduces to a vector whereas $W_1, W_2$ become scalars.
We study the following procedure inspired by how hard parameter sharing models are trained in practice.
\begin{enumerate}
%	\item Separate each dataset $(X_i, Y_i)$ randomly into a training set $(X_i^{tr}, Y_i^{tr})$ and a validation set $(X_i^{val}, Y_i^{val})$.
%	The size of each set is described below.
	\item Learn the shared layer $B$: minimize the training loss over $B$ and $W_1, \dots, W_t$, leading to a local minimum of $B$ that depends on $W_1, \dots, W_t$, denoted by $\hat{B} = \hat{B}(W_1, \dots, W_t)$.
		{\begin{align}\label{eq_mtl}
			f(B; W_1, \dots, W_t) = \sum_{k=1}^t \norm{X_k B W_k - Y_k}^2.
		\end{align}}
	\item Learn the output layers $W_1, W_2, \dots, W_t$: set $B = \hat{B}$ and minimize the training loss over $W_1, W_2, \dots, W_t$.
		{\begin{align}\label{eq_mtl_eval}
			g(W_1, \dots, W_t) = \sum_{k=1}^t \norm{X_k \hat{B} W_k - Y_k}^2.
		\end{align}}
\end{enumerate}
Let $\hat{\beta}_t^{\MTL}$ denote the multi-task learning estimator obtained from the procedure above.
By contrast, let $\hat{\beta}_t^{\STL} = (X_t^{\top}X_t)^{-1}X_t^{\top}{Y_t}$ denote the single-task learning estimator. % denoted by $L(\hat{\beta}_t^{\STL})$.
%where $B\in\real^{p\times r}$ and $W_k\in\real^r$ for every $1\le k\le t$.
%Following , we assume that $r < t$, because otherwise minimizing $f(\cdot)$ could result in $BW_i$ being the single-task optimum.

\smallskip
\noindent\textit{Remark.}
In general, the multi-task learning objective $f(\cdot)$ is non-convex with respect to $B$ and $W_1, \dots, W_t$.
Therefore, we first minimize $B$ in equation \eqref{eq_mtl} and then minimize $W_k$ given $B$ in equation \eqref{eq_mtl_eval}.
For our results later in Section \ref{sec_general} and \ref{sec_special}, we will identify tractable cases and provide guarantees to the above procedure.

\paragraph{Problem statement.}
%We focus on predicting a particular task, say the $t$-th task, without loss of generality.
For an estimator $\hat{\beta}$ of the target task model $\beta_t$, we define the prediction loss of the estimator $\hat{\beta}$ as
	{\begin{align*}
		\te(\hat{\beta}) = \exarg{x = \Sigma_t^{1/2} z}{({x}^{\top}\hat{\beta} - {x}^{\top}\beta_t)^2}
		= \bignorm{\Sigma_t^{1/2} (\hat{\beta} - \beta_t)}^2,
	\end{align*}}%
where $x = \Sigma_t^{1/2} z$ denotes a random feature vector with covariance $\Sigma_t$.
In the above equation, $x^{\top}\beta_t$ is the true label of $x$.
%= (\hat{\beta} - \beta_t)^{\top}\Sigma_t(\hat{\beta} - \beta_t)
We say that the source tasks provide a \textit{positive transfer} to the target task if the prediction loss of the MTL estimator is lower than that of the STL estimator, that is, if
	\[ L(\hat{\beta}_t^{\MTL}) > L(\hat{\beta}_t^{\STL}). \]
On the other hand, we say that the source tasks provide a \textit{negative transfer} to the target task if $L(\hat{\beta}_t^{\MTL}) < L(\hat{\beta}_t^{\STL})$.
Our goal is to study when the source tasks provide a positive transfer to the target task.
More specifically, we study how varying properties of task data including task similarity, sample ratio, and covariate shift affects information transfer in multi-task learning.



\subsection{Summary of Results}
\label{sec_prelim}

%We begin by defining our problem setup including the multi-task estimator we study.
%Then, we describe the bias-variance tradeoff of the multi-task estimator and connect the bias and variance of the estimator to \textit{task similarity}, \textit{sample size}, and \textit{covariate shift}.
%Finally, we show a tight concentration bound for the bias and variance quantities using random matrix theory.

%\subsection{Problem Formulation}

\textbf{Main results.} First, we develop tight bounds for the bias and variance of the multi-task estimator for two tasks by applying recent development in random matrix theory \cite{erdos2017dynamical,isotropic,Anisotropic}.
We observe that the variance of the multi-task estimator is \textit{always smaller} than single-task learning, because of added source task samples.
On the other hand, the bias of the multi-task estimator is \textit{always larger} than single-task learning, because of model distances.
Hence, the tradeoff between bias and variance determines whether the transfer is positive or negative.
We provide a sharp analysis of the \textit{variance} that scales with sample size and covariate shift.
We extend the analysis to the bias, which \textit{in addition} scales with {task similarity}.
Combining both, we analyze the bias-variance tradeoff for two tasks in Theorem \ref{thm_main_informal} and extend the analysis to many tasks with the same features in Theorem \ref{thm_many_tasks}.
%For the setting of two tasks, we show how the variance of the multi-task estimator  scales with sample size and covariate shift in the following result.
%\textit{Our first contribution} is to develop a concentration bound that arises naturally from the bias-variance tradeoff of $\hat{\beta}_t^{\MTL}$ for two tasks.
%Let $\hat{\beta}_t^{\STL}$ denote the single-task estimator.
%Without loss of generality, let the $t$-th task denote the target task.
%Importantly, the target task's data size is a fixed constant times $p$ in the high-dimensional setting.
%Hence adding more labeled data can help improve its test performance.
%$B\in\real^{p\times r}$
%$\set{W_i \in \real^{r}}_{i=1}^t$




%Concretely, we show a tight bound on the trace of $(X_1^{\top}X_1 + X_2^{\top}X_2)^{-1}$, which
%Theorem \ref{lem_cov_shift_informal} allows us to analyze the bias-variance tradeoff of the multi-task estimator for two settings:
%(i) two tasks with arbitrary covariate shift; (ii) many tasks with no covariate shift.

%We shall assume that each task data follows a linear model, i.e. $y_i = X_i \beta_i + \varepsilon_i$, $1\le i\le k$.
%Here $\beta_i\in\real^p$ is the model parameter for the $i$-th task.
%Each row of $X_i\in\real^{n_i\times p}$ is assumed to be drawn i.i.d. from a fixed
%distribution with covariance matrix $\Sigma_i$.

%We extend our result to the transfer learning
%in the setting of high-dimensional linear regression.
%by pooling source task representations into the shared body of the hard parameter sharing architecture, following
%setting of Taskonomy by Zamir et al. \cite{ZSSGM18}.
%We prove that the bias of the transfer learning estimator is given by the projection of $\beta_t$ to the orthogonal subspace spanned by $\set{\beta_i}_{i=1}^{t-1}$.
%These results are described more precisely in Section \ref{sec_main}.

Second, we explain the phenomena in Figure \ref{fig_model_shift_phasetrans} in isotropic and covariate shifted settings.
%We observe that negative transfer occurs as (a) \textit{task similarity}: tasks become more different; (b) \textit{data size}: source/target data size increases.
%\textbf{Task similarity:}
%\textbf{Data sizes:}
%\textbf{Covariate shift:}
%Furthermore, MTL performance is negatively affected when (c) \textit{covariate shift}: the covariance matrices of the two tasks become more different.
\squishlist
	\item We provide conditions to predict the effect of transfer as a parameter of model distance $\norm{\beta_1-\beta_2}$ (Section \ref{sec_similarity}).
	As model distance increases, the bias becomes larger, resulting in negative transfer.
%	Our result predicts most of the empirical observations in Figure \ref{fig_model_shift} correctly.
%	It is crucial that the concentration result in Theorem \ref{lem_cov_shift_informal} is sufficiently precise so that we can explain the transition phenomena in Figure \ref{fig_model_shift} and \ref{fig_size}.
%	The unexplained observations are caused by an error term from the bias.
%	We discuss these in Section \ref{sec_insight}.
	\item We provide conditions to predict transfer as a parameter of sample ratio $\rho_1/\rho_2$ (Section \ref{sec_data_size}).
	Adding source task samples helps initially by reducing variance, but hurts eventually due to bias.
	%namely adding more labeled data from the source task does not always improve performance (Proposition \ref{prop_data_size}).
	%Theorem \ref{lem_cov_shift_informal} allows us to compare MTL performance under different covariate shifts.
	\item For a special case of $\beta_1=\beta_2$, we show that MTL performs best when the singular values of $\Sigma_1^{1/2}\Sigma_2^{-1/2}$ are all equal  (Section \ref{sec_covshift}).
	Otherwise, the variance reduces less with covariate shift.
%	Our theoretical bound matches the empirical curve in Figure \ref{fig_covariate}.
\squishend
%In Section \ref{sec_insight}, we consider three components including task similarity, data size and covariate shift for a simplified isotropic setting of two tasks.
%We measure task similarity by how small is the distance between $\beta_1$ and $\beta_2$.
%Using our tool, we explain a transition from positive to negative transfer as task similarity decreases.
%		Furthermore, we show that negative transfer is more likely to occur when the source task labels are particularly noisy.
%		In Section \ref{sec_validate}, we validate the observation on text and image classification tasks.
%	In , we provide the trade-off between $\norm{\beta_1 - \beta_2}^2$ and a certain function $\Phi(\rho_1, \rho_2)$ to determine the type of transfer.
%We show that increasing the data size of the source task does not always improve performance for the target task in multi-task learning.
Along the way, we analyze the benefit of MTL for reducing labeled data to achieve comparable performance to STL, which has been empirically observed in Taskonomy by Zamir et al. \cite{ZSSGM18}.
%We show that covariate shift, measured by $\Sigma_1^{1/2}\Sigma_2^{-1/2}$, is another cause for suboptimal performance for $\hat{\beta}_t^{\MTL}$.
%		We show that as $n_1 / n_2$ becomes large, having no covariate shift between the source and target tasks yields the optimal performance for the target task.
%		On the other hand, when $n_1 / n_2$ is small, there are counter examples where having the same covariance matrix is not necessarily the optimal choice.

Our study also leads to several algorithmic consequences with practical interest.
First, we show that single-task learning results can help to predict positive or negative transfer for multi-task learning.
We validate this observation on ChestX-ray14 \cite{chexnet17} and sentiment analysis datasets \cite{LZWDA18}.
Second, we propose a new multi-task training schedule by incrementally adding task data batches to the training procedure.
This is inspired by our observation in Figure \ref{fig_size} where adding more source task data helps initially, but hurts eventually.
Using our incremental training schedule, we reduce the computational cost by $65\%$ compared to baseline multi-task training over six sentiment analysis datasets while keeping the accuracy the same.
Third, we provide a fine-grained insight on a covariance alignment procedure proposed in \cite{WZR20}.
We show that the alignment procedure provides more significant improvement when the source/target sample ratio is large.
Finally, we validate our three theoretical findings on sentiment analysis tasks.


We describe the main ideas for showing the generalization error of hard parameter sharing estimators.
There are two key ideas in the proof.
First, we study a bias-variance decomposition of hard parameter sharing, and show that the variance always reduces compared to single-task learning, while the bias always increases.
Second, we study the limiting bias and variance of hard parameter sharing, and derive their asymptotics using recent development from the random matrix theory literature.
The asymptotic results scale with key properties of task data such as sample size and covariance shift, and allow us to study the performance of multi-task learning by varying these properties, which will be the focus of Section \ref{sec_special}.



%This part introduces a positive variance reduction effect from adding the source labels.
%Hence, whether $\te(\hat{\beta}_t^{\MTL}) < \te(\hat{\beta}_t^{\STL})$ is determined precisely by the tradeoff between the negative effect of the bias term and the positive effect of the variance term!
%(i) the negative effect from model shift bias.
%(ii) the positive effect from variance reduction;


\subsection{Related Work}

We refer the interested readers to several excellent surveys on multi-task  learning for a comprehensive survey \cite{PY09,R17,ZY17,V20}.
Our setting is closely related to domain adaptation \cite{DM06,BB07,BC08,DH09,MMR09,CWB11,ZS13,NB17,ZD19}.
The important distinction is that we focus on predicting the target task using a hard parameter sharing model.
For such models, their output dimension plays an important role of regularization \cite{KD12}.
Linear models in multi-task learning have been studied in various settings, including representation learning \cite{BHKL19}, online learning \cite{CCG10,DCSP18}, and sparse regression \cite{LPVT11}.
Below, we describe several lines of work that are most related to this work.

\medskip
\noindent\textbf{Theoretical works.}
Some of the earliest works on multi-task learning are Baxter \cite{B00}, Ben-David and Schuller \cite{BS03}.
Mauer \cite{M06} studies generalization bounds for linear separation settings of MTL.
Ben-David et al. \cite{BBCK10} provides uniform convergence bounds that combines source and target errors optimally.
The benefit of learning multi-task representations has been studied for learning certain half-spaces \cite{MPR16} and sparse regression \cite{LPTV09,LPVT11}.
Our work is closely related to Wu et al. \cite{WZR20}.
While Wu et al. provide generalization bounds to show that adding more labeled helps learn the target task more accurately, their techniques cannot be used to explain when MTL outperforms STL.
\todo{spell out the challenge more explicitly}

% Adding a regularization over $B$, e.g. .
\medskip
\noindent\textbf{Methodological works.}
Ando and Zhang \cite{AZ05} introduces an alternating minimization framework for learning multiple tasks.
Argyriou et al. \cite{AEP08} present a convex algorithm which learns common sparse representations across a pool of related tasks.
Evgeniou et al. \cite{EMP05} develop a framework for multi-task learning in the context of kernel methods.
%\cite{KD12} observed that controlling the capacity can outperform the implicit capacity control of adding regularization over $B$.
The multi-task learning model that we have focused on uses the idea of hard parameter sharing \cite{C93,KD12,R17}.
We believe that our theoretical framework can apply to other approaches to multi-task learning.

\medskip
\noindent\textbf{Random matrix theory.}
The random matrix theory tool and related proof of our work fall into a paradigm of the so-called local law of random matrices \cite{erdos2017dynamical}.
For a sample covariance matrix $X^\top X$ with $\Sigma=\id$, such a local law was proved in \cite{isotropic}.
It was later extended to sample covariance matrices with non-identity $\Sigma$ \cite{Anisotropic}, and separable covariance matrices \cite{yang2019spiked}. On the other hand, one may derive the asymptotic result in Theorem \ref{lem_cov_shift_informal} with error $\oo(1)$ using the free addition of two independent random matrices in free probability theory \cite{nica2006lectures}. To the best of my knowledge, we do not find an {\it explicit result} for the sum of two sample covariance matrices with general covariates in the literature.



\paragraph{Organizations.}
The rest of this paper is organized as follows.
In Section \ref{set_prelim}, we present a preliminary of hard parameter sharing and random matrices. We describe several well-known facts.
In Section \ref{sec_example}, we describe illustrative examples of our technical results and discuss their implications.
In Section \ref{sec_same}, we consider the same covariates setting and show the generalization error of hard parameter sharing.
In Section \ref{sec_diff}, we consider the different covariates setting and show the bias and variance asymptotics using random matrix theory.


\section{Preliminaries}\label{sec_prelim}

\subsection{Notations}
We shall use $\oo(1)$ to mean a small positive quantity that converges to 0 as $p\to \infty$. 
We shall say an event $\Xi$ holds with high probability (w.h.p.) if $\P(\Xi)\ge 1- \oo(1)$, while we say $\Xi$ holds with overwhelming probability (w.o.p.) if for any large constant $D>0$, $\P(\Xi)\ge 1- p^{-D}$ for large enough $p$. For a matrix $X$, let $\lambda_{\min}(X)$ denote its smallest singular value and $\norm{X}$ denote its spectral norm.
We shall refer to random matrices of the form $\frac {X^\top X} n$ as sample covariance matrices following the standard notations in high-dimensional statistics.


Let $A = [a_1, a_2, \dots, a_t] \in \real^{r\times t}$ be a matrix notation that contains all the output layer parameters.

%\noindent\textbf{Positive/Negative transfer.}

\subsection{Hard Parameter Sharing}


\subsection{Random Matrices}

We introduce several standard assumptions on the random matrices we study.
\begin{assumption}\label{assume_rm}
	Let $\varphi > 4$ be a fixed value.
	Let $X = Z \Sigma^{1/2} \in\real^{n\times p}$ be a random matrix where $Z \in \real^{n\times p}$ consists of i.i.d. entries with zero mean and unit variance and $\Sigma \in \real^{p\times p}$ is a positive semidefinite matrix.
	We assume that $\Sigma$ and $Z$ satisfy the following (standard) conditions, respectively.
	\begin{enumerate}
		\item Bounded moments: for every entry of $Z$, we assume that its $c$-th moment exists, that is, there exist a fixed constant $C > 0$ such that
			\begin{align}\label{assmAhigh}
				\ex{\abs{Z_{i,j}}^{\varphi}} \le C, \text{ for any } 1\le i \le n \text{ and } 1\le j \le p.
			\end{align}
		\item Bounded singular values: the singular values of $\Sigma$ are bounded within a fixed range, that is, there exists two fixed values $C_{\max} > C_{\min} > 0$ such that all of $\Sigma$'s singular values are bounded within $C_{\min}$ and $C_{\max}$.
	\end{enumerate}
\end{assumption}

Under the above assumption, we state several well-known facts regarding the random matrix $X$.

\begin{fact}\label{lem_minv}
	Let $c_{\varphi}$ be any fixed value within $(0, \frac{\varphi - 4}{2\varphi})$.
	%Let $X  \in \real^{n\times p}$ be a random matrix that satisfies Assumption \ref{assume_rm}.
	%Let $\Sigma\in\real^{p\times p}$ denote the population covariance matrix of $X$.
	With high probability over the randomness of $X$, we have that
	\begin{enumerate}
		\item The sample covariance matrix $\frac{X^{\top}X}{n}$ is full rank.
		\item $\bigtr{\Sigma (X^{\top} X)^{-1}} = \frac{p}{n - p} + \OO(p^{-c})$, cf. Theorem 2.4 in \citet{isotropic}.
		\item The singular values of $Z^{\top}Z$ are greater than $(\sqrt{n} - \sqrt{p})^2 - n \cdot p^{-c}$ and less than $(\sqrt{n} + \sqrt{p})^2 + n \cdot p^{-c}$, cf. \citet[Theorem 2.10]{isotropic} and \citet[Lemma 3.12]{DY}.
	\end{enumerate}
\end{fact}

%Second, we assume that the spectral norm of $\Sigma_k$ is bounded by a fixed constant, for every $k = 1,\dots, t$.
%This is without loss of generality since we can always rescale $\Sigma_k$ so that its spectral norm satisfies the condition.
Finally, for the random noise component, we assume that all of its moments exist.
More precisely, there exists a fixed function $C(\cdot) : \mathbb{Z} \rightarrow \real^+$ such that for any $a = 1, 2, \dots, \infty$, we have that
\begin{align}\label{assmAhigh2}
	\ex{\abs{\varepsilon_{j}^{(i)}}^a} \le C(a), \text{ for any } 1\le i\le t \text{ and } 1\le j\le n_i.
\end{align}
Hence, for any value $\varphi > 4$, we get that Fact \ref{lem_minv} holds for $\varepsilon^{(i)}$, for all $i = 1, 2, \dots, t$.
Let $\e$ be an fixed value within $(0, 1/2)$.