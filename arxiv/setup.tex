\subsection{Setup and Main Results}

%We assume that there are multiple datasets that all follow (potentially different) linear models.
%In each dataset, suppose the feature vector is $x = {\Sigma}^{1/2} z$, where $z \in \real^p$ has i.i.d entries with zero mean and unit variance, and the population covariance matrix $\Sigma \in\real^{p \times p}$ is deterministic and positive semidefinite.
Suppose we have $t$ datasets.
For each dataset $i$ from $1$ to $t$, let $n_i$ denote its sample size.
Let $X^{(i)} \in \real^{n_i \times p}$ denote dataset $i$'s feature covariates.
We assume that the label vector $Y^{(i)} \in \real^{n_i}$ follows a linear model plus random noise.
We study the standard hard parameter sharing architecture to learn jointly from multiple datasets.
In this architecture, there is a shared feature representation layer $B\in\real^{p\times r}$ for all tasks and a separate output layer $A_i \in \real^r$ for every task $i$.
See Figure \ref{fig_intro_arch} for an illustration.
%The width of $B$, denoted by $r$, plays an important role in regularization.
%As observed in Proposition 1 of \citet{WZR20}, if $r \ge t$, there is no regularization effect.
%Hence, we assume that $r < t$ in our study.
%For example, when there are only two tasks, $r = 1$ and $B$ reduces to a vector whereas $W_1, W_2$ become scalars.
We study the following minimization problem:
%	\item Separate each dataset $(X_i, Y_i)$ randomly into a training set $(X_i^{tr}, Y_i^{tr})$ and a validation set $(X_i^{val}, Y_i^{val})$.
%	The size of each set is described below.
%	\item Learn the shared layer $B$: minimize the training loss over $B$ and $W_1, \dots, W_t$, leading to a local minimum of $B$ that depends on $W_1, \dots, W_t$, denoted by $\hat{B} = \hat{B}(W_1, \dots, W_t)$.
%\vspace{-0.1in}
\begin{align}\label{eq_mtl}
			f(A, B) = \sum_{i=1}^t \norm{X^{(i)} B A_i - Y^{(i)}}^2,
\end{align}
%\vspace{-0.1in}
where $A = [A_1, A_2, \dots, A_t] \in \real^{r \times t}$.
Given a solution $(\hat{A}, \hat{B})$ of $f(A, B)$ (which we will specify later), let $\hat{\beta}_i^{\MTL} = \hat{B} \hat{A}_i$ denote the HPS estimator for task $i$.
The critical questions here are:
(i) How well does the estimator work? In particular, how does the performance of the estimator scale with sample size?
(ii) For datasets with different sample sizes and covariate shifts, how do they affect the estimator?


\textbf{Main results.}
Our first result (Theorem \ref{thm_many_tasks}) applies to the multi-label prediction setting where all datasets have the same features, and we want to make multiple predictions (see e.g. \cite{hsu2009multi}).
We analyze the global minimizer of $f(A, B)$, and provide a sharp generalization bound on its (out-of-sample) prediction loss for any task.
This case is tractable even though in general, $f(A, B)$ is non-convex in $A$ and $B$ (e.g. matrix completion is a special case for suitably designed $X^{(i)}, Y^{(i)}$).
We show that the prediction loss of HPS admits a clean bias-variance decomposition.
Our results imply that in the same covariates setting where the sample sizes are also equal, HPS helps by reducing the variance compared to single-task learning, but hurts by increasing the bias.

Our second result (Theorem \ref{thm_main_RMT}) applies to two tasks with arbitrarily different sample size ratios and population covariance matrices.
We analyze the local minimizers of $f(A, B)$ and provide a sharp generalization bound on both tasks' prediction loss.
Despite its simplicity, we show several interesting phenomena by varying sample sizes and covariate shifts in this setting.
See Figure \ref{fig_intro_sample_size} for an illustration.
Consequently, using our precise loss estimates, we observe several qualitative properties of hard parameter sharing for varying datasets' properties.
%\begin{enumerate}

\textit{Sample efficiency (Example \ref{ex_same_cov})}:
	One advantage of combining multiple datasets is that the requirement for labeled data reduces compared to single-task learning, a phenomenon that \citet{ZSSGM18} has observed empirically.
	Our results further imply that HPS's sample efficiency depends on model-specific variances across tasks vs. the noise variance, and is generally high when the latter is large.

\textit{Sample size ratio (Example \ref{ex_sample_ratio})}: Increasing one task's sample size does not always help to reduce another task's loss. In a simplified setting, we find that the task loss either decreases first before increasing afterwards, or decreases monotonically depending on how fast the bias increases. These two trends result from different bias-variance tradeoffs. This result is surprising because previous generalization bounds in MTL typically scale down as the sample sizes of all tasks increase, thus do not apply for different sample sizes.

\textit{Covariate shift (Example \ref{ex_covshift})}: In addition to sample sizes, variance also scales with the covariate shift between different datasets. For a large sample size ratio, HPS's  variance is smallest when there is no covariate shift. Counterintuitively, for a small sample size ratio, having covariate shifts reduces variance through a complementary spectrum.
%\end{enumerate}

%First, we develop tight bounds for the bias and variance of the multi-task estimator for two tasks by applying recent development in random matrix theory \cite{erdos2017dynamical,isotropic,Anisotropic}.
%We observe that the variance of the multi-task estimator is \textit{always smaller} than single-task learning, because of added source task samples.
%On the other hand, the bias of the multi-task estimator is \textit{always larger} than single-task learning, because of model distances.
%Hence, the tradeoff between bias and variance determines whether the transfer is positive or negative.
%We provide a sharp analysis of the \textit{variance} that scales with sample size and covariate shift.
%We extend the analysis to the bias, which \textit{in addition} scales with {task similarity}.
%Combining both, we analyze the bias-variance tradeoff for two tasks in Theorem \ref{thm_main_informal} and extend the analysis to many tasks with the same features in Theorem \ref{thm_many_tasks}.
%For the setting of two tasks, we show how the variance of the multi-task estimator  scales with sample size and covariate shift in the following result.
%\textit{Our first contribution} is to develop a concentration bound that arises naturally from the bias-variance tradeoff of $\hat{\beta}_t^{\MTL}$ for two tasks.
%Let $\hat{\beta}_t^{\STL}$ denote the single-task estimator.
%Without loss of generality, let the $t$-th task denote the target task.
%Importantly, the target task's data size is a fixed constant times $p$ in the high-dimensional setting.
%Hence adding more labeled data can help improve its test performance.
%$B\in\real^{p\times r}$
%$\set{W_i \in \real^{r}}_{i=1}^t$

%Concretely, we show a tight bound on the trace of $(X_1^{\top}X_1 + X_2^{\top}X_2)^{-1}$, which
%Theorem \ref{lem_cov_shift_informal} allows us to analyze the bias-variance tradeoff of the multi-task estimator for two settings:
%(i) two tasks with arbitrary covariate shift; (ii) many tasks with no covariate shift.

%We shall assume that each task data follows a linear model, i.e. $y_i = X_i \beta_i + \varepsilon_i$, $1\le i\le k$.
%Here $\beta_i\in\real^p$ is the model parameter for the $i$-th task.
%Each row of $X_i\in\real^{n_i\times p}$ is assumed to be drawn i.i.d. from a fixed
%distribution with covariance matrix $\Sigma_i$.

%We extend our result to the transfer learning
%in the setting of high-dimensional linear regression.
%by pooling source task representations into the shared body of the hard parameter sharing architecture, following
%setting of Taskonomy by Zamir et al. \cite{ZSSGM18}.
%We prove that the bias of the transfer learning estimator is given by the projection of $\beta_t$ to the orthogonal subspace spanned by $\set{\beta_i}_{i=1}^{t-1}$.
%These results are described more precisely in Section \ref{sec_main}.

%Second, we explain the phenomena in Figure \ref{fig_model_shift_phasetrans} in isotropic and covariate shifted settings.
%We observe that negative transfer occurs as (a) \textit{task similarity}: tasks become more different; (b) \textit{data size}: source/target data size increases.
%\textbf{Task similarity:}
%\textbf{Data sizes:}
%\textbf{Covariate shift:}
%Furthermore, MTL performance is negatively affected when (c) \textit{covariate shift}: the covariance matrices of the two tasks become more different.
%\squishlist
%	\item We provide conditions to predict the effect of transfer as a parameter of model distance $\norm{\beta_1-\beta_2}$ (Section \ref{sec_similarity}).
%	As model distance increases, the bias becomes larger, resulting in negative transfer.
%	Our result predicts most of the empirical observations in Figure \ref{fig_model_shift} correctly.
%	It is crucial that the concentration result in Theorem \ref{lem_cov_shift_informal} is sufficiently precise so that we can explain the transition phenomena in Figure \ref{fig_model_shift} and \ref{fig_size}.
%	The unexplained observations are caused by an error term from the bias.
%	We discuss these in Section \ref{sec_insight}.
%	\item We provide conditions to predict transfer as a parameter of sample ratio $\rho_1/\rho_2$ (Section \ref{sec_data_size}).
%	Adding source task samples helps initially by reducing variance, but hurts eventually due to bias.
	%namely adding more labeled data from the source task does not always improve performance (Proposition \ref{prop_data_size}).
	%Theorem \ref{lem_cov_shift_informal} allows us to compare MTL performance under different covariate shifts.
%	\item For a special case of $\beta_1=\beta_2$, we show that MTL performs best when the singular values of $\Sigma_1^{1/2}\Sigma_2^{-1/2}$ are all equal  (Section \ref{sec_covshift}).
%	Otherwise, the variance reduces less with covariate shift.
%	Our theoretical bound matches the empirical curve in Figure \ref{fig_covariate}.
%\squishend
%In Section \ref{sec_insight}, we consider three components including task similarity, data size and covariate shift for a simplified isotropic setting of two tasks.
%We measure task similarity by how small is the distance between $\beta_1$ and $\beta_2$.
%Using our tool, we explain a transition from positive to negative transfer as task similarity decreases.
%		Furthermore, we show that negative transfer is more likely to occur when the source task labels are particularly noisy.
%		In Section \ref{sec_validate}, we validate the observation on text and image classification tasks.
%	In , we provide the trade-off between $\norm{\beta_1 - \beta_2}^2$ and a certain function $\Phi(\rho_1, \rho_2)$ to determine the type of transfer.
%We show that increasing the data size of the source task does not always improve performance for the target task in multi-task learning.
%Along the way, we analyze the benefit of MTL for reducing labeled data to achieve comparable performance to STL, which has been empirically observed in Taskonomy by Zamir et al. \cite{ZSSGM18}.
%We show that covariate shift, measured by $\Sigma_1^{1/2}\Sigma_2^{-1/2}$, is another cause for suboptimal performance for $\hat{\beta}_t^{\MTL}$.
%		We show that as $n_1 / n_2$ becomes large, having no covariate shift between the source and target tasks yields the optimal performance for the target task.
%		On the other hand, when $n_1 / n_2$ is small, there are counter examples where having the same covariance matrix is not necessarily the optimal choice.

%Our study also leads to several algorithmic consequences with practical interest.
%First, we show that single-task learning results can help to predict positive or negative transfer for multi-task learning.
%We validate this observation on ChestX-ray14 \cite{chexnet17} and sentiment analysis datasets \cite{LZWDA18}.

%Third, we provide a fine-grained insight on a covariance alignment procedure proposed in \cite{WZR20}.
%We show that the alignment procedure provides more significant improvement when the source/target sample ratio is large.
%Finally, we validate our three theoretical findings on sentiment analysis tasks.

% plus an identity matrix,  due to lack of freeness
%\HZ{add why this is challenging}

Finally, we discuss the practical implications of our work.
Our sample size ratio study implies a concrete progressive training procedure that gradually adds more data until performance drops.
For example, in the setting of Figure \ref{fig_intro_sample_size_b}, this procedure will stop right at the minimum of the local basin.
We conduct further studies of this procedure on six text classification datasets and observe that it reduces the computational cost by $65\%$ compared to a standard round-robin training procedure while keeping the average accuracy of all tasks simultaneously.

\textbf{Proof techniques.}
There are two main ideas in our analysis. The proof of our first result uses a geometric intuition that hard parameter sharing finds a ``rank-$r$'' approximation of the datasets.
We carefully keep track of the concentration error between the global minimum of $f(A, B)$ and its population version.
The proof of our second result is significantly more involved because of different sample sizes and covariate shifts. Using recently developed techniques from random matrix theory \cite{Anisotropic}, we show that the inverse of the sum of two sample covariance matrices with arbitrary covariate shifts converges to a deterministic diagonal matrix asymptotically (cf. Theorem \ref{thm_main_RMT}). % Moreover, we can obtain a sharp bound on the concentration error.
% to obtain a sharp estimate on the..., which is commonly referred to as the \emph{local law}.
%\HZ{add several sentences on the technical insight}
One limitation of our analysis is that in Example \ref{ex_sample_ratio}, there is an error term that can result in vacuous bounds equation for very small $n_1$ (cf. equation \eqref{cor_MTL_error})).
We believe that our analysis has provided important initial insight and it is an interesting question to tighten our result (see Section \ref{sec_conclude} for more discussion).

%This part introduces a positive variance reduction effect from adding the source labels.
%Hence, whether $\te(\hat{\beta}_t^{\MTL}) < \te(\hat{\beta}_t^{\STL})$ is determined precisely by the tradeoff between the negative effect of the bias term and the positive effect of the variance term!
%(i) the negative effect from model shift bias.
%(ii) the positive effect from variance reduction;


\subsection{Related Work}

There is a large body of both classical and recent works on multi-task learning.
We focus our discussion on theoretical works, and refer interested readers to several excellent surveys for general references \cite{PY09,R17,ZY17,V20}.
The early work of \citet{B00,BS03,M06} have sought for a study of multi-task learning from a theoretical perspective, often using uniform convergence or Rademacher complexity based techniques.
An influential paper by \citet{BBCK10} provides uniform convergence bounds that combine multiple datasets in certain settings.
One limitation of uniform convergence based techniques is that the results often assume that all  tasks have the same sample size, see e.g. \citet{B00,MPR16}.
Moreover, these techniques do not apply to the high-dimensional setting, because the results usually require a sample size at least $p \log p$.

Our proof techniques use the so-called local law of random matrices \cite{erdos2017dynamical}, which is a recent development in the random matrix theory literature.
\citet{isotropic} first proved such a local law for sample covariance matrices with isotropic covariance.
\citet{Anisotropic} later extended this result to arbitrary covariance setting.
%On the other hand, one may derive the asymptotic result in Theorem \ref{thm_main_RMT} with error $\oo(1)$ using the free addition of two independent random matrices in  theory .
These techniques provide almost sharp convergence rates to the asymptotic limit compared to other techniques such as free probability \cite{nica2006lectures}.
To the best of our knowledge, we are not aware of any previous results about the inverse of the sum of two sample covariance matrices with arbitrary covariate shifts.

The problem we study here is also related to high-dimensional prediction in transfer learning \cite{li2020transfer,bastani2020predicting} and distributed learning \cite{dobriban2018high}.
For example, \citet{li2020transfer} provides minimax optimal rates for predicting a target regression task given multiple sparse regression tasks.
One closely related work is \citet{WZR20}, which studied hard parameter sharing for two linear regression tasks.
\citet{WZR20} (and an earlier work by \citet{KD12}) observed that the shared layer size $r$ in hard parameter sharing plays a critical role of regularization.
%Linear models in multi-task learning have been studied in various settings, including online learning \cite{CCG10,DCSP18}, sparse regression \cite{LPTV09,LPVT11}, and representation learning \cite{BHKL19}.

%Our setting is closely related to domain adaptation \cite{DM06,BB07,BC08,DH09,MMR09,CWB11,ZS13,NB17,ZD19}.
%The important distinction is that we focus on predicting the target task using a hard parameter sharing model.
%For such models, their output dimension plays an important role of regularization \cite{KD12}.
%Below, we describe several lines of work that are most related to this work.

%Some of the earliest works on multi-task learning are Baxter , Ben-David and Schuller \cite{BS03}.
%Mauer \cite{M06} studies generalization bounds for linear separation settings of MTL.
%The benefit of learning multi-task representations has been studied for learning certain half-spaces \cite{MPR16} and sparse regression \cite{LPTV09,LPVT11}.
%Our work is closely related to Wu et al. \cite{WZR20}.
%While Wu et al. provide generalization bounds to show that adding more labeled helps learn the target task more accurately, their techniques cannot be used to explain when MTL outperforms STL.
%\todo{spell out the challenge more explicitly}

%Ando and Zhang \cite{AZ05} introduces an alternating minimization framework for learning multiple tasks.
%Argyriou et al. \cite{AEP08} present a convex algorithm which learns common sparse representations across a pool of related tasks.
%Evgeniou et al. \cite{EMP05} develop a framework for multi-task learning in the context of kernel methods.
%\cite{KD12} observed that controlling the capacity can outperform the implicit capacity control of adding regularization over $B$.
%The multi-task learning model that we have focused on uses the idea of hard parameter sharing \cite{C93,KD12,R17}.
%We believe that our theoretical framework can apply to other approaches to multi-task learning.



\textbf{Organizations.}
The rest of this paper is organized as follows.
In Section \ref{sec_same}, we present the bias-variance decomposition for hard parameter sharing.
In Section \ref{sec_diff}, we present our technical results that describe how varying sample sizes and covariate shifts impact hard parameter sharing using random matrix theory.
In Sections \ref{sec_simulation} and \ref{sec_text}, we validate our theory in both simulations and a real world classification task.
In Section \ref{sec_conclude}, we conclude the paper and describe several open questions.

\textbf{Notations.}
%Let $\cE \define [\varepsilon_1, \varepsilon_2, \dots, \varepsilon_t] \in \real^{n \times t}$ denote the random noise.
%We can also write $Y = XB^{\star} + \cE$.
%Let $A = [A_1, A_2, \dots, A_t] \in \real^{r\times t}$ be a matrix notation that contains all the output layer parameters.
For an $n\times p$ matrix $X$, let $\lambda_{\min}(X)$ denote its smallest singular value and $\norm{X}$ denote its spectral norm.
Let $\lambda_1(X) \ge \lambda_2(X) \ge \cdots \ge \lambda_{p\wedge n}(X)$ denote the singular values of $X$.
Let $X^+$ denote the Moore-Penrose psuedoinverse.
We refer to random matrices of the form $\frac {X^\top X} n$ as sample covariance matrices.
We say that an event $\Xi$ holds with high probability if the probability that $\Xi$ happens goes to $1$ as $p$ goes to infinity.
%We shall use $\oo(1)$ to mean a small positive quantity that converges to 0 as $p$ goes to infinity.
