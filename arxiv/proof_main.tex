\section{Proof of Main Results}

\subsection{Two Tasks}\label{app_proof_main_thm}

We now state several helper lemmas to get estimates on $L(\hat{\beta}_t^{\STL})$ and $L(\hat{\beta}_t^{\MTL})$ for $t=2$. The first lemma, which is a folklore result in random matrix theory, helps to determine the asymptotic limit of $\te(\hat{\beta}_t^{\STL})$ as $p\to \infty$. When the entries of $X$ are multivariate Gaussian, this lemma recovers the classical result for the mean of inverse Wishart distribution \cite{anderson1958introduction}. For general non-Gaussian random matrices, it can be obtained using Stieltjes transform method; see e.g., Lemma 3.11 of \cite{bai2009spectral}. Here we shall state a result obtained from Theorem 2.4 in \cite{isotropic}, which gives an almost sharp error bound.
\begin{lemma}\label{lem_minv}
Suppose $X$ satisfies assumption \ref{assm_secA1}. Let $A$ be any $p\times p$ matrix that is independent of $X$. We have that for any constant $\e>0$,
	\be\label{XXA}  \bigtr{(X^{\top}X)^{-1}A} = \frac{1}{\rho - 1} \frac1p\tr(\Sigma^{-1}A) +\bigo{ \|A\|p^{-1/2+\epsilon}} \ee
with high probability.
\end{lemma}


We shall refer to random matrices of the form $X^\top X$ as sample covariance matrices following the standard notations in high-dimensional statistics. The second lemma extends Lemma \ref{lem_minv} for a single sample covariance matrix to the sum of two independent sample covariance matrices. It is the main random matrix theoretical input of this paper.
%which deals with the inverse of the sum of two random matrices, which
%any is can be viewed as a special case of Theorem \ref{thm_model_shift}.

\begin{lemma}[Variance bound: Lemma \ref{lem_cov_shift_informal} restated]\label{lem_cov_shift}
	%Let $X_i\in\real^{n_i\times p}$ be a random matrix that contains i.i.d. row vectors with mean $0$ and variance $\Sigma_i\in\real^{p\times p}$, for $i = 1, 2$.
	Suppose $X_1=Z_1\Sigma_1^{1/2}\in \R^{n_1\times p}$ and $X_2=Z_2\Sigma_2^{1/2}\in \R^{n_2\times p}$ satisfy Assumption \ref{assm_secA1} with $\rho_1:=n_1/p>1$ and $\rho_2:=n_2/p>1$ being fixed constants.
	Denote by $M = \Sigma_1^{1/2}\Sigma_2^{-1/2}$ and let $\lambda_1, \lambda_2, \dots, \lambda_p$ be the singular values of $M$ in descending order. Let $A$ be any $p\times p$ matrix that is independent of $X_1$ and $X_2$. We have that for any constant $\e>0$,
	%When $n_1 = c_1 p$ and $n_2 = c_2 p$, we have that with high probability over the randomness of $X_1$ and $X_2$, the following equation holds
	\begin{align}\label{lem_cov_shift_eq}
		\bigtr{(X_1^{\top}X_1 + X_2^{\top}X_2)^{-1}A} = \frac{1}{\rho_1+\rho_2}\frac1p\bigtr{ (a_1 \Sigma_1 + a_2\Sigma_2)^{-1} A} +\bigo{\|A\| p^{-1/2+\epsilon}}
	\end{align}
with high probability, where $(a_1, a_2)$ is the solution to the following deterministic equations:
	\begin{align}
		a_1 + a_2 = 1- \frac{1}{\rho_1 + \rho_2},\quad a_1 + \frac1{\rho_1 + \rho_2}\cdot \frac{1}{p}\sum_{i=1}^p \frac{\lambda_i^2 a_1}{\lambda_i^2 a_1 + a_2} = \frac{\rho_1}{\rho_1 + \rho_2}. \label{eq_a12extra}
	\end{align}
\end{lemma}

Finally, the last lemma describes the asymptotic limit of $(X_1^{\top}X_1 + X_2^{\top}X_2)^{-1}\Sigma_2(X_1^{\top}X_1 + X_2^{\top}X_2)^{-1}$, which will be needed when we estimate the first term on the right-hand side of \eqref{eq_te_mtl_2task}.


%We will give the proof of this lemma in Section \ref{sec_maintools}.

\textbf{Proof overview.}
%We first describe the proof of Theorem \ref{lem_cov_shift_informal}.
The proofs of Lemma \ref{lem_cov_shift} and Lemma \ref{lem_cov_derivative} are based on the Stieltjes transform method (or the resolvent method) in random matrix theory \cite{bai2009spectral,tao2012topics,erdos2017dynamical}. Roughly speaking, we study the resolvent $R(z):=[\Sigma_2^{-1/2}( X_1^{\top}X_1 + X_2^{\top}X_2)\Sigma_2^{-1/2}-z]^{-1}$ for $z\in \C$ around $z=0$.
Using the methods in \cite{Anisotropic,yang2019spiked}, we find the asymptotic limit, say $R_\infty(z)$, of $R(z)$ for any $z$ as $p\to \infty$ with an almost optimal convergence rate. In particular, when $z=0$, $\tr[\Sigma_2^{-1/2}A\Sigma_2^{-1/2}R_\infty(0)]$ gives the limit in \eqref{lem_cov_shift_eq}. 
On the other hand, we can write 
$$\bignorm{\Sigma_2^{1/2} (X_1^{\top}X_1 + X_2^{\top}X_2)^{-1}\beta}^2= \beta^\top \Sigma_2^{-1/2}R'(0)\Sigma_2^{-1/2} \beta.$$ 
Hence its limit can be calculated through $R_\infty'(z)$, which gives the expression in \eqref{lem_cov_derv_eq}. The details can be found in Appendix \ref{sec_maintools}.

%The proof of Lemma \ref{lem_cov_shift} and Lemma \ref{lem_cov_derivative} is a main focus of Section \ref{sec_maintools}. 
We remark that one can probably derive the same asymptotic result using free probability theory (see e.g. \cite{nica2006lectures}), but our results \eqref{lem_cov_shift_eq} and \eqref{lem_cov_derv_eq} also give an almost sharp error bound $\bigo{ p^{-1/2+\epsilon}}$.




%\subsection{Proof for Two Tasks with General Covariates}\label{app_proof_main}

%To illustrate the idea, we observe that by using Lemma \ref{lem_minv}, we have that
%\[ \te(\hat{\beta}_t^{\STL}) = \frac{\sigma^2}{n_2 - p}\bigtr{\Sigma_2^{-1}}. \]
%We shall also derive the limit of $\te(\hat{\beta}_t^{\MTL})$.
%\todo{write a brief technical overview}

For the rest of this section, we shall state and prove the formal version of Theorem \ref{thm_main_informal}. %We shall consider the case where the entries of $\e_1$ and $\e_2$ have the same variance $\sigma_1^2=\sigma_2^2=\sigma^2$. 
%Moreover, we shall denote $\beta_1$ and $\beta_2$ by $\beta_1$ and $\beta_2$, standing for ``source task" and ``target task", respectively. 
First, we introduce several quantities that will be used in our statement, and they are also related to the quantities in Lemma \ref{lem_cov_shift} and Lemma \ref{lem_cov_derivative}. Given the optimal ratio $\hat v$, 
%let $\hat{M} = \hat{v} \Sigma_1^{1/2}\Sigma_2^{-1/2}$ denote the weighted covariate shift matrix, and ${\hat\lambda}_1\ge {\hat\lambda}_2 \ge \dots \ge {\hat\lambda}_p$ be the eigenvalues of $\hat{M}^{\top}\hat{M}$. Define 
let $(\hat a_1, \hat a_2)$ be the solution to the following system of deterministic equations,
	\be
		 \hat a_1 +  \hat a_2 = 1- \frac{1}{\rho_1 + \rho_2},\quad  \hat a_1 + \frac1{\rho_1 + \rho_2}\cdot \frac1p\sum_{i=1}^p \frac{ \hat v^2 \lambda_i^2 \hat a_1}{ \hat v^2 \lambda_i^2\hat a_1 +  \hat a_2} = \frac{\rho_1}{\rho_1 + \rho_2}.\label{eq_a2} \\
		 \ee
		 After obtaining $(\hat a_1,\hat a_2)$, we can solve the following linear equations to get $(\hat a_3,\hat a_4)$:
\begin{gather}
		\left(\rho_2 \hat a_2^{-2}- \hat b_0\right)\cdot \hat  a_3 - \hat b_1 \cdot \hat a_4
		=\hat b_0, \quad \left(\rho_1\hat a_1^{-2} - \hat b_2  \right)\cdot \hat a_4 - \hat b_1 \cdot \hat a_3 =\hat b_1 .\label{eq_a3} 
%		\left(\frac{n_1}{\hat a_1^2} -  \sum_{i=1}^p \frac{\hat \lambda_i^4   }{  (\hat a_2 + \hat \lambda_i^2\hat a_1)^2  }\right)\hat a_4 -\left(\sum_{i=1}^p \frac{\hat \lambda_i^2  }{  (\hat a_2 + \hat \lambda_i^2\hat a_1)^2  }\right)\hat a_3
%		= \sum_{i=1}^p \frac{\hat \lambda_i^2 }{  (\hat a_2 + \hat \lambda_i^2\hat a_1)^2  }. \label{eq_a4}
	\end{gather}
where we denoted
$$\hat b_k:= \frac1{p}\sum_{i=1}^p \frac{\hat v^{2k} \lambda_i^{2k}}{ (\hat a_2 +\hat v^2 \lambda_i^2\hat a_1)^2  },\quad k=0,1,2.$$
Then we introduce the following matrix
\be\label{defnpihat}\Pi \equiv \Pi(\hat v)= \frac{\rho_1^2}{(\rho_1 + \rho_2)^2}\cdot \hat v^2{M} \frac{(1 + \hat a_3)\id +\hat  a_4 \hat v^2 {M}^{\top} {M}}{(\hat  a_1 \hat v^2{M}^{\top} {M}+\hat a_2 )^2} {M}^{\top}.\ee
%which is defined in a way such that {\it in certain sense} it is the asymptotic limit of the random matrix 
%$$\hat v\Sigma_1^{1/2} (\hat{v}^2 X_1^{\top}X_1 + X_2^{\top}X_2)^{-1} \Sigma_2 (\hat{v}^2 X_1^{\top}X_1 + X_2^{\top}X_2)^{-1}\Sigma_1^{1/2}.$$ 
We introduce two factors that will appear often in our statements and discussions:
$$\al_-(\rho_1):=\left(1- \rho_1^{-1/2}\right)^2,\quad \al_+(\rho_1):=\left(1 + \rho_1^{-1/2}\right)^2.$$ 
In fact, $\al_-(\rho_1)$ and $\al_+(\rho_1)$ correspond to the largest and smallest singular values of $Z_1/\sqrt{n_1}$, respectively, as given by the famous Mar{\v c}enko-Pastur law \cite{MP}. In particular, as $\rho_1$ increases, both $\al_-$ and $\al_+$ will converge to 1 and $Z_1/\sqrt{n_1}$ will be more close to an isometry. Finally, we introduce the error term  
\be\label{eq_deltaextra}\delta\equiv \delta(\hat v):=\frac{\al_+^2(\rho_1) - 1 }{\al_-^{2}(\rho_1)\hat v^2\lambda_{\min}^2(M)} \cdot  \norm{\Sigma_1^{1/2}(\beta_1 - \hat{v}\beta_2)}^2.\ee
%where $\lambda_{\min}(\hat M)$ is the smallest singular value of $\hat M$. 
Note that this factor converges to 0 as $\rho_1$ increases.

%$$\delta:=\left[\frac{n_1 \lambda_1}{(\sqrt{n_1}-\sqrt{p})^2\lambda_p  +  (\sqrt{n_2}-\sqrt{p})^2}\right]^2\cdot \norm{\Sigma_1^{1/2}(\beta_1 - \hat{w}\beta_2)}^2.$$
%{\cor may be able to get a better bound, but the statement will be long}

Now we are ready to state our main result for two tasks with both covariate and model shift. It shows that the information transfer is determined by two deterministic quantities $\Delta_{\bias}$ and $\Delta_{\vari}$, which give the change of model shift bias and the change of variance, respectively.



\begin{theorem}[Theorem \ref{thm_main_informal} restated]\label{thm_model_shift}
%For $i=1,2$, let $Y_i = X_i\beta_i + \varepsilon_i$ be two independent data models, where $X_i$, $\beta_i$ and $\varepsilon_i$ are also independent of each other. Suppose that $X_i=Z_i\Sigma_i^{1/2}\in \R^{n_i\times p}$ satisfy Assumption \ref{assm_secA1} with $\rho_i:=n_i/p>1$ being fixed constants, and $\e_i\in \R^{n_i}$ are random vectors with i.i.d. entries with mean zero, variance $\sigma^2$ and all moments as in \eqref{assmAhigh}. 
Consider two data models $Y_i = X_i\beta_i + \varepsilon_i$, $i=1,2$, that satisfy Assumption \ref{assm_secA2} with $\sigma_1^2=\sigma_2^2=\sigma^2$. Then with high probability, we have
	\begin{align}
	 	\te(\hat{\beta}_{t}^{\MTL}) \le \te(\hat{\beta}_t^{\STL}) \quad \text{ when: } \ \ &\Delta_{\vari} - \Delta_{\bias} \ge   \delta(\hat v) \label{upper}\\
		\te(\hat{\beta}_t^{\MTL}) \ge \te(\hat{\beta}_t^{\STL}) \quad \text{ when: } \ \ &\Delta_{\vari} - \Delta_{\bias} \le - \delta(\hat v), \label{lower}
	\end{align}
	where
	\begin{align} %\bigtr{{\Sigma_2^{-1}}}
		\Delta_{\vari} &\define {\sigma^2}\bigbrace{\frac{1}{\rho_2 - 1} -  \frac{1}{\rho_1 + \rho_2}\cdot \frac1p \bigtr{(\hat a_1 \hat v^2 M^{\top}M + \hat a_2\id)^{-1}} } \label{Deltavarv} \\
		\Delta_{\bias} &\define (\beta_1 - \hat{v}\beta_2)^{\top} \Sigma_1^{1/2} \Pi (\hat v)\Sigma_1^{1/2} (\beta_1 - \hat{v}\beta_2). \label{Deltabetav}
	\end{align}
\end{theorem}

The proof of Theorem \ref{thm_model_shift} is based on Lemma \ref{lem_cov_shift}, Lemma \ref{lem_cov_derivative}, and the following bound on the singular values of $Z_1$: for any fixed $\e>0$, we have
\begin{align}
\al_-(\rho_1) - \OO(p^{-1/2+e})  \preceq {n_1^{-1}}{Z_1^T Z_1}  \preceq   \al_+(\rho_1) + \OO(p^{-1/2+e}) \quad \text{w.h.p.}  \label{eq_isometric}
\end{align}
%with high probability. 
In fact, $n_1^{-1}Z_1^TZ_1$ is a standard sample covariance matrix, and it is well-known that its eigenvalues are inside the support of the Marchenko-Pastur law $[\al_-(\rho_1)-\oo(1) ,\al_+(\rho_1)+\oo(1)]$ with probability $1-\oo(1)$ \cite{No_outside}. The estimate \eqref{eq_isometric} provides tight bounds for the concentration error. The result is formally stated in Lemma \ref{SxxSyy} below.

%\begin{remark}
The main error $\delta$ of Theorem \ref{thm_model_shift} comes from approximating $n_1^{-1}Z_1^TZ_1$ by $\id$ using \eqref{eq_isometric}; see the estimate \eqref{bounddelta-} below. In order to improve this estimate and obtain an exact asymptotic result, one needs to study the singular value distribution of the following random matrix:
$$(X_1^{\top}X_1)^{-1}X_2^{\top}X_2 +  \hat{v}^2 .$$
In fact, the eigenvalues of $\cal X:=(X_1^{\top}X_1)^{-1}X_2^{\top}X_2$ have been studied in the name of Fisher matrices; see e.g. \cite{Fmatrix}. However, since $\cal X$ is not symmetric, it is known that the singular values of $\cal X$ are different from its eigenvalues. To the best of our knowledge, the asymptotic singular value behavior of $\cal X$ is still unknown in random matrix theory literature, and the study of the singular values of $\cal X +\hat v^2$ will be even harder. We leave this problem to future study.
%\end{remark}





\begin{proof}[Proof of Theorem \ref{thm_model_shift}]
%\noindent
%To prove Theorem \ref{thm_cov_shift}, we study the spectrum of the random matrix model:
%$$Q= \Sigma_1^{1/2}  Z_1^{\top} Z_1 \Sigma_1^{1/2}  + \Sigma_2^{1/2}  Z_2^{\top} Z_2 \Sigma_2^{1/2} ,$$
%where $\Sigma_{1,2}$ are $p\times p$ deterministic covariance matrices, and $X_1=(x_{ij})_{1\le i \le n_1, 1\le j \le p}$ and $X_2=(x_{ij})_{n_1+1\le i \le n_1+n_2, 1\le j \le p}$ are $n_1\times p$ and $n_2 \times p$ random matrices, respectively, where the entries $x_{ij}$, $1 \leq i \leq n_1+n_2\equiv n$, $1 \leq j \leq p$, are real independent random variables satisfying
%\begin{equation}\label{eq_12moment} %\label{assm1}
%\mathbb{E} z_{ij} =0, \ \quad \ \mathbb{E} \vert z_{ij} \vert^2  = 1.
%\end{equation}
%\todo{A proof outline; including the following key lemma.}
Note that 
\begin{align*}
L(\hat{\beta}_t^{\STL}) - L(\hat{\beta}_t^{\MTL}) &=\sigma^2 \left(  \bigtr{(X_2^{\top}X_2)^{-1}\Sigma_2} -  \bigtr{( \hat{v}^2 X_1^{\top}X_1 + X_2^{\top}X_2)^{-1} \Sigma_2}\right) \\
&- \hat{v}^2 \bignorm{\Sigma_2^{1/2}(\hat{v}^2 X_1^{\top}X_1 + X_2^{\top}X_2)^{-1} X_1^{\top}X_1 (\beta_1 - \hat{v} \beta_2)}^2=:\delta_{\vari}(\hat v) - \delta_{\bias}(\hat v).
\end{align*}
We introduce the notation $\hat M \equiv \hat M(v)= v\Sigma_1^{1/2}\Sigma_2^{-1/2}$ for $v\in \R$. Then the proof is divided into the following four steps. 
\begin{itemize}
\item[(i)] We first consider $ \hat M(v)$ for a fixed $v\in \R$. Then we use Lemma \ref{lem_minv} and Lemma \ref{lem_cov_shift} to calculate the variance reduction $\delta_{\vari}(v)$, which will lead to the $\Delta_{\vari}$ term.
%$$\sigma^2 \cdot \bigtr{(X_2^{\top}X_2)^{-1}},\quad \sigma^2 \cdot \bigtr{({v}^2 X_1^{\top}X_1 + X_2^{\top}X_2)^{-1} \Sigma_2},$$
%and the difference between them 

\item[(ii)] Using the approximate isometry property of $Z_1$ in \eqref{eq_isometric}, we will bound the bias term $ \delta_{\bias}(v)$
%$${v}^2 \bignorm{\Sigma_2^{1/2}({v}^2 X_1^{\top}X_1 + X_2^{\top}X_2)^{-1} X_1^{\top}X_1 (\beta_1 - {v} \beta_2)}^2$$
through 
\be\label{deltabetapf}
\wt\delta_{\bias}(v):={v}^2 n_1^2\bignorm{\Sigma_2^{1/2}({v}^2 X_1^{\top}X_1 + X_2^{\top}X_2)^{-1} \Sigma_1 (\beta_1 - {v} \beta_2)}^2.\ee

\item[(iii)] We use Lemma \ref{lem_cov_derivative} to calculate \eqref{deltabetapf}, which will lead to the $\Delta_{\bias}$ term.

\item[(iv)] Finally we use a standard $\e$-net argument to extend the above results to $\hat M(\hat v)$ for a possibly random $\hat v$ which depends on $Y_1$ and $Y_2$.
\end{itemize}


\paragraph{Step I: Variance reduction.} Consider $\hat M(v)$ for any fixed $v\in \R$. Using Lemma \ref{lem_cov_shift}, we can obtain that for any constant $\e>0$, 
$$  \sigma^2 \cdot \bigtr{(X_2^{\top}X_2)^{-1}\Sigma_2} = \frac{\sigma^2}{\rho_2-1}\left( 1+ \OO(p^{-1/2+e})\right),$$
and 
$$ \sigma^2 \cdot \bigtr{( {v}^2 X_1^{\top}X_1 + X_2^{\top}X_2)^{-1} \Sigma_2} =   \frac {\sigma^2} {\rho_1 + \rho_2}\cdot \frac1p \bigtr{(\hat a_1 \hat M^{\top}\hat M + \hat a_2\id)^{-1}}\left( 1+ \OO(p^{-1/2+e})\right) ,$$
with high probability, where $\hat a_1$ and $\hat a_2$ satify \eqref{eq_a2} with $\hat v$ replaced by $v$. Combining them, we get 
\be\label{deltavaral-} \delta_{\vari}(v)=\Delta_{\vari}(v) +\OO( \sigma^2 p^{-1/2+e}) \quad \text{w.h.p.},
\ee 
where $\Delta_{\vari}(v)$ is defined as in \eqref{Deltavarv} but with $\hat v$ replaced by $v$.



\paragraph{Step II: Bounding the bias term.}
Next we use \eqref{eq_isometric} to approximate $\delta_{\bias}(v)$ with $\wt\delta_{\bias}(v)$. %in \eqref{deltabetapf}.  
%relate the first term in equation \eqref{eq_te_model_shift} to $\Delta_{\bias}$.
\begin{claim}\label{prop_model_shift}
	In the setting of Theorem \ref{thm_model_shift},
	we denote by $K = (v^2X_1^{\top}X_1 + X_2^{\top}X_1)^{-1}$, and
	\begin{align*}
		%\delta_1 &= v^2 \bignorm{\Sigma_2^{1/2} K X_1^{\top}X_1(\beta_1 - v\beta_2)}^2, \\
		%\delta_2 &= n_1^2\cdot v^2 \bignorm{\Sigma_2^{1/2}K\Sigma_1(\beta_1 - v\beta_2)}, \\
		\delta_{err}(v) := n_1^2 v^2 \bignorm{\Sigma_1^{1/2} K \Sigma_2 K \Sigma_1^{1/2}} \cdot \bignorm{\Sigma_1^{1/2} (\beta_1 - v\beta_2)}^2.
	\end{align*}
	Then we have w.h.p.
	\begin{align*}
		 \left| \delta_{\bias}(v)-\wt\delta_{\bias}(v)\right| 
		\le  \left( \al_+^2(\rho_1)-1 + \OO(p^{-1/2+\e})\right)\delta_{err}.
	\end{align*}
%	We have that
%	\begin{align*}
%		-2n_1^2\bigbrace{{2\sqrt{\frac{p}{n_1}}} + {\frac{p}{n_1}}} \delta_3
%		\le  \delta_1 - \delta_2
%		\le n_1^2\bigbrace{2\sqrt{\frac{p}{n_1}} + \frac{p}{n_1}}\bigbrace{2 + 2\sqrt{\frac{p}{n_1}} + \frac{p}{n_1}}\delta_3.
%	\end{align*}
%	For the special case when $\Sigma_1 = \id$ and $\beta_1 - \beta_2$ is i.i.d. with mean $0$ and variance $d^2$, we further have
%	\begin{align*}
%		\bigbrace{1 - \sqrt{\frac{p}{n_1}}}^4 \Delta_{\bias}
%		\le \bignorm{\Sigma_2^{1/2} (X_1^{\top}X_1 + X_2^{\top}X_2)^{-1}X_1^{\top}X_1(\beta_1 - \beta_2)}^2.
%	\end{align*}
\end{claim}

\begin{proof}
	%The proof follows by applying equation \eqref{eq_isometric}.
	%Recall that $X_1^{\top}X_1 = \Sigma_1^{1/2}Z_1^{\top}Z_1\Sigma_1^{1/2}$.
	Denote by $\cE = Z_1^{\top}Z_1 - {n_1}\id$. Then we can write
%	Let $\alpha = \bignorm{\Sigma_2^{1/2} K \Sigma_1 (\beta_1 - \hat{w}\beta_2)}^2$.
	%We have
	\begin{align}
%		& \bignorm{\Sigma_2^{1/2}(X_1^{\top}X_1 + X_2^{\top}X_2)^{-1}X_1^{\top}X_1(\beta_1 - \hat{w}\beta_2)}^2 \nonumber \\
		 \delta_{\bias}(v)-\wt\delta_{\bias}(v)&= {2v^2n_1}(\beta_1 - v\beta_2)^{\top}\Sigma_1^{1/2} \cE\left(\Sigma_1^{1/2}K \Sigma_2 K \Sigma_1^{1/2}\right) \Sigma_1^{1/2} (\beta_1 - v\beta_2) \nonumber
		\\
		&+ v^2\bignorm{\Sigma_2^{1/2} K \Sigma_1^{1/2}\cE \Sigma_1^{1/2}(\beta_1 - v\beta_2)}^2. \label{eq_lem_model_shift_1}
%		\le& n_1\bigbrace{{n_1^2}{} + \frac{2n_1}p(p + 2\sqrt{{n_1}p}) + (p + 2\sqrt{{n_1}p})^2} \alpha = n_1^2\bigbrace{1 + \sqrt{\frac{p}{n_1}}}^4 \alpha. \nonumber
	\end{align}
	Using \eqref{eq_isometric}, we can bound  
	$$\|\cal E\|\le \left( \al_+(\rho_1)-1 + \OO(p^{-1/2+\e})\right)n_1, \quad \text{w.h.p.}$$
	Thus we can estimate that 
	\begin{align*}
	| \delta_{\bias}(v)-\wt\delta_{\bias}(v)|&\le v^2 \left( 2n_1  \|\cal E\| +  \|\cal E\|^2 \right) \bignorm{\Sigma_1^{1/2} K \Sigma_2 K \Sigma_1^{1/2}} \bignorm{\Sigma_1^{1/2} (\beta_1 - v\beta_2)}^2 \\
	&=  v^2 \left[\left( n_1 + \|\cal E\|\right)^2 -n_1^2 \right] \bignorm{\Sigma_1^{1/2} K \Sigma_2 K \Sigma_1^{1/2}} \bignorm{\Sigma_1^{1/2} (\beta_1 - v\beta_2)}^2 \\
	& \le v^2 n_1^2 \left[ \al_+^2(\rho_1) + \OO(p^{-1/2+\e}) -1\right]\bignorm{\Sigma_1^{1/2} K \Sigma_2 K \Sigma_1^{1/2}} \bignorm{\Sigma_1^{1/2} (\beta_1 - v\beta_2)}^2,
	\end{align*}
	which concludes the proof by the definition of $\delta_\e$.	
%	we can bound the second term on the RHS of equation \eqref{eq_lem_model_shift_1} as
%	\begin{align*}
%		& \bigabs{(\beta_1 -  \beta_2)^{\top} \Sigma_1^{1/2} \cE \Sigma_1^{1/2} K \Sigma_2 K \Sigma_1 (\beta_1 - v\beta_2)}\le n_1  \|\cal E\| \cdot \bignorm{\Sigma_1^{1/2} K \Sigma_2 K \Sigma_1^{1/2}} \bignorm{\Sigma_1^{1/2} (\beta_1 - v\beta_2)}^2 \\
%		= & \bigabs{\bigtr{\cE \Sigma_1^{1/2}K\Sigma_2 K \Sigma_1(\beta_1 - \hat{w}\beta_2)(\beta_1 - \hat{w}\beta_2)^{\top} \Sigma_1^{1/2}}} \\
%		\le & \norm{\cE} \cdot \bignormNuclear{\Sigma_1^{1/2} K \Sigma_2 K \Sigma_1 (\beta_1 - \hat{w}\beta_2) (\beta_1 - \hat{w}\beta_2)^{\top} \Sigma_1^{1/2}} \\
%		\le & n_1 \bigbrace{2\sqrt{\frac{p}{n_1}} + \frac{p}{n_1}} \cdot \bignormNuclear{\Sigma_1^{1/2} K \Sigma_2 K \Sigma_1 (\beta_1 - \hat{w}\beta_2)(\beta_1 - \hat{w}\beta_2)^{\top} \Sigma_1^{1/2}} \tag{by equation \eqref{eq_isometric}} \\
%		\le   & n_1 \bigbrace{2\sqrt{\frac{p}{n_1}} + \frac{p}{n_1}} \bignorm{\Sigma_1^{1/2}K \Sigma_2 K \Sigma_1^{1/2}} \cdot \bignorm{\Sigma_1^{1/2}(\beta_1 - \hat{w}\beta_2)}^2 \tag{since the matrix inside is rank 1}
%	\end{align*}
%	The third term in equation \eqref{eq_lem_model_shift_1} can be bounded with
%	\begin{align*}
%		\bignorm{\Sigma_2^{1/2}K\Sigma_1^{1/2}\cE\Sigma_1^{1/2}(\beta_1 - v\beta_2)}^2
%		\le n_1^2 \bigbrace{2\sqrt{\frac{p}{n_1}} + \frac{p}{n_1}}^2 \bignorm{\Sigma_1^{1/2}K\Sigma_{2}K\Sigma_1^{1/2}} \cdot \bignorm{\Sigma_1^{1/2}(\beta_1 -  \beta_2)}^2.
%	\end{align*}
%	Combined together we have shown the right direction for $\delta_1 - \delta_2$.
%	For the left direction, we simply note that the third term in equation \eqref{eq_lem_model_shift_1} is positive.
%	And the second term is bigger than $-2n_1^2(2\sqrt{\frac{p}{n_1}} + \frac{p}{n_1}) \alpha$ using equation \eqref{eq_isometric}.
\end{proof}
Note by \eqref{eq_isometric}, we have with high probability,
\begin{align*}
&v^2 n_1^2 \Sigma_1^{1/2} K \Sigma_2 K \Sigma_1^{1/2} =n_1^2 \hat M (\hat M^\top Z_1^\top Z_1 \hat M + Z_2^\top Z_2)^{-2}\hat M^\top \\
&\preceq  n_1^2 \hat M \left[n_1 \al_-(\rho_1)\hat M^\top \hat M + n_2 \al_-(\rho_2) + \OO(p^{1/2+\e})\right]^{-2}\hat M^\top \\
&\preceq  \left[ \al_-^2(\rho_1) \hat M\hat M^\top + 2\frac{\rho_2}{\rho_1} \al_-(\rho_1)\al_-(\rho_2) + 2\left(\frac{\rho_2}{\rho_1}\right)^2 \al_-^2(\rho_2) (\hat M \hat M^\top )^{-1}\right]^{-1}+  \OO(p^{-1/2+\e}) \\
&\preceq [\al_-^2(\rho_1) \lambda_{\min}^2(\hat M)]^{-1}\cdot (1 - c)
\end{align*}
for some small enough constant $c>0$. Together with Claim \ref{prop_model_shift}, we get with high probability,
\be\label{bounddelta-}
\left| \delta_{\bias}(v)-\wt\delta_{\bias}(v)\right| 
		\le (1-c) \delta(v)
\ee
for some small constant $c>0$, where we recall $\delta(v)$ defined in \eqref{eq_deltaextra}.


\paragraph{Step III: The limit of $\wt\delta_{\bias}(v)$.} 
Using Lemma \ref{lem_cov_derivative} with $\Sigma_1$ and $M$ replaced by $v^2\Sigma_1$ and $\hat M$, we obtain that 
\begin{align*}
\wt\delta_{\bias}(v) &=\frac{\rho_1^2}{(\rho_1 + \rho_2)^2}\cdot v^2 (\beta_1-v\beta_2)^\top\Sigma_1 \Sigma_2^{-1/2}  \frac{(1 + \hat a_3)\id + \hat a_4 \hat{M}^{\top}\hat{M}}{( a_1 \hat{M}^{\top}\hat{M}+a_2 )^2} \Sigma_2^{-1/2} \Sigma_1(\beta_1-v\beta_2) +\OO(p^{-1/2+\e}) \\
&= (\beta_1 - {v}\beta_2)^{\top} \Sigma_1^{1/2} \Pi \Sigma_1^{1/2} (\beta_1 - {v}\beta_2) +\OO(p^{-1/2+\e}) =: \Delta_{\bias}(v) +\OO(p^{-1/2+\e}),
\end{align*}
with high probability. Together with and \eqref{deltavaral-} and \eqref{bounddelta-}, we obtain that w.h.p.,
\be\label{dicho_varbeta}
\begin{cases}\delta_{\vari}(v)>\delta_{\bias}(v), & \text{ if } \ \ \Delta_{\vari}(v) - \Delta_{\bias}(v) \ge   \delta(v),\\
\delta_{\vari}(v)<\delta_{\bias}(v),  & \text{ if }  \ \ \Delta_{\vari}(v) - \Delta_{\bias}(v) \le -  \delta(v).\end{cases}
\ee



\paragraph{Step IV: An $\e$-net argument.} Finally, it remains to extend the above result \eqref{dicho_varbeta} to $v=\hat v$, which is random and depends on $X_1$ and $X_2$. We first show that for any fixed constant $C_0>0$, there exists a high probability event $\Xi$ on which \eqref{dicho_varbeta} 
%\eqref{lem_cov_shift_eq} and \eqref{lem_cov_derv_eq} 
holds uniformly for all $v\in [-C_0, C_0]$. In fact, we consider $v$ belonging to a discrete set 
$$V:=\{v_k = kp^{-1}: -(C_0p +1)\le k \le C_0p +1\}.$$
Then using the arguments for the first three steps and a simple union bound, we get that
\eqref{dicho_varbeta} holds simultaneously for all $v\in V$ with high probability. On the other hand, by \eqref{eq_isometric} the event
$$\Xi_1:=\left\{ \al_-(\rho_1)/2 \preceq  \frac{Z_1^T Z_1}{n_1}  \preceq   2\al_+(\rho_1) ,\  \al_-(\rho_2)/2 \preceq  \frac{Z_2^T Z_2}{n_2}  \preceq   2\al_+(\rho_2)\right\}$$
holds with high probability. Now it is easy to check that on $\Xi_1$, for all $v_k \le v\le v_{k+1}$ we have the following estimates:
\begin{align*}
& |\delta_{\vari}(v) -\delta_{\vari}(v_k)|\lesssim p^{-1}\delta_{\vari}(v_k),\ \ |\delta_{\bias}(v) -\delta_{\bias}(v_k)|\lesssim p^{-1}\delta_{\bias}(v_k), \ \   |\delta(v)-\delta(v_k)|\lesssim p^{-1}\delta(v_k),\\
& |\Delta_{\bias}(v) -\Delta_{\bias}(v_k)|\lesssim p^{-1}\Delta_{\bias}(v_k),\ \ |\Delta_{\vari}(v) -\Delta_{\vari}(v_k)|\lesssim p^{-1}\Delta_{\vari}(v_k).
\end{align*}
Then a simple application of triangle inequality gives that the event 
$$\Xi_2=\{\eqref{dicho_varbeta} \text{ holds simultaneously for all }-C_0\le v \le C_0\}$$
holds with high probability. On the other hand, on $\Xi_1$ one can see that for any small constant $\e>0$, there exists a large enough constant $C_0>0$ depending on $\e$ such that
\begin{align*}
& |\delta_{\vari}(v) -\delta_{\vari}(C_0)|\le \e\delta_{\vari}(C_0),\quad |\delta_{\bias}(v) -\delta_{\bias}(C_0)|\le \e\delta_{\bias}(C_0), \quad  |\delta(v)-\delta(C_0)|\le \e\delta(C_0),\\
& |\Delta_{\bias}(v) -\Delta_{\bias}(C_0)|\le \e\Delta_{\bias}(C_0),\quad |\Delta_{\vari}(v) -\Delta_{\vari}(C_0)|\le \e\Delta_{\vari}(C_0),
\end{align*}
for all $v\ge C_0$. Similar estimates hold for $v\le -C_0$ if we replace $C_0$ with $-C_0$ in the above estimates. Together with the estimate at $\pm C_0$, we get that \eqref{dicho_varbeta} holds simultaneously for all $v\in \R$ on the high probability event $\Xi_1\cap \Xi_2$. This concludes the proof since $v$ must be one of the real values.
\end{proof}





\subsection{Multiple Tasks}\label{app_proof_many_tasks}

%As a remark, since the spectral norm of $U_r$ is less than $1$, we have that $\norm{U_r(i)} < 1$ for all $1 \le i \le t$. Compared to Theorem \ref{thm_main_informal}, we can get a simple expression for the two functions $\Delta_{vari}$ and $\Delta_{\bias}$. The proof of Theorem \ref{thm_many_tasks} can be found in Appendix \ref{app_proof_many_tasks}.



%In this section we consider the setting with $k$ many that have the same covariates.
%Since every task has the same number of data points as well as the same covariance, the only differences between different tasks are their models $\set{\beta_i}_{i=1}^k$.
%For this setting, we derive solutions for the multi-task training and the transfer learning setting that match our insights qualitatively from Section \ref{sec_denoise}.

\begin{proof}[Proof of Theorem \ref{thm_many_tasks}]
In this setting, we need to study the following loss function:
\begin{align}
	f(B; W_1, \dots, W_t) = \sum_{i=1}^t \bignorm{X B W_i - Y_i}^2. \label{eq_mtl_same_cov}
\end{align}
%In order to prove Theorem \ref{thm_many_tasks}, we will derive a closed form solution for equation \eqref{eq_mtl_same_cov}. \todo{check!}
For any fixed $W_1, W_2, \dots, W_t \in \R^r$, we can derive a closed form solution for $B$ as
	\begin{align*}
		\hat{B}(W_1, \dots, W_t) &= (X^{\top}X)^{-1} X^{\top} \bigbrace{\sum_{i=1}^t Y_i W_i^{\top}} (\cal W  \cal W^{\top})^{-1} \\
		&= (B^\star \cal W ^{\top}) (\cal W \cal W ^{\top})^{-1} + (X^{\top}X)^{-1}X^{\top} \bigbrace{\sum_{i=1}^t \varepsilon_i W_i^{\top}} (\cal W \cal W^{\top})^{-1},
	\end{align*}
	where we denote $\cal W \in\real^{r\times t}$ as $\cal W=[W_1, W_2, \dots, W_t]$.
	%Now we switch $\hat{B}$ back into equation \eqref{eq_mtl_same_cov} to
Then as in \eqref{approxvalid}, we pick $N$ independent samples of the training set for each task with $N\ge n^{1-\e_0}$, and use concentration to get the validation loss as
\be\label{eq_multival}g(\cal W)=  N\left[\val(\cal W) + t  \sigma^2 \right]\cdot \left( 1+\OO(p^{-(1-\e_0)/2+\e})\right).\ee
Here $\val(\cal W)$ is defined as
%it remains to consider minimizing the validation loss
	$$\val( \cal W):=\exarg{\varepsilon_j, \forall 1\le j\le t}{ \sum_{i=1}^t \bignorm{\Sigma^{1/2}( \hat B W_i - \beta_i)}^2} =  \delta_{\bias}(\cal W) + \delta_{\vari}(\cal W),$$
where the model shift bias term $\delta_{\bias}(\cal W) $ is given by
	\begin{align*}
		\delta_{\bias}(\cal W) :=\sum_{i=1}^t  \bignorm{\Sigma^{1/2}\bigbrace{(B^\star \cal W^\top) (\cal W\cal W^{\top})^{-1} W_i - \beta_i}}^2,
	\end{align*}
	and the variance term $\delta_{\vari}(\cal W)$ can be calculated as
	\begin{align*}
		\delta_{\vari}(\cal W):= \sigma^2 \cdot \bigtr{\Sigma (X^{\top}X)^{-1}}.
	\end{align*}
It suffices to minimize $\delta_{\bias}(\cal W)$ over $\cal W$, since both $t \sigma^2$ and $\delta_{\vari}$ do not depend on $\cal W$.

We denote $Q := \cal W^{\top} (\cal W\cal W^{\top})^{-1} \cal W \in\real^{k\times k}$, whose $(i,j)$-th entry is equal to $W_i^{\top} (\cal W\cal W^{\top})^{-1} W_j$.
	%Let $B^{\star} = [\beta_1, \beta_2, \dots, \beta_k] \in\real^{p \times k}$ denote the true model parameters.
	Now we can write $\delta_{\bias}(\cal W)$ succinctly as
	\begin{align*}
		\delta_{\bias}(\cal W) = \bignormFro{\Sigma^{1/2}B^{\star}  \bigbrace{Q -\id}}^2 .
	\end{align*}
	From this equation we can solve the minimizer optimally as $Q_0\equiv  \cal W_0^{\top} (\cal W_0\cal W_0^{\top})^{-1} \cal W_0=U_{r}U_r^{\top}$. On the other hand, let $\hat {\cal W}$ be the minimizer of $g$, and denote $\hat Q:= \hat{\cal W}^{\top} (\hat{\cal W}\hat{\cal W}^{\top})^{-1} \hat{\cal W} $. We claim that $\hat Q$ satisfies
	\be\label{Q-Q}\|Q_0^{-1}\hat Q - \id\| = \oo(1) \quad \text{w.h.p.}\ee
	In fact, if \eqref{Q-Q} does not hold, then using the condition $\lambda_{\min}((B^{\star})^\top\Sigma B^{\star})\gtrsim \sigma^2$ and that $\delta_{\vari}(\cal W)=\OO(\sigma^2)$ by \eqref{eq_isometric}, we obtain that
	$$   \val(\hat {\cal W}) + t  \sigma^2 > (\val( {\cal W}_0) + t \sigma^2 )\cdot (1+\oo(1)) \ \Rightarrow \ g( \hat{\cal W})>g( {\cal W}_0),$$
	that is, $\hat {\cal W}$ is not a minimizer. This leads to a contradiction.

	In sum, we have solved that $\hat{\beta}_i^{\MTL}=B^{\star}\left( U_r U_r(i) +\oo(1)\right)$. Inserting it into the definition of the test loss, we get that
	\begin{align*}
		\te(\hat{\beta}_t^{\MTL}) &= \bignorm{\Sigma^{1/2} \left((B^\star \hat{\cal W}^\top) (\hat{\cal W}\hat{\cal W}^{\top})^{-1} \hat W_t - \beta_t \right) }^2
		+ \sigma^2  \hat W_t^{\top} (\hat{\cal W}\hat{\cal W}^{\top})^{-1} \hat W_t \cdot \bigtr{\Sigma (X^{\top}X)^{-1}} \\
		&= \bignorm{\Sigma^{1/2} \bigbrace{B^{\star} U_r U_r(t)-\beta_t}}^2 + \oo\left(\|B^\star\|^2\right) + \sigma^2\norm{U_r(t)}^2 \bigtr{\Sigma (X^{\top}X)^{-1}} \cdot (1+\oo(1))\\
		&= \bignorm{\Sigma^{1/2} \bigbrace{B^{\star} U_r U_r(t)-\beta_t}}^2 + \frac{\sigma^2}{\rho-1}\norm{U_r(t)}^2 + \oo \left( \|B^\star\|^2 + \sigma^2\right),
	\end{align*}
	with high probability, where we used Lemma \ref{lem_minv} in the last step. Similar, by Lemma \ref{lem_minv} we have
	$$\te(\hat{\beta}_t^{\STL})=\frac{\sigma^2}{\rho-1} \cdot \left( 1+\oo(1)\right).$$
Combining the above two estimates, we conclude the proof.
\end{proof}
%From the above we can obtain three conceptual insights that are consistent with Section \ref{sec_denoise} and \ref{sec_insight}.
%\begin{itemize}
%	\item The de-noising effect of multi-task learning.
%	\item Multi-task training vs single-task training can be either positive or negative.
%	\item Transfer learning is better than the other two. And the improvement over multi-task training increases as the model distances become larger.
%\end{itemize}

