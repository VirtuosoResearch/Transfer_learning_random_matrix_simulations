\section{Proof of the General Case}\label{sec_proof_general}

\subsection{Proof of the Two-task Case}\label{app_proof_main_thm}

In this subsection, we provide the proof of Theorem \ref{thm_main_informal} using Lemma  \ref{lem_cov_shift} and \ref{lem_cov_derivative}.
First, we provide the steps for the bias-variance decomposition.
Using the local optimality condition of $B$, we obtain that
\begin{align*}
	 \hat{B}(W_1, W_2) &= (W_1^2 X_1^{\top}X_1 + W_2^2 X_2^{\top}X_2)^{-1} (W_1 X_1^{\top}Y_1 + W_2 X_2^{\top}Y_2)\\
	&= \frac{1}{W_2} \left( \frac{W_1^2}{W_2^2}  X_1^{\top}X_1 + X_2^{\top}X_2\right)^{-1} \left(\frac{W_1}{W_2} X_1^{\top}Y_1 + X_2^{\top}Y_2\right) \\
	&= \frac{1}{W_2}\left[\beta_2 + \left(\frac{W_1^2}{W_2^2} X_1^{\top}X_1 + X_2^{\top}X_2\right)^{-1}\bigbrace{X_1^{\top}X_1\left(\frac{W_1}{W_2}\beta_1 - \frac{W_1^2}{W_2^2} \beta_2\right) + \left(\frac{W_1}{W_2} X_1^{\top}\varepsilon_1 + X_2^{\top}\varepsilon_2\right)}\right].
\end{align*}
Recall that $\hat{\beta}_2^{\MTL} = \hat{B}(W_1, W_2) W_2$, and one can verify that equation \eqref{eq_bias_2task} and \eqref{eq_var_2task} hold.

During the proof we shall also give explicit expressions for $\Delta_{\bias}$ and $\Delta_{\vari}$, which might be of interest to some readers.
With $a_i$, $i=1,2,3,4$, given in Lemma \ref{lem_cov_shift} and Lemma \ref{lem_cov_derivative}, 
we define the following matrix
\be\label{defnpihat}\Pi \equiv \Pi(\hat v)= \frac{\rho_1^2}{(\rho_1 + \rho_2)^2}\cdot \hat v^2{M} \frac{(1 +   a_3)\id + \hat v^2 a_4 {M}^{\top} {M}}{( \hat v^2 a_1 {M}^{\top} {M}+ a_2 )^2} {M}^{\top}.\ee
%which is defined in a way such that {\it in certain sense} it is the asymptotic limit of the random matrix 
%$$\hat v\Sigma_1^{1/2} (\hat{v}^2 X_1^{\top}X_1 + X_2^{\top}X_2)^{-1} \Sigma_2 (\hat{v}^2 X_1^{\top}X_1 + X_2^{\top}X_2)^{-1}\Sigma_1^{1/2}.$$ 
Moreover, we introduce two factors that will appear often in our statements and discussions:
$$\al_-(\rho_1):=\left(1- \rho_1^{-1/2}\right)^2,\quad \al_+(\rho_1):=\left(1 + \rho_1^{-1/2}\right)^2.$$ 
In fact, $\al_-(\rho_1)$ and $\al_+(\rho_1)$ correspond to the largest and smallest singular values of $Z_1/\sqrt{n_1}$, respectively, as given by the famous Mar{\v c}enko-Pastur law \cite{MP}. In particular, as $\rho_1$ increases, both $\al_-$ and $\al_+$ will converge to 1 and $Z_1/\sqrt{n_1}$ will be more close to an isometry. Finally, we introduce the error term  
\be\label{eq_deltaextra} 
 \delta(\hat v):=\frac{\al_+^2(\rho_1) - 1 }{\al_-^{2}(\rho_1)\hat v^2\lambda_{\min}^2(M)} \cdot  \norm{\Sigma_1^{1/2}(\beta_1 - \hat{v}\beta_2)}^2.\ee
%where $\lambda_{\min}(\hat M)$ is the smallest singular value of $\hat M$. 
Note that this factor converges to 0 as $\rho_1$ increases. It is not hard to see that Theorem \ref{thm_main_informal} is an immediate consequence of the following lemma.

%$$\delta:=\left[\frac{n_1 \lambda_1}{(\sqrt{n_1}-\sqrt{p})^2\lambda_p  +  (\sqrt{n_2}-\sqrt{p})^2}\right]^2\cdot \norm{\Sigma_1^{1/2}(\beta_1 - \hat{w}\beta_2)}^2.$$
%{\cor may be able to get a better bound, but the statement will be long}





\begin{lemma}%[Theorem \ref{thm_main_informal} restated]
\label{thm_model_shift}
%For $i=1,2$, let $Y_i = X_i\beta_i + \varepsilon_i$ be two independent data models, where $X_i$, $\beta_i$ and $\varepsilon_i$ are also independent of each other. Suppose that $X_i=Z_i\Sigma_i^{1/2}\in \R^{n_i\times p}$ satisfy Assumption \ref{assm_secA1} with $\rho_i:=n_i/p>1$ being fixed constants, and $\e_i\in \R^{n_i}$ are random vectors with i.i.d. entries with mean zero, variance $\sigma^2$ and all moments as in \eqref{assmAhigh}. 
Consider two data models $Y_i = X_i\beta_i + \varepsilon_i$, $i=1,2$, that satisfy Assumption \ref{assm_secA2} with $\sigma_1^2=\sigma_2^2=\sigma^2$. Then with high probability, we have
	\begin{align}
	 	\te(\hat{\beta}_{t}^{\MTL}) \le \te(\hat{\beta}_t^{\STL}) \quad \text{ when: } \ \ &\Delta_{\vari} - \Delta_{\bias} \ge   \delta(\hat v), \label{upper}\\
		\te(\hat{\beta}_t^{\MTL}) \ge \te(\hat{\beta}_t^{\STL}) \quad \text{ when: } \ \ &\Delta_{\vari} - \Delta_{\bias} \le - \delta(\hat v), \label{lower}
	\end{align}
	where
	\begin{align} %\bigtr{{\Sigma_2^{-1}}}
		\Delta_{\vari} &\define {\sigma^2}\bigbrace{\frac{1}{\rho_2 - 1} -  \frac{1}{\rho_1 + \rho_2}\cdot \frac1p \bigtr{( \hat v^2 a_1 M^{\top}M +  a_2\id)^{-1}} } \label{Deltavarv} \\
		\Delta_{\bias} &\define (\beta_1 - \hat{v}\beta_2)^{\top} \Sigma_1^{1/2} \Pi (\hat v)\Sigma_1^{1/2} (\beta_1 - \hat{v}\beta_2). \label{Deltabetav}
	\end{align}
\end{lemma}

The proof of Lemma \ref{thm_model_shift} is based on Lemma \ref{lem_cov_shift}, Lemma \ref{lem_cov_derivative}, and the following bound on the singular values of $Z_1$: for any constant $\e>0$, we have
\begin{align}
\al_-(\rho_1) - \OO(p^{-1/2+e})  \preceq {n_1^{-1}}{Z_1^T Z_1}  \preceq   \al_+(\rho_1) + \OO(p^{-1/2+e}) \quad \text{w.h.p.}  \label{eq_isometric}
\end{align}
%with high probability. 
In fact, $n_1^{-1}Z_1^TZ_1$ is a standard sample covariance matrix, and it is well-known that its eigenvalues are all inside the support of the Marchenko-Pastur law $[\al_-(\rho_1)-\oo(1) ,\al_+(\rho_1)+\oo(1)]$ with probability $1-\oo(1)$ \cite{No_outside}. The estimate \eqref{eq_isometric} provides tight bounds on the concentration errors, and it will be formally proved in Lemma \ref{SxxSyy} below.

%\begin{remark}
The main error $\delta$ in Lemma \ref{thm_model_shift} comes from approximating $n_1^{-1}Z_1^TZ_1$ by $\id$ using \eqref{eq_isometric}; see the estimate \eqref{bounddelta-} below. In order to improve this estimate and obtain an exact asymptotic result, one needs to study the singular value distribution of the following random matrix:
$$(X_1^{\top}X_1)^{-1}X_2^{\top}X_2 +  \hat{v}^2 .$$
In fact, the eigenvalues of $\cal X:=(X_1^{\top}X_1)^{-1}X_2^{\top}X_2$ have been studied in the name of Fisher matrices; see e.g. \cite{Fmatrix}. However, since $\cal X$ is non-symmetric, it is known that the singular values of $\cal X$ are different from its eigenvalues. To the best of our knowledge, the asymptotic singular value behavior of $\cal X$ is still unknown in random matrix literature, and the study of the singular values of $\cal X +\hat v^2$ will be even harder. We leave this problem to future study.
%\end{remark}





\begin{proof}[Proof of Lemma \ref{thm_model_shift}]
%\noindent
%To prove Theorem \ref{thm_cov_shift}, we study the spectrum of the random matrix model:
%$$Q= \Sigma_1^{1/2}  Z_1^{\top} Z_1 \Sigma_1^{1/2}  + \Sigma_2^{1/2}  Z_2^{\top} Z_2 \Sigma_2^{1/2} ,$$
%where $\Sigma_{1,2}$ are $p\times p$ deterministic covariance matrices, and $X_1=(x_{ij})_{1\le i \le n_1, 1\le j \le p}$ and $X_2=(x_{ij})_{n_1+1\le i \le n_1+n_2, 1\le j \le p}$ are $n_1\times p$ and $n_2 \times p$ random matrices, respectively, where the entries $x_{ij}$, $1 \leq i \leq n_1+n_2\equiv n$, $1 \leq j \leq p$, are real independent random variables satisfying
%\begin{equation}\label{eq_12moment} %\label{assm1}
%\mathbb{E} z_{ij} =0, \ \quad \ \mathbb{E} \vert z_{ij} \vert^2  = 1.
%\end{equation}
%\todo{A proof outline; including the following key lemma.}
By \eqref{eq_te_model_shift} and \eqref{eq_te_var}, we can write
\begin{align*}
L(\hat{\beta}_t^{\STL}) - L(\hat{\beta}_t^{\MTL})=\delta_{\vari}(\hat v) - \delta_{\bias}(\hat v), 
\end{align*}
where we denote
\begin{align*}
\delta_{\vari}(\hat v)&=\sigma^2 \left(  \bigtr{(X_2^{\top}X_2)^{-1}\Sigma_2} -  \bigtr{( \hat{v}^2 X_1^{\top}X_1 + X_2^{\top}X_2)^{-1} \Sigma_2}\right), 
\end{align*}
and
\begin{align}\label{revise_deltabias}
 \delta_{\bias}(\hat v)&= \hat{v}^2 \bignorm{\Sigma_2^{1/2}(\hat{v}^2 X_1^{\top}X_1 + X_2^{\top}X_2)^{-1} X_1^{\top}X_1 (\beta_1 - \hat{v} \beta_2)}^2.
\end{align}
We introduce the notation $\hat M \equiv \hat M(v)= v\Sigma_1^{1/2}\Sigma_2^{-1/2}$ for $v\in \R$. Then the proof is divided into the following four steps. 
\begin{itemize}
\item[(i)] We first consider $ \hat M(v)$ for a fixed $v\in \R$. Then we use Lemma \ref{lem_minv} and Lemma \ref{lem_cov_shift} to calculate the variance reduction $\delta_{\vari}(v)$, which will lead to the $\Delta_{\vari}$ term.
%$$\sigma^2 \cdot \bigtr{(X_2^{\top}X_2)^{-1}},\quad \sigma^2 \cdot \bigtr{({v}^2 X_1^{\top}X_1 + X_2^{\top}X_2)^{-1} \Sigma_2},$$
%and the difference between them 

\item[(ii)] Using the approximate isometry property of $Z_1$ in \eqref{eq_isometric}, we will approximate the bias term $ \delta_{\bias}(v)$ by
%$${v}^2 \bignorm{\Sigma_2^{1/2}({v}^2 X_1^{\top}X_1 + X_2^{\top}X_2)^{-1} X_1^{\top}X_1 (\beta_1 - {v} \beta_2)}^2$$
%through 
\be\label{deltabetapf}
\wt\delta_{\bias}(v):={v}^2 n_1^2\bignorm{\Sigma_2^{1/2}({v}^2 X_1^{\top}X_1 + X_2^{\top}X_2)^{-1} \Sigma_1 (\beta_1 - {v} \beta_2)}^2.\ee

\item[(iii)] We use Lemma \ref{lem_cov_derivative} to calculate \eqref{deltabetapf}, which will lead to the $\Delta_{\bias}$ term.

\item[(iv)] Finally we use a standard $\e$-net argument to extend the above results to $\hat M(\hat v)$ for a possibly random $\hat v$ which depends on $Y_1$ and $Y_2$.
\end{itemize}


\paragraph{Step I: Variance reduction.} Consider $\hat M(v)$ for a fixed $v\in \R$. Using Lemma \ref{lem_cov_shift}, we can obtain that for any constant $\e>0$,  
$$  \sigma^2 \cdot \bigtr{(X_2^{\top}X_2)^{-1}\Sigma_2} = \frac{\sigma^2}{\rho_2-1}\left( 1+ \OO(p^{-1/2+e})\right),$$
and 
$$ \sigma^2 \cdot \bigtr{( {v}^2 X_1^{\top}X_1 + X_2^{\top}X_2)^{-1} \Sigma_2} =   \frac {\sigma^2} {\rho_1 + \rho_2}\cdot \frac1p \bigtr{( a_1(v) \hat M^{\top}\hat M + a_2(v)\id)^{-1}}\left( 1+ \OO(p^{-1/2+e})\right) ,$$
with high probability, where $a_1(v)$ and $a_2(v)$ satify \eqref{eq_a12extra} with $\hat v$ replaced by $v$. The subtraction of these two terms gives 
\be\label{deltavaral-} \delta_{\vari}(v)=\Delta_{\vari}(v) +\OO( \sigma^2 p^{-1/2+e}) \quad \text{w.h.p.},
\ee 
where $\Delta_{\vari}(v)$ is defined as in \eqref{Deltavarv} but with $\hat v$ replaced by $v$.



\paragraph{Step II: Bounding the bias term.}
Next we use \eqref{eq_isometric} to approximate $\delta_{\bias}(v)$ with $\wt\delta_{\bias}(v)$. %in \eqref{deltabetapf}.  
%relate the first term in equation \eqref{eq_te_model_shift} to $\Delta_{\bias}$.
\begin{claim}\label{prop_model_shift}
	In the setting of Lemma \ref{thm_model_shift},
	we denote $K(v) := (v^2X_1^{\top}X_1 + X_2^{\top}X_2)^{-1}$, and
	\begin{align*}
		%\delta_1 &= v^2 \bignorm{\Sigma_2^{1/2} K X_1^{\top}X_1(\beta_1 - v\beta_2)}^2, \\
		%\delta_2 &= n_1^2\cdot v^2 \bignorm{\Sigma_2^{1/2}K\Sigma_1(\beta_1 - v\beta_2)}, \\
		\delta_{err}(v) := n_1^2 v^2 \bignorm{\Sigma_1^{1/2} K(v) \Sigma_2 K(v) \Sigma_1^{1/2}} \cdot \bignorm{\Sigma_1^{1/2} (\beta_1 - v\beta_2)}^2.
	\end{align*}
	Then we have w.h.p.
	\begin{align*}
		 \left| \delta_{\bias}(v)-\wt\delta_{\bias}(v)\right| 
		\le  \left( \al_+^2(\rho_1)-1 + \OO(p^{-1/2+\e})\right)\delta_{err}(v).
	\end{align*}
%	We have that
%	\begin{align*}
%		-2n_1^2\bigbrace{{2\sqrt{\frac{p}{n_1}}} + {\frac{p}{n_1}}} \delta_3
%		\le  \delta_1 - \delta_2
%		\le n_1^2\bigbrace{2\sqrt{\frac{p}{n_1}} + \frac{p}{n_1}}\bigbrace{2 + 2\sqrt{\frac{p}{n_1}} + \frac{p}{n_1}}\delta_3.
%	\end{align*}
%	For the special case when $\Sigma_1 = \id$ and $\beta_1 - \beta_2$ is i.i.d. with mean $0$ and variance $d^2$, we further have
%	\begin{align*}
%		\bigbrace{1 - \sqrt{\frac{p}{n_1}}}^4 \Delta_{\bias}
%		\le \bignorm{\Sigma_2^{1/2} (X_1^{\top}X_1 + X_2^{\top}X_2)^{-1}X_1^{\top}X_1(\beta_1 - \beta_2)}^2.
%	\end{align*}
\end{claim}

\begin{proof}
	%The proof follows by applying equation \eqref{eq_isometric}.
	%Recall that $X_1^{\top}X_1 = \Sigma_1^{1/2}Z_1^{\top}Z_1\Sigma_1^{1/2}$.
We can calculate that
%	Let $\alpha = \bignorm{\Sigma_2^{1/2} K \Sigma_1 (\beta_1 - \hat{w}\beta_2)}^2$.
	%We have
	\begin{align}
%		& \bignorm{\Sigma_2^{1/2}(X_1^{\top}X_1 + X_2^{\top}X_2)^{-1}X_1^{\top}X_1(\beta_1 - \hat{w}\beta_2)}^2 \nonumber \\
		 \delta_{\bias}(v)-\wt\delta_{\bias}(v)&= {2v^2n_1}(\beta_1 - v\beta_2)^{\top}\Sigma_1^{1/2} \cE\left(\Sigma_1^{1/2}K \Sigma_2 K \Sigma_1^{1/2}\right) \Sigma_1^{1/2} (\beta_1 - v\beta_2) \nonumber
		\\
		&+ v^2\bignorm{\Sigma_2^{1/2} K \Sigma_1^{1/2}\cE \Sigma_1^{1/2}(\beta_1 - v\beta_2)}^2. \label{eq_lem_model_shift_1}
%		\le& n_1\bigbrace{{n_1^2}{} + \frac{2n_1}p(p + 2\sqrt{{n_1}p}) + (p + 2\sqrt{{n_1}p})^2} \alpha = n_1^2\bigbrace{1 + \sqrt{\frac{p}{n_1}}}^4 \alpha. \nonumber
	\end{align}
	where we denote $\cE: = Z_1^{\top}Z_1 - {n_1}\id$. 
  Using \eqref{eq_isometric}, we can bound  
	$$\|\cal E\|\le \left( \al_+(\rho_1)-1 + \OO(p^{-1/2+\e})\right)n_1, \quad \text{w.h.p.}$$
	Thus we can estimate that 
	\begin{align*}
	| \delta_{\bias}(v)-\wt\delta_{\bias}(v)|&\le v^2 \left( 2n_1  \|\cal E\| +  \|\cal E\|^2 \right) \bignorm{\Sigma_1^{1/2} K \Sigma_2 K \Sigma_1^{1/2}} \bignorm{\Sigma_1^{1/2} (\beta_1 - v\beta_2)}^2 \\
	&=  v^2 \left[\left( n_1 + \|\cal E\|\right)^2 -n_1^2 \right] \bignorm{\Sigma_1^{1/2} K \Sigma_2 K \Sigma_1^{1/2}} \bignorm{\Sigma_1^{1/2} (\beta_1 - v\beta_2)}^2 \\
	& \le v^2 n_1^2 \left[ \al_+^2(\rho_1) + \OO(p^{-1/2+\e}) -1\right]\bignorm{\Sigma_1^{1/2} K \Sigma_2 K \Sigma_1^{1/2}} \bignorm{\Sigma_1^{1/2} (\beta_1 - v\beta_2)}^2,
	\end{align*}
	which concludes the proof by the definition of $\delta_{err}$.	
%	we can bound the second term on the RHS of equation \eqref{eq_lem_model_shift_1} as
%	\begin{align*}
%		& \bigabs{(\beta_1 -  \beta_2)^{\top} \Sigma_1^{1/2} \cE \Sigma_1^{1/2} K \Sigma_2 K \Sigma_1 (\beta_1 - v\beta_2)}\le n_1  \|\cal E\| \cdot \bignorm{\Sigma_1^{1/2} K \Sigma_2 K \Sigma_1^{1/2}} \bignorm{\Sigma_1^{1/2} (\beta_1 - v\beta_2)}^2 \\
%		= & \bigabs{\bigtr{\cE \Sigma_1^{1/2}K\Sigma_2 K \Sigma_1(\beta_1 - \hat{w}\beta_2)(\beta_1 - \hat{w}\beta_2)^{\top} \Sigma_1^{1/2}}} \\
%		\le & \norm{\cE} \cdot \bignormNuclear{\Sigma_1^{1/2} K \Sigma_2 K \Sigma_1 (\beta_1 - \hat{w}\beta_2) (\beta_1 - \hat{w}\beta_2)^{\top} \Sigma_1^{1/2}} \\
%		\le & n_1 \bigbrace{2\sqrt{\frac{p}{n_1}} + \frac{p}{n_1}} \cdot \bignormNuclear{\Sigma_1^{1/2} K \Sigma_2 K \Sigma_1 (\beta_1 - \hat{w}\beta_2)(\beta_1 - \hat{w}\beta_2)^{\top} \Sigma_1^{1/2}} \tag{by equation \eqref{eq_isometric}} \\
%		\le   & n_1 \bigbrace{2\sqrt{\frac{p}{n_1}} + \frac{p}{n_1}} \bignorm{\Sigma_1^{1/2}K \Sigma_2 K \Sigma_1^{1/2}} \cdot \bignorm{\Sigma_1^{1/2}(\beta_1 - \hat{w}\beta_2)}^2 \tag{since the matrix inside is rank 1}
%	\end{align*}
%	The third term in equation \eqref{eq_lem_model_shift_1} can be bounded with
%	\begin{align*}
%		\bignorm{\Sigma_2^{1/2}K\Sigma_1^{1/2}\cE\Sigma_1^{1/2}(\beta_1 - v\beta_2)}^2
%		\le n_1^2 \bigbrace{2\sqrt{\frac{p}{n_1}} + \frac{p}{n_1}}^2 \bignorm{\Sigma_1^{1/2}K\Sigma_{2}K\Sigma_1^{1/2}} \cdot \bignorm{\Sigma_1^{1/2}(\beta_1 -  \beta_2)}^2.
%	\end{align*}
%	Combined together we have shown the right direction for $\delta_1 - \delta_2$.
%	For the left direction, we simply note that the third term in equation \eqref{eq_lem_model_shift_1} is positive.
%	And the second term is bigger than $-2n_1^2(2\sqrt{\frac{p}{n_1}} + \frac{p}{n_1}) \alpha$ using equation \eqref{eq_isometric}.
\end{proof}
Note by \eqref{eq_isometric}, we also obtain that with high probability,
\begin{align*}
&v^2 n_1^2 \Sigma_1^{1/2} K \Sigma_2 K \Sigma_1^{1/2} =n_1^2 \hat M (\hat M^\top Z_1^\top Z_1 \hat M + Z_2^\top Z_2)^{-2}\hat M^\top \\
&\preceq  n_1^2 \hat M \left[n_1 \al_-(\rho_1)\hat M^\top \hat M + n_2 \al_-(\rho_2) + \OO(p^{1/2+\e})\right]^{-2}\hat M^\top \\
&\preceq  \left[ \al_-^2(\rho_1) \hat M\hat M^\top + 2\frac{\rho_2}{\rho_1} \al_-(\rho_1)\al_-(\rho_2) + \left(\frac{\rho_2}{\rho_1}\right)^2 \al_-^2(\rho_2) (\hat M \hat M^\top )^{-1}\right]^{-1}+  \OO(p^{-1/2+\e}) \\
&\preceq [\al_-^2(\rho_1) \lambda_{\min}^2(\hat M)]^{-1}\cdot (1 - c)
\end{align*}
for a small enough constant $c>0$. Together with Claim \ref{prop_model_shift}, we get that with high probability,
\be\label{bounddelta-}
\left| \delta_{\bias}(v)-\wt\delta_{\bias}(v)\right| 
		\le (1-c) \delta(v)
\ee
for some small constant $c>0$, where recall that $\delta(v)$ was defined in \eqref{eq_deltaextra}.


\paragraph{Step III: The limit of $\wt\delta_{\bias}(v)$.} 
Using Lemma \ref{lem_cov_derivative}, we obtain that with high probability,
\begin{align*}
\wt\delta_{\bias}(v) &=\frac{\rho_1^2}{(\rho_1 + \rho_2)^2}\cdot v^2 (\beta_1-v\beta_2)^\top\Sigma_1 \Sigma_2^{-1/2}  \frac{(1 +  a_3(v))\id +v^2 a_4(v) {M}^{\top}{M}}{( v^2 a_1(v) {M}^{\top}{M}+a_2(v) )^2} \Sigma_2^{-1/2} \Sigma_1(\beta_1-v\beta_2) +\OO(p^{-1/2+\e}) \\
&= (\beta_1 - {v}\beta_2)^{\top} \Sigma_1^{1/2} \Pi(v) \Sigma_1^{1/2} (\beta_1 - {v}\beta_2) +\OO(p^{-1/2+\e}) =: \Delta_{\bias}(v) +\OO(p^{-1/2+\e}),
\end{align*}
where $a_3(v)$ and $a_4(v)$ satify \eqref{eq_a34extra} with $\hat v$ replaced by $v$, and $\Pi$ was defined in \eqref{defnpihat}. Together \eqref{deltavaral-} and \eqref{bounddelta-}, we obtain that w.h.p.,
\be\label{dicho_varbeta}
\begin{cases}\delta_{\vari}(v)>\delta_{\bias}(v), & \text{ if } \ \ \Delta_{\vari}(v) - \Delta_{\bias}(v) \ge   \delta(v),\\
\delta_{\vari}(v)<\delta_{\bias}(v),  & \text{ if }  \ \ \Delta_{\vari}(v) - \Delta_{\bias}(v) \le -  \delta(v).\end{cases}
\ee



\paragraph{Step IV: An $\e$-net argument.} Finally, it remains to extend the above result \eqref{dicho_varbeta} to $v=\hat v$, which is random and depends on $X_1$ and $X_2$. We first show that for any fixed constant $C_0>0$, there exists a high probability event $\Xi$ on which \eqref{dicho_varbeta} 
%\eqref{lem_cov_shift_eq} and \eqref{lem_cov_derv_eq} 
holds uniformly for all $v\in [-C_0, C_0]$. In fact, we first consider $v$ that belongs to a discrete set 
$$V:=\{v_k = kp^{-1}: -(C_0p +1)\le k \le C_0p +1\}.$$
Then using the arguments for the first three steps and a simple union bound, we get that
\eqref{dicho_varbeta} holds simultaneously for all $v\in V$ with high probability. On the other hand, by \eqref{eq_isometric} the event
$$\Xi_1:=\left\{ \al_-(\rho_1)/2 \preceq  \frac{Z_1^T Z_1}{n_1}  \preceq   2\al_+(\rho_1) ,\  \al_-(\rho_2)/2 \preceq  \frac{Z_2^T Z_2}{n_2}  \preceq   2\al_+(\rho_2)\right\}$$
holds with high probability. Now it is easy to check that on $\Xi_1$, the following estimates holds simultaneously for all $v_k \le v\le v_{k+1}$:
\begin{align*}
& |\delta_{\vari}(v) -\delta_{\vari}(v_k)|\lesssim p^{-1}\delta_{\vari}(v_k),\ \ |\delta_{\bias}(v) -\delta_{\bias}(v_k)|\lesssim p^{-1}\delta_{\bias}(v_k), \ \   |\delta(v)-\delta(v_k)|\lesssim p^{-1}\delta(v_k),\\
& |\Delta_{\bias}(v) -\Delta_{\bias}(v_k)|\lesssim p^{-1}\Delta_{\bias}(v_k),\ \ |\Delta_{\vari}(v) -\Delta_{\vari}(v_k)|\lesssim p^{-1}\Delta_{\vari}(v_k).
\end{align*}
Then a simple application of triangle inequality gives that the event 
$$\Xi_2=\{\eqref{dicho_varbeta} \text{ holds simultaneously for all }-C_0\le v \le C_0\}$$
holds with high probability. On the other hand, on $\Xi_1$ it is easy to check that for any small constant $\e>0$, there exists a large enough constant $C_0>0$ depending on $\e$ such that
\begin{align*}
& |\delta_{\vari}(v) -\delta_{\vari}(C_0)|\le \e\delta_{\vari}(C_0),\quad |\delta_{\bias}(v) -\delta_{\bias}(C_0)|\le \e\delta_{\bias}(C_0), \quad  |\delta(v)-\delta(C_0)|\le \e\delta(C_0),\\
& |\Delta_{\bias}(v) -\Delta_{\bias}(C_0)|\le \e\Delta_{\bias}(C_0),\quad |\Delta_{\vari}(v) -\Delta_{\vari}(C_0)|\le \e\Delta_{\vari}(C_0),
\end{align*}
for all $v\ge C_0$. Similar estimates hold for $v\le -C_0$ if we replace $C_0$ with $-C_0$ in the above estimates. Together with the estimate at $\pm C_0$, we get that \eqref{dicho_varbeta} holds simultaneously for all $v\in \R$ on the high probability event $\Xi_1\cap \Xi_2$. This concludes the proof of Lemma \ref{thm_model_shift}, since $\hat v$ must be one of the real values.
\end{proof}





\subsection{Proof of the Multiple-task Case}\label{app_proof_many_tasks}

%As a remark, since the spectral norm of $U_r$ is less than $1$, we have that $\norm{U_r(i)} < 1$ for all $1 \le i \le t$. Compared to Theorem \ref{thm_main_informal}, we can get a simple expression for the two functions $\Delta_{vari}$ and $\Delta_{\bias}$. The proof of Theorem \ref{thm_many_tasks} can be found in Appendix \ref{app_proof_many_tasks}.



%In this section we consider the setting with $k$ many that have the same covariates.
%Since every task has the same number of data points as well as the same covariance, the only differences between different tasks are their models $\set{\beta_i}_{i=1}^k$.
%For this setting, we derive solutions for the multi-task training and the transfer learning setting that match our insights qualitatively from Section \ref{sec_denoise}.
%Using Lemma \ref{lem_minv} and some concentration bounds, we can complete the proof of Theorem \ref{thm_many_tasks}.

\begin{proof}[Proof of Theorem \ref{thm_many_tasks}]
	We first introduce several notations.
	Let $\cal W = [W_1, W_2, \dots, W_t] \in \real^{r\times t}$ be a matrix that contains all the parameters of the output layers.
	Let $U_X U_X^{\top} = X (X^{\top} X)^{-1} X^{\top} \in\real^{n \times n}$ denote the projection matrix to $X$.

	Recall that all tasks have the same covariates in this setting.
	Therefore, the optimization objective equation \eqref{eq_mtl} becomes the following
	\begin{align}
		f(B; W_1, \dots, W_t) = \sum_{i=1}^t \bignorm{X B W_i - Y_i}^2, \label{eq_mtl_same_cov}
	\end{align}
	where $B \in \real^{p \times r}$ and $W_1, W_2, \dots, W_t \in \R^r$ for $1 < r < t$ by our assumption.
	Using the local optimality condition over $B$, that is $\nabla_B f = 0$, we obtain a closed form solution $\hat{B}$ that depends on the output layers as follows
	\begin{align*}
		\hat{B}(W_1, \dots, W_t) &= (X^{\top}X)^{-1} X^{\top} \bigbrace{\sum_{i=1}^t Y_i W_i^{\top}} (\cal W  \cal W^{\top})^{-1} \\
		&= (B^\star \cal W ^{\top}) (\cal W \cal W ^{\top})^{-1} + (X^{\top}X)^{-1}X^{\top}   \bigbrace{\sum_{i=1}^t \varepsilon_i W_i^{\top}} (\cal W \cal W^{\top})^{-1}.
	\end{align*}
	Plugging in the equation $\hat{B}$ back into equation \eqref{eq_mtl_same_cov} we obatin
	\begin{align}
		g(\cal W) = \sum_{j=1}^t \bignorm{X B^{\star} \cW^{\top} (\cW\cW^{\top})^{-1} W_j + U_X U_X^{\top} \bigbrace{\sum_{i=1}^t \varepsilon_i W_i^{\top}} (\cW\cW^{\top})^{-1} W_j - Y_i}^2. \label{eq_mtl_output_layer}
	\end{align}
	Recall that we obtain $\hat{\cW}$ by minimizing $\cW$ in the above equation.
	In order to derive the minimizer, our proof involves two steps.
	First, we consider the expectation of $g(\cW)$ over $\varepsilon_1, \varepsilon_2, \dots, \varepsilon_t$, and $X$.
	We show that the minimizer of $\ex{g(\cW)}$ has a simple closed form solution similar to principal component analysis.
	Second, we show that the concentration error between $g(\cW)$ and $\ex{g(\cW)}$ is small provided with $n$ samples.
	Intuitively, this is because $\cW$ only has $r \times t$ parameters, which is much smaller than $n$ by our assumption.
	Hence we use standard concentration bounds and $\epsilon$-net arguments to show that the concentration error is small for $g(\cW)$.
	To facilitate the analysis, we divide $g(\cW)$ into three parts based on their degree of dependence on the random noise:
	\begin{align*}
		Z_0 &= \sum_{j=1}^t \bignorm{XB^{\star} \cW^{\top} (\cW\cW^{\top})^{-1} W_j - X\beta_j}^2 = \bignorm{X B^{\star} (\cW^{\top} (\cW\cW^{\top})^{-1} \cW - \id_{t\times t})}^2, \\
		Z_1 &= \sum_{j=1}^t \inner{XB^{\star} \cW^{\top} (\cW\cW^{\top})^{-1}W_j - X\beta_j}{U_X U_X^{\top} \bigbrace{\sum_{i=1}^t \varepsilon_i W_i^{\top}} (\cW\cW^{\top})^{-1} W_j - \varepsilon_j} \\
				&= \sum_{j=1}^t \inner{-X B^{\star} \cW^{\top} (\cW \cW^{\top})^{-1} W_j + X\beta_j}{\varepsilon_j}, \\
		Z_2 &= \sum_{j=1}^t \bignorm{U_X U_X^{\top} \bigbrace{\sum_{i=1}^t \varepsilon_i W_i^{\top}} (\cW\cW^{\top})^{-1}W_j - \varepsilon_j}^2 \\
				&= \bigbrace{\sum_{j=1}^t \norm{\varepsilon_j}^2} - \sum_{1\le i,j\le t} \bigbrace{W_i^{\top} (\cW \cW^{\top})^{-1} W_j} \cdot \bigbrace{\varepsilon_i^{\top} U_X U_X^{\top} \varepsilon_j}.
	\end{align*}
	For the second equation of $Z_1$, we use the fact that $\sum_{j=1}^t X B^{\star} \cW^{\top} (\cW\cW^{\top})^{-1} W_j W_j^{\top} = X B^{\star} \cW^{\top} = \sum_{j=1}^t X \beta_j W_j^{\top}$.
	For the second equation of $Z_2$, we observe that $\sum_{j=1}^t (\cW\cW^{\top})^{-1} W_j W_j^{\top} = \id$.
	%Now we switch $\hat{B}$ back into equation \eqref{eq_mtl_same_cov} to
	%Then as in \eqref{approxvalid}, we pick $N$ independent samples of the training set for each task with $N\ge n^{1-\e_0}$, and use the concentration result, Lemma \ref{largedeviation}, to get the validation loss as

	\medskip
	\noindent\textbf{The optimal solution of $\ex{g(\cW)}$.}
	We observe that $\ex{g(\cW)}$ admits a bias-variance decomposition as follows.
	First, we take expectation over $\varepsilon_1, \dots, \varepsilon_t$ conditional on $X$ and obtain the following:
	\begin{align}
		\exarg{\varepsilon_1, \dots, \varepsilon_t}{g(\cW) \mid X} - Z_0
		= \exarg{\varepsilon_1, \dots, \varepsilon_t}{Z_2 \mid X}
%		=& \sum_{j=1}^t \bigbrace{{\ex{\norm{\varepsilon_j}^2}} - \exarg{\varepsilon_1, \dots, \varepsilon_t}{\bignorm{U_X U_X^{\top} \bigbrace{\sum_{i=1}^t \varepsilon_i W_t^{\top}} (\cW\cW^{\top})^{-1} W_j}^2 } }  \label{eq_empirical_2}\\
		= \sigma^2(n\cdot t - p\cdot r) \nonumber
	\end{align}
%	\begin{align}
%		\exarg{\varepsilon_1, \dots, \varepsilon_t}{g(\cW) \mid X}
%		=& \bignorm{X B^{\star} \cW^{\top} (\cW\cW^{\top})^{-1} \cW - X B^{\star}}^2 \nonumber \\
%		&+ \sum_{j=1}^t\bigbrace{\exarg{\varepsilon_1, \dots, \varepsilon_t}{\norm{\varepsilon_j}^2 - 2\inner{U_X U_X^{\top} \sum_{i=1}^t\bigbrace{\varepsilon_i W_i^{\top}} (\cW\cW^{\top})^{-1} W_j}{\varepsilon_j}} } \nonumber \\
%		&+ \sum_{j=1}^t \exarg{\varepsilon_1, \dots, \varepsilon_t}{\bignorm{U_X U_X^{\top}
%\bigbrace{\sum_{i=1}^t  \varepsilon_i W_i^{\top}} (\cW\cW^{\top})^{-1} W_j}^2} \nonumber \\
%		=& \norm{XB^{\star} (\cW^{\top}(\cW\cW^{\top})^{-1}\cW - \id)}^2	\label{eq_empirical_1}\\
%			&+ \sum_{j=1}^t \bigbrace{{\ex{\norm{\varepsilon_j}^2}} - \exarg{\varepsilon_1, \dots, \varepsilon_t}{\bignorm{U_X U_X^{\top} \bigbrace{\sum_{i=1}^t \varepsilon_i W_t^{\top}} (\cW\cW^{\top})^{-1} W_j}^2 } }  \label{eq_empirical_2}\\
%		=& \bignorm{X B^{\star} \cW^{\top} (\cW\cW^{\top})^{-1} \cW - XB^{\star}}^2 + \sigma^2(n\cdot t - p\cdot r) \nonumber
%	\end{align}
	The first equation uses the same fact the expectation of $Z_1$ over the random noise is zero.
	The second equation uses the fact that $\varepsilon_i$ and $\varepsilon_j$ are pairwise independent for any $i \neq j$, and the variance of every entry of $\varepsilon_i$ is $\sigma^2$ for every $i = 1,\dots, t$.
	We further take expectation over $X$ for $Z_0$ and obtain the following:
	\begin{align*}
		\ex{g(\cW)} = n \bignorm{\Sigma^{1/2} B^{\star} (\cW^{\top} (\cW \cW^{\top})^{-1} \cW - \id_{t\times t})}^2 + \sigma^2 (n\cdot t - p \cdot r).
	\end{align*}
	%\be\label{eq_multival}g(\cal W)=  N\left[\val(\cal W) + t  \sigma^2 \right]\cdot \left( 1+\OO(p^{-(1-\e_0)/2+\e})\right).\ee
\iffalse
{\color{red}
\begin{align*}
	f(W_1, \dots, W_t) = \sum_{i=1}^t \bignorm{X \hat B W_i - Y_i}^2 = \val(\cal W) \cdot \left( 1+\OO(p^{-1/2+\e})\right),
\end{align*}
where
\begin{align*}
	 \val(\cal W) &= \sum_{i=1}^t  \bignorm{X\bigbrace{(B^\star \cal W^\top) (\cal W\cal W^{\top})^{-1} W_i - \beta_i}}^2 + \E_{\e_i:1\le i \le t}\sum_{i=1}^t\left\| X(X^{\top}X)^{-1}X^{\top} \bigbrace{\sum_{i=1}^t \varepsilon_i W_i^{\top}} (\cal W \cal W^{\top})^{-1}W_i -\e_i\right\| \\
	 &= \sum_{i=1}^t  \bignorm{X\bigbrace{(B^\star \cal W^\top) (\cal W\cal W^{\top})^{-1} W_i - \beta_i}}^2 + \sigma^2 (nt-pr).
\end{align*}

}
\fi
%Here $\val(\cal W)$ is defined as
%it remains to consider minimizing the validation loss
%	$$\val( \cal W):=\exarg{\varepsilon_j, \forall 1\le j\le t}{ \sum_{i=1}^t \bignorm{\Sigma^{1/2}( \hat B W_i - \beta_i)}^2} =  \delta_{\bias}(\cal W) + \delta_{\vari} ,$$
%where the model shift bias term $\delta_{\bias}(\cal W) $ is given by
%	\begin{align*}
%		\delta_{\bias}(\cal W) :=\sum_{i=1}^t  \bignorm{\Sigma^{1/2}\bigbrace{(B^\star \cal W^\top) (\cal W\cal W^{\top})^{-1} W_i - \beta_i}}^2,
%	\end{align*}
%	and the variance term $\delta_{\vari} $ can be calculated as
%	\begin{align*}
%		\delta_{\vari} := \sigma^2 \cdot \bigtr{\Sigma (X^{\top}X)^{-1}}.
%	\end{align*}
	One can see that the above equation provides a bias-variance decomposition of $\ex{g(\cW)}$.
	The first part measures the bias of multi-task learning compared to single-task learning for all tasks.
	The second part measures the variance of multi-task learning and scales with $\sigma^2$.
	In order to minimize $\ex{g(\cW)}$, it suffices to minimize the bias equation over $\cal W$, since the variance equation does not depend on $\cal W$.
	%We denote $Q := \cal W^{\top} (\cal W\cal W^{\top})^{-1} \cal W \in\real^{t\times t}$, whose $(i,j)$-th entry is given by $W_i^{\top} (\cal W\cal W^{\top})^{-1} W_j$.
	%Let $B^{\star} = [\beta_1, \beta_2, \dots, \beta_k] \in\real^{p \times k}$ denote the true model parameters.
	%Now we can write $\delta_{\bias}(\cal W)$ succinctly as
	%\begin{align*}
	%	\delta_{\bias}(Q) \equiv \delta_{\bias}(\cal W) = \bignormFro{\Sigma^{1/2}B^{\star}  %\bigbrace{Q -\id}}^2 .
	%\end{align*}
	From the bias equation, we observe that the minimizer is equal to the best rank-$r$ (subspace) approximation to ${B^{\star}}^{\top} \Sigma B^{\star}$, which is denoted by $U_{r} U_r^{\top}$.

	\medskip
	\noindent\textbf{The concentration error between $g(\cW)$ and $\ex{g(\cW)}$.}
	Next, we show that the concentration errors of $Z_0, Z_1, Z_2$ are all lower order terms compared to $\ex{g(\cW)}$ and scale with $1/\sqrt{p}$.
	For $Z_1$, using Lemma \ref{largedeviation} in Appendix \ref{sec_maintools} and the fact that all moments of $\varepsilon_i$ exist, we conclude that
	\[ \abs{Z_1} \le \sum_{j=1}^t \sigma \cdot \norm{X B^{\star} \cW^{\top} (\cW\cW^{\top})^{-1} W_j - X \beta_j} \le \sigma \cdot \sqrt{t \cdot Z_0}, \]
	where the second part follows by using Cauchy-Shwartz inequality.
	For $Z_0$ and $Z_2$, \todo{add the spectral argument}
	%We provide tights bounds on the concentration error from the randomness of $\varepsilon_1, \dots, \varepsilon_t$, and $X$, respectively.
	%\begin{align}
	%	g(\cW) &= \exarg{\varepsilon_1, \dots, \varepsilon_t}{g(\cW) \mid X} \cdot (1 \pm \OO(p^{-1/2 + c})), \text{ and} \label{approxvalid_1} \\
	%	\exarg{\varepsilon_1, \dots, \varepsilon_t}{g(\cW) \mid X} &= \ex{g(\cW)} \cdot(1 + o(1)) \label{approxvalid_2}
	%\end{align}
	%Together, they imply that $g(\cW) = \ex{g(\cW)} \cdot (1 + o(1))$. Therefore, next we focus on proving equation \eqref{approxvalid_1} and \eqref{approxvalid_2}.

	%For equation \eqref{approxvalid_1}, we observe that the summands of $g(\cW)$ w.r.t. $\varepsilon_1, \dots, \varepsilon_t$ belong to either of the following three types:
	%\begin{enumerate}
	%	\item[(i)] $\varepsilon_i^{\top} A \varepsilon_i$, for any $i = 1,\dots, t$ and a fixed $A\in\real^{n\times n}$ that is independent of $\varepsilon_i$;
	%	\item[(ii)] $\varepsilon_i^{\top} A \varepsilon_j$, for any $i \neq j$ and a fixed $A$ that is independent of both $\varepsilon_i$ and $\varepsilon_j$;
	%	\item[(iii)] $\varepsilon_i^{\top} A$, for any $i = 1,\dots, t$ and a fixed $A\in\real^n$ that is independent of $\varepsilon_i$.
	%\end{enumerate}
	%For all three types, using Lemma \ref{largedeviation} in Appendix \ref{sec_maintools} and the fact that all moments of $\varepsilon_i$ exist, we conclude that
	%\begin{enumerate}
	%	\item[(i)] $\varepsilon_i^{\top} A \varepsilon_i = \sigma^2 (1 + \OO(p^{-1/2 + c})) \normFro{A}^2$;
	%	\item[(ii)] $\abs{}$
	%	\item[(iii)]
	%\end{enumerate}

	%We use the fact that our random vectors have i.i.d. entries.
%Before doing that, we first need to fix the setting for the following discussions, because we want to keep track of the error rate carefully instead of obtaining an asymptotic result only.
	%Recall that $Y_i = X_i\beta_i + \varepsilon_i$ and $\wt Y_i = \wt X_i\beta_i + %\wt\varepsilon_i$, $i=1,2$, all satisfy Assumption \ref{assm_secA2}. Then we rewrite %\eqref{eq_mtl_2tasktilde} as
%$$	g( v) = \sum_{i=1}^2\left\| \wt X_i\wt\beta_i  - \wt \e_i\right\|^2 , \quad \wt\beta:=\hat %B w_i-\beta_i.$$
	%Since $ \wt X_i\wt\beta$ and $ \wt \e_i$ are independent random vectors with i.i.d. centered entries, we can use the concentration result,  to get that for any constant $\e>0$,
	%\begin{align*}
		%\left|\left\| \wt X_i\wt\beta_i  - \wt \e_i\right\|^2 -  \exarg{\wt X_i,\wt{\e}_i} {\left\| \wt X_i\wt\beta_i  - \wt \e_i\right\|^2} \right| & =\left|\left\| \wt X_i\wt\beta_i  %- \wt \e_i\right\|^2 - N_i (\wt\beta_i^\top \Sigma_i \wt\beta_i + \sigma_i^2) \right| \\
%&\le N_i^{1/2+\e} (\wt\beta_i^\top \Sigma_i \wt\beta_i + \sigma_i^2),
	%$\end{align*}
	%with high probability. Thus we obtain that
	%$$g(v)= \left[\sum_{i=1}^2 N_i\left\|\Sigma_i^{1/2}( \hat B w_i - \beta_i) \right\|^2 + (N_1\sigma^2_1+N_2\sigma^2_2)\right]\cdot \left( 1+\OO(p^{-(1-\e_0)/2+\e})\right),$$
%where we also used $N_i\ge p^{-1+\e_0}$. Inserting \eqref{hatB} into the above expression and using
	% again the concentration result, Lemma \ref{largedeviation}, we get that
	%$$ \sum_{i=1}^2 N_i\left\|\Sigma_i^{1/2}( \hat B w_1 - \beta_i) \right\|^2 = \val(v)\cdot \left( 1+\OO(p^{-1/2+\e})\right)$$
%with high probability.
%-----old-------
%Suppose that the entries of $\e_1$ and $\e_2$ have variance $\sigma^2$.  Using a validation set that is sub-sampled from the original training dataset, we get a validation loss as follows
%\begin{align}
%		&\val(\hat{B}; w_1, w_2):= \exarg{\varepsilon_1,\e_2} \sum_{i=1}^2 \left\|\Sigma_i^{1/2}( \hat B w_1 - \beta_i) \right\|^2 \\
%	&=  n_1 \cdot \bignorm{\Sigma_1^{1/2}\left(\frac{w_1^2}{w_2^2} X_1^{\top}X_1 + X_2^{\top}X_2\right)^{-1}X_2^{\top}X_2\left (\beta_s - \frac{w_1}{w_2}\beta_t\right)}^2 \nonumber \\
%		&+ n_1 \sigma^2 \cdot \frac{w_1^2}{w_2^2} \bigtr{\left(\frac{w_1^2}{w_2^2}  X_1^{\top}X_1 + X_2^{\top}X_2\right)^{-1}\Sigma_1} \nonumber \\
%		&+ n_2 \cdot \frac{w_1^2}{w_2^2}\bignorm{\Sigma_2^{1/2}\left(\frac{w_1^2}{w_2^2} X_1^{\top}X_1 + X_2^{\top}X_2\right)^{-1}X_1^{\top}X_1\left(\beta_s - \frac{w_1}{w_2}\beta_t\right)}^2 \nonumber \\
%		&+ n_2 \sigma^2 \cdot \bigtr{\left(\frac{w_1^2}{w_2^2} X_1^{\top}X_1 + X_2^{\top}X_2\right)^{-1}\Sigma_2}. \label{eq_val_mtl}
%\end{align}
%\nc
%------------------
%Thus we conclude the proof.

	\medskip
	\noindent\textbf{Dealing with the empirical minimizer.}
	Let $\hat {\cal W}$ be the minimizer of $g$, and denote $\hat Q:= \hat{\cal W}^{\top} (\hat{\cal W}\hat{\cal W}^{\top})^{-1} \hat{\cal W} $. We claim that $\hat Q$ satisfies
	\be\label{Q-Q}\|Q_0^{-1}\hat Q - \id\|_F = \oo(1) \quad \text{w.h.p.}\ee
	In fact, if \eqref{Q-Q} does not hold, then using the condition $\lambda_{\min}((B^{\star})^\top\Sigma B^{\star})\gtrsim \sigma^2$ and that $\delta_{\vari}(\cal W)=\OO(\sigma^2)$ by  Lemma \ref{lem_minv}, we obtain that
	$$   \val(\hat {Q}) + t  \sigma^2 > (\val( Q_0) + t \sigma^2 )\cdot (1+\oo(1)) \ \Rightarrow \ g( \hat{\cal W})>g( {\cal W}_0),$$
	where $\cal W_0\in \R^{r\times t}$ is a matrix such that  $ \cal W_0^{\top} (\cal W_0\cal W_0^{\top})^{-1} \cal W_0=Q_0$. Hence $\hat {\cal W}$ is not a minimizer, which leads to a contradiction.

	\medskip
	\noindent\textbf{Putting both steps together.}
	In sum, we have solved that $\hat{\beta}_i^{\MTL}=B^{\star}\left( U_r v_i +\oo(1)\right)$. Inserting it into the definition of the test loss, we get that with high probability,
	\begin{align*}
		\te(\hat{\beta}_t^{\MTL}) &= \bignorm{\Sigma^{1/2} \left((B^\star \hat{\cal W}^\top) (\hat{\cal W}\hat{\cal W}^{\top})^{-1} \hat W_t - \beta_t \right) }^2
		+ \sigma^2  \hat W_t^{\top} (\hat{\cal W}\hat{\cal W}^{\top})^{-1} \hat W_t \cdot \bigtr{\Sigma (X^{\top}X)^{-1}} \\
		&= \bignorm{\Sigma^{1/2} \bigbrace{B^{\star} U_r v_t-\beta_t}}^2 + \oo\left(\|B^\star\|^2\right) + \sigma^2\norm{v_t}^2 \bigtr{\Sigma (X^{\top}X)^{-1}} \cdot (1+\oo(1))\\
		&= \bignorm{\Sigma^{1/2} \bigbrace{B^{\star} U_r v_t-\beta_t}}^2 + \frac{\sigma^2}{\rho-1}\norm{v_t}^2 + \oo \left( \|B^\star\|^2 + \sigma^2\right),
	\end{align*}
 where we used Lemma \ref{lem_minv} in the last step. On the other hand, by Lemma \ref{lem_minv} we have
	$$\te(\hat{\beta}_t^{\STL})=\frac{\sigma^2}{\rho-1} \cdot \left( 1+\oo(1)\right).$$
Combining the above two estimates, we conclude the proof.
\end{proof}
%From the above we can obtain three conceptual insights that are consistent with Section \ref{sec_denoise} and \ref{sec_insight}.
%\begin{itemize}
%	\item The de-noising effect of multi-task learning.
%	\item Multi-task training vs single-task training can be either positive or negative.
%	\item Transfer learning is better than the other two. And the improvement over multi-task training increases as the model distances become larger.
%\end{itemize}

