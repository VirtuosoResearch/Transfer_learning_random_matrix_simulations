\documentclass{article}

\usepackage{aistats2021_author_response}

\usepackage[utf8]{inputenc} % allow utf-8 input
\usepackage[T1]{fontenc}    % use 8-bit T1 fonts
\usepackage{hyperref}       % hyperlinks
\usepackage{url}            % simple URL typesetting
\usepackage{booktabs}       % professional-quality tables
\usepackage{amsfonts}       % blackboard math symbols
\usepackage{nicefrac}       % compact symbols for 1/2, etc.
\usepackage{microtype}      % microtypography
\usepackage{xcolor}         % define colors in text
\usepackage{xspace}         % fix spacing around commands


\begin{document}

	We thank all the reviewers for providing detailed comments.
	We are more than happy to address the weaknesses pointed out by the reviewers and improve our work.
	In this response, we first clarify several high-level questions and then respond to each reviewer's comments.

	\textbf{Clarifying the setting of Section 3.} Both R2 and R3 pointed out that our covariate shift analysis only applies to two tasks where the rank of the shared matrix $B$ is $1$.
	While this setting is only a special case of the two-layer linear neural net formulation, it corresponds to the important setting of transfer learning (specifically solving $f(B) = ||X^{(1)} B - Y^{(1)}||_F^2 + ||X^{(2)} B - Y^{(2)}||_F^2$).
	The transfer learning problem with two tasks is well motivated from a practical perspective, and to the best of our knowledge, our work provides the first upper bound in the transfer learning setting of linear regression under covariate shift.
	We think it is an excellent research question to study the multiple task setting under covariate shift.
	Several recent work has made progress in the multiple linear regression setting, including (Li et al., 2020), (Chen et al., 2020 arXiv:2009.11282), (Kong et al., 2020 arXiv:2002.08936).
	However, they all assume that the feature vectors are drawn from an isotropic Gaussian distribution.
	Therefore, while we acknowledge that our analysis only applies to two tasks, we believe our work has made an important contribution toward studying multi-task learning under covariate shift.

	\textbf{The gap between the theory and experiment in Fig 2.b?} Thanks to R2 and R3 for pointing out this gap.
	We have improved our result so that the theory matches experiment in Fig 2.b (see Figure \ref{fig_update} for the updated result).
	DESCRIBE THE IDEA.

	\textbf{Can any of the results in Section 3 be obtained from a straightforward application of a generalized MP theorem?}
	Thanks for raising this question. R3 is correct that the variance equation in Corollary 3.3 can be obtained from the MP law.
	However, to the best of our knowledge, the bias equation in Corollary 3.3 and Example 3.4 cannot be obtained from the MP law.
	ADD COMMENT TO THAT PAPER.

	\textbf{Response to R1.}

	\textbf{Response to R2.}

	\textbf{Response to R3.}

	\textbf{Response to R5.}
\end{document}
