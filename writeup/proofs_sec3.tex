\section{Supplementary Materials for the Technical Results}\label{app_proof_sec3}

From \cite{WZR20}, we know that we need to explicitly restrict the capacity $r$ of $B$ so that there is transfer between the two tasks.
for the rest of the section, we shall consider the case when $r=1$ we are considering the case of two tasks.
Here, equation \eqref{eq_mtl} simplifies to the following
\begin{align}\label{eq_mtl_2task}
	f(B; w_1, w_2) = \bignorm{X_1 B w_1 - Y_1}^2 + \bignorm{X_2 B w_2 - Y_2}^2,
\end{align}
where $B\in\real^p$ and $w_1, w_2$ are both real numbers. To solve the above problem, suppose that $w_1, w_2$ are fixed, by local optimality, we find the optimal $B$ as
\begin{align*}
	& \hat{B}(w_1, w_2) \\
	=& (w_1^2 X_1^{\top}X_1 + w_2^2 X_2^{\top}X_2)^{-1} (w_1 X_1^{\top}Y_1 + w_2 X_2^{\top}Y_2) \\
	=& \frac{1}{w_2} \left( \frac{w_1^2}{w_2^2}  X_1^{\top}X_1 + X_2^{\top}X_2\right)^{-1} \left(\frac{w_1}{w_2} X_1^{\top}Y_1 + X_2^{\top}Y_2\right) \\
	=& \frac{1}{w_2}\bigbrace{\beta_t + \left(\frac{w_1^2}{w_2^2} X_1^{\top}X_1 + X_2^{\top}X_2\right)^{-1}\bigbrace{X_1^{\top}X_1\left(\frac{w_1}{w_2}\beta_s - \frac{w_1^2}{w_2^2} \beta_t\right) + \left(\frac{w_1}{w_2} X_1^{\top}\varepsilon_1 + X_2^{\top}\varepsilon_2\right)}}.
\end{align*}
As a remark, when $w_1 = w_2 = 1$, we recover the linear regression estimator.
The advantage of using $f(B; w_1, w_2)$ is that if $\theta_1$ is a scaling of $\theta_2$, then this case can be solved optimally using equation \eqref{eq_mtl_2task} \cite{KD12}.

%For the discussions below, we assume that the entries of $\e_1$ and $\e_2$ all have the same variance $\sigma^2$. This holds for most parts of our discussion, except in Proposition \ref{prop_var_transition}. We will derive different expressions for the validation loss and the test error 

\textbf{Defining the multi-task learning estimator.} Suppose that the entries of $\e_1$ and $\e_2$ have variance $\sigma^2$.  Using a validation set that is sub-sampled from the original training dataset, we get a validation loss as follows
\begin{align}
		\val(\hat{B}; w_1, w_2)
	&=  n_1 \cdot \bignorm{\Sigma_1^{1/2}\left(\frac{w_1^2}{w_2^2} X_1^{\top}X_1 + X_2^{\top}X_2\right)^{-1}X_2^{\top}X_2\left (\beta_s - \frac{w_1}{w_2}\beta_t\right)}^2 \nonumber \\
		&+ n_1 \sigma^2 \cdot \frac{w_1^2}{w_2^2} \bigtr{\left(\frac{w_1^2}{w_2^2}  X_1^{\top}X_1 + X_2^{\top}X_2\right)^{-1}\Sigma_1} \nonumber \\
		&+ n_2 \cdot \frac{w_1^2}{w_2^2}\bignorm{\Sigma_2^{1/2}\left(\frac{w_1^2}{w_2^2} X_1^{\top}X_1 + X_2^{\top}X_2\right)^{-1}X_1^{\top}X_1\left(\beta_s - \frac{w_1}{w_2}\beta_t\right)}^2 \nonumber \\
		&+ n_2 \sigma^2 \cdot \bigtr{\left(\frac{w_1^2}{w_2^2} X_1^{\top}X_1 + X_2^{\top}X_2\right)^{-1}\Sigma_2}. \label{eq_val_mtl}
\end{align}
Let $\hat{w_1}/\hat{w_2}$ be the global minimizer of $\val(\hat{B}; w_1, w_2)$.
We will define the multi-task learning estimator for the target task as
	\[ \hat{\beta}_t^{\MTL} = \hat{w}_{2}\hat{B}(\hat{w}_1, \hat{w}_2). \]

The intuition for deriving $\hat{\beta}_t^{\MTL}$ is akin to performing multi-task training in practice.
Let $\hat{v} = \hat{w_1} / \hat{w_2}$ for the simplicity of notation.
The test loss of using $\hat{\beta}_t^{\MTL}$ for the target task is
\begin{align}
	\te(\hat{\beta}_t^{\MTL}) &=~ \hat{v}^2 \bignorm{\Sigma_2^{1/2}(\hat{v}^2 X_1^{\top}X_1 + X_2^{\top}X_2)^{-1} X_1^{\top}X_1 (\beta_s - \hat{v} \beta_t)}^2 \nonumber \\
			&+~  \sigma^2 \cdot \bigtr{(\hat{v}^2 X_1^{\top}X_1 + X_2^{\top}X_2)^{-1} \Sigma_2}. \label{eq_te_mtl_2task}
\end{align}

Our goal is to study under model and covariate shifts, whether multi-task learning helps learn the target task better than single-task learning.
The baseline where we solve the target task with its own data is
\begin{align*}
	L(\hat{\beta}_t^{\STL}) = \sigma^2 \cdot \bigtr{\Sigma_2(X_2^{\top}X_2)^{-1}}, \text{ where } \hat{\beta}_t^{\STL} = (X_2^{\top}X_2)^{-1} X_2^{\top}Y_2.
\end{align*}
%Clearly, whether $\hat{\beta}_t^{\MTL}$ outperforms $\hat{\beta}_t^{\STL}$ depends on the covariate matrices, the difference of the task models, and the number of per-task data points.

We now state several helper lemmas to get a bound on the variance of $\hat{\beta}_t^{\STL}$ and $\hat{\beta}_t^{\MTL}$. Before that, we first need to fix the setting for the discussions in this section.

\begin{assumption}\label{assm_secA1}
We will consider two $n\times p$ random matrices of the form $X=Z\Sigma^{1/2}$, where $\Sigma$  is a $p\times p$ deterministic positive definite symmetric matrices, and $Z=(z_{ij})$ is an $n\times p$ random matrix with real i.i.d. entries with mean zero and variance one. Note that the rows of $X$ are i.i.d. centered random vectors with covariance matrix $\Sigma$. For simplicity, we assume that all the moments of $z_{ij}$ exists, that is, for any $k\in \N$,
\begin{equation}\label{assmAhigh}
\mathbb{E} |z_{ij}|^k \le C_k ,\quad 1\le i \le n, 1\le j \le p,
\end{equation}
for some constant $C_k>0$. We assume that $n=\rho p$ for some fixed constant $\rho>1$. Without loss of generality, after a rescaling we can assume that the norm of $\Sigma$ is bounded by a constant $C>0$. Moreover, we assume that $\Sigma$ is well-conditioned: $\kappa(\Sigma)\le C$, where $\kappa(\cdot)$ denotes the condition number. 
\end{assumption}

Here we have assumed \eqref{assmAhigh} solely for simplicity of representation. if the entries of $Z$ only have finite $a$-th moment for some $a>4$, then all the results below still hold except that we need to replace $\OO(p^{-\frac12+\e})$ with $\OO( p^{-\frac12+\frac2a +\epsilon})$ in the error bounds. We will not get deeper into this issue in this section, but refer the reader to Corollary \ref{main_cor} below. 


The first lemma, which is a folklore result in random matrix theory, helps to determine the asymptotic limit of $\te(\hat{\beta}_t^{\STL})$, as $p\to \infty$. When the entries of $X$ are multivariate Gaussian, this lemma recovers the classical result for the mean of inverse Wishart distribution \cite{anderson1958introduction}. For general non-Gaussian random matrices, it can be obtained from Stieltjes transform method; see e.g., Lemma 3.11 of \cite{bai2009spectral}. Here we shall state a result obtained from Theorem 2.4 in \cite{isotropic}, which gives an almost sharp error bound. Throughout the appendix, we shall say an event $\Xi$ holds with high probability (whp) if for any fixed $D>0$, $\P(\Xi)\ge 1- p^{-D}$ for large enough $p$.

\begin{lemma}\label{lem_minv} 
Suppose $X$ satisfies assumption \ref{assm_secA1}. Let $A$ be any $p\times p$ matrix that is independent of $X$. We have that for any constant $\e>0$,
	\be\label{XXA}  \bigtr{(X^{\top}X)^{-1}A} = \frac{1}{\rho - 1} \frac1p\tr(\Sigma^{-1}A) +\bigo{ \|A\|p^{-1/2+\epsilon}} \ee
with high probability.
\end{lemma}


We shall refer to random matrices of the form $X^\top X$ as sample covariance matrices following the standard notations in high-dimensional statistics. The second lemma extends Lemma \ref{lem_minv} for a single sample covariance matrices to the sum of two independent sample covariance matrices. It is the main random matrix theoretical input of this paper. 
%which deals with the inverse of the sum of two random matrices, which 
%any is can be viewed as a special case of Theorem \ref{thm_model_shift}.

\begin{lemma}\label{lem_cov_shift}
	%Let $X_i\in\real^{n_i\times p}$ be a random matrix that contains i.i.d. row vectors with mean $0$ and variance $\Sigma_i\in\real^{p\times p}$, for $i = 1, 2$.
	Suppose $X_1=Z_1\Sigma_1^{1/2}\in \R^{n_1\times p}$ and $X_2=Z_2\Sigma_2^{1/2}\in \R^{n_2\times p}$ satisfy Assumption \ref{assm_secA1} with $\rho_1:=n_1/p>1$ and $\rho_2:=n_2/p>1$. 
	Denote by $M = \Sigma_1^{1/2}\Sigma_2^{-1/2}$ and let $\lambda_1, \lambda_2, \dots, \lambda_p$ be the singular values of $M^{\top}M$ in descending order. Let $A$ be any $p\times p$ matrix that is independent of $X_1$ and $X_2$. We have that for any constant $\e>0$,
	%When $n_1 = c_1 p$ and $n_2 = c_2 p$, we have that with high probability over the randomness of $X_1$ and $X_2$, the following equation holds
	\begin{align}\label{lem_cov_shift_eq}
		\bigtr{(X_1^{\top}X_1 + X_2^{\top}X_2)^{-1}A} = \frac{1}{\rho_1+\rho_2}\cdot \frac1p\bigtr{ (a_1 \Sigma_1 + a_2\Sigma_2)^{-1} A} +\bigo{\|A\| p^{-1/2+\epsilon}},
	\end{align}
with high probability, where $(a_1, a_2)$ is the solution to the following deterministic equations:
	\begin{align}
		a_1 + a_2 = 1- \frac{1}{\rho_1 + \rho_2},\quad a_1 + \frac1{\rho_1 + \rho_2}\cdot \frac{1}{p}\sum_{i=1}^p \frac{\lambda_i^2 a_1}{\lambda_i^2 a_1 + a_2} = \frac{\rho_1}{\rho_1 + \rho_2}. \label{eq_a12extra}
	\end{align}
\end{lemma}

Finally, the last lemma describes the asymptotic limit of $(X_1^{\top}X_1 + X_2^{\top}X_2)^{-2}$.

\begin{lemma}\label{lem_cov_derivative}
In the setting of Lemma \ref{lem_cov_shift}, let $\beta \in \R^p$ be any unit vector that is independent of $X_1$ and $X_2$. We have that for any constant $\e>0$,
\begin{equation}\label{lem_cov_derv_eq}
\begin{split}
&(n_1+n_2)^2\bignorm{\Sigma_2^{1/2} (X_1^{\top}X_1 + X_2^{\top}X_2)^{-1}\beta}^2 \\
&= \beta^{\top} \Sigma_2^{-1/2}  \frac{(1 + a_3)\id + a_4 {M}^{\top}{M}}{(a_2 + a_1 {M}^{\top}{M})^2} \Sigma_2^{-1/2} \beta +\OO(p^{-1/2+\e}),
\end{split}
\end{equation}
with high probability, where $a_{3}$ and $a_4$ satisfy the following system of linear equations: 
\begin{gather}\label{eq_a34extra}
		\left(\rho_2  a_2^{-2}-  b_0\right)\cdot  a_3 - b_1 \cdot  a_4
		= b_0, \quad \left(\rho_1 a_1^{-2} -  b_2  \right)\cdot  a_4 -  b_1 \cdot  a_3 = b_1 .  
%		\left(\frac{n_1}{\hat a_1^2} -  \sum_{i=1}^p \frac{\hat \lambda_i^4   }{  (\hat a_2 + \hat \lambda_i^2\hat a_1)^2  }\right)\hat a_4 -\left(\sum_{i=1}^p \frac{\hat \lambda_i^2  }{  (\hat a_2 + \hat \lambda_i^2\hat a_1)^2  }\right)\hat a_3
%		= \sum_{i=1}^p \frac{\hat \lambda_i^2 }{  (\hat a_2 + \hat \lambda_i^2\hat a_1)^2  }. \label{eq_a4}
	\end{gather}
Here $b_0$, $b_1$ and $b_2$ are defined as
$$ b_k:= \frac1{p}\sum_{i=1}^p \frac{ \lambda_i^{2k}}{ ( a_2 + \lambda_i^2 a_1)^2  },\quad k=0,1,2.$$
\end{lemma}
%We will give the proof of this lemma in Section \ref{sec_maintools}.

The proof of Lemma \ref{lem_cov_shift} and Lemma \ref{lem_cov_derivative} is a main focus of Section \ref{sec_maintools}. We remark that one can probably derive the same asymptotic result using free probability theory (see e.g. \cite{nica2006lectures}), but our results \eqref{lem_cov_shift_eq} and \eqref{lem_cov_derv_eq} also give an almost sharp error bound $\bigo{ p^{-1/2+\epsilon}}$.


