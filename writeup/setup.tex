\subsection{Setup and Main Results}\label{sec_setup}

We assume that for every row $x$ of $X_i$, we have $\ex{xx^{\top}} = \Sigma_i.$
We also write $x = \Sigma_i^{1/2} z_i$, where $z_i$ is a random vector with mean $0$ and variance $1$.
We will designate the $k$-th task as the target.
Our goal is to come up with an estimator $\hat{\beta}$ to provide accurate predictions for the target task, provided with the other auxiliary task data.
Concretely, we focus on the test error for the target task:
%\begin{itemize}
%	\item Estimation error for the target model $\beta_t$: we consider their distance
%		\[ \err(\hat{\beta}) \define \exarg{\varepsilon_i, \forall 1\le i\le k}{\bignorm{\hat{\beta} - \beta_t}^2}. \]
		\begin{align*}
			\te_k(\hat{\beta}) &\define \exarg{x \sim \Sigma_k}{\exarg{{\varepsilon_i, \forall 1\le i\le k}}{(x^{\top}\hat{\beta} - x^{\top}\beta_t)^2}} \\
			&= \exarg{\varepsilon_i, \forall 1\le i\le k}{(\hat{\beta} - \beta_t)^{\top}\Sigma_k(\hat{\beta} - \beta_t)}.
		\end{align*}
%\end{itemize}

\todo{show that $\te_k(\hat{\beta}_t^{\TL})$ is less than both $\te_k(\hat{\beta}_t^{\MTL})$ and $\te_k(\hat{\beta}_t^{\STL})$.}

\paragraph{Hypothesis on Heterogeneous Task Data}

Our hypothesis is that the heterogeneity among the multiple tasks can be categorized into two classes, \textit{covariate shift} and \textit{model shift}. %\todo{Add some references to add spice on these takes.}
We consider two natural questions within each category.
\begin{itemize}
	\item \textbf{Model shift.}
	In general the single-task models can also be different across different tasks.
We shall argue that in addition to the bias and variance terms of generalization error, model shift introduces a third term which is the bias caused by model shift.

	Under model shift, when do we get positive vs. negative transfer?
	How does the type of transfer depend on the number of data points, the distance of the task models etc?
	\item \textbf{Covariate shift.}
	The covariance matrices $\Sigma_i$ may be be different across tasks, i.e. having different spectrum or singular vectors.
This is also known as covariate shift in the literature.
	How does covariate shift affect the rate of information transfer?
	For example, is it better to have the same covariance matrix or not?
\end{itemize}


\paragraph{The High-Dimensional Setting.}
%\todo{setup motivation and notations}
We would like to get insight on how covariate and model shifts affect the rate of transfer.
We will consider the high-dimensional setting where for the target task, its number of data points is a small constant times $p$.
This setting captures a wide range of applications of multi-task learning where we would like to use auxiliary task data to help train tasks with limited labeled data.
Furthermore, this setting is particularly suited to our study since there is need for adding more data to help learn the target task.

For the case of two tasks, we can get precise rates using random matrix theory.
For the sake of clarity, we call task 1 the source task and task 2 the target task,
i.e. $\beta_1 = \beta_s$ and $\beta_2 = \beta_t$.
We introduce the following notations for the high-dimensional setting
\[ c_{n_1} \define \frac{n_1}{p} \to c_1, \quad c_{n_2} \define \frac{n_2}p \to c_2, \quad \text{as } \ n_1, n_2\to \infty, \]
for some constants $c_1, c_2 \in (1,\infty)$.
A crucial quantity is what we call the \textit{covariate shift} matrix $M = \Sigma_1^{1/2}\Sigma_2^{-1/2}$.
Let $\lambda_1, \lambda_2, \dots, \lambda_p$ denote the singular values of $M$.
% \gamma_n:= \frac{p} {n_1 + n_2} \to \gamma,
