\section{A Technical Tool to Analyze the Bias-Variance Trade-off}
\label{sec_main}

We first describe the bias-variance tradeoff in our setting more formally.
Then, we develop a key lemma for analyzing the tradeoff using random matrix theory.
%We develop a technical tool to derive $\te(\hat{\beta}_t^{\MTL})$ that only depends on the qualities of task data such as data sizes and covariance matrices, for two tasks with general covariances.
%Then, we extend the result to more than two tasks that share the same features but have different labels.
Finally, we apply the key lemma to provide a sharp analysis of the bias-variance tradeoff of multi-task and transfer learning in our setting.

\subsection{The Bias-Variance Tradeoff}\label{sec_setup}

Recall that we have $t$ labeled tasks available, denoted by $(X_1, Y_1), (X_2, Y_2), \dots, (X_t, Y_t)$, where $X_i\in\real^{n_i\times p}$ and $Y_i\in\real^{n_i}$ for $1\le i\le t$.
%Following \cite{HMRT19,BLLT20}, we assume that for each task $i = 1,2,\dots,t$,  every feature vector is generated as $x = \Sigma_i^{1/2} z$, where $z\in\real^p$ is a random vector with i.i.d. entries of mean zero and unit variance and $\Sigma_i\in\real^{p\times p}$ is a positive semidefinite matrix.
Without loss of generality, let the $t$-th task denote the target task.
For an estimator $\hat{\beta}\in\real^p$, we define the out-of-sample (prediction) loss as
	\begin{align*}
		\te_t(\hat{\beta}) \define \exarg{z}{\exarg{\varepsilon_t}{({(\Sigma_t^{1/2} z)}^{\top}\hat{\beta} - {(\Sigma_t^{1/2})}^{\top}\beta_t)^2}}
		= \exarg{\varepsilon_t}{(\hat{\beta} - \beta_t)^{\top}\Sigma_t(\hat{\beta} - \beta_t)}.
	\end{align*}
The single-task estimator $\hat{\beta}_t^{\STL}$ is given by $(X_t^{\top}X_t)^{-1}X_t^{\top}Y_t$.
The bias-variance trade-off \cite{HTF09} says
	\[ \te_t(\hat{\beta}) =
		\bignorm{\exarg{\varepsilon_t}{\hat{\beta}} - \beta_t}^2 + \exarg{\varepsilon_t}{\bignorm{\hat{\beta} - \exarg{\varepsilon_t}{\hat{\beta}}}^2}. \]
In order to study the trade-off between model-shift bias and variance reduction, we need tight concentration bounds to quantify both effects.
For this purpose, we consider the high-dimensional regime where $n_i$ is a fixed constant $\rho_i > 1$ times $p$ for every $1\le i\le t$, and $p$ is large.
%Recall that $n_i = \rho_i \cdot p$ and we assume $\rho_i > 1$ is a fixed constant for every $1\le i\le t$.

We focus on a setting where $\rho_t$ is a small constant.
This setting captures the need for adding more labeled data to reduce the test error of the target task.
A well-known result for this setting states that $\te_t(\hat{\beta}_t^{\STL}) = \sigma^2 \cdot \tr[(X_t^{\top}X_t)^{-1}\Sigma_t]$ is concentrated around $\frac {\sigma^2} {\rho_t - 1}$ (e.g. Chapter 6 of \cite{S07}), which scales with the data size and noise level of the target task.
However, this result only applies to the single-task setting.
Therefore, our goal is to extend this result to the multi-task setting.


\textbf{Notations.}
When there is no ambiguity, we drop the subscript $t$ from $\te_t(\hat{\beta}_t^{\MTL})$ to $\te(\hat{\beta}_t^{\MTL})$ for simplicity.
We refer to the first task as the source task when there are only two tasks.
We call $M = \Sigma_1^{1/2}\Sigma_2^{-1/2}$ the covariate shift matrix.

\subsection{A Key Lemma using Random Matrix Theory}\label{label_rmt}

To illustrate our intuition, we begin by considering the setting of two tasks with general covariance matrices.
Recall that $\hat{\beta}_t^{\MTL}$ is defined as $BW_t$ after solving equation \eqref{eq_mtl}.
We decompose the test error of $\hat{\beta}_{t}^{\MTL}$ on the target task into two parts (to be derived in Appendix \ref{app_proof_sec3}) as follows
\begin{align}
	\te(\hat{\beta}_t^{\MTL}) =& ~ \hat{v}^2 \bignorm{\Sigma_2^{1/2} (\hat{v}^2 X_1^{\top}X_1 + X_2^{\top}X_2)^{-1}X_1^{\top}X_1 (\beta_1 - \hat{v}\beta_2)}^2 \label{eq_te_model_shift} \\
	&+ \sigma^2\cdot \bigtr{(\hat{v}^2 X_1^{\top}X_1 + X_2^{\top}X_2)^{-1}\Sigma_2}, \label{eq_te_var}
\end{align}
where $\hat{v} = W_2 / W_1$ denotes the ratio of the output layer weights.
The scaling term $\hat{v}$ corresponds to the intuition that the shared module $B$ learns a subspace for both tasks.
The output layer of each task scales the direction of $B$ suitably to fit the task.

We term equation \eqref{eq_te_model_shift} as \textit{model-shift bias}, which captures how similar $\beta_1$ and $\beta_2$ are.
This part introduces a negative effect to multi-task learning.
Next, it is not hard to verify that equation \eqref{eq_te_var}, the variance of $\hat{\beta}_t^{\MTL}$, is always smaller than the variance of $\hat{\beta}_t^{\STL}$.
This part introduces a positive variance reduction effect to performing multi-task learning.
Hence, whether $\te(\hat{\beta}_t^{\MTL})$ is lower than $\te(\hat{\beta}_t^{\STL})$ is determined by the tradeoff between two effects:
(i) the positive effect from variance reduction;
(ii) the negative effect from model shift bias.
In order to analyze the tradeoff, we present the following lemma, which provides a tight bound for equation \eqref{eq_te_var}.

\begin{lemma}[Informal statement of Lemma \ref{lem_cov_shift}]\label{lem_cov_shift_informal}
	In the setting of two tasks, let $M = \Sigma_1^{1/2}\Sigma_2^{-1/2}$ and $\lambda_1, \lambda_2, \dots, \lambda_p$ be the singular values of $M^{\top}M$ in descending order.
	For any constant $\e>0$, w.h.p. over the randomness of $X_1, X_2$, we have that
	\begin{align*}
		\bigtr{(X_1^{\top}X_1 + X_2^{\top}X_2)^{-1}\Sigma_2} = \frac{1}{\rho_1+\rho_2}\cdot \frac1p\bigtr{ (a_1 \Sigma_1 + a_2\Sigma_2)^{-1} \Sigma_2} +\bigo{\|\Sigma_2\| p^{-1/2+\epsilon}},
	\end{align*}
where $(a_1, a_2)$ is the solution to the following deterministic equations:
	\begin{align*}
		a_1 + a_2 = 1- \frac{1}{\rho_1 + \rho_2},\quad a_1 + \frac1{\rho_1 + \rho_2}\cdot \frac{1}{p}\sum_{i=1}^p \frac{\lambda_i^2 a_1}{\lambda_i^2 a_1 + a_2} = \frac{\rho_1}{\rho_1 + \rho_2}.
	\end{align*}
\end{lemma}

As a remark, when there is only the target task, $\rho_1$ equals zero.
Hence $a_2 = 1 - 1/ \rho_2, a_1 = 0$ and Lemma \ref{lem_cov_shift_informal} states that $\tr[(X_2^{\top}X_2)^{-1}\Sigma_2] = \sigma^2 / (\rho_2 - 1)$, which is to a well-known result in random matrix theory \cite{S07}.

\subsection{A Sharp Analysis of the Tradeoff for Multi-Task and Transfer Learning}

\textbf{Multi-task learning.}
We use Lemma \ref{lem_cov_shift_informal} to provide a sharp bias-variance tradeoff for two settings:
(i) two tasks with general covariance matrices;
(ii) any number of tasks that have the same features.
This setting is prevalent in applications of multi-task learning to image classification, where there are multiple prediction labels/tasks for every image \cite{chexnet17,EA20}.
We state our result for two tasks as follows.
%The formal statement is stated in Theorem \ref{thm_many_tasks} and its proof can be found in Appendix \ref{app_proof_many_tasks}.
%The technical crux of our approach is to derive the asymptotic limit of $\te(\hat{\beta}_t^{\MTL})$ in the high-dimensional setting, when $p$ approaches infinity.
%We derive a precise limit of $\bigtr{(X_1^{\top}X_1 + X_2^{\top}X_2)^{-1}\Sigma_2}$, which is a deterministic function that only depends on $\Sigma_1, \Sigma_2$ and $n_1/p, n_2/p$ (see Lemma \ref{lem_cov_shift} in Appendix \ref{app_proof_main} for the result).
%Based on the result, we show how to determine positive versus negative transfer as follows.

\begin{theorem}[Informal statement of Theorem \ref{thm_model_shift}]\label{thm_main_informal}
	For the setting of two tasks, let $\delta > 0$ be a desired error margin and $\rho_1 > ??$.
	There exists two deterministic functions $\Delta_{\beta}$ and $\Delta_{\vari}$ that only depend on $\set{\hat{v}, \Sigma_1^{}, n_1, n_2, \beta_1, \beta_2}$ such that
	\squishlist
		\item If $\Delta_{\vari} - \Delta_{\beta} \ge \delta$, then whp $\te(\hat{\beta}_t^{\MTL}) < \te(\hat{\beta}_t^{\STL})$.
		\item If $\Delta_{\vari} - \Delta_{\beta} \le \delta$, then whp $\te(\hat{\beta}_t^{\MTL}) > \te(\hat{\beta}_t^{\STL})$.
	\squishend
\end{theorem}

Theorem \ref{thm_main_informal} shows nearly tight bounds on the trade-off between model-shift bias and variance reduction.
%determined by the covariate shift matrix and the model shift.
The bounds get tighter and tighter as $\rho_1$ increases.
While the general form of $\Delta_{\vari}$ and $\Delta_{\beta}$ can be quite complex, we will show that they provide nice interpretation for simplified settings later on in Section \ref{sec_insight}.
The proof of Theorem \ref{thm_main_informal}  is presented in Appendix \ref{app_proof_main}.
The result for more than two tasks is deferred to Appendix \ref{app_proof_many_tasks}.

\textbf{Transfer learning.}
We extend the intuition behind Theorem \ref{thm_main_informal} to transfer learning settings.
We provide an analysis of the transfer function of Taskonomy \cite{ZSSGM18} using our setup.
Specifically, the source task encoder consists of the representations learnt from one or more source tasks.
The transfer function then tries to fit the target task data to the source task encoder.
For more details, we refer the reader to Figure 4 in Taskonomy.

We map the procedure to our setup as follows.
First, we obtain the single-task estimator $\hat{\beta}_i$ from the source tasks, for $1\le i \le t-1$.
This forms the shared representation $B = [\hat{\beta}_1,\hat{\beta}_2,\dots,\hat{\beta}_{t-1}]$.
Then, we learn the output layer $W_t$ on the target task by minimizing the following objective
\begin{align}
	g(W_t) = \bignorm{X_t B W_t - Y_t}^2.
\end{align}
After solving $W_t$, we use $\hat{\beta}_t^{\TL} = B W_t$ as the estimator for the target task.
By comparing $\te(\hat{\beta}_t^{\TL})$ to $\te(\hat{\beta}_t^{\STL})$, we observe a similar trade-off between model-shift bias and variance reduction for this setting.
%We use our tools to compare $\te(B W_t)$ to $\te(\beta_t^{\STL})$.
The formal statement is presented in Theorem \ref{prop_taskonomy} and its proof in Appendix \ref{app_proof_sec4}.


