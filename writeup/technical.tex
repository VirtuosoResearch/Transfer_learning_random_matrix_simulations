\section{Technical Tools: Quantifying Model Shift Bias versus Variance Trade-off}\label{sec_main}

In this part, we consider the case of two tasks to show  establish the intuition that adding more data helps in multi-task learning by reducing the variance of the estimator.
We achieve this through tight generalization bounds obtained from random matrix theory.
For the case of two tasks, we identify three factors that determine the type of transfer between tasks: model distance, covariate shift matrix, and data ratio.

Recall that the test error of $\hat{\beta}_{t}^{\MTL}$ consists of two parts
\begin{align}
	\te(\hat{\beta}_t^{\MTL}) = \hat{w}^2 \bignorm{\Sigma_2^{1/2} (\hat{w}^2 X_1^{\top}X_1 + X_2^{\top}X_2)^{-1}X_1^{\top}X_1 (\beta_s - \hat{w}\beta_t)}^2 + \sigma^2\cdot \bigtr{(\hat{w}^2 X_1^{\top}X_1 + X_2^{\top}X_2)^{-1}\Sigma_2}. \label{eq_te_model_shift}
\end{align}
It is not hard to show that the variance of $\hat{\beta}_t^{\MTL}$ is reduced compared to $\hat{\beta}_t^{\STL}$ (following the argument of Proposition \ref{prop_monotone}), i.e.
\[ \sigma^2\cdot \bigtr{(\hat{w}^2 X_1^{\top}X_1 + X_2^{\top}X_2)^{-1}\Sigma_2} \le \sigma^2\cdot \bigtr{(X_2^{\top}X_2)^{-1}\Sigma_2}. \]
Because of model shift however, i.e. $\beta_s \neq \beta_t$.
We can no longer guarantee that $\te(\hat{\beta}_{t}^{\MTL}) \le \te(\hat{\beta}_t^{\STL})$.
The main result of this part show deterministic conditions under which we get positive or negative transfer.
And the conditions depend only on the covariate shift matrix $M$, the difference of the task models, and the number of per-task data points.
In order to characterize $\te(\hat{\beta}_t^{\MTL})$ and $\te(\hat{\beta}_t^{\STL})$, the technical crux of our approach relies on deriving the limit of the trace of matrix inverse in the high-dimensional setting.
To illustrate the idea, we observe that by using Lemma \ref{lem_minv}, we have that
\[ \te(\hat{\beta}_t^{\STL}) = \frac{\sigma^2}{n_2 - p}\bigtr{\Sigma_2^{-1}}. \]
We shall also derive the limit of $\te(\hat{\beta}_t^{\MTL})$.
\todo{write a brief technical overview}

Let ${M} = \hat{w} \Sigma_1^{1/2}\Sigma_2^{-1/2}$ denote the weighted covariate shift matrix. Denote by ${\lambda}_1\ge {\lambda}_2 \ge \dots \ge {\lambda}_p$ the singular values of ${M}^{\top}{M}$. Let $(a_1, a_2)$ be the solutions to the following system of equations
	\be
		 a_1 + a_2 = 1- \frac{p}{n_1 + n_2},~ a_1 + \sum_{i=1}^p \frac{a_1}{(n_1 + n_2)(a_1 + a_2/ \lambda_i^2)} = \frac{n_1}{n_1 + n_2}.\label{eq_a2} \\
		 \ee
		 After obtaining $(a_1,a_2)$, we can solve the following linear equations to get $(a_3,a_4)$:
\begin{gather}
		\left(\frac{n_2}{a_2^2}- \sum_{i=1}^p \frac{1}{ (a_2 + \lambda_i^2a_1)^2  }\right) a_3 -  \left(\sum_{i=1}^p \frac{  \lambda_i^2 }{ (  a_2 + \lambda_i^2a_1)^2  }\right)a_4
		= \sum_{i=1}^p \frac{1 }{ (  a_2 + \lambda_i^2a_1)^2  }, \label{eq_a3} \\
		\left(\frac{n_1}{a_1^2} -  \sum_{i=1}^p \frac{\lambda_i^4   }{  (a_2 + \lambda_i^2a_1)^2  }\right)a_4 -\left(\sum_{i=1}^p \frac{\lambda_i^2  }{  (a_2 + \lambda_i^2a_1)^2  }\right)a_3
		= \sum_{i=1}^p \frac{\lambda_i^2 }{  (a_2 + \lambda_i^2a_1)^2  }. \label{eq_a4}
	\end{gather}
Then we introduce the following matrix
$$Z = \frac{n_1^2}{(n_1 + n_2)^2}\cdot{M} \frac{(1 + a_3)\id + a_4 {M}^{\top}{M}}{(a_2 + a_1 {M}^{\top}{M})^2} {M}^{\top},$$
which can be regarded as the asymptotic limit of $\hat w \Sigma_2^{1/2} (\hat{w}^2 X_1^{\top}X_1 + X_2^{\top}X_2)^{-1}\Sigma_1^{1/2}$. Finally we introduce
$$\delta:=\left[\frac{n_1 \lambda_1}{(\sqrt{n_1}-\sqrt{p})^2\lambda_p  +  (\sqrt{n_2}-\sqrt{p})^2}\right]^2\cdot \norm{\Sigma_1^{1/2}(\beta_s - \hat{w}\beta_t)}^2.$$
{\cor may be able to get a better bound, but the statement will be long}

We now state our main result for two tasks with both covariate and model shift in the following theorem.

\begin{theorem}\label{thm_model_shift}
	Let $n_1, n_2$ be the number of data points for the source, target task, respectively.
	Let $\hat{w}$ denote the optimal solution for the ratio $w_1/w_2$ in equation \eqref{eq_val_mtl}.
	%Let ${M} = \hat{w} \Sigma_1^{1/2}\Sigma_2^{-1/2}$ denote the weighted covariate shift matrix.
	%Denote by ${\lambda}_1, {\lambda}_2, \dots, {\lambda}_p$ the singular values of ${M}^{\top}{M}$.
	The information transfer is solely determined by two deterministic quantities $\Delta_{\beta}$ and $\Delta_{\vari}$, which show the change of model shift bias and variace, respectively.
	With high probability we have
	\begin{align}
	 	\te(\hat{\beta}_{t}^{\MTL}) \le \te(\hat{\beta}_t^{\STL}) \text{ when: } &\Delta_{\vari} - \Delta_{\beta} \ge \left(\bigbrace{1 + \sqrt{\frac{p} {n_1}}}^4 - 1 \right) \delta \label{upper}\\
		\te(\hat{\beta}_t^{\MTL}) \ge \te(\hat{\beta}_t^{\STL}) \text{ when: } &\Delta_{\vari} - \Delta_{\beta} \le -2\bigbrace{2\sqrt{\frac {p}{n_1}} + \frac{p}{n_1}} \delta, \label{lower}
	\end{align}
	where
	\begin{align*} %\bigtr{{\Sigma_2^{-1}}}
		\Delta_{\vari} &\define {\sigma^2}\bigbrace{\frac{p}{n_2 - p} -  \frac{1}{n_1 + n_2} \bigtr{(a_1 M^{\top}M + a_2\id)^{-1}} } \\
		\Delta_{\beta} &\define (\beta_s - \hat{w}\beta_t)^{\top} \Sigma_1^{1/2} Z \Sigma_1^{1/2} (\beta_s - \hat{w}\beta_t).
	\end{align*}
%	{\color{blue}To write the bounds in the form \eqref{upper} and \eqref{lower}, we have to define $\wh w$ as the minimizer of
%	$$\frac{\sigma^2}{n_1 + n_2} \bigtr{\frac1{a_1(w) M(w)^{\top}M(w) + a_2(w)\id}}+ \Delta_\beta(w).$$
%	Otherwise, the two bounds has to be written into the form
%	\begin{align*}
%	{\sigma^2}\frac{p}{n_2 - p}\ge \min_w \left\{\frac{\sigma^2}{n_1 + n_2} \bigtr{\frac1{a_1(w) M(w)^{\top}M(w) + a_2(w)\id}} + \Delta_\beta(w)+ \left(\bigbrace{1 + \sqrt{\frac{p} {n_1}}}^4 - 1 \right) \delta(w)\right\},
%	\end{align*}
%	and
%	\begin{align*}
%	{\sigma^2}\frac{p}{n_2 - p}\le \min_w \left\{\frac{\sigma^2}{n_1 + n_2} \bigtr{\frac1{a_1(w) M(w)^{\top}M(w) + a_2(w)\id}}+ \Delta_\beta(w) - 2\bigbrace{2\sqrt{\frac {p}{n_1}} + \frac{p}{n_1}} \delta(w)\right\}.
%	\end{align*}
%	}
\end{theorem}





Theorem \ref{thm_model_shift} shows upper and lower bounds that guarantee positive transfer, which is determined by the change of variance $\Delta_{\vari}$ and a certain model shift bias parameter $\Delta_{\beta}$ determined by the covariate shift matrix and the model shift.
The bounds get tighter and tighter as $n_1 / p$ increases.


\subsection{Extension to Many Tasks of the Same Covariates}

In this section we consider the setting with $k$ many that have the same covariates.
Since every task has the same number of data points as well as the same covariance, the only differences between different tasks are their models $\set{\beta_i}_{i=1}^k$.
For this setting, we derive solutions for the multi-task training and the transfer learning setting that match our insights qualitatively from Section \ref{sec_denoise}.


\section{Extension to Transfer Learning}

We study the transfer function of taskonomy \cite{ZSSGM18}.
The algorithm is as follows.
First, we obtain the single-task estimator $\hat{\beta}_i$ from every task, for $1\le i \le k$.
This forms the shared representation $B$ in Algorithm \ref{alg_estimator}.
Then, we learn the output layer on the target task.
We use our tools to analyze this setting as follows.