\section{Results for Two Tasks}\label{sec_defspike}


We would like to get insight on how covariate and model shifts result in slower rates of transfer.
For the case of two tasks, we can get precise rates in the low-dimensional setting using random matrix theory.
Since there are only two tasks, we call task 1 the source task and task 2 the target task,
i.e. $\beta_1 = \beta_s$ and $\beta_2 = \beta_t$.
%We denote task 1 as the source, i.e. $\beta_1 = \beta_s$.

\subsection{Covariate Shift}

First, we show that if $\beta_s$ and $\beta_t$ are equal, then combining the source and target task always helps.
%The key quantity is to look at:
The estimation error using the source and target together is
\begin{align}\label{eq_two_task}
  e(\hat{\beta}_{s,t}) = \sigma^2 \cdot \tr[(X_1^{\top}X_1 + X_2^{\top} X_2)^{-1}].
\end{align}
The estimation error using the target alone is
\begin{align}\label{eq_target_task}
	e(\hat{\beta}_t) = \sigma^2 \cdot \tr[(X_2^{\top} X_2)^{-1}].
\end{align}

\begin{proposition}\label{prop_monotone}
  When there is no model shift, adding the source task data always reduces the estimation error, i.e.
	\[ e(\hat{\beta}_{s,t}) \le e(\hat{\beta}_t). \]
\end{proposition}

\begin{proof}
	This is simply because $\bigtr{(X_1^{\top}X_1 + X_2^{\top}X_2)^{-1}} \le \bigtr{(X_1^{\top}X_1)^{-1}}$.
\end{proof}

Next, we calculate the amount of improvement by comparing equation \eqref{eq_two_task} to equation \eqref{eq_target_task}.
We can get a precise result that only depends on the covariance matrices $\Sigma_1, \Sigma_2$ and the number of data points $n_1, n_2$ as follows.
We shall consider the high-dimensional setting such that
$$\gamma_n:= \frac{p} {n_1 + n_2} \to \gamma,\quad c_n:= \frac{n_1}{n_1 + n_2} \to c, \quad \text{as } \ n_1, n_2\to \infty, $$
for some constants $\gamma\in (0,\infty)$ and $c \in (0,1)$.

\begin{lemma}\label{lem_model_shift}
	Let $\lambda_1, \lambda_2, \dots, \lambda_p$ denote the singular values of $\Sigma_1^{-1/2}\Sigma_2^{1/2}$.
	When there is no model shift, the amount of reduction on the estimation error is given as
	\begin{align*}
		e(\hat{\beta}_t) - e(\hat{\beta}_{s,t})
		= \sigma^2 p \cdot \bigtr{\frac 1 {(n_2 - p) \Sigma_2} - \frac 1 {(n_1 + n_2)a_1 \Sigma_1 + (n_1 + n_2)a_2 \Sigma_2}},
	\end{align*}
	where $a_1, a_2$ are the solutions of the following equations
	\begin{gather}
		 a_1 + a_2 = 1- \frac{p}{n_1 + n_2} \label{m35reduced_1}\\
		 a_1 +\frac1{n_1 + n_2}\sum_{i=1}^p \frac{a_1}{a_1 + \lambda_i^2[(1-\frac{p}{n_1 + n_2})-a_1]} = \frac{n_1}{n_1 + n_2}. \label{m35reduced_2}
	\end{gather}
\end{lemma}

As a remark, we see that Proposition \ref{prop_monotone} follows from Lemma \ref{lem_model_shift} because
\begin{align*}
	e(\hat{\beta}_{s,t}) \le e(\hat{\beta}_t)
	\Leftarrow~ & (n_2 - p)\Sigma_2 \preceq (n_1 + n_2) a_1 \Sigma_1 + (n_1 + n_2)a_2 \Sigma \\
	\Leftrightarrow~ & \zeroMatrix \preceq (n_1 + n_2) a_1 \Sigma_1 + (n_1 - (n_1 + n_2)\cdot a_1) \Sigma_2,
\end{align*}
which is true since $a_1 \le n_1 / (n_1 + n_2)$ by equation \eqref{m35reduced_2}.

\paragraph{Example 1: when $\Sigma_1 = \Sigma_2$.}
In this case, we have $\lambda_i = 1$ for all $1\le i\le p$.
And $a_1 + a_2 = 1 - p / (n_1 + n_2)$.
Hence
\[ e(\hat{\beta}_t) - e(\hat{\beta}_{s,t}) = \sigma^2 p \cdot \bigbrace{\frac{1}{n_2 - p} - \frac{1}{n_1 + n_2 - p}} \bigtr{\frac 1 {\Sigma}}. \]

\paragraph{Example 2: when $\Sigma_2 \neq \Sigma_2$.}
\todo{bla}

\bigskip
To prove Lemma \ref{lem_model_shift}, we study the spectrum of the random matrix model:
$$Q= \Sigma_1^{1/2}  Z_1^T Z_1 \Sigma_1^{1/2}  + \Sigma_2^{1/2}  Z_2^T Z_2 \Sigma_2^{1/2} ,$$
where $\Sigma_{1,2}$ are $p\times p$ deterministic covariance matrices, and $X_1=(x_{ij})_{1\le i \le n_1, 1\le j \le p}$ and $X_2=(x_{ij})_{n_1+1\le i \le n_1+n_2, 1\le j \le p}$ are $n_1\times p$ and $n_2 \times p$ random matrices, respectively, where the entries $x_{ij}$, $1 \leq i \leq n_1+n_2\equiv n$, $1 \leq j \leq p$, are real independent random variables satisfying
\begin{equation}\label{eq_12moment} %\label{assm1}
\mathbb{E} z_{ij} =0, \ \quad \ \mathbb{E} \vert z_{ij} \vert^2  = n^{-1}.
\end{equation} 
For now, we assume that the random variables $x_{ij}$ are i.i.d. Gaussian, but we know that universality holds for generally distributed entries. %have arbitrarily high moments, 
%in the sense that for any fixed $k\in \mathbb N$, there is a constant $\mu_k>0$ such that
%\begin{equation}\label{eq_highmoment} %\label{eqn:subgaus}
%\max_{i,j}\left(\mathbb E|x_{ij}|^k\right)^{1/k} \le \mu_k n^{-1/2},  %\var \left(h_{xy}\right)^{1/2}
%\end{equation}
%for all $n$. %For simplicity, we assume that $k$ is a finite fixed integer, the strengths $d_1 > d_2 > \cdots > d_k >0$ are fixed constants, and $ \bu_i$,  $\bv_i$ are deterministic unit vectors. 


\begin{lemma}
	In the setting of Lemma \ref{lem_model_shift}, we have with high probability $1-{\rm o}(1)$,
\begin{align*}
\tr ((X_1^{\top}X_1 + X_2^{\top}X_2)^{-1}) = \frac{p}{n_1+n_2}\cdot \tr \left[ \frac{1}{a_1\Sigma_1 + a_2\Sigma_2} \right] +\bigo{\frac{p}{n^{1-\epsilon}}}.
\end{align*}
	for any constant $\epsilon>0$. %, where $a_{3,4}$ are found using equations in  \eqref{m35reduced}.
	In particular, when $n_1 = 0$, we have that $a_1 = 0$ and $a_2 = (n_2-p) / n_2$, hence
	\[ \bigtr{(X_2^{\top}X_2)^{-1}} = \frac{p}{n_2-p}\bigtr{\frac 1 {\Sigma_2}} + \bigo{\frac{p}{n_2^{1-\varepsilon}}}. \]
\end{lemma}

We assume that $ \Sigma_1^{-1/2}\Sigma_2$ has eigendecomposition
\be  \label{eigen}
\Sigma_1^{-1/2}\Sigma_2^{1/2} = ODO^T ,\quad D=\text{diag}(d_1, \cdots, d_p).
\ee
Then by the rotational invariance of Gaussian matrices, we have
$$\wt Q \overset{d}{=}\Sigma_1^{1/2} O \wt Q O^T \Sigma_1^{1/2},\quad \wt Q:=   X_1^T X_1  + D X_2^T X_2 D .$$
Thus we study the spectrum of $\wt Q$ instead. We define $\cal G(z):= (\wt Q-z)^{-1}$ for $z\in \C_+$. With some random matrix tools, we have that 
$$\cal G(z) \approx \diag\left( \frac{1}{-z\left( 1+ m_3(z) + d_i^2 m_4(z)\right)}\right)_{1\le i \le p}= \frac{1}{-z\left( 1+ m_3(z) + D^2 m_4(z)\right)} $$
{\cob in certain sense}. Here $m_{3,4}(z)$ satisfy the following self-consistent equations
%$$\frac{1}{G_{ii}} \approx -z \left( 1+m_3 + d_i^2m_4 \right), \quad \frac{1}{G_{\mu\mu}} = -z(1+m_1), \ \ \mu\in \cal I_1,\quad \frac{1}{G_{\nu\nu}} = -z(1+m_2), \ \ \nu\in \cal I_2,$$
%$$m_1= \frac1n\sum_{i}G_{ii}, \quad m_2= \frac1n\sum_{i}d_i^2 G_{ii}, \quad m_3 = \frac1n\sum_{\mu\in \cal I_1} G_{\mu\mu},\quad m_4 = \frac1n\sum_{\mu\in \cal I_2} G_{\mu\mu}.$$
\begin{align}\label{m34}
\frac{n_1}{n}\frac1{m_3} = - z +\frac1n\sum_{i=1}^p \frac1{  1+m_3 + d_i^2m_4  } ,\quad \frac{n_2}{n}\frac1{m_4} = - z +\frac1n\sum_{i=1}^p \frac{d_i^2 }{  1+m_3 + d_i^2m_4  } 
\end{align}
When $z\to 0$, we shall have
$$m_3(z)= - \frac{a_3}{z} + \OO(1), \quad m_4(z)= - \frac{a_4}{z} + \OO(1),\quad a_3,a_4>0.$$
Then for $z\to0$, the equations in \eqref{m34} are reduced to
\begin{align}\label{m35}
\frac{n_1}{n}\frac{1}{a_3} = 1 +\frac1n\sum_{i=1}^p \frac{1}{a_3 + d_i^2a_4  } ,\quad \frac{n_2}{n}\frac1{a_4} = 1 +\frac1n\sum_{i=1}^p \frac{d_i^2 }{  a_3 + d_i^2 a_4 }. 
\end{align}
First, it is easy to see that these equations are equivalent to
\begin{align} a_3 + a_4 = 1- \gamma_n, \quad a_3 +\frac1n\sum_{i=1}^p \frac{a_3}{a_3 + d_i^2[(1-\gamma_n)-a_3]  }=c_n  .\end{align}
Furthermore, we have
\begin{align*}
\tr (Q^{-1}) &= \lim_{z\to 0}\tr \left[\Sigma_1^{-1/2} O \cal G(z)O^T \Sigma_1^{-1/2}\right]
=\tr \left[\Sigma_1^{-1/2} O  \left( \frac{1}{a_3 + D^2 a_4 }\right) O^T \Sigma_1^{-1/2}\right] \\
&=\tr \left[\Sigma_1^{-1/2}  \frac{1}{a_3+ \Sigma_1^{-1}\Sigma_2 a_4} \Sigma_1^{-1/2}\right]=\tr \left[ \frac{1}{a_3\Sigma_1 + a_4\Sigma_2 } \right].
\end{align*}

\subsection{Model Shift}

In this case, $\beta_s$ and $\beta_t$ are different.
If we put the two tasks together, we get
\begin{align}
	\hat{\beta}_{s,t} &= (X_1^{\top}X_1 + X_2^{\top}X_2)^{-1} (X_1^{\top}Y_1 + X_2^{\top}Y_2) \\
	&= (X_1^{\top}X_1 + X_2^{\top}X_2)^{-1} \bigbrace{(X_1^{\top}X_1\beta_s + X_2^{\top}X_2\beta_t) + (X_1^{\top}\varepsilon_1 + X_2^{\top}\varepsilon_2)}
\end{align}
Hence
\begin{align}
	\exarg{\varepsilon}{\bignorm{\hat{\beta}_{s,t} - \beta_t}}
	= \bignorm{(X_1^{\top}X_1 + X_2^{\top}X_2)^{-1} X_1^{\top}X_1 (\beta_s - \beta_t)}^2
	+ \sigma^2 \bigtr{(X_1^{\top}X_1 + X_2^{\top}X_2)^{-1}}
\end{align}

We are now interested in the following quantity 
$$ (\beta_s - \beta_t)^{\top}X_1^{\top}X_1 \frac{1}{X_1^{\top}X_1 + X_2^{\top}X_2}  \Sigma_2 \frac{1}{X_1^{\top}X_1 + X_2^{\top}X_2} X_1^{\top}X_1 (\beta_s - \beta_t) $$
We consider the same setting as in previous subsection: 
$$ X_1^{\top}X_1:=\Sigma_1^{1/2}  Z_1^T Z_1 \Sigma_1^{1/2} ,\quad X_2^{\top}X_2= \Sigma_2^{1/2}  Z_2^T Z_2 \Sigma_2^{1/2},$$
where $z_{ij}$, $1 \leq i \leq n_1+n_2\equiv n$, $1 \leq j \leq p$, are real independent random variables satisfying \eqref{eq_12moment}. For now, we assume that the random variables $z_{ij}$ are i.i.d. Gaussian, but we know that universality holds for generally distributed entries. Assume that $p/n_1$ is a small number such that $Z_1^TZ_1$ is roughly an isometry, that is, under \eqref{eq_12moment}, 
$$ \left\| Z_1^T Z_1 -  \frac{n_1}{n} I \right\| \le 2\sqrt{\frac{p}{n}} + {\frac{p}{n}} .$$ 
Then we have
$$  \left\|X_1^{\top}X_1 (\beta_s - \beta_t)  - \frac{n_1}{n}(\beta_s - \beta_t)\right\|_2 \le C \sqrt{\frac{p}{n}} \left\| (\beta_s - \beta_t)\right\|_2. $$
It remain to study the following expression 
\begin{align*}
\frac{1}{X_1^{\top}X_1 + X_2^{\top}X_2}  \Sigma_2 \frac{1}{X_1^{\top}X_1 + X_2^{\top}X_2}  = \Sigma_2^{-1/2}\left(\frac{1}{A^T Z_1^{\top}Z_1 A + Z_2^{\top}Z_2} \right)^2 \Sigma_2^{-1/2} \\
\stackrel{d}{=} \Sigma_2^{-1/2}V \left(\frac{1}{\Lambda Z_1^{\top}Z_1 \Lambda + Z_2^{\top}Z_2} \right)^2 V^T \Sigma_2^{-1/2},
\end{align*}
where 
\be  \label{eigen2}
A:=\Sigma_1^{1/2}\Sigma_2^{-1/2} = U\Lambda V^T ,\quad \Lambda=\text{diag}(\lambda_1, \cdots, \lambda_p).
\ee
Using 
$$\left(\frac{1}{\Lambda Z_1^{\top}Z_1 \Lambda + Z_2^{\top}Z_2} \right)^2 =\left. \frac{\dd }{\dd z}\right|_{z=0}\frac{1}{\Lambda Z_1^{\top}Z_1 \Lambda + Z_2^{\top}Z_2 - z} ,$$
we  need to study the resolvent of 
$$G(z) = \left( \Lambda Z_1^{\top}Z_1 \Lambda + Z_2^{\top}Z_2 - z \right)^{-1}.$$
Its local law can be studied as in previous subsection (be careful we need to switch the roles of $Z_1$ and $Z_2$). More precisely,  we have that 
$$ G(z) \approx \diag\left( \frac{1}{-z\left( 1+ m_3(z) + \lambda_i^2 m_4(z)\right)}\right)_{1\le i \le p}= \frac{1}{-z\left( 1+ m_3(z) + \Lambda^2 m_4(z)\right)} .$$
Here $m_{3,4}(z)$ satisfy the following self-consistent equations
%$$\frac{1}{G_{ii}} \approx -z \left( 1+m_3 + d_i^2m_4 \right), \quad \frac{1}{G_{\mu\mu}} = -z(1+m_1), \ \ \mu\in \cal I_1,\quad \frac{1}{G_{\nu\nu}} = -z(1+m_2), \ \ \nu\in \cal I_2,$$
%$$m_1= \frac1n\sum_{i}G_{ii}, \quad m_2= \frac1n\sum_{i}d_i^2 G_{ii}, \quad m_3 = \frac1n\sum_{\mu\in \cal I_1} G_{\mu\mu},\quad m_4 = \frac1n\sum_{\mu\in \cal I_2} G_{\mu\mu}.$$
\begin{align}\label{m34shift}
\frac{n_2}{n}\frac1{m_3} = - z +\frac1n\sum_{i=1}^p \frac1{  1+m_3 + \lambda_i^2m_4  } ,\quad \frac{n_1}{n}\frac1{m_4} = - z +\frac1n\sum_{i=1}^p \frac{\lambda_i^2 }{  1+m_3 + \lambda_i^2m_4  } .
\end{align}
Then we calculate the derivatives of $u_3:=zm_3$ and $u_4:=zm_4$ with respect to $z$:
\begin{align}\label{dotm34}
\frac{n_2}{n}\frac1{u_3^2}u'_3 =  \frac1n\sum_{i=1}^p \frac{1+u'_3 + \lambda_i^2 u'_4}{ ( z+u_3 + \lambda_i^2u_4)^2  } ,\quad \frac{n_1}{n}\frac1{u_4^2}u'_4 =   \frac1n\sum_{i=1}^p \frac{\lambda_i^2\left(1+u'_3 + \lambda_i^2 u'_4\right) }{  (z+u_3 + \lambda_i^2u_4)^2  } .
\end{align}
We can solve the above equations to get $u'_3$ and $u'_4$. Then we have 
$$ G^2(z) \approx   \frac{1 +  u_3'(z) +\Lambda^2 u'_4(z)}{ \left( z+ u_3(z) + \Lambda^2  u_4(z)\right)^2}  $$
{\cob in certain sense}.

We now simplify the expressions for $z\to 0$ case. When $z\to 0$, we shall have
$$u_3(z)= -  {a_3} + \OO(z), \quad u_4(z)= -  a_4 + \OO(z), \quad a_3,a_4 >0.$$
For $z\to0$, the equations in \eqref{m34shift} are reduced to
\begin{align}\label{m35shift}
\frac{n_2}{n}\frac{1}{a_3} = 1 +\frac1n\sum_{i=1}^p \frac{1}{a_3 + \lambda_i^2a_4  } ,\quad \frac{n_1}{n}\frac1{a_4} = 1 +\frac1n\sum_{i=1}^p \frac{\lambda_i^2 }{  a_3 + \lambda_i^2 a_4 }. 
\end{align}
It is easy to see that these equations are equivalent to
\begin{align} a_3 + a_4 = 1- \gamma_n, \quad a_3 +\frac1n\sum_{i=1}^p \frac{a_3}{a_3 + \lambda_i^2[(1-\gamma_n)-a_3]  }=\frac{n_2}{n}  .\end{align}
The equations in \eqref{dotm34} reduce to 
\be \label{dotm34red}
\begin{split}
\left(\frac{n_2}{n}\frac1{a_3^2}- \frac1n\sum_{i=1}^p \frac{1 }{ (  a_3 + \lambda_i^2a_4)^2  }\right) b_3 -  \left(\frac1n\sum_{i=1}^p \frac{  \lambda_i^2 }{ (  a_3 + \lambda_i^2a_4)^2  }\right)b_4 =  \frac1n\sum_{i=1}^p \frac{1 }{ (  a_3 + \lambda_i^2a_4)^2  } ,\\
 %\frac{n_1}{n}\frac1{a_4^2}b_4 =   \frac1n\sum_{i=1}^p \frac{\lambda_i^2\left(1+b_3 + \lambda_i^2 b_4\right) }{  (a_3 + \lambda_i^2a_4)^2  } , \\
 \left( \frac{n_1}{n}\frac1{a_4^2} -  \frac1n\sum_{i=1}^p \frac{\lambda_i^4   }{  (a_3 + \lambda_i^2a_4)^2  }\right)b_4 -\left( \frac1n\sum_{i=1}^p \frac{\lambda_i^2  }{  (a_3 + \lambda_i^2a_4)^2  }\right)b_3 =   \frac1n\sum_{i=1}^p \frac{\lambda_i^2 }{  (a_3 + \lambda_i^2a_4)^2  } .
\end{split}
\ee
where we denote $b_3:=u_3'(0)$ and $b_4:=u_4'(0)$. Thus we have
$$\left(\frac{1}{\Lambda Z_1^{\top}Z_1 \Lambda + Z_2^{\top}Z_2} \right)^2  = G^2(0) \approx   \frac{ 1 + b_3 + \Lambda^2  b_4}{\left( a_3 + \Lambda^2  a_4\right)^2} .$$
This gives that
\begin{align*}
& \frac{1}{X_1^{\top}X_1 + X_2^{\top}X_2}  \Sigma_2 \frac{1}{X_1^{\top}X_1 + X_2^{\top}X_2} =\Sigma_2^{-1/2}V \left(\frac{1}{\Lambda Z_1^{\top}Z_1 \Lambda + Z_2^{\top}Z_2} \right)^2 V^T \Sigma_2^{-1/2}\\
& \approx  \Sigma_2^{-1/2}V \frac{ 1 + b_3 + \Lambda^2  b_4}{\left( a_3 + \Lambda^2  a_4\right)^2}V^T \Sigma_2^{-1/2}= \Sigma_2^{-1/2}  \frac{ 1 + b_3 +  b_4 A^\top A }{\left( a_3 +   a_4 A^\top A\right)^2}\Sigma_2^{-1/2} .
\end{align*}
Hence we have 
\begin{align*}
& (\beta_s - \beta_t)^{\top}X_1^{\top}X_1 \frac{1}{X_1^{\top}X_1 + X_2^{\top}X_2}  \Sigma_2 \frac{1}{X_1^{\top}X_1 + X_2^{\top}X_2} X_1^{\top}X_1 (\beta_s - \beta_t) \\
 & \approx (\beta_s - \beta_t)^{\top}\Sigma_1 \Sigma_2^{-1/2}  \frac{ 1 + b_3 +  b_4 A^\top A }{\left( a_3 +   a_4 A^\top A\right)^2}\Sigma_2^{-1/2} \Sigma_1(\beta_s - \beta_t).
\end{align*}

 
%Then it is equivalent to study 
%$$Q:=  ( X_1^T X_1 )^{-1}   D X_2^T X_2 D ,$$
%which has the same nonzero eigenvalues of 
%$$\mathcal Q:=  X_2 D ( X_1^T X_1 )^{-1} D X_2^T,$$
%i.e. $\mathcal Q$ has the same nonzero eigenvalues of $Q$, but has $(n_2-p)$ more zero eigenvalues. 

%----------The following is the results on the eigenvalues, not the singular values-------------
%We are interested in the eigenvalues of 
%$$(X_1^{\top}X_1)^{-1} (X_2^{\top}X_2),$$
%which is called a generalized Fisher matrix. As in the previous setting, we write out their covariance explicitly and consider
%$$  (\Sigma_1^{1/2}  Z_1^T Z_1 \Sigma_1^{1/2})^{-1}   \Sigma_2^{1/2}  Z_2^T Z_2 \Sigma_2^{1/2} ,$$
%where $\Sigma_{1,2}$ are $p\times p$ deterministic covariance matrices, and $X_1=(x_{ij})_{1\le i \le n_1, 1\le j \le p}$ and $X_2=(x_{ij})_{n_1+1\le i \le n_1+n_2, 1\le j \le p}$ are $n_1\times p$ and $n_2 \times p$ random matrices, respectively, where the entries $x_{ij}$, $1 \leq i \leq n_1+n_2\equiv n$, $1 \leq j \leq p$, are real independent random variables satisfying \eqref{eq_12moment}. 
%We can study it using the linearization matrix ({\color{red}for my own purpose right now})
%$$ H = \begin{pmatrix} - z I & X_2D & 0 \\ DX_2^T & 0 & X_1^T \\ 0 & X_1 & I \end{pmatrix}, \quad G(z):=H(z)^{-1}.$$ 
%Then the $(1,1)$-th block is equal to $\cal G(z):=(\mathcal Q-z)^{-1}$. We denote 
%\begin{equation}\label{m1m}m_1(z):=\frac1n\tr \cal G(z), \quad m(z)= \frac1p \left[ nm_1(z) + \frac{n_2-p}{z}\right].\ee
%We can show that it satisfies the following self-consistent equations together with another $m_3(z)$:
%\begin{align}\label{m13}
%\frac{n_2}{n}\frac1{m_1} = - z +\frac1n\sum_{i=1}^p \frac{d_i^2}{ m_3 + d_i^2m_1  } ,\quad \frac{n_1}{n}\frac1{m_3} = 1 +\frac1n\sum_{i=1}^p \frac{1}{   m_3 + d_i^2m_1  } .
%\end{align}
%For these two equations, we can obtain one single equation for $m_1(z)$:
%\begin{align}\label{m1}
%m_3= 1-\frac pn + zm_1, \quad \frac{n_2}{n}\frac1{m_1} = - z +\frac1n\sum_{i=1}^p \frac{d_i^2}{ 1-\frac pn + (z + d_i^2)m_1  }  .
%\end{align}
%One can solve the above equation for $m_1$ with positive imaginary parts, and then calculate $m(z)$ using \eqref{m1m}. 
%
%With $m(z)$, we can define
%$$\rho_c(z):=\frac1\pi \lim_{\eta\downarrow 0} m(z).$$
%It will be a compact supported probability density which gives the eigenvalue distribution of $Q$. Moreover, we have  
%$$\frac{1}{p}\tr \frac1{(\Sigma_1^{1/2}  X_1^T X_1 \Sigma_1^{1/2})^{-1}   \Sigma_2^{1/2}  X_2^T X_2 \Sigma_2^{1/2} +1}\left(\frac1{(\Sigma_1^{1/2}  X_1^T X_1 \Sigma_1^{1/2})^{-1}   \Sigma_2^{1/2}  X_2^T X_2 \Sigma_2^{1/2} +1}\right)^T \approx \int \frac{\rho_c(x)}{(x+1)^2}{\rm d}x.$$
%We know that 
%$$ m(z)=\int \frac{\rho_c(x)}{ x-z}{\rm d}x.$$
%Hence we have
%$$ m'(-1)=\int \frac{\rho_c(x)}{(x+1)^2}{\rm d}x.$$
%
%Moreover, the right edge $\lambda_+$ of $\rho_c$ gives the location of the largest eigenvalue, while the left edge $\lambda_-$  of $\rho_c$ gives the location of the smallest eigenvalue. They will provide the upper and lower bounds on the operator norm: 
%$$\frac{1}{1+\lambda_+}\le \frac1{(\Sigma_1^{1/2}  X_1^T X_1 \Sigma_1^{1/2})^{-1}   \Sigma_2^{1/2}  X_2^T X_2 \Sigma_2^{1/2} +1}\le \frac1{1+\lambda_-}.$$
%The edges of the spectrum can be determined by solving the following equations of $(x,m_1)$:
%$$\frac{n_2}{n}\frac1{m_1} = - x +\frac1n\sum_{i=1}^p \frac{d_i^2}{ 1-\frac pn + (x + d_i^2)m_1  } ,\quad  \frac{n_2}{n}\frac1{m_1^2} = \frac1n\sum_{i=1}^p \frac{d_i^2 (x+d_i^2)}{\left[ 1-\frac pn + (x + d_i^2)m_1 \right]^2 } .$$
%The solutions for $x$ give the locations for the edges of the spectrum.  
%--------------------------------------


