\subsection{Proof for Two Tasks with General Covariates}\label{app_proof_main}

%To illustrate the idea, we observe that by using Lemma \ref{lem_minv}, we have that
%\[ \te(\hat{\beta}_t^{\STL}) = \frac{\sigma^2}{n_2 - p}\bigtr{\Sigma_2^{-1}}. \]
%We shall also derive the limit of $\te(\hat{\beta}_t^{\MTL})$.
%\todo{write a brief technical overview}

In this section, we state and prove the formal version of Theorem \ref{thm_main_informal}, which covers the two tasks case with $t=2$. In this section, we consider the case where the entries of $\e_1$ and $\e_2$ have the same variance $\sigma_1^2=\sigma_2^2=\sigma^2$. 
%Moreover, we shall denote $\beta_1$ and $\beta_2$ by $\beta_1$ and $\beta_2$, standing for ``source task" and ``target task", respectively. 


First, we  introduce several quantities that will be used in our statement, and they are also related to the quantities in Lemma \ref{lem_cov_shift} and Lemma \ref{lem_cov_derivative}. Given the optimal ratio $\hat v$, let $\hat{M} = \hat{v} \Sigma_1^{1/2}\Sigma_2^{-1/2}$ denote the weighted covariate shift matrix, and ${\hat\lambda}_1\ge {\hat\lambda}_2 \ge \dots \ge {\hat\lambda}_p$ be the eigenvalues of $\hat{M}^{\top}\hat{M}$. Define $(\hat a_1, \hat a_2)$ as the solution to the following system of deterministic equations,
	\be
		 \hat a_1 +  \hat a_2 = 1- \frac{1}{\rho_1 + \rho_2},\quad  \hat a_1 + \frac1{\rho_1 + \rho_2}\cdot \frac1p\sum_{i=1}^p \frac{ \hat \lambda_i^2 \hat a_1}{ \hat \lambda_i^2\hat a_1 +  \hat a_2} = \frac{\rho_1}{\rho_1 + \rho_2}.\label{eq_a2} \\
		 \ee
		 After obtaining $(\hat a_1,\hat a_2)$, we can solve the following linear equations to get $(\hat a_3,\hat a_4)$:
\begin{gather}
		\left(\rho_2 \hat a_2^{-2}- \hat b_0\right)\cdot \hat  a_3 - \hat b_1 \cdot \hat a_4
		=\hat b_0, \quad \left(\rho_1\hat a_1^{-2} - \hat b_2  \right)\cdot \hat a_4 - \hat b_1 \cdot \hat a_3 =\hat b_1 .\label{eq_a3} 
%		\left(\frac{n_1}{\hat a_1^2} -  \sum_{i=1}^p \frac{\hat \lambda_i^4   }{  (\hat a_2 + \hat \lambda_i^2\hat a_1)^2  }\right)\hat a_4 -\left(\sum_{i=1}^p \frac{\hat \lambda_i^2  }{  (\hat a_2 + \hat \lambda_i^2\hat a_1)^2  }\right)\hat a_3
%		= \sum_{i=1}^p \frac{\hat \lambda_i^2 }{  (\hat a_2 + \hat \lambda_i^2\hat a_1)^2  }. \label{eq_a4}
	\end{gather}
where we denoted
$$\hat b_k:= \frac1{p}\sum_{i=1}^p \frac{\hat \lambda_i^{2k}}{ (\hat a_2 +\hat  \lambda_i^2\hat a_1)^2  },\quad k=0,1,2.$$
Then we introduce the following matrix
\be\label{defnpihat}\Pi = \frac{\rho_1^2}{(\rho_1 + \rho_2)^2}\cdot{\hat M} \frac{(1 + \hat a_3)\id +\hat  a_4 \hat {M}^{\top}\hat {M}}{(\hat  a_1 \hat {M}^{\top}\hat {M}+\hat a_2 )^2} \hat {M}^{\top}.\ee
%which is defined in a way such that {\it in certain sense} it is the asymptotic limit of the random matrix 
%$$\hat v\Sigma_1^{1/2} (\hat{v}^2 X_1^{\top}X_1 + X_2^{\top}X_2)^{-1} \Sigma_2 (\hat{v}^2 X_1^{\top}X_1 + X_2^{\top}X_2)^{-1}\Sigma_1^{1/2}.$$ 
We introduce two factors that will appear often in our statements and discussions:
$$\al_-(\rho_1):=\left(1- \rho_1^{-1/2}\right)^2,\quad \al_+(\rho_1):=\left(1 + \rho_1^{-1/2}\right)^2.$$ 
In fact, $\al_-(\rho_1)$ and $\al_+(\rho_1)$ correspond to the largest and smallest singular values of $Z_1/\sqrt{n_1}$, respectively, as given by the famous Mar{\v c}enko-Pastur law \cite{MP}. In particular, as $\rho_1$ increases, both $\al_-$ and $\al_+$ will converge to 1 and $Z_1/\sqrt{n_1}$ will be more close to an isometry. Finally, we introduce the error term  
\be\label{eq_deltaextra}\delta\equiv \delta(\hat v):=\frac{\al_+^2(\rho_1) - 1 }{\al_-^{2}(\rho_1)\lambda_{\min}^2(\hat M)} \cdot  \norm{\Sigma_1^{1/2}(\beta_1 - \hat{v}\beta_2)}^2,\ee
where $\lambda_{\min}(\hat M)$ is the smallest singular value of $\hat M$. Note that this factor converges to 0 as $\rho_1$ increases.

%$$\delta:=\left[\frac{n_1 \lambda_1}{(\sqrt{n_1}-\sqrt{p})^2\lambda_p  +  (\sqrt{n_2}-\sqrt{p})^2}\right]^2\cdot \norm{\Sigma_1^{1/2}(\beta_1 - \hat{w}\beta_2)}^2.$$
%{\cor may be able to get a better bound, but the statement will be long}

Now we are ready to state our main result for two tasks with both covariate and model shift. It shows that the information transfer is determined by two deterministic quantities $\Delta_{\bias}$ and $\Delta_{\vari}$, which give the change of model shift bias and the change of variance, respectively.



\begin{theorem}\label{thm_model_shift}
%For $i=1,2$, let $Y_i = X_i\beta_i + \varepsilon_i$ be two independent data models, where $X_i$, $\beta_i$ and $\varepsilon_i$ are also independent of each other. Suppose that $X_i=Z_i\Sigma_i^{1/2}\in \R^{n_i\times p}$ satisfy Assumption \ref{assm_secA1} with $\rho_i:=n_i/p>1$ being fixed constants, and $\e_i\in \R^{n_i}$ are random vectors with i.i.d. entries with mean zero, variance $\sigma^2$ and all moments as in \eqref{assmAhigh}. 
Consider two data models $Y_i = X_i\beta_i + \varepsilon_i$, $i=1,2$, that satisfy Assumption \ref{assm_secA2}. With high probability, we have
	\begin{align}
	 	\te(\hat{\beta}_{t}^{\MTL}) \le \te(\hat{\beta}_t^{\STL}) \quad \text{ when: } \ \ &\Delta_{\vari} - \Delta_{\bias} \ge   \delta \label{upper}\\
		\te(\hat{\beta}_t^{\MTL}) \ge \te(\hat{\beta}_t^{\STL}) \quad \text{ when: } \ \ &\Delta_{\vari} - \Delta_{\bias} \le - \delta, \label{lower}
	\end{align}
	where
	\begin{align} %\bigtr{{\Sigma_2^{-1}}}
		\Delta_{\vari} &\define {\sigma^2}\bigbrace{\frac{1}{\rho_2 - 1} -  \frac{1}{\rho_1 + \rho_2}\cdot \frac1p \bigtr{(\hat a_1 \hat M^{\top}\hat M + \hat a_2\id)^{-1}} } \label{Deltavarv} \\
		\Delta_{\bias} &\define (\beta_1 - \hat{v}\beta_2)^{\top} \Sigma_1^{1/2} \Pi \Sigma_1^{1/2} (\beta_1 - \hat{v}\beta_2). \label{Deltabetav}
	\end{align}
\end{theorem}

For the isotropic model in Section \ref{sec_insight}, we actually have an easier and sharper bound than Theorem \ref{thm_model_shift} as follows. 
\begin{lemma}\label{prop_model_shift_tight}
		In the setting of Theorem \ref{thm_model_shift}, assume that $\Sigma_1 =\id$,
		$\beta_2$ is a random vector with i.i.d. entries with mean $0$, variance $\kappa^2$ and all moments, and $\beta_1$ is a random vector such that $(\beta_1 - \beta_2)$ is a random vector with i.i.d. entries with mean $0$, variance $d^2$ and all moments. Denote
		$\Delta^\star_{\bias} := \bigbrace{(1 - \hat{v})^2 \kappa^2 + d^2} \bigtr{\Pi}.$
	Then we have
		\begin{align*}
			\te(\hat{\beta}_t^{\MTL}) \le \te(\hat{\beta}_t^{\STL})\quad \text{ when: }\ \  & \Delta_{\vari} \ge  (\al_+^2(\rho_1)+\oo(1))\cdot  \Delta^\star_{\bias}, \\
			\te(\hat{\beta}_t^{\MTL}) \ge \te(\hat{\beta}_t^{\STL})\quad \text{ when: }\ \  & \Delta_{\vari} \le (\al_-^2(\rho_1)-\oo(1))\cdot  \Delta^\star_{\bias}.
		\end{align*}
\end{lemma}

Now we give the proof of Theorem \ref{thm_model_shift} based on Lemma \ref{lem_cov_shift} and Lemma \ref{lem_cov_derivative}. 

\begin{proof}[Proof of Theorem \ref{thm_model_shift}]
%\noindent
%To prove Theorem \ref{thm_cov_shift}, we study the spectrum of the random matrix model:
%$$Q= \Sigma_1^{1/2}  Z_1^{\top} Z_1 \Sigma_1^{1/2}  + \Sigma_2^{1/2}  Z_2^{\top} Z_2 \Sigma_2^{1/2} ,$$
%where $\Sigma_{1,2}$ are $p\times p$ deterministic covariance matrices, and $X_1=(x_{ij})_{1\le i \le n_1, 1\le j \le p}$ and $X_2=(x_{ij})_{n_1+1\le i \le n_1+n_2, 1\le j \le p}$ are $n_1\times p$ and $n_2 \times p$ random matrices, respectively, where the entries $x_{ij}$, $1 \leq i \leq n_1+n_2\equiv n$, $1 \leq j \leq p$, are real independent random variables satisfying
%\begin{equation}\label{eq_12moment} %\label{assm1}
%\mathbb{E} z_{ij} =0, \ \quad \ \mathbb{E} \vert z_{ij} \vert^2  = 1.
%\end{equation}
%\todo{A proof outline; including the following key lemma.}
Note that 
\begin{align*}
L(\hat{\beta}_t^{\STL}) - L(\hat{\beta}_t^{\MTL}) &=\sigma^2 \left(  \bigtr{(X_2^{\top}X_2)^{-1}\Sigma_2} -  \bigtr{( \hat{v}^2 X_1^{\top}X_1 + X_2^{\top}X_2)^{-1} \Sigma_2}\right) \\
&- \hat{v}^2 \bignorm{\Sigma_2^{1/2}(\hat{v}^2 X_1^{\top}X_1 + X_2^{\top}X_2)^{-1} X_1^{\top}X_1 (\beta_1 - \hat{v} \beta_2)}^2=:\delta_{\vari}(\hat v) - \delta_{\bias}(\hat v).
\end{align*}
The proof is divided into the following four steps. 
\begin{itemize}
\item[(i)] We first consider $\hat M \equiv \hat M(v)= v\Sigma_1^{1/2}\Sigma_2^{-1/2}$ for a fixed $v\ge 0$. Then we use Lemma \ref{lem_minv} and Lemma \ref{lem_cov_shift} to calculate the variance reduction $\delta_{\vari}(v)$, which will lead to the $\Delta_{\vari}$ term.
%$$\sigma^2 \cdot \bigtr{(X_2^{\top}X_2)^{-1}},\quad \sigma^2 \cdot \bigtr{({v}^2 X_1^{\top}X_1 + X_2^{\top}X_2)^{-1} \Sigma_2},$$
%and the difference between them 

\item[(ii)] Using the approximate isometry property of $X_1$ (see \eqref{eq_isometric} below), we will bound the bias term $ \delta_{\bias}(v)$
%$${v}^2 \bignorm{\Sigma_2^{1/2}({v}^2 X_1^{\top}X_1 + X_2^{\top}X_2)^{-1} X_1^{\top}X_1 (\beta_1 - {v} \beta_2)}^2$$
through 
\be\label{deltabetapf}
\wt\delta_{\bias}(v):={v}^2 n_1^2\bignorm{\Sigma_2^{1/2}({v}^2 X_1^{\top}X_1 + X_2^{\top}X_2)^{-1} \Sigma_1 (\beta_1 - {v} \beta_2)}^2.\ee

\item[(iii)] We use Lemma \ref{lem_cov_derivative} to calculate \eqref{deltabetapf}, which will lead to the $\Delta_{\bias}$ term.

\item[(iv)] Finally we use a standard $\e$-net argument to extend the above results to $\hat M= \hat v\Sigma_1^{1/2}\Sigma_2^{-1/2}$ for a possibly random $\hat v$ which depends on $Y_1$ and $Y_2$.
\end{itemize}


\paragraph{Step I: Variance reduction.} Let $\hat M= v\Sigma_1^{1/2}\Sigma_2^{-1/2}$ for any fixed constant $v\ge 0$. Using Lemma \ref{lem_cov_shift}, we can obtain that for any constant $\e>0$, 
$$  \sigma^2 \cdot \bigtr{(X_2^{\top}X_2)^{-1}\Sigma_2} = \frac{\sigma^2}{\rho_2-1}\left( 1+ \OO(p^{-1/2+e})\right),$$
and 
$$ \sigma^2 \cdot \bigtr{( {v}^2 X_1^{\top}X_1 + X_2^{\top}X_2)^{-1} \Sigma_2} =   \frac {\sigma^2} {\rho_1 + \rho_2}\cdot \frac1p \bigtr{(\hat a_1 \hat M^{\top}\hat M + \hat a_2\id)^{-1}}\left( 1+ \OO(p^{-1/2+e})\right) ,$$
with high probability, where $\hat a_1$ and $\hat a_2$ satify \eqref{eq_a2}. Combining them, we get 
\be\label{deltavaral-} \delta_{\vari}(v)=\Delta_{\vari}(v) +\OO( \sigma^2 p^{-1/2+e}) \quad \text{whp},
\ee 
where $\Delta_{\vari}(v)$ is defined as in \eqref{Deltavarv} but with $\hat v$ replaced by $v$.



\paragraph{Step II: Bounding the bias term.}
In this step, we shall use the following the following bounds on the singular values of $Z_1$: for any fixed $\e>0$, we have
\begin{align}
\al_-(\rho_1) - \OO(p^{-1/2+e})  \preceq \frac{Z_1^T Z_1}{n_1}  \preceq   \al_+(\rho_1) + \OO(p^{-1/2+e})   \label{eq_isometric}
\end{align}
with high probability. In fact, $Z_1^TZ_1$ is a standard sample covariance matrix, and it is well-known that its nonzero eigenvalues are all inside the support of the Marchenko-Pastur law $[\al_-(\rho_1)-\oo(1) ,\al_+(\rho_1)+\oo(1)]$ with probability $1-\oo(1)$ \cite{No_outside}. For the estimate \eqref{eq_isometric} we used \cite[Theorem 2.10]{isotropic} to get a stronger probability bound.


Next we shall use \eqref{eq_isometric} to approximate $\delta_{\bias}(v)$ with $\wt\delta_{\bias}(v)$ in \eqref{deltabetapf}.  
%relate the first term in equation \eqref{eq_te_model_shift} to $\Delta_{\bias}$.
\begin{lemma}\label{prop_model_shift}
	In the setting of Theorem \ref{thm_model_shift},
	we denote by $K = (v^2X_1^{\top}X_1 + X_2^{\top}X_1)^{-1}$, and
	\begin{align*}
		%\delta_1 &= v^2 \bignorm{\Sigma_2^{1/2} K X_1^{\top}X_1(\beta_1 - v\beta_2)}^2, \\
		%\delta_2 &= n_1^2\cdot v^2 \bignorm{\Sigma_2^{1/2}K\Sigma_1(\beta_1 - v\beta_2)}, \\
		\delta_{\e}(v) := n_1^2 v^2 \bignorm{\Sigma_1^{1/2} K \Sigma_2 K \Sigma_1^{1/2}} \cdot \bignorm{\Sigma_1^{1/2} (\beta_1 - v\beta_2)}^2.
	\end{align*}
	Then we have whp,
	\begin{align*}
		 \left| \delta_{\bias}(v)-\wt\delta_{\bias}(v)\right| 
		\le  \left( \al_+^2(\rho_1)-1 + \OO(p^{-1/2+\e})\right)\delta_\e.
	\end{align*}
%	We have that
%	\begin{align*}
%		-2n_1^2\bigbrace{{2\sqrt{\frac{p}{n_1}}} + {\frac{p}{n_1}}} \delta_3
%		\le  \delta_1 - \delta_2
%		\le n_1^2\bigbrace{2\sqrt{\frac{p}{n_1}} + \frac{p}{n_1}}\bigbrace{2 + 2\sqrt{\frac{p}{n_1}} + \frac{p}{n_1}}\delta_3.
%	\end{align*}
%	For the special case when $\Sigma_1 = \id$ and $\beta_1 - \beta_2$ is i.i.d. with mean $0$ and variance $d^2$, we further have
%	\begin{align*}
%		\bigbrace{1 - \sqrt{\frac{p}{n_1}}}^4 \Delta_{\bias}
%		\le \bignorm{\Sigma_2^{1/2} (X_1^{\top}X_1 + X_2^{\top}X_2)^{-1}X_1^{\top}X_1(\beta_1 - \beta_2)}^2.
%	\end{align*}
\end{lemma}

\begin{proof}
	%The proof follows by applying equation \eqref{eq_isometric}.
	%Recall that $X_1^{\top}X_1 = \Sigma_1^{1/2}Z_1^{\top}Z_1\Sigma_1^{1/2}$.
	Denote by $\cE = Z_1^{\top}Z_1 - {n_1}\id$. Then we can write
%	Let $\alpha = \bignorm{\Sigma_2^{1/2} K \Sigma_1 (\beta_1 - \hat{w}\beta_2)}^2$.
	%We have
	\begin{align}
%		& \bignorm{\Sigma_2^{1/2}(X_1^{\top}X_1 + X_2^{\top}X_2)^{-1}X_1^{\top}X_1(\beta_1 - \hat{w}\beta_2)}^2 \nonumber \\
		 \delta_{\bias}(v)-\wt\delta_{\bias}(v)&= {2v^2n_1}(\beta_1 - v\beta_2)^{\top}\Sigma_1^{1/2} \cE\left(\Sigma_1^{1/2}K \Sigma_2 K \Sigma_1^{1/2}\right) \Sigma_1^{1/2} (\beta_1 - v\beta_2) \nonumber
		\\
		&+ v^2\bignorm{\Sigma_2^{1/2} K \Sigma_1^{1/2}\cE \Sigma_1^{1/2}(\beta_1 - v\beta_2)}^2. \label{eq_lem_model_shift_1}
%		\le& n_1\bigbrace{{n_1^2}{} + \frac{2n_1}p(p + 2\sqrt{{n_1}p}) + (p + 2\sqrt{{n_1}p})^2} \alpha = n_1^2\bigbrace{1 + \sqrt{\frac{p}{n_1}}}^4 \alpha. \nonumber
	\end{align}
	Using \eqref{eq_isometric}, we can bound  
	$$\|\cal E\|\le \left( \al_+(\rho_1)-1 + \OO(p^{-1/2+\e})\right)n_1, \quad \text{whp}.$$
	Thus we can estimate that 
	\begin{align*}
	| \delta_{\bias}(v)-\wt\delta_{\bias}(v)|&\le v^2 \left( 2n_1  \|\cal E\| +  \|\cal E\|^2 \right) \bignorm{\Sigma_1^{1/2} K \Sigma_2 K \Sigma_1^{1/2}} \bignorm{\Sigma_1^{1/2} (\beta_1 - v\beta_2)}^2 \\
	&=  v^2 \left[\left( n_1 + \|\cal E\|\right)^2 -n_1^2 \right] \bignorm{\Sigma_1^{1/2} K \Sigma_2 K \Sigma_1^{1/2}} \bignorm{\Sigma_1^{1/2} (\beta_1 - v\beta_2)}^2 \\
	& \le v^2 n_1^2 \left[ \al_+^2(\rho_1) + \OO(p^{-1/2+\e}) -1\right]\bignorm{\Sigma_1^{1/2} K \Sigma_2 K \Sigma_1^{1/2}} \bignorm{\Sigma_1^{1/2} (\beta_1 - v\beta_2)}^2,
	\end{align*}
	which concludes the proof by the definition of $\delta_\e$.	
%	we can bound the second term on the RHS of equation \eqref{eq_lem_model_shift_1} as
%	\begin{align*}
%		& \bigabs{(\beta_1 -  \beta_2)^{\top} \Sigma_1^{1/2} \cE \Sigma_1^{1/2} K \Sigma_2 K \Sigma_1 (\beta_1 - v\beta_2)}\le n_1  \|\cal E\| \cdot \bignorm{\Sigma_1^{1/2} K \Sigma_2 K \Sigma_1^{1/2}} \bignorm{\Sigma_1^{1/2} (\beta_1 - v\beta_2)}^2 \\
%		= & \bigabs{\bigtr{\cE \Sigma_1^{1/2}K\Sigma_2 K \Sigma_1(\beta_1 - \hat{w}\beta_2)(\beta_1 - \hat{w}\beta_2)^{\top} \Sigma_1^{1/2}}} \\
%		\le & \norm{\cE} \cdot \bignormNuclear{\Sigma_1^{1/2} K \Sigma_2 K \Sigma_1 (\beta_1 - \hat{w}\beta_2) (\beta_1 - \hat{w}\beta_2)^{\top} \Sigma_1^{1/2}} \\
%		\le & n_1 \bigbrace{2\sqrt{\frac{p}{n_1}} + \frac{p}{n_1}} \cdot \bignormNuclear{\Sigma_1^{1/2} K \Sigma_2 K \Sigma_1 (\beta_1 - \hat{w}\beta_2)(\beta_1 - \hat{w}\beta_2)^{\top} \Sigma_1^{1/2}} \tag{by equation \eqref{eq_isometric}} \\
%		\le   & n_1 \bigbrace{2\sqrt{\frac{p}{n_1}} + \frac{p}{n_1}} \bignorm{\Sigma_1^{1/2}K \Sigma_2 K \Sigma_1^{1/2}} \cdot \bignorm{\Sigma_1^{1/2}(\beta_1 - \hat{w}\beta_2)}^2 \tag{since the matrix inside is rank 1}
%	\end{align*}
%	The third term in equation \eqref{eq_lem_model_shift_1} can be bounded with
%	\begin{align*}
%		\bignorm{\Sigma_2^{1/2}K\Sigma_1^{1/2}\cE\Sigma_1^{1/2}(\beta_1 - v\beta_2)}^2
%		\le n_1^2 \bigbrace{2\sqrt{\frac{p}{n_1}} + \frac{p}{n_1}}^2 \bignorm{\Sigma_1^{1/2}K\Sigma_{2}K\Sigma_1^{1/2}} \cdot \bignorm{\Sigma_1^{1/2}(\beta_1 -  \beta_2)}^2.
%	\end{align*}
%	Combined together we have shown the right direction for $\delta_1 - \delta_2$.
%	For the left direction, we simply note that the third term in equation \eqref{eq_lem_model_shift_1} is positive.
%	And the second term is bigger than $-2n_1^2(2\sqrt{\frac{p}{n_1}} + \frac{p}{n_1}) \alpha$ using equation \eqref{eq_isometric}.
\end{proof}
Note by \eqref{eq_isometric}, we have with high probability,
\begin{align*}
&v^2 n_1^2 \Sigma_1^{1/2} K \Sigma_2 K \Sigma_1^{1/2} = \hat M\frac{1}{(\hat M^\top Z_1^\top Z_1 \hat M + Z_2^\top Z_2)^2}\hat M^\top \\
&\preceq  n_1^2 \hat M\frac{1}{\left[n_1 \al_-(\rho_1)\hat M^\top \hat M + n_2 \al_-(\rho_2) + \OO(p^{1/2+\e})\right]^2}\hat M^\top \\
&\preceq  \left[ \al_-^2(\rho_1) \hat M\hat M^\top + 2\frac{\rho_2}{\rho_1} \al_-(\rho_1)\al_-(\rho_2) + 2\left(\frac{\rho_2}{\rho_1}\right)^2 \al_-^2(\rho_2) (\hat M \hat M^\top )^{-1}\right]^{-1}+  \OO(p^{-1/2+\e}) \\
&\prec [\al_-^2(\rho_1) \lambda_{\min}^2(\hat M)]^{-1}\cdot (1 - c)
\end{align*}
for some small enough constant $c>0$. Together with Lemma \ref{prop_model_shift}, we get with high probability,
\be\label{bounddelta-}
\left| \delta_{\bias}(v)-\wt\delta_{\bias}(v)\right| 
		\le (1-c) \delta(v)
\ee
for some small constant $c>0$, where recall $\delta(v)$ defined in \eqref{eq_deltaextra}.


\paragraph{Step III: The limit of $\wt\delta_{\bias}(v)$.} 
Using Lemma \ref{lem_cov_derivative} with $\Sigma_1$ and $M$ replaced by $v^2\Sigma_1$ and $\hat M$, we obtain that 
\begin{align*}
\wt\delta_{\bias}(v) &=\frac{\rho_1^2}{(\rho_1 + \rho_2)^2}\cdot v^2 (\beta_1-v\beta_2)^\top\Sigma_1 \Sigma_2^{-1/2}  \frac{(1 + \hat a_3)\id + \hat a_4 \hat{M}^{\top}\hat{M}}{( a_1 \hat{M}^{\top}\hat{M}+a_2 )^2} \Sigma_2^{-1/2} \Sigma_1(\beta_1-v\beta_2) +\OO(p^{-1/2+\e}) \\
&= (\beta_1 - {v}\beta_2)^{\top} \Sigma_1^{1/2} \Pi \Sigma_1^{1/2} (\beta_1 - {v}\beta_2) +\OO(p^{-1/2+\e}) =: \Delta_{\bias}(v) +\OO(p^{-1/2+\e}),
\end{align*}
with high probability. Together with and \eqref{deltavaral-} and \eqref{bounddelta-}, we obtain that whp,
\be\label{dicho_varbeta}
\begin{cases}\delta_{\vari}(v)>\delta_{\bias}(v), & \text{ if } \ \ \Delta_{\vari}(v) - \Delta_{\bias}(v) \ge   \delta(v),\\
\delta_{\vari}(v)<\delta_{\bias}(v),  & \text{ if }  \ \ \Delta_{\vari}(v) - \Delta_{\bias}(v) \le -  \delta(v).\end{cases}
\ee



\paragraph{Step IV: An $\e$-net argument.} Finally, it remains to extend the above result to $v=\hat v$, which is random and depends on $X_1$ and $X_2$. We first show that for any fixed constant $C_0>0$, there exists a high probability event $\Xi$ on which \eqref{dicho_varbeta} 
%\eqref{lem_cov_shift_eq} and \eqref{lem_cov_derv_eq} 
holds uniformly for all $v\in [0, C_0]$. In fact, for a large constant $C_1>0$, we consider $v$ belonging to a discrete set 
$$V:=\{v_k = kp^{-1}: 1\le k \le C_0p +1\}.$$
Then using the arguments for the first three steps and a simple union bound, we get that
\eqref{dicho_varbeta} holds simultaneously for all $v\in V$ with high probability. On the other hand, by \eqref{eq_isometric} the event
$$\Xi_1:=\left\{ \al_-(\rho_1)/2 \preceq  \frac{Z_1^T Z_1}{n_1}  \preceq   2\al_+(\rho_1) ,\  \al_-(\rho_2)/2 \preceq  \frac{Z_2^T Z_2}{n_2}  \preceq   2\al_+(\rho_2)\right\}$$
holds with high probability. Now it is easy to check that on $\Xi_1$, for all $v_k \le v\le v_{k+1}$ we have the following estimates:
\begin{align*}
& |\delta_{\vari}(v) -\delta_{\vari}(v_k)|\lesssim p^{-1}\delta_{\vari}(v_k),\ \ |\delta_{\bias}(v) -\delta_{\bias}(v_k)|\lesssim p^{-1}\delta_{\bias}(v_k), \ \   |\delta(v)-\delta(v_k)|\lesssim p^{-1}\delta(v_k),\\
& |\Delta_{\bias}(v) -\Delta_{\bias}(v_k)|\lesssim p^{-1}\Delta_{\bias}(v_k),\ \ |\Delta_{\vari}(v) -\Delta_{\vari}(v_k)|\lesssim p^{-1}\Delta_{\vari}(v_k).
\end{align*}
Then a simple application of triangle inequality gives that the event 
$$\Xi_2=\{\eqref{dicho_varbeta} \text{ holds simultaneously for all }0\le v \le C_0\}$$
holds with high probability. On the other hand, on $\Xi_1$ one can see that for any small constant $\e>0$,
\begin{align*}
& |\delta_{\vari}(v) -\delta_{\vari}(C_0)|\le \e\delta_{\vari}(C_0),\quad |\delta_{\bias}(v) -\delta_{\bias}(C_0)|\le \e\delta_{\bias}(C_0), \quad  |\delta(v)-\delta(C_0)|\le \e\delta(C_0),\\
& |\Delta_{\bias}(v) -\Delta_{\bias}(C_0)|\le \e\Delta_{\bias}(C_0),\quad |\Delta_{\vari}(v) -\Delta_{\vari}(C_0)|\le \e\Delta_{\vari}(C_0),
\end{align*}
for all $v\ge C_0$ as long as $C_0$ is chosen large enough depending on $\e$. Together with the estimate at $C_0$, we get that \eqref{dicho_varbeta} holds simultaneously for all $v\ge 0$ on the high probability event $\Xi_1\cap \Xi_2$. This concludes the proof since $v$ must be one of the positive values.
\end{proof}

\begin{remark}
One can see from the above proof that the main error, $\delta$, of Theorem \ref{thm_model_shift} comes from approximating $\delta_{\bias}$ by $\wt\delta_{\bias}$ in \eqref{bounddelta-}. In order to improve this estimate and obtain an exact asymptotic result as for the $\delta_{\vari}$ term, one needs to study the singular value distribution of the following random matrix:
$$(X_1^{\top}X_1)^{-1}X_2^{\top}X_2 +  {v}^2 .$$
In fact, the eigenvalues of $\cal X:=(X_1^{\top}X_1)^{-1}X_2^{\top}X_2$ have been studied in the name of Fisher matrices; see e.g. \cite{Fmatrix}. However, since $\cal X$ is not symmetric, it is known that the singular values of $\cal X$ are different from its eigenvalues. To the best of our knowledge, the asymptotic singular value behavior of $\cal X$ is still unknown in random matrix theory literature, and the study of the singular values of $\cal X + v^2$ will be even harder. We leave this problem to future study.
\end{remark}

By replacing \eqref{bounddelta-} with a tighter bound in Step II of the above proof, we can conclude the proof of Lemma \ref{prop_model_shift_tight}.

\begin{proof}[Proof of Lemma \ref{prop_model_shift_tight}]
For any fixed $v\ge 0$, $\beta_1 - v\beta_2$ is a random vector with i.i.d. entries with mean $0$ and variance $(1-v)^2\kappa^2 + d^2$. Then using the concentration result, Lemma \ref{largedeviation}, we get that for any constant $\e>0$,
\begin{align}
&\left|\delta_{\bias}(v)- [(1-v)^2\kappa^2 + d^2]\tr (\cal K^\top \cal K)\right| \nonumber\\
&=  \left|(\beta_1 - {v} \beta_2)^\top \cal K^\top \cal K (\beta_1 - {v} \beta_2) - [(1-v)^2\kappa^2 + d^2]\tr (\cal K^\top \cal K)\right| \nonumber\\
&\le p^\e [(1-v)^2\kappa^2 + d^2] \left\{\tr\left[(\cal K^\top \cal K)^2\right]\right\}^{1/2} \lesssim p^{1/2+\e} [(1-v)^2\kappa^2 + d^2],\label{anotherbeta}
\end{align}
where we denoted $\cal K:=v\Sigma_2^{1/2}({v}^2 X_1^{\top}X_1 + X_2^{\top}X_2)^{-1} X_1^{\top}X_1$, and in the last step we used $\|\cal K\|=\OO(1)$ by \eqref{eq_isometric}. Now for $\tr (\cal K^\top \cal K)$, we rewrite it as
\begin{align*}
v^2 [(1-v)^2\kappa^2 + d^2]\tr \left[ ({v}^2 X_1^{\top}X_1 + X_2^{\top}X_2)^{-1}\Sigma_2({v}^2 X_1^{\top}X_1 + X_2^{\top}X_2)^{-1}(X_1^\top X_1)^2 \right].
\end{align*}
Recalling that $\Sigma_1=\id$ and bounding $(X_1^\top X_1)^2=(Z_1^\top Z_1)^2$ using \eqref{eq_isometric} again, we obtain that
\be\label{anotherwtbeta} \delta^\star_{\bias}(v) \cdot (\al_-^2(\rho_1)-\OO(p^{-1/2+\e})) \le  [(1-v)^2\kappa^2 + d^2]\tr (\cal K^\top \cal K) \le \delta^\star_{\bias}(v) \cdot  (\al_+^2(\rho_1)+\OO(p^{-1/2+\e})),\ee
where 
$$\delta^\star_{\bias}(v):=n_1^2 v^2 [(1-v)^2\kappa^2 + d^2]\tr \left[ ({v}^2 X_1^{\top}X_1 + X_2^{\top}X_2)^{-1}\Sigma_2({v}^2 X_1^{\top}X_1 + X_2^{\top}X_2)^{-1}\right].$$
Note that $\delta^\star_{\bias}(v) \sim 1$, hence combining \eqref{anotherbeta} and \eqref{anotherwtbeta} we get 
\be\label{replaceest}
  \delta^\star_{\bias}(v) \cdot (\al_-^2(\rho_1)-\OO(p^{-1/2+\e})) \le  \delta_{\bias}(v)\le \delta^\star_{\bias}(v) \cdot  (\al_+^2(\rho_1)+\OO(p^{-1/2+\e})).
\ee
Now we can replace the estimate \eqref{bounddelta-} with this stronger estimate, and repeat all the other parts of the proof of Theorem \ref{thm_model_shift} to conclude Lemma \ref{prop_model_shift_tight}. In particular, one can calculate $  \delta^\star_{\bias}(v)$ using Lemma \ref{lem_cov_derivative} and get the $  \Delta^\star_{\bias}(v)$ term, We omit the details.
\end{proof}



\subsection{Proof for Many Tasks with the Same Covariate}\label{app_proof_many_tasks}


\begin{theorem}\label{thm_many_tasks}
%Suppose $X=Z\Sigma^{1/2}\in \R^{n\times p}$ satisfy Assumption \ref{assm_secA1} with $\rho:=n/p>1$ being some fixed constant. Consider data models  $Y_i = X\beta_i + \varepsilon_i$, $i=1,2,\cdots, t$, where $\e_i\in \R^{n}$ are random vectors with i.i.d. entries with mean zero, variance $\sigma^2$ and all moments as in \eqref{assmAhigh}. Moreover, assume that $X$, $\beta_i$ and $\e_i$ are all independent of each other.  
	%Let $n = c \cdot p$.
	%Let $X\in\real^{n\times p}$ and $Y_i = X\beta_i + \varepsilon_i$, for $i = 1,\dots,k$.
	Consider $t$ data models $Y_i = X\beta_i + \varepsilon_i$, $i=1,2,\cdots, t$, that satisfy Assumption \ref{assm_secA2}.
	Let $U_r U_r^{\top}$ denote the best rank-$r$ subspace approximation of $(B^{\star})^\top\Sigma B^{\star}$, where $B^\star := [{\beta}_1,{\beta}_2,\dots,{\beta}_{t}]$ and $U_r\in\real^{t\times r}$. Suppose $(B^{\star})^\top\Sigma B^{\star}$ is of full rank in the sense that $\lambda_{\min}((B^{\star})^\top\Sigma B^{\star})\gtrsim \sigma^2$. Let $U_r(i)$ denote the $i$-th row vector of $U_r$.
	We have the following 
	\begin{itemize}
		\item We have $\te(\hat{\beta}_t^{\MTL}) < \te(\hat{\beta}_t^{\STL})$ with high probability if
		$$\left(1 - \norm{U_r(t)}^2 \right)\frac{\sigma^2}{\rho - 1} > \norm{\Sigma (B^{\star} U_r U_r(t) - \beta_t)}^2 +  \oo \left( \|B^\star\|^2 + \sigma^2\right).$$ 
		
		\item We have $\te(\hat{\beta}_t^{\MTL}) > \te(\hat{\beta}_t^{\STL})$ with high probability if
		$$\left(1 - \norm{U_r(t)}^2\right)\frac{\sigma^2}{\rho - 1} < \norm{\Sigma(B^{\star} U_r U_r(t) - \beta_t)}^2 -  \oo \left( \|B^\star\|^2 + \sigma^2\right).$$ 

	\end{itemize}
\end{theorem}
%As a remark, since the spectral norm of $U_r$ is less than $1$, we have that $\norm{U_r(i)} < 1$ for all $1 \le i \le t$. Compared to Theorem \ref{thm_main_informal}, we can get a simple expression for the two functions $\Delta_{vari}$ and $\Delta_{\bias}$. The proof of Theorem \ref{thm_many_tasks} can be found in Appendix \ref{app_proof_many_tasks}.



%In this section we consider the setting with $k$ many that have the same covariates.
%Since every task has the same number of data points as well as the same covariance, the only differences between different tasks are their models $\set{\beta_i}_{i=1}^k$.
%For this setting, we derive solutions for the multi-task training and the transfer learning setting that match our insights qualitatively from Section \ref{sec_denoise}.

\begin{proof}[Proof of Theorem \ref{thm_many_tasks}]
In this setting, we need to study the following loss function:
\begin{align}
	f(B; W_1, \dots, W_t) = \sum_{i=1}^t \bignorm{X B W_i - Y_i}^2. \label{eq_mtl_same_cov}
\end{align}
%In order to prove Theorem \ref{thm_many_tasks}, we will derive a closed form solution for equation \eqref{eq_mtl_same_cov}. \todo{check!} 
For any fixed $W_1, W_2, \dots, W_t \in \R^r$, we can derive a closed form solution for $B$ as
	\begin{align*}
		\hat{B}(W_1, \dots, W_t) &= (X^{\top}X)^{-1} X^{\top} \bigbrace{\sum_{i=1}^t Y_i W_i^{\top}} (\cal W  \cal W^{\top})^{-1} \\
		&= (B^\star \cal W ^{\top}) (\cal W \cal W ^{\top})^{-1} + (X^{\top}X)^{-1}X^{\top} \bigbrace{\sum_{i=1}^t \varepsilon_i W_i^{\top}} (\cal W \cal W^{\top})^{-1},
	\end{align*}
	where we denote $\cal W \in\real^{r\times t}$ as $\cal W=[W_1, W_2, \dots, W_t]$.
	%Now we switch $\hat{B}$ back into equation \eqref{eq_mtl_same_cov} to
Then as in \eqref{approxvalid}, we pick $N$ independent samples of the training set for each task with $N\ge n^{1-\e_0}$, and use concentration to get the validation loss as
\be\label{eq_multival}\wt f(\hat B; \cal W)=  N\left[\val(\cal W) + t  \sigma^2 \right]\cdot \left( 1+\OO(p^{-(1-\e_0)/2+\e})\right).\ee
Here $\val(\cal W)$ is defined as 
%it remains to consider minimizing the validation loss 
	$$\val( \cal W):=\exarg{\varepsilon_j, \forall 1\le j\le t}{ \sum_{i=1}^t \bignorm{\Sigma^{1/2}( \hat B W_i - \beta_i)}^2} =  \delta_{\bias}(\cal W) + \delta_{\vari}(\cal W),$$ 
where the model shift bias term $\delta_{\bias}(\cal W) $ is given by
	\begin{align*}
		\delta_{\bias}(\cal W) :=\sum_{i=1}^t  \bignorm{\Sigma^{1/2}\bigbrace{(B^\star \cal W^\top) (\cal W\cal W^{\top})^{-1} W_i - \beta_i}}^2,
	\end{align*}
	and the variance term $\delta_{\vari}(\cal W)$ can be calculated as 
	\begin{align*}
		\delta_{\vari}(\cal W):= \sigma^2 \cdot \bigtr{\Sigma (X^{\top}X)^{-1}}.
	\end{align*}
It suffices to minimize $\delta_{\bias}(\cal W)$ over $\cal W$, since both $t N \sigma^2$ and $\delta_{\vari}(\cal W)$ do not depend on the weights. 

We denote $Q := \cal W^{\top} (\cal W\cal W^{\top})^{-1} \cal W \in\real^{k\times k}$, whose $(i,j)$-th entry is equal to $W_i^{\top} (\cal W\cal W^{\top})^{-1} W_j$. 
	%Let $B^{\star} = [\beta_1, \beta_2, \dots, \beta_k] \in\real^{p \times k}$ denote the true model parameters.
	Now we can write $\delta_{\bias}(\cal W)$ succinctly as 
	\begin{align*}
		\delta_{\bias}(\cal W)= \bignormFro{\Sigma^{1/2}B^{\star}  \bigbrace{Q -\id}}^2 .
	\end{align*}
	From this equation we can solve the minimizer optimally as $Q_0=U_{r}U_r^{\top}$. On the other hand, let $\hat {\cal W}$ be the minimizer of $\wt f$, and denote $\hat Q:= \hat{\cal W}^{\top} (\hat{\cal W}\hat{\cal W}^{\top})^{-1} \hat{\cal W} $. We claim that $\hat Q$ satisfies
	\be\label{Q-Q}\|Q_0^{-1}\hat Q - \id\| = \oo(1) \quad \text{whp}.\ee
	In fact, if \eqref{Q-Q} does not hold, then using the condition $\lambda_{\min}((B^{\star})^\top\Sigma B^{\star})\gtrsim \sigma^2$ and that $\delta_{\vari}(\cal W)=\OO(\sigma^2)$ by \eqref{eq_isometric}, we obtain that 
	$$   \val(\hat {\cal W}) + t  \sigma^2 > (\val( {\cal W}_0) + t \sigma^2 )\cdot (1+\oo(1)) \ \Rightarrow \ \wt f(\hat B; \hat{\cal W})>\wt f(\hat B; {\cal W}_0),$$
	that is, $\hat {\cal W}$ is not a minimizer. This leads to a contradiction. 
	
	In sum, we have solved that $\hat{\beta}_i^{\MTL}=B^{\star}\left( U_r U_r(i) +\oo(1)\right)$. Inserting it into the definition of the test loss, we get that
	\begin{align*}
		\te(\hat{\beta}_t^{\MTL}) &= \bignorm{\Sigma^{1/2} \left((B^\star \hat{\cal W}^\top) (\hat{\cal W}\hat{\cal W}^{\top})^{-1} \hat W_t - \beta_2 \right) }^2
		+ \sigma^2  \hat W_t^{\top} (\hat{\cal W}\hat{\cal W}^{\top})^{-1} \hat W_t \cdot \bigtr{\Sigma (X^{\top}X)^{-1}} \\
		&= \bignorm{\Sigma^{1/2} \bigbrace{B^{\star} U_r U_r(t)-\beta_2}}^2 + \oo\left(\|B^\star\|^2\right) + \sigma^2\norm{U_r(t)}^2 \bigtr{\Sigma (X^{\top}X)^{-1}} \cdot (1+\oo(1))\\
		&= \bignorm{\Sigma^{1/2} \bigbrace{B^{\star} U_r U_r(t)-\beta_2}}^2 + \frac{\sigma^2}{\rho-1}\norm{U_r(t)}^2 + \oo \left( \|B^\star\|^2 + \sigma^2\right),
	\end{align*}
	with high probability, where we used Lemma \ref{lem_minv} in the last step. Similar, by Lemma \ref{lem_minv} we have
	$$\te(\hat{\beta}_t^{\MTL})=\frac{\sigma^2}{\rho-1} \cdot \left( 1+\oo(1)\right).$$
Combining the above two estimates, we conclude the proof.
\end{proof}
%From the above we can obtain three conceptual insights that are consistent with Section \ref{sec_denoise} and \ref{sec_insight}.
%\begin{itemize}
%	\item The de-noising effect of multi-task learning.
%	\item Multi-task training vs single-task training can be either positive or negative.
%	\item Transfer learning is better than the other two. And the improvement over multi-task training increases as the model distances become larger.
%\end{itemize}

