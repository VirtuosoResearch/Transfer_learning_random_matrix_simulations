%Since all tasks share the same underlying model $\beta$, we use a simplified objective as follows.
%\begin{align}
%	\label{eq_mtl_basic}
%	f(w) = \sum_{i=1}^k \norm{X_i w - Y_i}^2.
%\end{align}
%{\bf The distributed learning problem.}
%    \begin{align}
%     f(w) = \sum_{i=1}^k \normFro{X_i w - Y_i}^2. \label{eq_dist}
%   \end{align}
%Equation \eqref{eq_mtl_basic} is simplified from equation \eqref{eq_mtl} by setting $A_i$ to be 1 for all tasks. %can select a model from the subspace of $B$ to fit $(X_i, Y_i)$.

As a warm up, we show that when $\beta_s = \beta_t$, then the transfer is always positive.

\begin{proposition}\label{prop_monotone}
	Suppose that $n > p$.
  When there is no model shift, i.e. $\beta_s = \beta_t$, adding the source task data always reduces the estimation error and the test error for the target task, i.e.
	\begin{align}
%		\err(\hat{\beta}_{t}^{\MTL})  &\le \err(\hat{\beta}_t^{\STL}), \text{ and} \label{eq_mono_e}\\
		\te(\hat{\beta}_{t}^{\MTL}) &\le \te(\hat{\beta}_t^{\STL}). \label{eq_mono_te}
	\end{align}
\end{proposition}

\begin{proof}
%	Equation \eqref{eq_mono_e} is simply because
%		\[ \err(\hat{\beta}_{s,t}) = \bigtr{(X_1^{\top}X_1 + X_2^{\top}X_2)^{-1}} \le \bigtr{(X_1^{\top}X_1)^{-1}} = \err(\hat{\beta}_t). \]
%	Equation \eqref{eq_mono_te} follows because
	Recall that $\hat{\beta}_t^{\MTL} = \hat{w} \cdot \hat{B}$.
	By the optimality of $\hat{w}$, we have by setting $w = 1$ in equation \eqref{eq_te_mtl}
	\begin{align*}
		\te(\hat{\beta}_t^{\MTL}) &\le \sigma^2 \cdot \bigtr{(X_1^{\top}X_1 + X_2^{\top}X_2)^{-1}\Sigma_2} \\
		&= \sigma^2 \cdot \bigtr{\bigbrace{\Sigma_2^{-1/2}X_1^{\top}X_1\Sigma_2^{-1/2} + \Sigma_2^{-1/2}X_2^{\top}X_2\Sigma^{-1/2}}^{-1}} \\
		&\le \bigtr{\bigbrace{\Sigma_2^{-1/2}X_2^{\top}X_2\Sigma_2^{-1/2}}^{-1}}
			= \bigtr{(X_2^{\top}X_2)^{-1} \Sigma_2} = \te(\hat{\beta}_t^{\STL}),
	\end{align*}
	which concludes the proof.
\end{proof}

As a remark, we can derive a similar result for the estimation error as well. The details are omitted.


In the above, the first is a bias term introduced by the model shift between the source and target tasks.
The second is a variance term, which decreases monotonically as we add more and more source task data.
Let $\hat{w}$ denote the minimizer of $te(w_2 \hat{B})$ over $w\in\real$.
Note that we can obtain the value of $\te(w_1\hat{B})$ through a validation set.
We will denote $\hat{\beta}_t^{\TL} = w_2 \hat{B}(\hat{w})$.

There are two reasons for studying this setting.
First, this is akin to performing transfer learning using the hard parameter sharing architecture.
Another way to obtain $\hat{\beta}_t^{\TL}$ is that once we minimize the multi-task validation loss, we further minimize the validation loss on the target dataset, which is also known as fine-tuning in practice.
%	\begin{align}
%		\left(\frac{n_2}{n_1+n_2}\frac1{a_1^2}- \frac1{n_1+n_2}\sum_{i=1}^p \frac{1}{ (a_1 + \lambda_i^2a_2)^2  }\right) a_3 -  \left(\frac1{n_1+n_2}\sum_{i=1}^p \frac{  \lambda_i^2 }{ (  a_1 + \lambda_i^2a_2)^2  }\right)a_4 &=  \frac1{n_1+n_2}\sum_{i=1}^p \frac{1 }{ (  a_1 + \lambda_i^2a_2)^2  } , \label{eq_a3} \\
%		\left( \frac{n_1}{n_1+n_2}\frac1{a_2^2} -  \frac1n\sum_{i=1}^p \frac{\lambda_i^4   }{  (a_1 + \lambda_i^2a_2)^2  }\right)a_4 -\left( \frac1 {n_1+n_2}\sum_{i=1}^p \frac{\lambda_i^2  }{  (a_1 + \lambda_i^2a_2)^2  }\right)a_3 &=   \frac1 {n_1+n_2}\sum_{i=1}^p \frac{\lambda_i^2 }{  (a_1 + \lambda_i^2a_2)^2  }. \label{eq_a4}
%	\end{align}
	%We have that
	%\begin{align*}
	%	&~ {\bignorm{\Sigma_2^{1/2} (X_1^{\top}X_1 + X_2^{\top}X_2)^{-1} X_1^{\top}X_1 %(\beta_s - \beta_t)}} \\
	%	&\le ~ \left[(\beta_s - \beta_t)^{\top}\Sigma_1^{1/2}M\frac{(1 + a_3)\id + a_4 M^{\top}M}{(a_1 + a_2 M^{\top}M)} M^{\top}\Sigma_1^{1/2} (\beta_s - \beta_t)\right]^{1/2} \\
	%	&+ \left\|M\frac{(1 + a_3)\id + a_4 M^{\top}M}{(a_1 + a_2 M^{\top}M)} M^{\top}\right\|_{op}^{1/2} \|\Sigma_1^{1/2} (\beta_s - \beta_t)\|_2 \left( 2\sqrt{\frac{p} {n_1}} + \frac{p}{n_1}\right),
	%\end{align*}
	%and
	%\begin{align*}
	%	&~ {\bignorm{\Sigma_2^{1/2} (X_1^{\top}X_1 + X_2^{\top}X_2)^{-1} X_1^{\top}X_1 (\beta_s - \beta_t)}} \\
	%	&\ge ~ \left[(\beta_s - \beta_t)^{\top}\Sigma_1^{1/2}M\frac{(1 + a_3)\id + a_4 M^{\top}M}{(a_1 + a_2 M^{\top}M)} M^{\top}\Sigma_1^{1/2} (\beta_s - \beta_t)\right]^{1/2} \\
	%	&- \left\|M\frac{(1 + a_3)\id + a_4 M^{\top}M}{(a_1 + a_2 M^{\top}M)} M^{\top}\right\|_{op}^{1/2} \|\Sigma_1^{1/2} (\beta_s - \beta_t)\|_2 \left( 2\sqrt{\frac{p} {n_1}} + \frac{p}{n_1}\right).
	%\end{align*}
	%Here in the error term $p$ can be replaced with the rank of $\Sigma_1$. Without any extra information, we can only use the fact the rank of $\Sigma_1$ is at most $p$.



%Then it is equivalent to study
%$$Q:=  ( X_1^T X_1 )^{-1}   D X_2^T X_2 D ,$$
%which has the same nonzero eigenvalues of
%$$\mathcal Q:=  X_2 D ( X_1^T X_1 )^{-1} D X_2^T,$$
%i.e. $\mathcal Q$ has the same nonzero eigenvalues of $Q$, but has $(n_2-p)$ more zero eigenvalues.

%----------The following is the results on the eigenvalues, not the singular values-------------
%We are interested in the eigenvalues of
%$$(X_1^{\top}X_1)^{-1} (X_2^{\top}X_2),$$
%which is called a generalized Fisher matrix. As in the previous setting, we write out their covariance explicitly and consider
%$$  (\Sigma_1^{1/2}  Z_1^T Z_1 \Sigma_1^{1/2})^{-1}   \Sigma_2^{1/2}  Z_2^T Z_2 \Sigma_2^{1/2} ,$$
%where $\Sigma_{1,2}$ are $p\times p$ deterministic covariance matrices, and $X_1=(x_{ij})_{1\le i \le n_1, 1\le j \le p}$ and $X_2=(x_{ij})_{n_1+1\le i \le n_1+n_2, 1\le j \le p}$ are $n_1\times p$ and $n_2 \times p$ random matrices, respectively, where the entries $x_{ij}$, $1 \leq i \leq n_1+n_2\equiv n$, $1 \leq j \leq p$, are real independent random variables satisfying \eqref{eq_12moment}.
%We can study it using the linearization matrix ({\color{red}for my own purpose right now})
%$$ H = \begin{pmatrix} - z I & X_2D & 0 \\ DX_2^T & 0 & X_1^T \\ 0 & X_1 & I \end{pmatrix}, \quad G(z):=H(z)^{-1}.$$
%Then the $(1,1)$-th block is equal to $\cal G(z):=(\mathcal Q-z)^{-1}$. We denote
%\begin{equation}\label{m1m}m_1(z):=\frac1n\tr \cal G(z), \quad m(z)= \frac1p \left[ nm_1(z) + \frac{n_2-p}{z}\right].\ee
%We can show that it satisfies the following self-consistent equations together with another $m_3(z)$:
%\begin{align}\label{m13}
%\frac{n_2}{n}\frac1{m_1} = - z +\frac1n\sum_{i=1}^p \frac{d_i^2}{ m_3 + d_i^2m_1  } ,\quad \frac{n_1}{n}\frac1{m_3} = 1 +\frac1n\sum_{i=1}^p \frac{1}{   m_3 + d_i^2m_1  } .
%\end{align}
%For these two equations, we can obtain one single equation for $m_1(z)$:
%\begin{align}\label{m1}
%m_3= 1-\frac pn + zm_1, \quad \frac{n_2}{n}\frac1{m_1} = - z +\frac1n\sum_{i=1}^p \frac{d_i^2}{ 1-\frac pn + (z + d_i^2)m_1  }  .
%\end{align}
%One can solve the above equation for $m_1$ with positive imaginary parts, and then calculate $m(z)$ using \eqref{m1m}.
%
%With $m(z)$, we can define
%$$\rho_c(z):=\frac1\pi \lim_{\eta\downarrow 0} m(z).$$
%It will be a compact supported probability density which gives the eigenvalue distribution of $Q$. Moreover, we have
%$$\frac{1}{p}\tr \frac1{(\Sigma_1^{1/2}  X_1^T X_1 \Sigma_1^{1/2})^{-1}   \Sigma_2^{1/2}  X_2^T X_2 \Sigma_2^{1/2} +1}\left(\frac1{(\Sigma_1^{1/2}  X_1^T X_1 \Sigma_1^{1/2})^{-1}   \Sigma_2^{1/2}  X_2^T X_2 \Sigma_2^{1/2} +1}\right)^T \approx \int \frac{\rho_c(x)}{(x+1)^2}{\rm d}x.$$
%We know that
%$$ m(z)=\int \frac{\rho_c(x)}{ x-z}{\rm d}x.$$
%Hence we have
%$$ m'(-1)=\int \frac{\rho_c(x)}{(x+1)^2}{\rm d}x.$$
%
%Moreover, the right edge $\lambda_+$ of $\rho_c$ gives the location of the largest eigenvalue, while the left edge $\lambda_-$  of $\rho_c$ gives the location of the smallest eigenvalue. They will provide the upper and lower bounds on the operator norm:
%$$\frac{1}{1+\lambda_+}\le \frac1{(\Sigma_1^{1/2}  X_1^T X_1 \Sigma_1^{1/2})^{-1}   \Sigma_2^{1/2}  X_2^T X_2 \Sigma_2^{1/2} +1}\le \frac1{1+\lambda_-}.$$
%The edges of the spectrum can be determined by solving the following equations of $(x,m_1)$:
%$$\frac{n_2}{n}\frac1{m_1} = - x +\frac1n\sum_{i=1}^p \frac{d_i^2}{ 1-\frac pn + (x + d_i^2)m_1  } ,\quad  \frac{n_2}{n}\frac1{m_1^2} = \frac1n\sum_{i=1}^p \frac{d_i^2 (x+d_i^2)}{\left[ 1-\frac pn + (x + d_i^2)m_1 \right]^2 } .$$
%The solutions for $x$ give the locations for the edges of the spectrum.
%--------------------------------------





%\section{The Case of Two Tasks}\label{sec_defspike}
%We denote task 1 as the source, i.e. $\beta_1 = \beta_s$.
%\subsection{Covariate Shift}
%\subsection{Model Shift}
	Moreover, when $n_1 = 0$, we have %we have that $a_1 = 0$ and $a_2 = (n_2-p) / n_2$, hence
	\begin{align}\label{lem_cov_shift_eq2} \bigtr{(X_2^{\top}X_2)^{-1}\Sigma_2} = \frac{p}{n_2-p} + \bigo{n^{-1/2+\epsilon}}, \end{align}
which is a well-known result for inverse Whishart matrices {\color{red}add some references}.
 %, where $a_{3,4}$ are found using equations in  \eqref{m35reduced}.
