\section{Supplementary Materials for Section \ref{sec_main}}

Let ${M} = \hat{w} \Sigma_1^{1/2}\Sigma_2^{-1/2}$ denote the weighted covariate shift matrix. Denote by ${\lambda}_1\ge {\lambda}_2 \ge \dots \ge {\lambda}_p$ the singular values of ${M}^{\top}{M}$. Let $(a_1, a_2)$ be the solutions to the following system of equations
	\be
		 a_1 + a_2 = 1- \frac{p}{n_1 + n_2},~ a_1 + \sum_{i=1}^p \frac{a_1}{(n_1 + n_2)(a_1 + a_2/ \lambda_i^2)} = \frac{n_1}{n_1 + n_2}.\label{eq_a2} \\
		 \ee
		 After obtaining $(a_1,a_2)$, we can solve the following linear equations to get $(a_3,a_4)$:
\begin{gather}
		\left(\frac{n_2}{a_2^2}- \sum_{i=1}^p \frac{1}{ (a_2 + \lambda_i^2a_1)^2  }\right) a_3 -  \left(\sum_{i=1}^p \frac{  \lambda_i^2 }{ (  a_2 + \lambda_i^2a_1)^2  }\right)a_4
		= \sum_{i=1}^p \frac{1 }{ (  a_2 + \lambda_i^2a_1)^2  }, \label{eq_a3} \\
		\left(\frac{n_1}{a_1^2} -  \sum_{i=1}^p \frac{\lambda_i^4   }{  (a_2 + \lambda_i^2a_1)^2  }\right)a_4 -\left(\sum_{i=1}^p \frac{\lambda_i^2  }{  (a_2 + \lambda_i^2a_1)^2  }\right)a_3
		= \sum_{i=1}^p \frac{\lambda_i^2 }{  (a_2 + \lambda_i^2a_1)^2  }. \label{eq_a4}
	\end{gather}
Then we introduce the following matrix
$$Z = \frac{n_1^2}{(n_1 + n_2)^2}\cdot{M} \frac{(1 + a_3)\id + a_4 {M}^{\top}{M}}{(a_2 + a_1 {M}^{\top}{M})^2} {M}^{\top},$$
which can be regarded as the asymptotic limit of $\hat w \Sigma_2^{1/2} (\hat{w}^2 X_1^{\top}X_1 + X_2^{\top}X_2)^{-1}\Sigma_1^{1/2}$. Finally we introduce
$$\delta:=\left[\frac{n_1 \lambda_1}{(\sqrt{n_1}-\sqrt{p})^2\lambda_p  +  (\sqrt{n_2}-\sqrt{p})^2}\right]^2\cdot \norm{\Sigma_1^{1/2}(\beta_s - \hat{w}\beta_t)}^2.$$
{\cor may be able to get a better bound, but the statement will be long}

We now state our main result for two tasks with both covariate and model shift in the following theorem.

\begin{theorem}\label{thm_model_shift}
	Let $n_1, n_2$ be the number of data points for the source, target task, respectively.
	Let $\hat{w}$ denote the optimal solution for the ratio $w_1/w_2$ in equation \eqref{eq_val_mtl}.
	%Let ${M} = \hat{w} \Sigma_1^{1/2}\Sigma_2^{-1/2}$ denote the weighted covariate shift matrix.
	%Denote by ${\lambda}_1, {\lambda}_2, \dots, {\lambda}_p$ the singular values of ${M}^{\top}{M}$.
	The information transfer is solely determined by two deterministic quantities $\Delta_{\beta}$ and $\Delta_{\vari}$, which show the change of model shift bias and variace, respectively.
	With high probability we have
	\begin{align}
	 	\te(\hat{\beta}_{t}^{\MTL}) \le \te(\hat{\beta}_t^{\STL}) \text{ when: } &\Delta_{\vari} - \Delta_{\beta} \ge \left(\bigbrace{1 + \sqrt{\frac{p} {n_1}}}^4 - 1 \right) \delta \label{upper}\\
		\te(\hat{\beta}_t^{\MTL}) \ge \te(\hat{\beta}_t^{\STL}) \text{ when: } &\Delta_{\vari} - \Delta_{\beta} \le -2\bigbrace{2\sqrt{\frac {p}{n_1}} + \frac{p}{n_1}} \delta, \label{lower}
	\end{align}
	where
	\begin{align*} %\bigtr{{\Sigma_2^{-1}}}
		\Delta_{\vari} &\define {\sigma^2}\bigbrace{\frac{p}{n_2 - p} -  \frac{1}{n_1 + n_2} \bigtr{(a_1 M^{\top}M + a_2\id)^{-1}} } \\
		\Delta_{\beta} &\define (\beta_s - \hat{w}\beta_t)^{\top} \Sigma_1^{1/2} Z \Sigma_1^{1/2} (\beta_s - \hat{w}\beta_t).
	\end{align*}
\end{theorem}



\subsection{Proof of Theorem \ref{thm_model_shift}}\label{sec pfmain}

\noindent\todo{A proof outline; including the following key lemma.}
To prove Theorem \ref{thm_cov_shift}, we study the spectrum of the random matrix model:
$$Q= \Sigma_1^{1/2}  Z_1^{\top} Z_1 \Sigma_1^{1/2}  + \Sigma_2^{1/2}  Z_2^{\top} Z_2 \Sigma_2^{1/2} ,$$
where $\Sigma_{1,2}$ are $p\times p$ deterministic covariance matrices, and $X_1=(x_{ij})_{1\le i \le n_1, 1\le j \le p}$ and $X_2=(x_{ij})_{n_1+1\le i \le n_1+n_2, 1\le j \le p}$ are $n_1\times p$ and $n_2 \times p$ random matrices, respectively, where the entries $x_{ij}$, $1 \leq i \leq n_1+n_2\equiv n$, $1 \leq j \leq p$, are real independent random variables satisfying
\begin{equation}\label{eq_12moment} %\label{assm1}
\mathbb{E} z_{ij} =0, \ \quad \ \mathbb{E} \vert z_{ij} \vert^2  = 1.
\end{equation}

The proof of Theorem \ref{thm_model_shift} involves two parts.

\paragraph{Part I: Bounding the bias from model shift.}
We relate the first term in equation \eqref{eq_te_model_shift} to $\Delta_{\beta}$.
\begin{proposition}\label{prop_model_shift}
	In the setting of Theorem \ref{thm_model_shift},
	denote by $K = (\hat{w}^2X_1^{\top}X_1 + X_2^{\top}X_1)^{-1}$, and
	\begin{align*}
		\delta_1 &= \hat{w}^2 \bignorm{\Sigma_2^{1/2} K X_1^{\top}X_1(\beta_s - \hat{w}\beta_t)}^2, \\
		\delta_2 &= n_1^2\cdot \hat{w}^2 \bignorm{\Sigma_2^{1/2}K\Sigma_1(\beta_s - \hat{w}\beta_t)}, \\
		\delta_3 &= n_1^2\cdot \hat{w}^2 \bignorm{\Sigma_1^{1/2} K \Sigma_2 K \Sigma_1^{1/2}} \cdot \bignorm{\Sigma_1^{1/2} (\beta_s - \hat{w}\beta_t)}^2.
	\end{align*}
	We have that
	\begin{align*}
		-2n_1^2\bigbrace{{2\sqrt{\frac{p}{n_1}}} + {\frac{p}{n_1}}} \delta_3
		\le  \delta_1 - \delta_2
		\le n_1^2\bigbrace{2\sqrt{\frac{p}{n_1}} + \frac{p}{n_1}}\bigbrace{2 + 2\sqrt{\frac{p}{n_1}} + \frac{p}{n_1}}\delta_3.
	\end{align*}
	For the special case when $\Sigma_1 = \id$ and $\beta_s - \beta_t$ is i.i.d. with mean $0$ and variance $d^2$, we further have
	\begin{align*}
		\bigbrace{1 - \sqrt{\frac{p}{n_1}}}^4 \Delta_{\beta}
		\le \bignorm{\Sigma_2^{1/2} (X_1^{\top}X_1 + X_2^{\top}X_2)^{-1}X_1^{\top}X_1(\beta_s - \beta_t)}^2.
	\end{align*}
\end{proposition}

\begin{proof}
	The proof follows by applying equation \eqref{eq_isometric}.
	Recall that $X_1^{\top}X_1 = \Sigma_1^{1/2}Z_1^{\top}Z_1\Sigma_1^{1/2}$.
	Denote by $\cE = Z_1^{\top}Z_1 - {n_1}\id$.
	Let
%	Let $\alpha = \bignorm{\Sigma_2^{1/2} K \Sigma_1 (\beta_s - \hat{w}\beta_t)}^2$.
	We have
	\begin{align}
%		& \bignorm{\Sigma_2^{1/2}(X_1^{\top}X_1 + X_2^{\top}X_2)^{-1}X_1^{\top}X_1(\beta_s - \hat{w}\beta_t)}^2 \nonumber \\
		\delta_1 = \delta_2 + {2\hat{w}^2n_1}(\beta_s - \hat{w}\beta_t)^{\top}\Sigma_1^{1/2}\cE\Sigma_1^{1/2}K \Sigma_2 K \Sigma_1 (\beta_s - \hat{w}\beta_t)
		+ \hat{w}^2\bignorm{\Sigma_2^{1/2} K \Sigma_1^{1/2}\cE \Sigma_1^{1/2}(\beta_s - \hat{w}\beta_t)}^2 \label{eq_lem_model_shift_1}
%		\le& n_1\bigbrace{{n_1^2}{} + \frac{2n_1}p(p + 2\sqrt{{n_1}p}) + (p + 2\sqrt{{n_1}p})^2} \alpha = n_1^2\bigbrace{1 + \sqrt{\frac{p}{n_1}}}^4 \alpha. \nonumber
	\end{align}
	Here we use the following on the second term in equation \eqref{eq_lem_model_shift_1}
	\begin{align*}
		& \bigabs{(\beta_s - \hat{w}\beta_t)^{\top} \Sigma_1^{1/2} \cE \Sigma_1^{1/2} K \Sigma_2 K \Sigma_1 (\beta_s - \hat{w}\beta_t)} \\
		= & \bigabs{\bigtr{\cE \Sigma_1^{1/2}K\Sigma_2 K \Sigma_1(\beta_s - \hat{w}\beta_t)(\beta_s - \hat{w}\beta_t)^{\top} \Sigma_1^{1/2}}} \\
		\le & \norm{\cE} \cdot \bignormNuclear{\Sigma_1^{1/2} K \Sigma_2 K \Sigma_1 (\beta_s - \hat{w}\beta_t) (\beta_s - \hat{w}\beta_t)^{\top} \Sigma_1^{1/2}} \\
		\le & n_1 \bigbrace{2\sqrt{\frac{p}{n_1}} + \frac{p}{n_1}} \cdot \bignormNuclear{\Sigma_1^{1/2} K \Sigma_2 K \Sigma_1 (\beta_s - \hat{w}\beta_t)(\beta_s - \hat{w}\beta_t)^{\top} \Sigma_1^{1/2}} \tag{by equation \eqref{eq_isometric}} \\
		\le   & n_1 \bigbrace{2\sqrt{\frac{p}{n_1}} + \frac{p}{n_1}} \bignorm{\Sigma_1^{1/2}K \Sigma_2 K \Sigma_1^{1/2}} \cdot \bignorm{\Sigma_1^{1/2}(\beta_s - \hat{w}\beta_t)}^2 \tag{since the matrix inside is rank 1}
	\end{align*}
	The third term in equation \eqref{eq_lem_model_shift_1} can be bounded with
	\begin{align*}
		\bignorm{\Sigma_2^{1/2}K\Sigma_1^{1/2}\cE\Sigma_1^{1/2}(\beta_s - \hat{w}\beta_t)}^2
		\le n_1^2 \bigbrace{2\sqrt{\frac{p}{n_1}} + \frac{p}{n_1}}^2 \bignorm{\Sigma_1^{1/2}K\Sigma_{2}K\Sigma_1^{1/2}} \cdot \bignorm{\Sigma_1^{1/2}(\beta_s - \hat{w}\beta_t)}^2.
	\end{align*}
	Combined together we have shown the right direction for $\delta_1 - \delta_2$.
	For the left direction, we simply note that the third term in equation \eqref{eq_lem_model_shift_1} is positive.
	And the second term is bigger than $-2n_1^2(2\sqrt{\frac{p}{n_1}} + \frac{p}{n_1}) \alpha$ using equation \eqref{eq_isometric}.
\end{proof}




\paragraph{Part II: The limit of $\bignorm{\Sigma_2^{1/2} (\hat{w}^2X_1^{\top}X_1 + X_2^{\top}X_2)^{-1}\Sigma_1 (\beta_s - \hat{w}\beta_t)}^2$ using random matrix theory.}
We consider the same setting as in previous subsection:
$$ X_1^{\top}X_1:=\Sigma_1^{1/2}  Z_1^T Z_1 \Sigma_1^{1/2} ,\quad X_2^{\top}X_2= \Sigma_2^{1/2}  Z_2^T Z_2 \Sigma_2^{1/2},$$
where $z_{ij}$, $1 \leq i \leq n_1+n_2\equiv n$, $1 \leq j \leq p$, are real independent random variables satisfying \eqref{eq_12moment}. For now, we assume that the random variables $z_{ij}$ are i.i.d. Gaussian, but we know that universality holds for generally distributed entries. Assume that $p/n_1$ is a small number such that $Z_1^TZ_1$ is roughly an isometry, that is, under \eqref{eq_12moment},
%\todo{
%\begin{align}\left\| Z_1^T Z_1 -  \frac{n_1}{n} I \right\| \le 2\sqrt{\frac{p}{n}} + {\frac{p}{n}} .
%\end{align}}
{\color{blue}
If we assume the variances of the entries of $Z_1$ are $1$, then we have
\begin{align}
- {n_1} \left(2\sqrt{\frac{p}{n_1}} - {\frac{p}{n_1}}\right)  \le {Z_1^T Z_1 -  n_1 \id}  \le {n_1} \left(2\sqrt{\frac{p}{n_1}} + {\frac{p}{n_1}}\right) . \label{eq_isometric}
\end{align}
}

\begin{lemma}\label{lem_cov_derivative}
	In the setting of Theorem \ref{thm_model_shift}, we have with high probability $1-{\rm o}(1)$,
\begin{equation}\label{lem_cov_derv_eq}
\begin{split}
&\wh w^{2}(n_1+n_2)^2\bignorm{\Sigma_2^{1/2} (\hat{w}X_1^{\top}X_1 + X_2^{\top}X_2)^{-1}\Sigma_1 (\beta_s - {w}\beta_t)}^2 \\
&= (\beta_s - {w}\beta_t)^{\top} \Sigma_1^{1/2} {M} \frac{(1 + a_3)\id + a_4 {M}^{\top}{M}}{(a_2 + a_1 {M}^{\top}{M})^2} {M}^{\top} \Sigma_1^{1/2} (\beta_s - \hat{w}\beta_t) +\OO(n^{-1/2+\e}),
\end{split}
\end{equation}
	for any constant $\epsilon>0$. %, where $a_{3,4}$ are found using equations in  \eqref{m35reduced}
\end{lemma}
We will give the proof of this lemma in Section \ref{sec_maintools}.

\begin{proof}[Proof of Proposition \ref{prop_model_shift_tight}]
the proof for tighter bound ........... 
\end{proof}

{\cor add some arguments with $\e$-net.}


\subsection{Proof of Theorem ???}

\paragraph{Extension to Many Tasks of the Same Covariates}

In this section we consider the setting with $k$ many that have the same covariates.
Since every task has the same number of data points as well as the same covariance, the only differences between different tasks are their models $\set{\beta_i}_{i=1}^k$.
For this setting, we derive solutions for the multi-task training and the transfer learning setting that match our insights qualitatively from Section \ref{sec_denoise}.

Concretely we will consider the following problem.
\begin{align}
	f(B; W_1, \dots, W_k) = \sum_{i=1}^k \bignorm{X B W_i - Y_i}^2. \label{eq_mtl_same_cov}
\end{align}
By fixing $W_1, W_2, \dots, W_k$, we can derive a closed form solution for $B$ as
\begin{align*}
	\hat{B}(W_1, \dots, W_k) &= (X^{\top}X)^{-1} X^{\top} \bigbrace{\sum_{i=1}^k Y_i W_i^{\top}} (Z Z^{\top})^{-1} \\
	&= \sum_{i=1}^k \bigbrace{\beta_i W_i^{\top}} (ZZ^{\top})^{-1} + (X^{\top}X)^{-1}X^{\top} \bigbrace{\sum_{i=1}^k \varepsilon_i W_i^{\top}} (ZZ^{\top})^{-1}
\end{align*}
where we denote $Z\in\real^{r\times k}$ as the $k$ vectors $W_1, W_2, \dots, W_k$ stacked together.
%Now we switch $\hat{B}$ back into equation \eqref{eq_mtl_same_cov} to
Similar to Section \ref{sec_setup}, we consider minimizing the validation loss over $W_1, W_2, \dots, W_k$ provided with $\hat{B}$.

\paragraph{Jointly minimizing over all tasks.}
Denote by $\varepsilon(W) = \sum_{i=1}^k \varepsilon_i W_i^{\top}$.
We shall decompose the validation loss $\val(\hat{B}; W_1, \dots, W_k)$ into two parts.
The first part is the model shift bias, which is equal to
\begin{align*}
	\sum_{j=1}^k \bigbrace{\bignorm{\Sigma^{1/2}\bigbrace{\sum_{i=1}^k(\beta_i W_i^{\top}) (ZZ^{\top})^{-1} W_j - \beta_j}}^2}
\end{align*}
The second part is the variance, which is equal to
\begin{align*}
	& \sum_{j=1}^k \exarg{\varepsilon_i, \forall i}{\bigbrace{\bigbrace{\sum_{i=1}^k \varepsilon_i W_i^{\top}} (ZZ^{\top})^{-1} W_j}^2} \cdot {\bigtr{\Sigma(X^{\top}X)^{-1}}} \\
	=& \sigma^2 \cdot \bigtr{\Sigma (X^{\top}X)^{-1}}.
\end{align*}
Therefore we shall focus on the minimizer for the model shift bias since the variance part does not depend the weights.
Let us denote $Q\in\real^{k\times k}$ where the $(i,j)$-th entry is equal to $W_i^{\top} (ZZ^{\top})^{-1} W_j$, for any $1\le i, j\le k$.
Let $B^{\star} = [\beta_1, \beta_2, \dots, \beta_k] \in\real^{p \times k}$ denote the true model parameters.
We can now write the validation loss succinctly as follows.
\begin{align*}
	\val(\hat{B}; W_1, \dots, W_k) = \sum_{j=1}^k \bignorm{\Sigma^{1/2} \bigbrace{B^{\star} Q - \beta_j}}^2 + \sigma^2 \cdot \bigtr{\Sigma (X^{\top}X)^{-1}}
\end{align*}
From the above we can solve for $Q$ optimally as \todo{this}.

\paragraph{Minimizing over the target task alone.}
If we only minimize over the the validation loss for the target task, we shall get the following.
\begin{align*}
	\val_j(\hat{B} W_j) = \bignorm{\Sigma^{1/2}\bigbrace{\sum_{i=1}^k W_i^{\top} (ZZ^{\top})^{-1}W_j \beta_i - \beta_j}}^2
	+ \sigma^2 \cdot W_j^{\top} (ZZ^{\top})^{-1} W_j \cdot \bigtr{\Sigma (X^{\top}X)^{-1}}.
\end{align*}

From the above we can obtain three conceptual insights that are consistent with Section \ref{sec_denoise} and \ref{sec_insight}.
\begin{itemize}
	\item The de-noising effect of multi-task learning.
	\item Multi-task training vs single-task training can be either positive or negative.
	\item Transfer learning is better than the other two. And the improvement over multi-task training increases as the model distances become larger.
\end{itemize}

