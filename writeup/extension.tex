\section{Extensions to the Sparse Setting}

\todo{Describe a straightforward extension of our results here through isometry.}
One important case is the sparse case, where we assume that the $\beta_i$'s are sparse in the sense
that each $\beta_i$ is in a subspace of dimension $k_i\ll p$. In other words,
$$\beta_i = \sum_{j=1}^{k_i} a^{(i)}_j v^{(i)}_j,$$
where $\{v^{(i)}_j\}$ is a set of $k_i$ orthonormal vectors. For $X_i= Z_i \Sigma_i^{1/2}$, we have
$$ X_i \beta_i = Z_i \Sigma_i^{1/2}\beta_i=  Z_i \Sigma_i^{1/2}P_i\beta_i,\quad P_i:= \sum_{j=1}^{k_i}v_j^{(i)}(v_j^{(i)})^\top.$$
Then we have
$$\frac{1}{n_i}\mathbb E\left[P_i \Sigma_i^{1/2} Z_i^T Z_i \Sigma_i^{1/2}P_i\right]=P_i \Sigma_i P_i.$$
Hence our analysis will be valid with projected covariance matrices $P_i \Sigma_i P_i$. Moreover, due to the sparsity assumption, it is natural to have restricted isometry property:
$$\frac{1}{n_i} P_i \Sigma_i^{1/2} Z_i^T Z_i \Sigma_i^{1/2}P_i \approx P_i \Sigma_i P_i.$$
Finally, for our analysis it is not necessary to assume that $\beta_T$ is sparse for the target.

