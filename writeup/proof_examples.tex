\section{Supplementary Materials for the Theoretical Implications}

\subsection{Proofs for Section \ref{sec_similarity}}\label{app_proof_31}

For the ratio $w=w_1/w_2$, we define the function
\begin{align}
	\val(w) &= n_1\left[d^2 +\left( w-1\right)^2\kappa^2\right]\cdot \tr\left[( w^2X_1^{\top}X_1 +X_2^{\top}X_2)^{-2} (X_2^{\top}X_2)^2\right] \nonumber \\
	& + n_2w^2\left[d^2 +\left( w-1\right)^2\kappa^2\right]\cdot \tr\left[( w^2X_1^{\top}X_1 +X_2^{\top}X_2)^{-2} (X_1^{\top}X_1)^2\right] \nonumber\\
			& + (n_1 w^2 +n_2)\sigma^2 \cdot \bigtr{(w^2X_1^{\top}X_1  + X_2^{\top}X_2)^{-1} }. \nonumber
\end{align}
Under the setting of Lemma \ref{prop_model_shift_tight}, again using concentration for random vectors with i.i.d. entries, Lemma \ref{largedeviation}, we can obtain that for the validation loss in \eqref{eq_val_mtl},
\be\label{boundv-w}\val(\hat{B}; w_1, w_2)= \val(w) \left( 1+\OO(p^{-1/2+\e})\right)\ee
with high probability for any constant $\e>0$. Thus for the following discussions, it suffices focus on the behavior of $\val (w)$. Let $\hat w$ the minimizer of $\val(w)$. The proof will consist of two main steps.
\begin{itemize}
	\item First, we show that the optimal ratio $\hat{w}$ is close to $1$. Then \eqref{boundv-w} gives that $|\hat v - \hat w|=\OO(p^{-1/2+\e})$ whp.
	\item Second, we plug $\hat{v}$ back into $\te(\hat{\beta}_t^{\MTL})$ and use Lemma \ref{prop_model_shift_tight} to show the result.
\end{itemize}
%For the minimizer $\hat w$ of $\val(w)$, we have a similar result as in Proposition \ref{thm_cov_shift}.
For the first step, we will prove the following result.
\begin{lemma}\label{lem_hat_v}
%Suppose the assumptions of Lemma \ref{prop_model_shift_tight} hold. Assume that $ \kappa^2 \sim pd^2 \sim \sigma^2$ are of the same order. 
We have that the minimizer for $\val(w)$ satisfies
	\be\label{hatw_add1}|\hat w -1|\le C\left(\frac{d^2}{\kappa^2} + \frac{\sigma^2}{p\kappa^2}\right)\ee
	for some constant $C>0$.
	%is $\hat{w} = 1 \pm \bigo{\frac 1 {n_1+n_2}}$ \todo{(figure out the constants)}
\end{lemma}
\begin{proof}
First it is easy to observe that $\val(w)< \val(-w)$ for $w> 0$. Hence it suffices to assume that $w\ge 0$.

We first consider the case $w\ge 1$. We write
\begin{align}
	\val(w) &= n_1\left[\frac{d^2}{w^4} +\frac{\left( w-1\right)^2}{w^4}\kappa^2\right]\cdot \tr\left[( X_1^{\top}X_1 +w^{-2}X_2^{\top}X_2)^{-2} (X_2^{\top}X_2)^2\right] \nonumber \\
	& + n_2\left[\frac{d^2}{w^2} +\frac{\left( w-1\right)^2}{w^2}\kappa^2\right]\cdot \tr\left[( X_1^{\top}X_1 +w^{-2}X_2^{\top}X_2)^{-2} (X_1^{\top}X_1)^2\right] \nonumber\\
			& + n_1\sigma^2  \cdot \bigtr{(X_1^{\top}X_1  + w^{-2}X_2^{\top}X_2)^{-1} }+ n_2\sigma^2 \cdot \bigtr{(w^2X_1^{\top}X_1  + X_2^{\top}X_2)^{-1} }.\nonumber%+  \sigma^2n_2 \cdot \bigtr{(X_1^{\top}X_1  + w^{-2}X_2^{\top}X_2)^{-1} }. \nonumber
\end{align}
Notice that 
$$\tr\left[( X_1^{\top}X_1 +w^{-2}X_2^{\top}X_2)^{-2} (X_i^{\top}X_i)^2\right],\ i=1,2, \quad \text{and} \quad \bigtr{(X_1^{\top}X_1  + w^{-2}X_2^{\top}X_2)^{-1} }$$
are increasing functions in $w$. Hence taking derivative of $\val(w)$ with respect to $w$, we obtain that
\begin{align*}
\val'(w) \ge &~ n_1\left[ \frac{2(w-1)(2-w)}{w^5}\kappa^2 - \frac{4d^2}{w^5}\right] \tr\left[( X_1^{\top}X_1 +w^{-2}X_2^{\top}X_2)^{-2} (X_2^{\top}X_2)^2\right]   \\
+&~ n_2\left[\frac{2\left( w-1\right)}{w^3}\kappa^2 - \frac{2d^2}{w^3} \right]\cdot \tr\left[( X_1^{\top}X_1 +w^{-2}X_2^{\top}X_2)^{-2} (X_1^{\top}X_1)^2\right] \nonumber\\
		-&~ 2 n_2 \frac{\sigma^2}{w^3} \cdot \bigtr{(X_1^{\top}X_1 + w^{-2}X_2^{\top}X_2)^{-2} X_1^\top X_1  } = n_2 \bigtr{(X_1^{\top}X_1 + w^{-2}X_2^{\top}X_2)^{-2} \cal A },
\end{align*}
where the matrix $\cal A$ is
\begin{align*}
\cal A :&= \frac{n_1}{n_2}\left[ \frac{2(w-1)(2-w)}{w^5}\kappa^2 - \frac{4d^2}{w^5}\right](X_2^{\top}X_2)^2 + \left[\frac{2\left( w-1\right)}{w^3}\kappa^2 - \frac{2d^2}{w^3} \right](X_1^{\top}X_1)^2 - 2 \frac{\sigma^2}{w^3}X_1^\top X_1.
\end{align*}
Using the estimate \eqref{eq_isometric}, we get that $\cal A$ is lower bounded as
\begin{align*}
\cal A \succeq&~ - \frac{4d^2}{w^5}n_1n_2 (\al_+(\rho_2)+\oo(1))^2 + \left[\frac{2\left( w-1\right)}{w^3}\kappa^2 - \frac{2d^2}{w^3} \right]n_1^2 (\al_-(\rho_1)-\oo(1))^2 \\
&~ - 2 \frac{\sigma^2}{w^3}n_1(\al_{+}(\rho_1)+\oo(1)) \succ 0,
\end{align*}
as long as
$$w> w_1:=1 +\frac{d^2}{\kappa^2}+ \frac{\sigma^2}{n_1\kappa^2}\frac{\al_{+}(\rho_1)+\oo(1)}{\al_{-}^2(\rho_1)} + \frac{2d^2}{\kappa^2}\frac{\rho_2(\al_+^2(\rho_2)+\oo(1))}{\rho_1\al_-^2(\rho_1) }.$$
Hence $\val'(w)>0$ on $(w_1,\infty)$, i.e. $\val(w)$ is strictly increasing for $w>w_1$. Hence we must have $\hat w\le w_1$. 

Then we consider the case $w\le 1$, and the proof is similar as above. Taking derivative of $\val(w)$, we obtain that
\begin{align}
	\val'(w) &\le n_1 \left[2\left( w-1\right) \kappa^2\right]\cdot \tr\left[( w^2X_1^{\top}X_1 +X_2^{\top}X_2)^{-2} (X_2^{\top}X_2)^2\right] \nonumber \\
	& + n_2\left[2wd^2 +2w\left( w-1\right)(2w-1)\kappa^2\right]\cdot \tr\left[( w^2X_1^{\top}X_1 +X_2^{\top}X_2)^{-2} (X_1^{\top}X_1)^2\right] \nonumber\\
	&~ + n_1(2 w\sigma^2) \cdot \bigtr{(w^2X_1^{\top}X_1 + X_2^{\top}X_2)^{-2} X_2^\top X_2  }\\
			& = n_1 \bigtr{(w^2X_1^{\top}X_1  + X_2^{\top}X_2)^{-1} \cal B} , \nonumber
\end{align}
where the matrix $\cal B$ is
$$\cal B= 2\left( w-1\right) \kappa^2  (X_2^{\top}X_2)^2+\frac{n_2}{n_1}\left[2wd^2 +2w\left( w-1\right)(2w-1)\kappa^2\right](X_1^{\top}X_1)^2 + 2 w\sigma^2 X_2^\top X_2 .$$
Using the estimate \eqref{eq_isometric}, we get that $\cal B$ is upper bounded as
\begin{align*}
\cal B \preceq - 2(1-w)\kappa^2 n_2^2 (\al_-(\rho_2) -\oo(1))^2 +2w d^2 n_1n_2 (\al_+(\rho_1) +\oo(1))^2 + 2w\sigma^2 n_2 (\al_+(\rho_2)+\oo(1)) \prec 0,
\end{align*}
as long as
$$w< w_2:=1 -   \frac{d^2}{\kappa^2}\frac{\rho_1(\al_+(\rho_1) +\oo(1))^2}{\rho_2 \al_-^2(\rho_2) } -  \frac{\sigma^2}{n_2\kappa^2}\frac{\al_{+}(\rho_2)+\oo(1)}{\al_{-}^2(\rho_2)} .$$
Hence $\val'(w)<0$ on $[0,w_2)$, i.e. $\val(w)$ is strictly decreasing for $w<w_2$. Hence we must have $\hat w\le w_2$. 

In sum, we obtain that $w_2\le w\le w_1$. Note that under our assumptions, we have 
$$\max(|w_1 -1|, |w_2 -1|) =\OO\left(\frac{d^2}{\kappa^2} + \frac{\sigma^2}{p\kappa^2}\right),$$
which concludes the proof.
\end{proof}

Combining \eqref{hatw_add1} and \eqref{boundv-w}, we obtain that whp,
\be\label{hatv_add1} 
|\hat v - 1|= \OO\left(\cal E\right), \quad \cal E:=\frac{d^2}{\kappa^2} + \frac{\sigma^2}{p\kappa^2} + p^{-1/2+\e} 
\ee
Inserting it into \eqref{eq_te_mtl_2task} and using a similar concentration result as in \eqref{boundv-w}, we get whp,
\begin{align}
\te(\hat{\beta}_t^{\MTL})&=(1+\OO(\cal E))\cdot \left[d^2 + \OO\left(\cal E^2 \kappa^2\right)\right] \tr\left[(X_1^{\top}X_1 +X_2^{\top}X_2)^{-2} (X_1^{\top}X_1)^2\right]\nonumber\\ 
&+(1+\OO(\cal E))\cdot \sigma^2  \bigtr{(X_1^{\top}X_1  + X_2^{\top}X_2)^{-1} }. \nonumber
\end{align}
In order to study the phenomenon of bias-variance trade-off, we need the bias term with $d^2$ and the variance term with $\sigma^2$ to be of the same order. With estimate \eqref{eq_isometric}, we see that 
$$\tr\left[(X_1^{\top}X_1 +X_2^{\top}X_2)^{-2} (X_1^{\top}X_1)^2\right] \sim p,\quad \bigtr{(X_1^{\top}X_1  + X_2^{\top}X_2)^{-1} } \sim \frac{p}{n_1+n_2}.$$
Hence we need to choose that $p\cdot d^2 \sim \sigma^2$. On the other hand, we want the error term $\cal E^2 \kappa^2$ to be much smaller than $d^2$, which leads to the condition $p^{-1+2\e}\kappa^2  \ll d^2 \ll \kappa^2$. The above considerations lead to the following choices of parameters: there exists a constant $c>0$ such that
\be\label{choiceofpara}
pd^2 \sim \sigma^2 \sim 1,\quad p^{-1+c}\kappa^2  \le d^2 \le p^c \kappa^2.
\ee
Under this choice, we can write 
\be\label{simple1}
\begin{split}
\te(\hat{\beta}_t^{\MTL})&=(1+\OO(n^{-\e}))\cdot d^2 \tr\left[(X_1^{\top}X_1 +X_2^{\top}X_2)^{-2} (X_1^{\top}X_1)^2\right] \\ 
&+(1+\OO(n^{-\e}))\cdot \sigma^2  \bigtr{(X_1^{\top}X_1  + X_2^{\top}X_2)^{-1} }  
\end{split}
\ee
whp for some constant $\e>0$. 


With \eqref{simple1} and Lemma \ref{prop_model_shift_tight}, we can prove Proposition \ref{prop_dist_transition}, which gives a transition threshold with respect to the ratio between the model bias and the noise level. With slight abuse of notations, we shall write $\hat a_i$, $\hat b_k$ and $\hat M$ as $a_i$, $b_k$ and $M$, respectively, throughout the rest of this section. 


\begin{proof}[Proof of Proposition \ref{prop_dist_transition}]
	In the setting of Proposition \ref{prop_dist_transition}, we have $M = \Sigma_1^{1/2}\Sigma_2^{-1/2} = \id$. Then solving equations \eqref{eq_a2} and \eqref{eq_a3} with $\hat \lambda_i=1$, we get that
	\begin{align}
		 a_1 = \frac{\rho_1(\rho_1 + \rho_2 - 1)}{(\rho_1 + \rho_2)^2} ,\quad
		& a_2 = \frac{\rho_2(\rho_1 + \rho_2 - 1)}{(\rho_1 + \rho_2)^2} , \label{simplesovlea12}\\
		 a_3 = \frac{\rho_2}{(\rho_1 + \rho_2)(\rho_1 + \rho_2 - 1)}, \quad
		& a_4 = \frac{\rho_1}{(\rho_1 + \rho_2)(\rho_1 + \rho_2 - 1)}.\label{simplesovlea34}
	\end{align}
	Using Lemma \ref{lem_minv} and Lemma \ref{lem_cov_shift}, we can track the reduction of variance from $\hat{\beta}_t^{\MTL}$ to $\hat{\beta}_t^{\STL}$ as 
\be\label{Deltavar}
\begin{split}
\delta_{\vari}&:=\sigma^2  \bigtr{(X_2^{\top}X_2)^{-1} }  - (1+\OO(n^{-\e}))\cdot \sigma^2  \bigtr{(X_1^{\top}X_1  + X_2^{\top}X_2)^{-1} } \\
&=\Delta_{\vari}\cdot (1+\OO(n^{-\e})) 
\end{split}
\ee
with high probability, where 
	\begin{align*}
		\Delta_{\vari} &\define \sigma^2 \bigbrace{\frac{1}{\rho_2 - 1} - \frac{1}{\rho_1 + \rho_2}\cdot\frac{1}{a_1+ a_2} } =\sigma^2  \cdot \frac{\rho_1}{(\rho_2-1)(\rho_1 + \rho_2 -1)}.
	\end{align*}
	%where we use equation \eqref{eq_a12} and Lemma \ref{lem_hat_v}.
	Next for the model shift bias
	$$\delta_\beta:=(1+\OO(n^{-\e}))\cdot d^2 \tr\left[(X_1^{\top}X_1 +X_2^{\top}X_2)^{-2} (X_1^{\top}X_1)^2\right], $$
	we can get from Lemma \ref{prop_model_shift_tight} (or rather the proof of Lemma \ref{prop_model_shift_tight}) that
\be\label{Deltabeta} 
\al_-^2(\rho_1) - \oo(1)  \le \frac{\delta_\beta}{ \Delta_{\beta}} \le \al_+^2(\rho_1) +  \oo(1) , \ee
	where 
	$$\Delta_{\beta}:=pd^2 \cdot \frac{\rho_1^2}{(\rho_1+\rho_2)^2} \cdot \frac{1 + a_3 + a_4}{(a_1 + a_2)^2}= pd^2 \cdot \frac{\rho_1^2 (\rho_1+\rho_2)}{(\rho_1 + \rho_2 - 1)^3}.$$	
%\begin{align*}
%		\Delta_{\beta} &\define \hat{v}^2 \bignorm{(\hat{v}^2 X_1^{\top}X_1 + X_2^{\top}X_2)^{-1} X_1^{\top}X_1 (\beta_s - \hat{v}\beta_t)}^2 \\
%		&= d^2 \cdot \bignormFro{(\hat{v}^2 X_1^{\top}X_1 + X_2^{\top}X_2)^{-1}X_1^{\top}X_1}^2 + \bigo{p^{-1/2 + \varepsilon}d^2}
%	\end{align*}
%	Using Lemma \ref{prop_model_shift_tight}, we get an upper and lower bound on $\Delta_{\beta}$ as
%	\be\label{Deltabeta} \bigbrace{1 - \sqrt{\frac{1}{c_1}}}^4 \le \Delta_{\beta} / \bigbrace{p\cdot d^2 \cdot \frac{c_1^2}{(c_1+c_2)^2} \cdot \frac{1 + a_3 + a_4}{(a_1 + a_2)^2} + \bigo{p^{1/2+\e}d^2} } \le \bigbrace{1 + \sqrt{\frac{1}{c_1}}}^4. \ee	
%	Hence we obtain that
%	\begin{align*}
%		\frac{1 + a_3 + a_4}{(a_1 + a_2)^2}
%		= \frac{(c_1 + c_2)^3}{(c_1 + c_2 - 1)^3} + \bigo{p^{-1/2+\varepsilon}}.
%	\end{align*}
With \eqref{Deltavar} and \eqref{Deltabeta}, we conclude that if
\be\label{upper101}pd^2 \cdot \frac{\rho_1^2 (\rho_1+\rho_2)}{(\rho_1 + \rho_2 - 1)^3} \cdot \left( \al_+^2(\rho_1) +  \oo(1) \right) < \sigma^2  \cdot \frac{\rho_1}{(\rho_2-1)(\rho_1 + \rho_2 -1)},\ee
	we have that $\delta_{\vari} > \delta_{\beta}$, which implies that $\te(\hat{\beta}_t^{\MTL})$ is lower than $\te(\hat{\beta}_t^{\STL})$. We can simplify \eqref{upper101} to
	\[   d^2  < \frac{\sigma^2}{p}\frac{(\rho_1 + \rho_2 - 1)^2}{\rho_1(\rho_1 + \rho_2) (\rho_2 - 1)} \cdot \left(\bigbrace{1 + \sqrt{\frac 1 {\rho_1}}}^{-4} - \oo(1)\right), \]
Plugging into $\rho_1>40$, we obtain the first statement of Proposition \ref{prop_dist_transition}.	
	
	
On the other hand, if
\be\label{upper102}pd^2 \cdot \frac{\rho_1^2 (\rho_1+\rho_2)}{(\rho_1 + \rho_2 - 1)^3} \cdot \left( \al_-^2(\rho_1) -  \oo(1) \right) > \sigma^2  \cdot \frac{\rho_1}{(\rho_2-1)(\rho_1 + \rho_2 -1)},\ee
	we have that $\delta_{\vari} < \delta_{\beta}$, which implies that $\te(\hat{\beta}_t^{\MTL})$ is larger than $\te(\hat{\beta}_t^{\STL})$. We can simplify \eqref{upper102} to
	\[   d^2  > \frac{\sigma^2}{p}\frac{(\rho_1 + \rho_2 - 1)^2}{\rho_1(\rho_1 + \rho_2) (\rho_2 - 1)} \cdot \left(\bigbrace{1 - \sqrt{\frac 1 {\rho_1}}}^{-4} +\oo(1)\right), \]
Plugging into $\rho_1>40$, we obtain the second statement of Proposition \ref{prop_dist_transition}.
\end{proof}

Then we prove Proposition \ref{prop_data_size}, which describes the effect of source task data size on the information transfer. 

\begin{proof}[Proof of Proposition \ref{prop_data_size}]
Recall that under the setting of Example \ref{ex_basic}, \eqref{Deltavar} and \eqref{Deltabeta} hold.  Then we get that $\te(\hat{\beta}_t^{\MTL})< \te(\hat{\beta}_t^{\STL})$ whp if
\be\label{pos1}\left(pd^2 \frac{c_1(c_1+c_2)(c_2-1)}{(c_1+c_2-1)^2} + \bigo{p^{1/2+\e}d^2} \right)\bigbrace{1 + \sqrt{\frac{1}{c_1}}}^4 <  \sigma^2 \bigbrace{1 + \bigo{p^{-1/2 + \varepsilon}}};\ee
otherwise if
\be\label{neg1}\left(pd^2 \frac{c_1(c_1+c_2)(c_2-1)}{(c_1+c_2-1)^2} + \bigo{p^{1/2+\e}d^2} \right)\bigbrace{1 - \sqrt{\frac{1}{c_1}}}^4 >  \sigma^2 \bigbrace{1 + \bigo{p^{-1/2 + \varepsilon}}},\ee
then we have $\te(\hat{\beta}_t^{\MTL})> \te(\hat{\beta}_t^{\STL})$ whp.

Now we prove the first statement of Proposition \ref{prop_data_size}. Notice that the function
$$ \frac{c_1(c_1+c_2)(c_2-1)}{(c_1+c_2-1)^2} =(c_2-1) \left(1 +\frac{c_2-2}{c_1}+\frac{1}{c_1(c_1+c_2)}\right)^{-1} $$
is strictly increasing with respect to $c_1$ as long as $c_2\ge 3$. In particular, by taking $c_1\to \infty$, we get the bound:
$$\frac{c_1(c_1+c_2)(c_2-1)}{(c_1+c_2-1)^2} \cdot \bigbrace{1 + \sqrt{\frac{1}{c_1}}}^4 <  (c_2-1) \bigbrace{1 + \sqrt{\frac{1}{c_1}}}^4.$$
 Hence we conclude that $\te(\hat{\beta}_t^{\MTL}) < \te(\hat{\beta}_t^{\STL})$ whp as long as 
 $$pd^2 +\oo(1) \le \bigbrace{1 + \sqrt{\frac{1}{c_1}}}^{-4}\frac{\sigma^2}{c_2-1} .$$ 
 This gives the first statement by taking $c_1 > a$. 

The second statement can be proved in a similar way. Suppose $c_1 > a$ and 
$pd^2 > (1 - a^{-1/2})^{-4}\frac{\sigma^2}{c_2-1}$. If $c_1 > \frac{(c_2-2) \sigma^2}{(1-a^{-1/2})^4(1-(a+c_2-2)^{-2})(c_2 - 1) pd^2 - \sigma^2}$, we have
\begin{align*}
 & pd^2 \cdot \frac{c_1(c_1+c_2)(c_2-1)}{(c_1+c_2-1)^2}  \cdot \bigbrace{1 - \sqrt{\frac{1}{c_1}}}^4  \\
 &>  \bigbrace{1 - \sqrt{\frac{1}{c_1}}}^4 \left( 1-\frac{1}{(c_1+c_2-2)^2}\right)\cdot \frac{ pd^2(c_2-1) c_1 }{c_1+(c_2-2)}  >\sigma^2\cdot (1+\oo(1)) , 
 \end{align*}
 where in the first step we used that 
 $$ \frac{(c_1+c_2)(c_1+c_2-2)}{(c_1+c_2-1)^2}   > 1-\frac{1}{(c_1+c_2-2)^2}.$$
This shows that \eqref{neg1} holds as long as $p$ is large enough, and hence $\te(\hat{\beta}_t^{\MTL})> \te(\hat{\beta}_t^{\STL})$ holds. On the other hand, if $c_1 < \frac{(c_2-2)\sigma^2}{(1+a^{-1/2})^4(c_2 - 1) pd^2 - \sigma^2}$, then we have 
 \begin{align*}
 pd^2 \cdot \frac{c_1(c_1+c_2)(c_2-1)}{(c_1+c_2-1)^2}  \cdot \bigbrace{1 + \sqrt{\frac{1}{c_1}}}^4 < \bigbrace{1 + \sqrt{\frac{1}{c_1}}}^4 \cdot \frac{pd^2(c_2-1) c_1 }{c_1+(c_2-2)}  < \sigma^2\cdot (1-\oo(1)).
 \end{align*}
 This shows that \eqref{pos1} holds as long as $p$ is large enough, and hence $\te(\hat{\beta}_t^{\MTL})< \te(\hat{\beta}_t^{\STL})$ holds.
 \end{proof}
 

\begin{proof}[Proof of Proposition \ref{prop_covariate}]
Let 
$$M_0:=\argmin_{M\in \cal S_{\mu}}(\te(\hat{\beta}_t^{\MTL}))(M).$$ 
We now calculate $(\te(\hat{\beta}_t^{\MTL}))(M_0)$. In this case, as in Lemma \ref{lem_hat_v} we also have that 
\be\label{hatvM}|\hat v-1|=\OO(p^{-1}) \ee 
in the setting Proposition \ref{prop_covariate}. In fact, Lemma \ref{lem_hat_v} was proved assuming that $M=\id$, but its proof can be easily extended to the case with general $M\in \cal S_{\mu}$ by using that $\lambda(M)\in [\mu_{\min},\mu_{\max}]$. We omit the details here. 

Now using \eqref{hatvM}, Lemma \ref{lem_cov_shift} and Lemma \ref{prop_model_shift_tight}, we get that whp,
\begin{align*}
(\te(\hat{\beta}_t^{\MTL}))(M_0)= \frac{\sigma^2}{c_1+c_2}\cdot \frac1p\tr\left( \frac{1}{a_1M_0^\top M_0 + a_2}\right)  + \Delta_\beta(M_0),
\end{align*}
where  $ \Delta_{\beta}(M_0)$ satisfies 
\be \nonumber
\begin{split}
&\left| \Delta_{\beta}(M_0) - \frac{d^2\cdot c_1^2}{(c_1 + c_2)^2} \tr\left[{M_0} \frac{(1 + a_3)\id + a_4 {M_0}^{\top}{M_0}}{(a_2 + a_1 {M_0}^{\top}{M_0})^2} {M_0}^{\top}\right]\right| \\
&\le \left(\bigbrace{1 + \sqrt{\frac{1}{c_1}}}^4-1\right)\left( \frac{c_1\mu_{\max}}{(\sqrt{c_1}-1)^2 \mu_{\min} + (\sqrt{c_2}-1)^2}\right)^2 \cdot d^2\tr\left(\Sigma_1\right). 
\end{split}
\ee
From equation \eqref{eq_a2}, we get
$$a_1\ge \frac{c_1-1}{c_1+c_2},\quad a_2\le \frac{c_2}{c_1+c_2}.$$
Then solving equations \eqref{eq_a3} and \eqref{eq_a4}, we get that
\begin{align*}
0\le a_3&= \frac{ \frac{1}{n_2}\sum_{i=1}^p \frac{a_2^2 }{ (  a_2 + \lambda_i^2a_1)^2  }\left(1 - \frac{1}{n_1} \sum_{i=1}^p \frac{\lambda_i^4 a_1^2  }{  (a_2 + \lambda_i^2a_1)^2  }\right) +\left( \frac{1}{n_1}\sum_{i=1}^p \frac{\lambda_i^2 a_1^2}{  (a_2 + \lambda_i^2a_1)^2  }\right) \left(\frac{1}{n_2}\sum_{i=1}^p \frac{  \lambda_i^2 a_2^2 }{ (  a_2 + \lambda_i^2a_1)^2  }\right) } {\left(1- \frac1{n_2}\sum_{i=1}^p \frac{a_2^2}{ (a_2 + \lambda_i^2a_1)^2  }\right) \left(1 - \frac{1}{n_1} \sum_{i=1}^p \frac{\lambda_i^4 a_1^2  }{  (a_2 + \lambda_i^2a_1)^2  }\right) - \left(\frac{1}{n_2}\sum_{i=1}^p \frac{  \lambda_i^2 a_2^2 }{ (  a_2 + \lambda_i^2a_1)^2  }\right) \left(\frac{1}{n_1}\sum_{i=1}^p \frac{\lambda_i^2 a_1^2 }{  (a_2 + \lambda_i^2a_1)^2  }\right)} \\
&\le \frac{c_2^{-1}}{1-c_1^{-1}-c_2^{-1}}
\end{align*}
and
\begin{align*}
0\le a_4&= \frac{\frac{1}{n_1}\sum_{i=1}^p \frac{\lambda_i^2 a_1^2}{  (a_2 + \lambda_i^2a_1)^2  }}{\left(1- \frac1{n_2}\sum_{i=1}^p \frac{a_2^2}{ (a_2 + \lambda_i^2a_1)^2  }\right) \left(1 - \frac{1}{n_1} \sum_{i=1}^p \frac{\lambda_i^4 a_1^2  }{  (a_2 + \lambda_i^2a_1)^2  }\right) - \left(\frac{1}{n_2}\sum_{i=1}^p \frac{  \lambda_i^2 a_2^2 }{ (  a_2 + \lambda_i^2a_1)^2  }\right) \left(\frac{1}{n_1}\sum_{i=1}^p \frac{\lambda_i^2 a_1^2 }{  (a_2 + \lambda_i^2a_1)^2  }\right)} \\
&\le  \frac{c_1^{-1}\cdot \mu_{\min}^{-2}}{1-c_1^{-1}-c_2^{-1}},
\end{align*}
where we also used that 
$$\left(\sum_{i=1}^p \frac{\lambda_i^2  }{  (a_2 + \lambda_i^2a_1)^2  }\right)^2 \le \sum_{i=1}^p \frac{\lambda_i^4 }{  (a_2 + \lambda_i^2a_1)^2  }\cdot \sum_{i=1}^p \frac{1}{  (a_2 + \lambda_i^2a_1)^2  }$$
by Cauchy-Schwarz inequality.

Combining the above estimates, we get that 
\begin{align}\label{approximateteM}
(\te(\hat{\beta}_t^{\MTL}))(M_0) =  \te(M_0) + \cal E,
\end{align}
where
$$\te(M_0):=\frac{\sigma^2}{c_1+c_2}\cdot \frac1p\tr\left( \frac{1}{a_1M_0^\top M_0}\right)  +\frac{d^2\cdot c_1^2}{(c_1 + c_2)^2} \tr\left(\frac{1 + a_3}{a_1^2 M_0^{\top}{M_0}} \right),$$
and the error satisfies
$$|\cal E|\le C\left( \frac{c_2}{c_1} +c_1^{-1/2}\right)\te(M_0),$$
Here the constant $C>0$ depends only on $\mu_{\max}$, $\mu_{\min}$ and $\|\Sigma_1\|$, but otherwise does not depend on $c_1$ and $c_2$. 

%With these two bounds, we can easily conclude \eqref{approxteM}. 
%
%We have that the test error satisfies
%\be\label{approxteM}  te(M)\left(1 -  \frac{n_2}{n_1-p} \frac{1}{\lambda_p^2 + \frac{n_2}{n_1-p}}\right)  \le  \frac{\sigma^2}{n_1+n_2}\tr\left( \frac{1}{a_1M^\top M + a_2}\right) \le te(M),\ee
%where $\lambda_p$ is the smallest singular value of $p$ and
%$$te(M):= \frac{\sigma^2}{a_1(n_1+n_2)}\tr\left( \frac{1}{M^\top M}\right) .$$
%Moreover, for all $M$ satisfying \eqref{GMcons}, the minimum of $te(M)$ is attained when $M= a\id$.


Finally using AM-GM inequality, we observe that 
$$\tr\left( \frac{1}{M^\top M}\right) = \sum_{i=1}^p\frac{1}{\lambda_i}$$
is minimized when $\lambda_1 = \cdots\lambda_p=\mu$ under the restriction $\prod_{i=1}^p\lambda_i\le \mu^p$. Hence we get that 
$$ \te (M_0)\le \te(\mu \id).$$
Together with \eqref{approximateteM}, we conclude \eqref{formular_covariate0}.
%, we conclude that the sum $\sum_{i=1}^p\lambda_i^{-1}$ is smallest when $\lambda_1=\cdots=\lambda_p = a$.
\end{proof}

\begin{proof}[Theoretical justification of Example \ref{ex_covariate}]
In the setting of Example \ref{ex_covariate}, equations in \eqref{eq_a2} become
\be\label{compleeq} a_1 + a_2 = 1 - \frac{p}{n_1 + n_2},  \ \ a_1 + \frac{p}{2(n_1 + n_2)}\cdot \bigbrace{\frac{a_1}{a_1 + \lambda^2 a_2} + \frac{a_1}{a_1 + \frac{a_2}{\lambda^2}}} = \frac{n_1}{n_1 + n_2}. \ee
It's not hard to verify that there is only one valid solution $(a_1,a_2)$ to \eqref{compleeq}. After solving these, we get the test error for the target task as follows.
\be\label{testcomple}
te(\lambda)=\frac{p}{2(n_1 + n_2)} \cdot \bigbrace{\frac{1}{\frac{a_1}{\lambda^2} + a_2} + \frac{1}{a_1\lambda^2 + a_2}}.\ee

First we notice that the curves in Figure \ref{fig_te_complement} all cross at the point $n_1=n_2$. In fact, if $n_1=n_2$, then it is easy to observe that $a_1=a_2=(1-\gamma)/2$ is the solution to equation \eqref{compleeq}, where we denote $ \gamma=p/(n_1+n_2)$. Then for any $\lambda$, the test error in \eqref{testcomple} takes the value
$$te(\lambda)= \frac{\gamma}{2}\frac{1}{(1-\gamma)/2}=\frac{p}{n_1+n_2-p}.$$

%Second, from Figure \ref{fig_te_complement} we observe that the complementary cases with $\lambda>1$ is better than the case without covariate shift (i.e. $M=\id$ case) when $n_1<n_2$. On the other hand, if we have enough source task data such that $n_1>n_2$, then it is always better to have no covariate shift.
This phenomenon can be also explained using our theory. With \eqref{compleeq}, we can write
$$te(\lambda)=\frac{\gamma}{2} \cdot \bigbrace{\frac{1}{\frac{a_1}{\lambda^2} + (1-\gamma - a_1)} + \frac{1}{a_1\lambda^2 + (1-\gamma - a_1)}}.$$
We can compute that
\begin{align*}
te(\lambda) - te(1)&= \frac{\gamma}{2(1-\gamma)} (\lambda^2-1)a_1\cdot \bigbrace{  \frac{1}{ -a_1(\lambda^2-1)+(1-\gamma)\lambda^2 } - \frac{1}{a_1(\lambda^2-1) + (1-\gamma)}} \\
&= \frac{\gamma}{2(1-\gamma)} (\lambda^2-1)^2 a_1\cdot  \frac{2a_1 - (1-\gamma) }{[-a_1(\lambda^2-1)+(1-\gamma)\lambda^2 ][a_1(\lambda^2-1) + (1-\gamma)]} .
\end{align*}
If $n_1>n_2$, we have $a_1>(1-\gamma)/2$ (because $a_1>a_2$ as observed from the equation \eqref{compleeq}), and hence $te(\lambda)>te(1)$. Otherwise if $n_1< n_2$, we have $a_1< (1-\gamma)/2$, and hence $te(\lambda)< te(1)$.
\end{proof}


\iffalse
Note that for the case of $k$ tasks with the same covariates, since there is no covariate shift and the data ratio is always equal to one, the main factor is model distance.

\paragraph{A precise bound when there is no model shift.}
As Proposition \ref{prop_monotone} shows, if $\beta_s$ and $\beta_t$ are equal, then adding the source task dataset always helps learn the target task.
The goal of this section is to understand how covariate shift affects the rate of transfer. \todo{add conceptual msg}

%The key quantity is to look at:
%The estimator using the source and target together from minimizing \eqref{eq_mtl_basic} is
%\[ \hat{\beta}_{s,t} = (X_1^{\top} X_1 + X_2^{\top} X_2)^{-1} (X_1^{\top}Y_1 + X_2^{\top}Y_2)\]
%The estimation error of $\hat{\beta}_{s,t}$ is
%\begin{align}\label{eq_two_task}
%  \err(\hat{\beta}_{s,t}) = \sigma^2 \cdot \tr[(X_1^{\top}X_1 + X_2^{\top} X_2)^{-1}].
%\end{align}
%The estimation error using the target alone is
%\begin{align}\label{eq_target_task}
%	\err(\hat{\beta}_t) = \sigma^2 \cdot \tr[(X_2^{\top} X_2)^{-1}].
%\end{align}
%The improvement of estimation error from adding the source task is then given by
%$\err(\hat{\beta}_t) - \err(\hat{\beta}_{s,t})$.
%For the test error on the target task, the improvement from adding the source task is
%\[ \te(\hat{\beta}_t) - \te(\hat{\beta}_{s,t}) = \sigma^2\cdot\bigtr{\bigbrace{(X_2^{\top}X_2)^{-1} - (X_1^{\top}X_1 + X_2^{\top}X_2)^{-1}}\cdot\Sigma_2}. \]

%We calculate the amount of improvement by comparing equation \eqref{eq_two_task} to equation \eqref{eq_target_task}.
A simple observation here is that when $\beta_s = \beta_t$, the optimal $\hat{w}$ for minimizing equation \eqref{eq_te_mtl} is equal to $1$.
Based on this observation, we can get a more precise result than Theorem \ref{thm_model_shift} on the improvement of adding the source task data that only depends on the covariance matrices $\Sigma_1, \Sigma_2$ and the number of data points $n_1, n_2$.



\begin{proposition}[Transfer rate without model shift]\label{thm_cov_shift}
Suppose $\beta_s = \beta_t$ and $\|\beta_t\|_2^2\sim p\sigma^2$ (i.e. the $l^2$-norm of the vector $\beta_t$ is of the same order as that of the error vector). Assume that the condition numbers of $\Sigma_1$, $\Sigma_2$ and $M:=\Sigma_1^{1/2}\Sigma_2^{-1/2}$ are all bounded by a constant $C>0$. Then we have that the optimal ratio for $W_1/W_2$ in equation \eqref{eq_te_mtl} satisfies
	$$1\le \hat{w} \le 1+\OO(p^{-1}).$$%is $\hat{w} = 1 \pm \bigo{\frac 1 {n_1+n_2}}$ \todo{(figure out the constants)}
	%where $M:=\Sigma_1^{1/2}\Sigma_2^{-1/2}$.
Moreover, we have
	\begin{align}\label{tehatw1}
		%\err(\hat{\beta}^{\TL}_{s,t}) &= \sigma^2 \cdot \bigtr{\frac 1 {(n_1 + n_2)a_1\Sigma_1 + (n_1 + n_2)a_2\Sigma_2}} \\
		\te(\hat{\beta}^{\TL}_{t}) &= \sigma^2 \cdot \bigtr{\bigbrace{(n_1 + n_2)a_1 M^\top M  + (n_1 + n_2)a_2\id}^{-1}} \cdot \left(1+ \bigo{p^{-1}}\right),
	\end{align}
where $a_1, a_2$ are the solutions to equations \eqref{eq_a2}. %\cor $w_0$ is close to 1 if the signal strength $\beta_t$ is much larger than the noise strength \nc
\end{proposition}


\begin{proof}
We abbreviate $\val(w_2\hat{B}(w)):=\val(w)$. Note that $\val(w)\le \val(-w)$ for $w\ge 0$. Hence we have $\hat w\ge 0$. Moreover, we notice that $\val (w) < \val (1)$ for all $0\le w < 1$. Thus we have $\hat w\ge 1$. It suffices to consider the case with $w> 1$. Under the assumption on $\beta_s$ and $\beta_t$, we can write
\begin{align}
	\val(w) =&~  \left( 1-\frac1w\right)^2 \left\|( M^\top Z_1^{\top}Z_1 M + w^{-2}Z_2^{\top}Z_2)^{-1} M^\top Z_1^{\top} Z_1 \Sigma_1^{1/2}\beta_t \right\|^2 \nonumber \\
			&~ + \frac{\sigma^2}{w^2} \cdot \bigtr{( M^\top Z_1^{\top}Z_1 M + w^{-2}Z_2^{\top}Z_2)^{-1} }. \nonumber
\end{align}
Since
\begin{align*}
&\left\|( M^\top Z_1^{\top}Z_1 M + w^{-2}Z_2^{\top}Z_2)^{-1} M^\top Z_1^{\top} Z_1 \Sigma_1^{1/2}\beta_t \right\|^2 \\
&= \tr \left[ ( M^\top Z_1^{\top}Z_1 M + w^{-2}Z_2^{\top}Z_2)^{-2}M^\top Z_1^{\top} Z_1 \Sigma_1^{1/2}\beta_t\beta_t^\top \Sigma_1^{1/2}Z_1^{\top} Z_1  M\right]
\end{align*}
is increasing with respect to $w$, then the derivative of $\val(w)$ can be bounded from below as
\begin{align*}
\val'(w) \ge &~ 2\frac{w-1}{w^3} \left\|( M^\top Z_1^{\top}Z_1 M + w^{-2}Z_2^{\top}Z_2)^{-1} M^\top Z_1^{\top} Z_1 \Sigma_1^{1/2}\beta_t \right\|^2   \\
			&~ - 2 \frac{\sigma^2}{w^3} \cdot \bigtr{(M^\top Z_1^{\top}Z_1 M +w^{-2}  Z_2^{\top}Z_2)^{-1} M^\top Z_1^{\top}Z_1 M (M^\top Z_1^{\top}Z_1 M + w^{-2} Z_2^{\top}Z_2)^{-1}} \\
\ge &~ 2\frac{w-1}{w^3} \left\|( M^\top Z_1^{\top}Z_1 M +  Z_2^{\top}Z_2)^{-1} M^\top Z_1^{\top} Z_1 \Sigma_1^{1/2}\beta_t \right\|^2   - 2 \frac{\sigma^2}{w^3} \cdot \bigtr{(M^\top Z_1^{\top}Z_1 M)^{-1}}.
			%\\ =& ~ 2\frac{d^2}{w^3} \tr\left[( M^\top Z_1^{\top}Z_1 M + w^{-2}Z_2^{\top}Z_2)^{-1} M^\top Z_1^{\top} \left[\left( w-1\right)\left(Z_1 \Sigma_1 Z_1^{\top}\right) - \frac{\sigma^2}{d^2}\id \right] Z_1 M ( M^\top Z_1^{\top}Z_1 M + w^{-2}Z_2^{\top}Z_2)^{-1}\right]  .
\end{align*}
Hence $\val'(w)\ge 0$ if $w>1+ \e_0$, where
$$\e_0:= \frac{\sigma^2  \bigtr{(M^\top Z_1^{\top}Z_1 M)^{-1}} }{\left\|( M^\top Z_1^{\top}Z_1 M + Z_2^{\top}Z_2)^{-1} M^\top Z_1^{\top} Z_1 \Sigma_1^{1/2}\beta_t \right\|^2}.$$
In other words, $\val(w)$ is strictly increasing function on $[1+\e_0,\infty]$. Thus we get that $\hat w$ satisfies
\be\label{hatw1}1\le w \le 1+\e_0.\ee Using \eqref{eq_isometric}, we get that
$$\e_0=\OO(\sigma^2/\|\beta_t\|_2^2)=\OO(p^{-1}).$$

Finally, plugging \eqref{hatw1} into the expression $\te(\hat{\beta}^{\TL}_{t}) $, we obtain \eqref{tehatw1}.
%$$1\le \hat{w} \le w_0:=1 +\frac{\sigma^2  \bigtr{(M^\top Z_1^{\top}Z_1 M)^{-1}} }{\left\|( M^\top Z_1^{\top}Z_1 M + Z_2^{\top}Z_2)^{-1} M^\top Z_1^{\top} Z_1 \Sigma_1^{1/2}\beta_t \right\|^2},$$
\end{proof}



As a remark, we see that Proposition \ref{prop_monotone} follows from Theorem \ref{thm_cov_shift}.
The amount of reduction on test error for the target task is given as
	\begin{align*}
%		\err(\hat{\beta}_t) - \err(\hat{\beta}_{s,t})
%		&= \sigma^2 p \cdot \bigtr{\frac 1 {(n_2 - p) \Sigma_2} - \frac 1 {(n_1 + n_2)a_1 \Sigma_1 + (n_1 + n_2)a_2 \Sigma_2}}, \\
		\te(\hat{\beta}_t) - \te(\hat{\beta}_{s,t})
		&= \sigma^2 \cdot \bigbrace{\frac p {n_2 - p} -  \bigtr{\bigbrace{(n_1 + n_2)a_1\Sigma_2^{-1/2}\Sigma_1\Sigma_2^{-1/2} + (n_1 + n_2)a_2\id}^{-1}}}.
	\end{align*}
Because
\begin{align*}
	\te(\hat{\beta}_{s,t}) \le \te(\hat{\beta}_t)
	\Leftarrow~ & (n_2 - p)\Sigma_2 \preceq (n_1 + n_2) a_1 \Sigma_1 + (n_1 + n_2)a_2 \Sigma_2 \\
	\Leftrightarrow~ & \zeroMatrix \preceq (n_1 + n_2) a_1 \Sigma_1 + (n_1 - (n_1 + n_2)\cdot a_1) \Sigma_2,
\end{align*}
which is true since $a_1 \le n_1 / (n_1 + n_2)$ by equation \eqref{eq_a2}.
The proof for $\te(\hat{\beta}_{s,t}) \le \te(\hat{\beta}_t)$ follows by multiplying $\Sigma_2^{-1/2}$ on both sides of the inequalities above.

\medskip
\fi

\subsection{Proofs for Section \ref{sec_benefit}}\label{app_proof_data}

With similar techniques, we then show Proposition \ref{prop_var_transition}, which gives a transition threshold with respect to the difference between the noise levels of the two tasks. 

\begin{proof}[Proof of Proposition \ref{prop_var_transition}]
In the setting of Proposition \ref{prop_var_transition}, the validation loss and the test error become
\begin{align*}
		\val(\hat{B}; w_1, w_2)
	&=  n_1 \cdot \bignorm{\Sigma_1^{1/2}\left(\frac{w_1^2}{w_2^2} X_1^{\top}X_1 + X_2^{\top}X_2\right)^{-1}X_2^{\top}X_2\left (\beta_s - \frac{w_1}{w_2}\beta_t\right)}^2 \nonumber \\
		&+ n_1 \sigma^2 \cdot \frac{w_1^2}{w_2^2} \bigtr{\left(\frac{w_1^2}{w_2^2}  X_1^{\top}X_1 + X_2^{\top}X_2\right)^{-2} \left(\sigma_1^2 \frac{w_1^2}{w_2^2}  X_1^{\top}X_1 + \sigma_2^2  X_2^{\top}X_2\right)} \nonumber \\
		&+ n_2 \cdot \frac{w_1^2}{w_2^2}\bignorm{\Sigma_2^{1/2}\left(\frac{w_1^2}{w_2^2} X_1^{\top}X_1 + X_2^{\top}X_2\right)^{-1}X_1^{\top}X_1\left(\beta_s - \frac{w_1}{w_2}\beta_t\right)}^2 \nonumber \\
		&+ n_2 \sigma^2 \cdot \bigtr{\left(\frac{w_1^2}{w_2^2} X_1^{\top}X_1 + X_2^{\top}X_2\right)^{-2} \left(\sigma_1^2 \frac{w_1^2}{w_2^2}  X_1^{\top}X_1 + \sigma_2^2  X_2^{\top}X_2\right)}, 
\end{align*}
and 
\begin{align*}
	\te(\hat{\beta}_t^{\MTL}) &=~ \hat{v}^2 \bignorm{ (\hat{v}^2 X_1^{\top}X_1 + X_2^{\top}X_2)^{-1} X_1^{\top}X_1 (\beta_s - \hat{v} \beta_t)}^2 \nonumber \\
			&~ + \sigma_2^2 \cdot \bigtr{(\hat{v}^2 X_1^{\top}X_1 + X_2^{\top}X_2)^{-1} } + (\sigma_1^2 -\sigma_2^2) \cdot \bigtr{(\hat{v}^2 X_1^{\top}X_1 + X_2^{\top}X_2)^{-2} \hat v^2 X_1^\top X_1},
\end{align*}
where $\hat v=\hat{w_1}/\hat{w_2}$ is the global minimizer of $\val(\hat{B}; w_1, w_2)$. Again using concentration of random vector with i.i.d. entries, Lemma \ref{largedeviation}, we can rewrite $\te(\hat{\beta}_t^{\MTL})$ as
\begin{align*}
	\te(\hat{\beta}_t^{\MTL}) &=~ \hat{v}^2 \left[d^2 +\left( w-1\right)^2\kappa^2\right]\bigtr{ (\hat{v}^2 X_1^{\top}X_1 + X_2^{\top}X_2)^{-2} (X_1^{\top}X_1)^2} \cdot \left(1+\OO(p^{-1/2+\e})\right)\nonumber \\
	& + \sigma_2^2 \cdot \bigtr{(\hat{v}^2 X_1^{\top}X_1 + X_2^{\top}X_2)^{-1} } + (\sigma_1^2 -\sigma_2^2)  \cdot \bigtr{(\hat{v}^2 X_1^{\top}X_1 + X_2^{\top}X_2)^{-2} \hat v^2X_1^\top X_1}
\end{align*}
with high probability for any constant $\e>0$. 



In the current setting, we can also show that  \eqref{hatv_add1}  holds for $\hat v$. 
Since the proof is almost the same as the one for Lemma \ref{lem_hat_v}, we omit the details. 
%$$|\hat w-1|=\OO(p^{-1}).$$
%as in \eqref{hatw_add1}. 
%We omit the details of the proof, since it is almost the same as the one in the proof of Lemma \ref{lem_hat_v}. 
Thus under the choice parameters in \eqref{choiceofpara}, $\te(\hat{\beta}_t^{\MTL}) $ can be simplified as in \eqref{simple1}: 
\be \label{simple2}
\begin{split}
\te(\hat{\beta}_t^{\MTL})&=(1+\OO(n^{-\e}))\cdot d^2 \tr\left[(X_1^{\top}X_1 +X_2^{\top}X_2)^{-2} (X_1^{\top}X_1)^2\right] \\ 
&+(1+\OO(n^{-\e}))\cdot \sigma_2^2  \bigtr{(X_1^{\top}X_1  + X_2^{\top}X_2)^{-1} }  \\
&+(1+\OO(n^{-\e}))\cdot (\sigma_1^2-\sigma_2^2)  \bigtr{(X_1^{\top}X_1  + X_2^{\top}X_2)^{-2} X_1^\top X_1} .
\end{split}
\ee
Then we write 
$$ \te(\hat{\beta}_t^{\STL})-\te(\hat{\beta}_t^{\MTL}) =\delta_{\vari} - \delta_\beta - \delta_{\vari}^{(2)},$$
where 
$$\delta_{\vari}:=\sigma_2^2  \bigtr{(X_2^{\top}X_2)^{-1} }  - (1+\OO(n^{-\e}))\cdot \sigma_2^2  \bigtr{(X_1^{\top}X_1  + X_2^{\top}X_2)^{-1} }$$
satisfies \eqref{Deltavar} but with $\sigma^2$ replaced with $\sigma_2^2$, 
$$\delta_\beta:=(1+\OO(n^{-\e}))\cdot d^2 \tr\left[(X_1^{\top}X_1 +X_2^{\top}X_2)^{-2} (X_1^{\top}X_1)^2\right]$$
satisfies \eqref{Deltabeta}, and 
\begin{align*}	
	\delta_{\vari}^{(2)}:=(1+\OO(n^{-\e}))\cdot (\sigma_1^2 -\sigma_2^2) \bigtr{(X_1^{\top}X_1 + X_2^{\top}X_2)^{-2} X_1^\top X_1} .
\end{align*}
%$$	\te(\hat{\beta}_t^{\MTL}) = \left[\Delta_\beta + \Delta_{diff}+  \sigma_2^2 \cdot \tr(X_1^{\top}X_1 + X_2^{\top}X_2)^{-1} \right]\cdot \left(1+\OO(p^{-1/2})\right), $$
%where $\Delta_\beta:= \sigma_2^2 \cdot \bigtr{(\hat{v}^2 X_1^{\top}X_1 + X_2^{\top}X_2)^{-1} }$ has 
To estimate this new term, we use the same arguments as in the proof of Lemma \ref{prop_model_shift_tight}: we first replace $X_1^\top X_1$ with $n_1\id$ up to some error using \eqref{eq_isometric}, and then apply Lemma \ref{lem_cov_derivative} to calcualte $\bigtr{(X_1^{\top}X_1 + X_2^{\top}X_2)^{-2}}$. This process leads to the following estimates on $\delta_{\vari}^{(2)}$:
\be\label{Deltavar2} 
\al_-(\rho_1) - \oo(1)  \le \frac{\delta_{\vari}^{(2)}}{ \Delta_{\vari}^{(2)}} \le \al_+(\rho_1) +  \oo(1) , \ee
where 
$$ \Delta_{\vari}^{(2)}:=(\sigma_1^2 -\sigma_2^2) \frac{\rho_1 (\rho_1+\rho_2)}{(\rho_1+\rho_2-1)^3}.$$
Next we compare $\delta_{\vari}$ with $\delta_\beta + \delta_{\vari}^{(2)}$. Our main goal is to see how the extra $\delta_{\vari}^{(2)}$ affects the information transfer in this case.

%Using \eqref{eq_isometric} and Lemma \ref{lem_cov_derivative}, a similar bound holds for $\Delta_{diff}$ as in \eqref{Deltabeta}:
%$$\bigbrace{1 - \sqrt{\frac{1}{c_1}}}^2\le \frac{\Delta_{diff}}{\sigma_1^2 -\sigma_2^2}\cdot \left( \frac{c_1 (c_1+c_2)}{(c_1+c_2-1)^3} + \OO(p^{-1/2+\e})\right)^{-1}\le \bigbrace{1 + \sqrt{\frac{1}{c_1}}}^2.$$

Note that the condition $d^2 < \frac {\sigma_2^2} {2p} \cdot \Phi(\rho_1, \rho_2)$ means the we have $\delta_{\vari}>\delta_{\beta} $ by Proposition \ref{prop_dist_transition}. Hence if $\sigma_1^2\le \sigma_2^2$, then $\delta_{\vari}^{(2)}<0$ and we always have $\delta_{\vari} > \delta_{\beta}+ \delta_{\vari}^{(2)}$, which gives $\te(\hat{\beta}_t^{\MTL})<\te(\hat{\beta}_t^{\STL})$. 

It remains to consider the case $\sigma_1^2 \ge \sigma_2^2$. 

\noindent{\bf Positive transfer.} By \eqref{Deltavar}, \eqref{Deltabeta} and \eqref{Deltavar2} above, if
\be\label{cond sigma1}
\begin{split} 
&\sigma_2^2  \cdot \frac{\rho_1}{(\rho_2-1)(\rho_1 + \rho_2 -1)} \cdot (1-\oo(1)) \\
&>pd^2 \cdot \frac{\rho_1^2 (\rho_1+\rho_2)}{(\rho_1 + \rho_2 - 1)^3}\bigbrace{1 + \sqrt{\frac{1}{\rho_1}}}^4 + (\sigma_1^2 -\sigma_2^2)\cdot \frac{\rho_1 (\rho_1+\rho_2)}{(\rho_1+\rho_2-1)^3} \bigbrace{1 + \sqrt{\frac{1}{\rho_1}}}^2 , 
\end{split}
\ee
then we have $\delta_{\vari} > \delta_{\beta} + \delta_{\vari}^{(2)}$ whp, which gives $\te(\hat{\beta}_t^{\MTL})<\te(\hat{\beta}_t^{\STL})$. We can solve \eqref{cond sigma1} to get
\begin{align*}
\sigma_1^2 < - pd^2 \cdot \rho_1\bigbrace{1 + \sqrt{\frac{1}{\rho_1}}}^2 \cdot (1-\oo(1))+\sigma_2^2 \left[ 1+ \rho_1\Phi(\rho_1, \rho_2)\bigbrace{1 + \sqrt{\frac{1}{\rho_1}}}^{-2}\right]\cdot (1-\oo(1)).
\end{align*}
Plugging into $\rho_1>50$, we obtain the first claim of Proposition \ref{prop_var_transition} for positive transfer.


\noindent{\bf Negative transfer.} On the other hand, if 
\be\label{cond sigma12}
\begin{split} 
&\sigma_2^2 \cdot \frac{\rho_1}{(\rho_2-1)(\rho_1 + \rho_2 -1)}\cdot \left(1 + \oo(1)\right) \\
&< pd^2 \cdot \frac{\rho_1^2 (\rho_1+\rho_2)}{(\rho_1 + \rho_2 - 1)^3}\bigbrace{1 - \sqrt{\frac{1}{\rho_1}}}^4 + (\sigma_1^2 -\sigma_2^2)\cdot \frac{\rho_1 (\rho_1+\rho_2)}{(\rho_1+\rho_2-1)^3} \bigbrace{1 - \sqrt{\frac{1}{\rho_1}}}^2 , 
\end{split}
\ee
then we have $\delta_{\vari} < \delta_{\beta} + \delta_{\vari}^{(2)}$ whp, which gives $\te(\hat{\beta}_t^{\MTL})>\te(\hat{\beta}_t^{\STL})$. We can solve \eqref{cond sigma12} to get
\begin{align*}
\sigma_1^2 > - pd^2 \cdot \rho_1\bigbrace{1 - \sqrt{\frac{1}{\rho_1}}}^2 \cdot (1+\oo(1))+\sigma_2^2 \left[ 1+ \rho_1\Phi(\rho_1, \rho_2)\bigbrace{1 - \sqrt{\frac{1}{\rho_1}}}^{-2}\right]\cdot (1+\oo(1)).
\end{align*}
Plugging into $\rho_1>50$, we obtain the second claim of Proposition \ref{prop_var_transition} for negative transfer.
\end{proof}

%\textbf{Remark.} Furthermore, as a function of $c_1$ over the range $[??, \infty]$, the maximum of $\te(\hat{\beta}_t^{\STL}) - \te(\hat{\beta}_t^{\MTL})$ is attained when $c_1 = {c_2\sigma^2}/{\max(2(c_2 - 1)pd^2 -\sigma^2, 0)}$. {\cor (cannot get this because we only have some bounds. If we let $c_1\to \infty$, then the curve of $\te(\hat{\beta}_t^{\STL}) - \te(\hat{\beta}_t^{\MTL})$ already becomes flat, and it is meaningless to discuss the minimum of this function at this point?)}

Then we prove Proposition \ref{prop_data_efficiency}, which gives precise upper and lower bounds on the data efficiency ratio.

\begin{proof}[Proof of Proposition \ref{prop_data_efficiency}]
Suppose we have reduced number of datapoints---$\al n_1$ for task 1 and $\al n_2$ for task 2 with $n_1=\rho_1 p$ and $n_2=\rho_2 p$. 
%For this proof, we introduce the notations 
%$$\al_{\pm}^{(1)}\equiv \al_{\pm}^{(1)}(\al) = \left ( 1- (\al\rho_1)^{-1/2}\right)^2,\quad \al_{\pm}^{(2)}\equiv \al_{\pm}^{(2)}(\al) = \left ( 1- (\al\rho_2)^{-1/2}\right)^2.$$
In the setting of Proposition \ref{prop_data_efficiency}, we still have \eqref{hatw_add1}. 
%and by symmetry it suffices to focus on one of tasks, say task 2. 
Then using Lemmas \ref{lem_cov_shift} and \ref{prop_model_shift_tight}, and recalling \eqref{simplesovlea12} and \eqref{simplesovlea34}, we get that whp,
\be\label{reduceproof1}
\begin{split}
\te_i(\hat \beta(\al)) =\sigma^2\left( \frac{1}{\al (\rho_1+\rho_2) - 1}+ \bigo{p^{-1/2 + \varepsilon}}\right)+ \Delta^{(i)}_{\beta}, \quad i=1,2,
\end{split}
\ee
%and
%\be\label{reduceproof1}
%\begin{split}
%\te_2(\hat \beta(\al)) =\sigma^2\left( \frac{1}{\al ( \rho_1+\rho_2) - 1}+ \Delta^{(2)}_{\beta}+ \bigo{p^{-1/2 + \varepsilon}}\right),
%\end{split}
%\ee
where $ \Delta^{(i)}_{\beta}$ satisfies 
\be \nonumber
\bigbrace{1 - \sqrt{\frac{1}{\al \rho_i}}}^4 \le \Delta^{(i)}_{\beta} /  \left[ pd^2 \cdot \frac{(\al\rho_i)^2\cdot \al (\rho_1+\rho_2)}{[\al ( \rho_1+\rho_2) - 1]^3}\cdot\left(1+ \bigo{p^{-1/2+\e}}\right)\right] \le \bigbrace{1 + \sqrt{\frac{1}{\al \rho_i}}}^4. 
\ee
On the other hand, using Lemma \ref{lem_minv}, we have whp
\be\label{reduceproof2} 
\te_i(\hat{\beta}_t^{\STL}) = \frac{\sigma^2}{\rho_i -1} \left( 1+ \bigo{p^{-1/2 + \varepsilon}}\right),\quad i=1,2.
\ee
Comparing \eqref{reduceproof1} and \eqref{reduceproof2}, we immediately obtain a trivial lower bound $\al^*\ge \al_l -\oo(1)$, where
$$\al_l:=\frac1{\rho_1+\rho_2}\left[\frac{2(\rho_1-1)(\rho_2-1)}{\rho_1+\rho_2 -2}+1\right] \ge \frac{\min(\rho_1,\rho_2)}{\rho_1+\rho_2}.$$
In fact, one can see that if $\al\le \al_l$, then we have 
$$ \frac{2\sigma^2}{\al (\rho_1+\rho_2) - 1} \ge \frac{\sigma^2}{\rho_1-1}+\frac{\sigma^2}{\rho_2-1},$$
that is $\te_1(\hat{\beta}(\alpha)) + \te_2(\hat{\beta}(\alpha))$ is larger than $\te_1(\hat{\beta}_t^{\STL}) + \te_2(\hat{\beta}_t^{\STL})$ even if we do not take into account the model shift bias terms $ \Delta^{(i)}_{\beta}$. Next we try to get more precise bounds on $\al^*$. In the following discussions, we only consider $\al$ such that $\al\rho \ge \al_l \rho \ge \min(\rho_1,\rho_2)$.

%Comparing \eqref{reduceproof1} and \eqref{reduceproof2}, we observe that $\te_2(\hat \beta(\al))  \ge \te_2(\hat{\beta}_t^{\STL}) $ for $\al \le 1/2-\oo(1)$, which gives $\al^* \ge 1/2 - \oo(1)$.

\noindent{\bf The upper bound.} From \eqref{reduceproof1} and \eqref{reduceproof2}, we see that $\al^*\le \al$ if $\al$ satisfies
\be\label{solval1}
\begin{split}
&(1+\oo(1)) \cdot \sum_{i=1}^2 pd^2 \frac{(\al\rho_i)^2\cdot \al (\rho_1+\rho_2)}{[\al ( \rho_1+\rho_2) - 1]^3}  \bigbrace{1 + \sqrt{\frac{1}{\al \rho_i}}}^4 \\
&\le \frac{\sigma^2}{\rho_1 -1}+\frac{\sigma^2}{\rho_2 -1} - \frac{2\sigma^2}{\al (\rho_1+\rho_2) - 1} . 
\end{split}
\ee
For the sum on the left-hand side, we can rewrite it as
\begin{align*}
 pd^2 \frac{1}{[1 - (\al \rho)^{-1}]^3} \sum_{i=1}^2 \bigbrace{\sqrt{\frac{\rho_i}{\rho_1+\rho_2}} + \sqrt{\frac{1}{\al \rho}}}^4,
\end{align*}
where $\rho:=\rho_1+\rho_2$. 

In order to solve \eqref{solval1}, we now consider the case $\min(\rho_1,\rho_2)\ge 200$. With some basic calculations, one can show that in this case
$$ \frac{1}{[1 - (\al \rho)^{-1}]^3} \sum_{i=1}^2 \bigbrace{\sqrt{\frac{\rho_i}{\rho_1+\rho_2}} + \sqrt{\frac{1}{\al \rho}}}^4 <  \frac{\rho_1^2 + \rho_2^2}{(\rho_1+\rho_2)^2} + 0.32.$$
Thus the following inequality implies \eqref{solval1}:
\be\label{solval1add}
\begin{split}
&\left(  \frac{\rho_1^2 + \rho_2^2}{(\rho_1+\rho_2)^2} + 0.32\right) pd^2 < \frac{\sigma^2}{\rho_1 -1}+\frac{\sigma^2}{\rho_2 -1} - \frac{2\sigma^2}{\al (\rho_1+\rho_2) - 1} .
\end{split}
\ee
In particular, if 
\be\nonumber
\begin{split}
&\left(  \frac{\rho_1^2 + \rho_2^2}{(\rho_1+\rho_2)^2} + 0.32\right) pd^2 < \frac{\sigma^2}{\rho_1 -1}+\frac{\sigma^2}{\rho_2 -1} - \frac{2\sigma^2}{(\rho_1+\rho_2) - 1} ,
\end{split}
\ee
that is, we have positive transfer when using all the data, then we can solve from \eqref{solval1add} the following upper bound on $\al^*$:
\begin{align*}
\al^* &<  \frac1{\rho_1+\rho_2}\left[\frac{2 }{\frac{1}{\rho_1-1}+\frac1{\rho_2-1}  -  \left(  \frac{\rho_1^2 + \rho_2^2}{(\rho_1+\rho_2)^2} + 0.32\right)\frac{pd^2}{\sigma^2}}+ 1\right] \\
& < \frac1{\rho_1+\rho_2}\left[\frac{2 }{\frac{1}{\rho_1}+\frac1{\rho_2}  - \left(  \frac{\rho_1^2 + \rho_2^2}{(\rho_1+\rho_2)^2} + \frac13\right)\frac{pd^2}{\sigma^2}}+ 1\right] .
\end{align*}


\noindent{\bf The lower bound.} From \eqref{reduceproof1} and \eqref{reduceproof2}, we see that $\al^*\ge \al$ if $\al$ satisfies
\be\label{solval2}
\begin{split}
&(1-\oo(1)) \cdot \sum_{i=1}^2 pd^2 \frac{(\al\rho_i)^2\cdot \al (\rho_1+\rho_2)}{[\al ( \rho_1+\rho_2) - 1]^3}  \bigbrace{1 - \sqrt{\frac{1}{\al \rho_i}}}^4 \\
&\ge \frac{\sigma^2}{\rho_1 -1}+\frac{\sigma^2}{\rho_2 -1} - \frac{2\sigma^2}{\al (\rho_1+\rho_2) - 1} . 
\end{split}
\ee
We then follow similar arguments as the above proof for the upper bound.

In order to solve \eqref{solval2}, we consider the case $\min(\rho_1,\rho_2)\ge 200$. With some basic calculations, one can show that the sum on the left-hand side of \eqref{solval2} satisfies 
$$ \frac{1}{[1 - (\al \rho)^{-1}]^3} \sum_{i=1}^2 \bigbrace{\sqrt{\frac{\rho_i}{\rho_1+\rho_2}} - \sqrt{\frac{1}{\al \rho}}}^4 >  \frac{\rho_1^2 + \rho_2^2}{(\rho_1+\rho_2)^2} -0.26 .$$
Thus the following inequality implies \eqref{solval2}:
\be\label{solval2add}
\begin{split}
&\left( \frac{\rho_1^2 + \rho_2^2}{(\rho_1+\rho_2)^2} -0.26\right) pd^2 > \frac{\sigma^2}{\rho_1 -1}+\frac{\sigma^2}{\rho_2 -1} - \frac{2\sigma^2}{\al (\rho_1+\rho_2) - 1} .
\end{split}
\ee
There are two cases: if 
\be\nonumber
\begin{split}
&\left( \frac{\rho_1^2 + \rho_2^2}{(\rho_1+\rho_2)^2} -0.26\right) pd^2 \ge \frac{\sigma^2}{\rho_1 -1}+\frac{\sigma^2}{\rho_2 -1},
\end{split}
\ee
then we always have negative transfer for all choice of $0\le \al \le 1$; otherwise, we can solve from \eqref{solval2add} the following lower bound on $\al^*$:
\begin{align*}
\al^* &>  \frac1{\rho_1+\rho_2}\left[\frac{2 }{\frac{1}{\rho_1-1}+\frac1{\rho_2-1}  - \left( \frac{\rho_1^2 + \rho_2^2}{(\rho_1+\rho_2)^2} -0.26\right) \frac{pd^2}{\sigma^2}}+ 1\right] \\
& >\frac1{\rho_1+\rho_2}\left[\frac{2 }{\frac{1}{\rho_1}+\frac1{\rho_2}  - \left( \frac{\rho_1^2 + \rho_2^2}{(\rho_1+\rho_2)^2} -\frac13\right) \frac{pd^2}{\sigma^2}}+ 1\right] .
\end{align*}
This concludes the proof.
\end{proof}



%\subsection{Proofs for Section \ref{sec_covariate}}\label{app_covariate}


