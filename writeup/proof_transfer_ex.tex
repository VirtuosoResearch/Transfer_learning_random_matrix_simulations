\section{Supplementary Materials for Section \ref{sec_insight}}

From \cite{WZR20}, we know that we need to explicitly restrict the capacity $r$ of $B$ so that there is transfer between the two tasks.
for the rest of the section, we shall consider the case when $r=1$ we are considering the case of two tasks.
Here, equation \eqref{eq_mtl} simplifies to the following
\begin{align}\label{eq_mtl_2task}
	f(B; w_1, w_2) = \bignorm{X_1 B w_1 - Y_1}^2 + \bignorm{X_2 B w_2 - Y_2}^2,
\end{align}
where $B\in\real^p$ and $w_1, w_2$ are both real numbers.
To solve the above, suppose that $w_1, w_2$ are fixed, by local optimality, we solve $B$ as
\begin{align*}
	& \hat{B}(w_1, w_2) \\
	=& (w_1^2 X_1^{\top}X_1 + w_2^2 X_2^{\top}X_2)^{-1} (w_1 X_1^{\top}Y_1 + w_2 X_2^{\top}Y_2) \\
	=& \frac{1}{w_2} ((\frac{w_1}{w_2})^2 X_1^{\top}X_1 + X_2^{\top}X_2)^{-1} (\frac{w_1}{w_2} X_1^{\top}Y_1 + X_2^{\top}Y_2) \\
	=& \frac{1}{w_2}\bigbrace{\beta_t + ((\frac{w_1}{w_2})^2 X_1^{\top}X_1 + X_2^{\top}X_2)^{-1}\bigbrace{X_1^{\top}X_1(\frac{w_1}{w_2}\beta_s - w^2\beta_t) + (\frac{w_1}{w_2} X_1^{\top}\varepsilon_1 + X_2^{\top}\varepsilon_2)}}.
\end{align*}
As a remark, when $w_1 = w_2 = 1$, we recover the linear regression estimator.
The advantage of using $f(B; w_1, w_2)$ is that if $\theta_1$ is a scaling of $\theta_2$, then this case can be solved optimally using equation \eqref{eq_mtl_2task} \cite{KD12}.

\textbf{Defining the multi-task learning estimator.}
Using a validation set that is sub-sampled from the original training dataset, we get a validation loss as follows
\begin{align}
		\val(\hat{B}; w_1, w_2)
	= & n_1 \cdot \bignorm{\Sigma_1^{1/2}((\frac{w_1}{w_2})^2 X_1^{\top}X_1 + X_2^{\top}X_2)^{-1}X_2^{\top}X_2(\beta_s - \frac{w_1}{w_2}\beta_t)}^2 \nonumber \\
		&+ n_1 \sigma^2 \cdot \bigtr{((\frac{w_1}{w_2})^2 X_1^{\top}X_1 + X_2^{\top}X_2)^{-1}\Sigma_1} \nonumber \\
		&+ n_2 \cdot w^2\bignorm{\Sigma_2^{1/2}((\frac{w_1}{w_2})^2 X_1^{\top}X_1 + X_2^{\top}X_2)^{-1}X_1^{\top}X_1(\beta_s - \frac{w_1}{w_2}\beta_t)}^2 \nonumber \\
		&+ n_2 \cdot \sigma^2 \cdot \bigtr{((\frac{w_1}{w_2})^2 X_1^{\top}X_1 + X_2^{\top}X_2)^{-1}\Sigma_2}. \label{eq_val_mtl}
\end{align}
Let $\hat{w_1}/\hat{w_2}$ be the global minimizer of $\val(\hat{B}; w_1, w_2)$.
We will define the multi-task learning estimator for the target task as
	\[ \hat{\beta}_t^{\MTL} = \hat{w}_{2}\hat{B}(\hat{w}_1, \hat{w}_2). \]

The intuition for deriving $\hat{\beta}_t^{\MTL}$ is akin to performing multi-task training in practice.
Let $\hat{v} = \hat{w_1} / \hat{w_2}$ for the simplicity of notation.
The test loss of using $\hat{\beta}_t^{\MTL}$ for the target task is
\begin{align}
	\te(\hat{\beta}_t^{\MTL}) =&~ \hat{v}^2 \bignorm{\Sigma_2^{1/2}(\hat{v}^2 X_1^{\top}X_1 + X_2^{\top}X_2)^{-1} X_1^{\top}X_1 (\beta_s - \hat{v} \beta_t)}^2 \nonumber \\
			&~ + \sigma^2 \cdot \bigtr{(\hat{v}^2 X_1^{\top}X_1 + X_2^{\top}X_2)^{-1} \Sigma_2}. \label{eq_te_mtl_2task}
\end{align}

Our goal is to study under model and covariate shifts, whether multi-task learning helps learn the target task better than single-task learning.
The baseline where we solve the target task with its own data is
\begin{align*}
	te(\hat{\beta}_t^{\STL}) = \sigma^2 \cdot \bigtr{(X_1^{\top}X_1 + X_2^{\top}X_2)^{-1}}, \text{ where } \hat{\beta}_t^{\STL} = (X_2^{\top}X_2)^{-1} X_2^{\top}Y_2.
\end{align*}
%Clearly, whether $\hat{\beta}_t^{\MTL}$ outperforms $\hat{\beta}_t^{\STL}$ depends on the covariate matrices, the difference of the task models, and the number of per-task data points.

We state several helper lemmas to get a bound on the variance of $\hat{\beta}_t^{\STL}$ and $\hat{\beta}_t^{\MTL}$.
The first lemma, which is a folklore result in random matrix theory, helps determine the asymptotic limit of $\te(\hat{\beta}_t^{\STL})$, as $p$ goes to infinity.
\begin{lemma}\label{lem_minv}[\todo{ref?}]
	Let $X\in\real^{n\times p}$ be a random matrix that contains i.i.d. row vectors with mean $0$ and covariance $\Sigma\in\real^{p\times p}$.
	In the setting when $n = c p$ we have that as $p$ goes to infinity,
	\[ \bigtr{(X^{\top}X)^{-1}\Sigma} = \frac{1}{c - 1}. \]
\end{lemma}

The second lemma, which deals the inverse of the sum of two random matrices, can be viewed as a special case of Theorem \ref{thm_model_shift}.
\begin{lemma}\label{lem_cov_shift}
	Let $X_i\in\real^{n_i\times p}$ be a random matrix that contains i.i.d. row vectors with mean $0$ and variance $\Sigma_i\in\real^{p\times p}$, for $i = 1, 2$.
	Denote by $M = \Sigma_1^{1/2}\Sigma_2^{-1/2}$ and let $\lambda_1, \lambda_2, \dots, \lambda_p$ be the singular values of $M^{\top}M$ in decreasing order.
	When $n_1 = c_1 p$ and $n_2 = c_2 p$, we have that with high probability over the randomness of $X_1$ and $X_2$, the following equation holds
	\begin{align}%\label{lem_cov_shift_eq}
		\bigtr{(X_1^{\top}X_1 + X_2^{\top}X_2)^{-1}\Sigma_2} = \frac{1}{c_1+c_2}\cdot \bigtr{ \frac{1}{p}\cdot(a_1 M^\top M + a_2)^{-1}} +\bigo{ n^{-1/2+\epsilon}},
	\end{align}
	for any constant $\epsilon>0$, where $a_1, a_2$ are solutions to the following deterministic equations:
	\begin{align}
		a_1 + a_2 = 1- \frac{1}{c_1 + c_2},~ a_1 + \frac{1}{p}\cdot\sum_{i=1}^p \frac{a_1}{(c_1 + c_2)(a_1 + a_2/ \lambda_i^2)} = \frac{c_1}{c_1 + c_2}. \label{eq_a12}
	\end{align}
\end{lemma}
We will give the proof of Lemma \ref{lem_cov_shift} in Section \ref{sec_maintools}.

Finally, we can get a bound tighter than Theorem \ref{thm_model_shift} as follows.
\todo{restate this}
\begin{lemma}\label{prop_model_shift_tight}
		In the setting of Theorem \ref{thm_model_shift}, assume that $\Sigma_1 =\id$,
		$\beta_t$ is i.i.d. with mean $0$ and variance $\kappa^2$ and $\beta_s - \beta_t$ is i.i.d. with mean $0$ and variance $d^2$.
		We set $\Delta_{\beta} = \bigbrace{(1 - \hat{w})^2 \kappa^2 + d^2)} \bigtr{Z}$
		and we have
		\begin{align*}
			\te(\hat{\beta}_t^{\MTL}) \le \te(\hat{\beta}_t^{\STL}) \text{ when: } & \Delta_{\vari} \ge \bigbrace{1 + \sqrt{\frac{p}{n_1}}}^4 \Delta_{\beta}, \\
			\te(\hat{\beta}_t^{\MTL}) \ge \te(\hat{\beta}_t^{\STL}) \text{ when: } & \Delta_{\vari} \le \bigbrace{1 - \sqrt{\frac{p}{n_1}}}^4 \Delta_{\beta}.
		\end{align*}
\end{lemma}

\subsection{Proof of Proposition \ref{prop_dist_transition} and Proposition \ref{prop_var_transition}}\label{app_proof_31}

The proof will consist of two main steps.
\begin{itemize}
	\item First, we show that $\hat{v}$ is close to $1$.
	\item Second, we plug $\hat{v}$ back into $\te(\hat{\beta}_t^{\MTL})$ to show the result.
\end{itemize}

We denote
\begin{align}
	\val(w) &= n_1\left[d^2 +\left( w-1\right)^2\kappa^2\right]\cdot \tr\left[( w^2X_1^{\top}X_1 +X_2^{\top}X_2)^{-2} (X_2^{\top}X_2)^2\right] \nonumber \\
	& + n_2w^2\left[d^2 +\left( w-1\right)^2\kappa^2\right]\cdot \tr\left[( w^2X_1^{\top}X_1 +X_2^{\top}X_2)^{-2} (X_1^{\top}X_1)^2\right] \nonumber\\
			& + (n_1+n_2)\sigma^2 \cdot \bigtr{(w^2X_1^{\top}X_1  + X_2^{\top}X_2)^{-1} }. \nonumber
\end{align}
Under the setting of Proposition \ref{prop_model_shift_tight}, using concentration of random vector with i.i.d. Gaussian entries, Lemma \ref{largedeviation}, we have that
$$\val(\hat{B}; w_1, w_2)= \val(w) \left( 1+\OO(p^{-1/2})\right)\quad \text{with probability }1-\oo(1).$$
Thus it suffices to study the behavior of $\val (w)$.
For the minimizer $\hat w$ of $\val(w)$, we have a similar result as in Proposition \ref{thm_cov_shift}.

\begin{lemma}\label{lem_hat_v}
Suppose the assumptions of Proposition \ref{prop_model_shift_tight} hold. Assume that $ \kappa^2 \sim pd^2 \sim \sigma^2$ are of the same order. Then we have that the optimal ratio for $\val(w)$ satisfies
	$$|\hat w -1|=\OO(p^{-1}).$$%is $\hat{w} = 1 \pm \bigo{\frac 1 {n_1+n_2}}$ \todo{(figure out the constants)}
\end{lemma}
\begin{proof}
The proof is also similar to the one for Proposition \ref{thm_cov_shift}. First it is easy to observe that $\val(w)\le \val(-w)$ for $w\ge 0$. Hence it suffices to consider the $w\ge 0$ case.

We first consider the case $w\ge 1$. We write
\begin{align}
	\val(w) &= n_1\left[\frac{d^2}{w^4} +\frac{\left( w-1\right)^2}{w^4}\kappa^2\right]\cdot \tr\left[( X_1^{\top}X_1 +w^{-2}X_2^{\top}X_2)^{-2} (X_2^{\top}X_2)^2\right] \nonumber \\
	& + n_2\left[\frac{d^2}{w^2} +\frac{\left( w-1\right)^2}{w^2}\kappa^2\right]\cdot \tr\left[( X_1^{\top}X_1 +w^{-2}X_2^{\top}X_2)^{-2} (X_1^{\top}X_1)^2\right] \nonumber\\
			& + \sigma^2(n_1+n_2) \cdot \bigtr{(w^{2}X_1^{\top}X_1  + X_2^{\top}X_2)^{-1} } .\nonumber%+  \sigma^2n_2 \cdot \bigtr{(X_1^{\top}X_1  + w^{-2}X_2^{\top}X_2)^{-1} }. \nonumber
\end{align}

Taking derivative of $\val(w)$ with respect to $w$, we obtain that
\begin{align*}
\val'(w) \ge &~ n_1\left[ \frac{2(w-1)(2-w)}{w^5}\kappa^2 - \frac{4d^2}{w^5}\right] \tr\left[( X_1^{\top}X_1 +w^{-2}X_2^{\top}X_2)^{-2} (X_2^{\top}X_2)^2\right]   \\
+&~ n_2\left[\frac{2\left( w-1\right)}{w^3}\kappa^2 - \frac{2d^2}{w^3} \right]\cdot \tr\left[( X_1^{\top}X_1 +w^{-2}X_2^{\top}X_2)^{-2} (X_1^{\top}X_1)^2\right] \nonumber\\
		-&~ 2(n_1+n_2) \frac{\sigma^2}{w^3} \cdot \bigtr{(X_1^{\top}X_1 + w^{-2}X_2^{\top}X_2)^{-2} X_1^\top X_1  } = n_1 \bigtr{(X_1^{\top}X_1 + w^{-2}X_2^{\top}X_2)^{-2} \cal A },
\end{align*}
where the matrix $\cal A$ is
\begin{align*}
\cal A :&= \left[ \frac{2(w-1)(2-w)}{w^5}\kappa^2 - \frac{4d^2}{w^5}\right](X_2^{\top}X_2)^2 + \frac{n_2}{n_1} \left[\frac{2\left( w-1\right)}{w^3}\kappa^2 - \frac{2d^2}{w^3} \right](X_1^{\top}X_1)^2. - 2\frac{n_1+n_2}{n_1}\frac{\sigma^2}{w^3}X_1^\top X_1
\end{align*}
Using the estimate \eqref{eq_isometric}, we get that $\cal A$ is lower bounded as
\begin{align*}
\cal A \succeq - \frac{4d^2}{w^5}(\sqrt{n_2}+\sqrt{p})^4 + \frac{n_2}{n_1} \left[\frac{2\left( w-1\right)}{w^3}\kappa^2 - \frac{2d^2}{w^3} \right](\sqrt{n_1}-\sqrt{p})^4 - 2\frac{n_1+n_2}{n_1}\frac{\sigma^2}{w^3}(\sqrt{n_1}+\sqrt{p})^2\succ 0,
\end{align*}
as long as
$$w> w_1:=1 +\frac{d^2}{\kappa^2}+ \frac{\sigma^2}{\kappa^2}\frac{(n_1+n_2)(\sqrt{n_1}+\sqrt{p})^2}{n_2(\sqrt{n_1}-\sqrt{p})^4}+ \frac{2d^2}{\kappa^2}\frac{n_1(\sqrt{n_2}+\sqrt{p})^4}{n_2(\sqrt{n_1}-\sqrt{p})^4}.$$
Hence $\val'(w)>0$ on $(w_1,\infty)$, i.e. $\val(w)$ is strictly increasing for $w>w_1$. Hence we must have $\hat w\le w_1$. Note that under our assumptions, we have $w_1=1+\OO(p^{-1})$.

Then we consider the case $w\le 1$. Taking derivative of $\val(w)$ with respect to $w$, we obtain that
\begin{align}
	\val'(w) &\le n_1 \left[2\left( w-1\right) \kappa^2\right]\cdot \tr\left[( w^2X_1^{\top}X_1 +X_2^{\top}X_2)^{-2} (X_2^{\top}X_2)^2\right] \nonumber \\
	& + n_2\left[2d^2w +2w\left( w-1\right)(2w-1)\kappa^2\right]\cdot \tr\left[( w^2X_1^{\top}X_1 +X_2^{\top}X_2)^{-2} (X_1^{\top}X_1)^2\right] \nonumber\\
			& = n_1 \bigtr{(w^2X_1^{\top}X_1  + X_2^{\top}X_2)^{-1} \cal B} , \nonumber
\end{align}
where the matrix $\cal B$ is
$$\cal B= 2\left( w-1\right) \kappa^2  (X_2^{\top}X_2)^2+\frac{n_2}{n_1}\left[2d^2w +2w\left( w-1\right)(2w-1)\kappa^2\right](X_1^{\top}X_1)^2.$$
Using the estimate \eqref{eq_isometric}, we get that $\cal B$ is upper bounded as
\begin{align*}
\cal B \preceq - 2(1-w)\kappa^2(\sqrt{n_2}-\sqrt{p})^4 +2d^2 w\frac{n_2}{n_1}(\sqrt{n_1}+\sqrt{p})^4 \prec 0,
\end{align*}
as long as
$$w< w_2:=1 -   \frac{d^2}{\kappa^2}\frac{n_2(\sqrt{n_1}+\sqrt{p})^4}{n_1(\sqrt{n_2}-\sqrt{p})^4}.$$
Hence $\val'(w)<0$ on $[0,w_2)$, i.e. $\val(w)$ is strictly increasing for $w<w_2$. Hence we must have $\hat w\le w_2$. Note that under our assumptions, we have $w_2=1-\OO(p^{-1})$.
\end{proof}



Based on Lemma \ref{lem_hat_v}, we can prove Proposition \ref{prop_dist_transition}.
\begin{proof}[Proof of Proposition \ref{prop_dist_transition}]
	Since $\Sigma_1 = \Sigma_2$, we know that $M = \Sigma_1^{1/2}\Sigma_2^{-1/2} = \hat{v} \id$.
	Using Lemma \ref{lem_minv} and \ref{lem_cov_shift}, we can track the reduction of variance from $\hat{\beta}_t^{\MTL}$ to $\hat{\beta}_t^{\STL}$ as whp
	\begin{align}
		\Delta_{\vari} &\define \sigma^2 \bigbrace{\frac{1}{c_2 - 1} - \frac{1}{c_1 + c_2}\cdot\frac{1}{a_1\hat{v}^2 + a_2} + \bigo{p^{-1/2+\varepsilon}}} \nonumber\\
		&= \sigma^2 \bigbrace{\frac{c_1}{(c_2-1)(c_1 + c_2 -1)} + \bigo{p^{-1/2 + \varepsilon}}},\label{Deltavar}
	\end{align}
	where we use equation \eqref{eq_a12} and Lemma \ref{lem_hat_v}.
	Next we consider the model shift bias
	\begin{align*}
		\Delta_{\beta} &\define \hat{v}^2 \bignorm{(\hat{v}^2 X_1^{\top}X_1 + X_2^{\top}X_2)^{-1} X_1^{\top}X_1 (\beta_s - \hat{v}\beta_t)}^2 \\
		&= d^2 \cdot \bignormFro{(\hat{v}^2 X_1^{\top}X_1 + X_2^{\top}X_2)^{-1}X_1^{\top}X_1}^2 + \bigo{p^{-1/2 + \varepsilon}d^2}
	\end{align*}
	Using Lemma \ref{prop_model_shift_tight}, we get an upper and lower bound on $\Delta_{\beta}$ as
	\be\label{Deltabeta} \bigbrace{1 - \sqrt{\frac{1}{c_1}}}^4 \le \Delta_{\beta} / \bigbrace{p\cdot d^2 \cdot \frac{c_1^2}{(c_1+c_2)^2} \cdot \frac{1 + a_3 + a_4}{(a_1 + a_2)^2} + \bigo{p^{1/2+\e}d^2} } \le \bigbrace{1 + \sqrt{\frac{1}{c_1}}}^4. \ee
	By solving equations \eqref{eq_a12}, \eqref{eq_a3} and \eqref{eq_a4}, we get
	\begin{align*}
		& a_1 = \frac{c_1(c_1 + c_2 - 1)}{(c_1 + c_2)^2} + O(p^{-1/2 + \varepsilon}),
		a_2 = \frac{c_2(c_1 + c_2 - 1)}{(c_1 + c_2)^2} + O(p^{-1/2 + \varepsilon}), \\
		& a_3 = \frac{c_2}{(c_1 + c_2)(c_1 + c_2 - 1)} + O(p^{-1/2 + \varepsilon}),
		a_4 = \frac{c_1}{(c_1 + c_2)(c_1 + c_2 - 1)} + O(p^{-1/2 + \varepsilon}).
	\end{align*}
	Hence we obtain that
	\begin{align*}
		\frac{1 + a_3 + a_4}{(a_1 + a_2)^2}
		= \frac{(c_1 + c_2)^3}{(c_1 + c_2 - 1)^3} + \bigo{p^{-1/2+\varepsilon}}.
	\end{align*}
	To sum up, we have shown that when
	\[ \frac{p\cdot d^2}{\sigma^2} < \frac{(c_1 + c_2 - 1)^2}{c_1(c_1 + c_2) (c_2 - 1)} \cdot \bigbrace{1 + \sqrt{\frac 1 {c_1}}}^{-4}, \]
	we have that $\Delta_{\vari} > \Delta_{\beta}$, which implies that $\te(\hat{\beta}_t^{\MTL})$ is lower than $\te(\hat{\beta}_t^{\STL})$.
	When
	\[ \frac{p\cdot d^2}{\sigma^2} > \frac{(c_1 + c_2 - 1)^2}{c_1 (c_1 + c_2) (c_2 - 1)} \cdot \bigbrace{1 - \sqrt{\frac 1 {c_1}}}^{-4}, \]
	we have that $\Delta_{\vari} < \Delta_{\beta}$, which implies that $\te(\hat{\beta}_t^{\MTL})$ is higher than $\te(\hat{\beta}_t^{\STL})$.
	The result follows by taking $c_1 = c_2 = c$ for the above equation.
\end{proof}


\subsection{Proof of Proposition \ref{prop_data_size}}


Recall that under the setting of Example \ref{ex_basic}, \eqref{Deltavar} and \eqref{Deltabeta} hold.  Then we get that $\te(\hat{\beta}_t^{\MTL})< \te(\hat{\beta}_t^{\STL})$ whp if
\be\label{pos1}\left(pd^2 \cdot \frac{c_1(c_1+c_2)(c_2-1)}{(c_1+c_2-1)^2} + \bigo{p^{1/2+\e}d^2} \right)\cdot \bigbrace{1 + \sqrt{\frac{1}{c_1}}}^4 <  \sigma^2 \bigbrace{1 + \bigo{p^{-1/2 + \varepsilon}}};\ee
otherwise if
\be\label{neg1}\left(pd^2 \cdot \frac{c_1(c_1+c_2)(c_2-1)}{(c_1+c_2-1)^2} + \bigo{p^{1/2+\e}d^2} \right)\cdot \bigbrace{1 - \sqrt{\frac{1}{c_1}}}^4 >  \sigma^2 \bigbrace{1 + \bigo{p^{-1/2 + \varepsilon}}},\ee
then we have $\te(\hat{\beta}_t^{\MTL})> \te(\hat{\beta}_t^{\STL})$ whp.

Now we prove the first statement of Proposition \ref{prop_data_size}. Notice that the function
$$ \frac{c_1(c_1+c_2)(c_2-1)}{(c_1+c_2-1)^2} =(c_2-1) \left(1 +\frac{c_2-2}{c_1}+\frac{1}{c_1(c_1+c_2)}\right)^{-1} $$
is strictly increasing with respect to $c_1$ as long as $c_2\ge 3$. In particular, for $c_1\ge 100$, we get the bound by taking $c_1\to \infty$:
$$\frac{c_1(c_1+c_2)(c_2-1)}{(c_1+c_2-1)^2} \cdot \bigbrace{1 + \sqrt{\frac{1}{c_1}}}^4 < 1.1^4 (c_2-1).$$
 Hence we conclude that for $c_1\ge 100$, we always have $\te(\hat{\beta}_t^{\MTL}) < \te(\hat{\beta}_t^{\STL})$ whp as long as $pd^2 \le \frac{2\sigma^2}{3(c_2-1)}$. This gives the first statement. 

The second statement can be proved in a similar way. Suppose $c_1 \ge 110$ and $pd^2 > \frac{3\sigma^2}{2(c_2-1)}$. If $c_1 > \frac{(c_2-2) \sigma^2}{2(c_2 - 1) pd^2/3 - \sigma^2}$, we have
\begin{align*}
 pd^2 \cdot \frac{c_1(c_1+c_2)(c_2-1)}{(c_1+c_2-1)^2}  \cdot \bigbrace{1 - \sqrt{\frac{1}{c_1}}}^4 > \frac{2}{3} pd^2(c_2-1)\cdot \frac{c_1 }{c_1+(c_2-2)}  >\sigma^2, 
 \end{align*}
 where in the first step we used that 
 $$ \frac{(c_1+c_2)(c_1+c_2-2)}{(c_1+c_2-1)^2}  \cdot \bigbrace{1 - \sqrt{\frac{1}{c_1}}}^4 > \frac23$$
 for $c_1 \ge 110$. This shows that \eqref{neg1} holds as long as $p$ is large enough, and hence $\te(\hat{\beta}_t^{\MTL})> \te(\hat{\beta}_t^{\STL})$ holds. On the other hand, if $100\le c_1 \le \frac{(c_2-2) \sigma^2}{3(c_2 - 1) pd^2/2 - \sigma^2}$, then we have 
 \begin{align*}
 pd^2 \cdot \frac{c_1(c_1+c_2)(c_2-1)}{(c_1+c_2-1)^2}  \cdot \bigbrace{1 + \sqrt{\frac{1}{c_1}}}^4 < \frac{3}{2} pd^2(c_2-1)\cdot \frac{c_1 }{c_1+(c_2-2)}  < \sigma^2, 
 \end{align*}
 where in the first step we used that 
 $$ \frac{(c_1+c_2)(c_1+c_2-2)}{(c_1+c_2-1)^2}  \cdot \bigbrace{1 + \sqrt{\frac{1}{c_1}}}^4 < \frac32$$
 for $c_1 \ge 100$. This shows that \eqref{pos1} holds as long as $p$ is large enough, and hence $\te(\hat{\beta}_t^{\MTL})< \te(\hat{\beta}_t^{\STL})$ holds.
 



\subsection{Proof of ???}
Note that for the case of $k$ tasks with the same covariates, since there is no covariate shift and the data ratio is always equal to one, the main factor is model distance.

\paragraph{A precise bound when there is no model shift.}
As Proposition \ref{prop_monotone} shows, if $\beta_s$ and $\beta_t$ are equal, then adding the source task dataset always helps learn the target task.
The goal of this section is to understand how covariate shift affects the rate of transfer. \todo{add conceptual msg}

%The key quantity is to look at:
%The estimator using the source and target together from minimizing \eqref{eq_mtl_basic} is
%\[ \hat{\beta}_{s,t} = (X_1^{\top} X_1 + X_2^{\top} X_2)^{-1} (X_1^{\top}Y_1 + X_2^{\top}Y_2)\]
%The estimation error of $\hat{\beta}_{s,t}$ is
%\begin{align}\label{eq_two_task}
%  \err(\hat{\beta}_{s,t}) = \sigma^2 \cdot \tr[(X_1^{\top}X_1 + X_2^{\top} X_2)^{-1}].
%\end{align}
%The estimation error using the target alone is
%\begin{align}\label{eq_target_task}
%	\err(\hat{\beta}_t) = \sigma^2 \cdot \tr[(X_2^{\top} X_2)^{-1}].
%\end{align}
%The improvement of estimation error from adding the source task is then given by
%$\err(\hat{\beta}_t) - \err(\hat{\beta}_{s,t})$.
%For the test error on the target task, the improvement from adding the source task is
%\[ \te(\hat{\beta}_t) - \te(\hat{\beta}_{s,t}) = \sigma^2\cdot\bigtr{\bigbrace{(X_2^{\top}X_2)^{-1} - (X_1^{\top}X_1 + X_2^{\top}X_2)^{-1}}\cdot\Sigma_2}. \]

%We calculate the amount of improvement by comparing equation \eqref{eq_two_task} to equation \eqref{eq_target_task}.
A simple observation here is that when $\beta_s = \beta_t$, the optimal $\hat{w}$ for minimizing equation \eqref{eq_te_mtl} is equal to $1$.
Based on this observation, we can get a more precise result than Theorem \ref{thm_model_shift} on the improvement of adding the source task data that only depends on the covariance matrices $\Sigma_1, \Sigma_2$ and the number of data points $n_1, n_2$.



\begin{proposition}[Transfer rate without model shift]\label{thm_cov_shift}
Suppose $\beta_s = \beta_t$ and $\|\beta_t\|_2^2\sim p\sigma^2$ (i.e. the $l^2$-norm of the vector $\beta_t$ is of the same order as that of the error vector). Assume that the condition numbers of $\Sigma_1$, $\Sigma_2$ and $M:=\Sigma_1^{1/2}\Sigma_2^{-1/2}$ are all bounded by a constant $C>0$. Then we have that the optimal ratio for $W_1/W_2$ in equation \eqref{eq_te_mtl} satisfies
	$$1\le \hat{w} \le 1+\OO(p^{-1}).$$%is $\hat{w} = 1 \pm \bigo{\frac 1 {n_1+n_2}}$ \todo{(figure out the constants)}
	%where $M:=\Sigma_1^{1/2}\Sigma_2^{-1/2}$.
Moreover, we have
	\begin{align}\label{tehatw1}
		%\err(\hat{\beta}^{\TL}_{s,t}) &= \sigma^2 \cdot \bigtr{\frac 1 {(n_1 + n_2)a_1\Sigma_1 + (n_1 + n_2)a_2\Sigma_2}} \\
		\te(\hat{\beta}^{\TL}_{t}) &= \sigma^2 \cdot \bigtr{\bigbrace{(n_1 + n_2)a_1 M^\top M  + (n_1 + n_2)a_2\id}^{-1}} \cdot \left(1+ \bigo{p^{-1}}\right),
	\end{align}
where $a_1, a_2$ are the solutions to equations \eqref{eq_a2}. %\cor $w_0$ is close to 1 if the signal strength $\beta_t$ is much larger than the noise strength \nc
\end{proposition}


\begin{proof}
We abbreviate $\val(w_2\hat{B}(w)):=\val(w)$. Note that $\val(w)\le \val(-w)$ for $w\ge 0$. Hence we have $\hat w\ge 0$. Moreover, we notice that $\val (w) < \val (1)$ for all $0\le w < 1$. Thus we have $\hat w\ge 1$. It suffices to consider the case with $w> 1$. Under the assumption on $\beta_s$ and $\beta_t$, we can write
\begin{align}
	\val(w) =&~  \left( 1-\frac1w\right)^2 \left\|( M^\top Z_1^{\top}Z_1 M + w^{-2}Z_2^{\top}Z_2)^{-1} M^\top Z_1^{\top} Z_1 \Sigma_1^{1/2}\beta_t \right\|^2 \nonumber \\
			&~ + \frac{\sigma^2}{w^2} \cdot \bigtr{( M^\top Z_1^{\top}Z_1 M + w^{-2}Z_2^{\top}Z_2)^{-1} }. \nonumber
\end{align}
Since
\begin{align*}
&\left\|( M^\top Z_1^{\top}Z_1 M + w^{-2}Z_2^{\top}Z_2)^{-1} M^\top Z_1^{\top} Z_1 \Sigma_1^{1/2}\beta_t \right\|^2 \\
&= \tr \left[ ( M^\top Z_1^{\top}Z_1 M + w^{-2}Z_2^{\top}Z_2)^{-2}M^\top Z_1^{\top} Z_1 \Sigma_1^{1/2}\beta_t\beta_t^\top \Sigma_1^{1/2}Z_1^{\top} Z_1  M\right]
\end{align*}
is increasing with respect to $w$, then the derivative of $\val(w)$ can be bounded from below as
\begin{align*}
\val'(w) \ge &~ 2\frac{w-1}{w^3} \left\|( M^\top Z_1^{\top}Z_1 M + w^{-2}Z_2^{\top}Z_2)^{-1} M^\top Z_1^{\top} Z_1 \Sigma_1^{1/2}\beta_t \right\|^2   \\
			&~ - 2 \frac{\sigma^2}{w^3} \cdot \bigtr{(M^\top Z_1^{\top}Z_1 M +w^{-2}  Z_2^{\top}Z_2)^{-1} M^\top Z_1^{\top}Z_1 M (M^\top Z_1^{\top}Z_1 M + w^{-2} Z_2^{\top}Z_2)^{-1}} \\
\ge &~ 2\frac{w-1}{w^3} \left\|( M^\top Z_1^{\top}Z_1 M +  Z_2^{\top}Z_2)^{-1} M^\top Z_1^{\top} Z_1 \Sigma_1^{1/2}\beta_t \right\|^2   - 2 \frac{\sigma^2}{w^3} \cdot \bigtr{(M^\top Z_1^{\top}Z_1 M)^{-1}}.
			%\\ =& ~ 2\frac{d^2}{w^3} \tr\left[( M^\top Z_1^{\top}Z_1 M + w^{-2}Z_2^{\top}Z_2)^{-1} M^\top Z_1^{\top} \left[\left( w-1\right)\left(Z_1 \Sigma_1 Z_1^{\top}\right) - \frac{\sigma^2}{d^2}\id \right] Z_1 M ( M^\top Z_1^{\top}Z_1 M + w^{-2}Z_2^{\top}Z_2)^{-1}\right]  .
\end{align*}
Hence $\val'(w)\ge 0$ if $w>1+ \e_0$, where
$$\e_0:= \frac{\sigma^2  \bigtr{(M^\top Z_1^{\top}Z_1 M)^{-1}} }{\left\|( M^\top Z_1^{\top}Z_1 M + Z_2^{\top}Z_2)^{-1} M^\top Z_1^{\top} Z_1 \Sigma_1^{1/2}\beta_t \right\|^2}.$$
In other words, $\val(w)$ is strictly increasing function on $[1+\e_0,\infty]$. Thus we get that $\hat w$ satisfies
\be\label{hatw1}1\le w \le 1+\e_0.\ee Using \eqref{eq_isometric}, we get that
$$\e_0=\OO(\sigma^2/\|\beta_t\|_2^2)=\OO(p^{-1}).$$

Finally, plugging \eqref{hatw1} into the expression $\te(\hat{\beta}^{\TL}_{t}) $, we obtain \eqref{tehatw1}.
%$$1\le \hat{w} \le w_0:=1 +\frac{\sigma^2  \bigtr{(M^\top Z_1^{\top}Z_1 M)^{-1}} }{\left\|( M^\top Z_1^{\top}Z_1 M + Z_2^{\top}Z_2)^{-1} M^\top Z_1^{\top} Z_1 \Sigma_1^{1/2}\beta_t \right\|^2},$$
\end{proof}



As a remark, we see that Proposition \ref{prop_monotone} follows from Theorem \ref{thm_cov_shift}.
The amount of reduction on test error for the target task is given as
	\begin{align*}
%		\err(\hat{\beta}_t) - \err(\hat{\beta}_{s,t})
%		&= \sigma^2 p \cdot \bigtr{\frac 1 {(n_2 - p) \Sigma_2} - \frac 1 {(n_1 + n_2)a_1 \Sigma_1 + (n_1 + n_2)a_2 \Sigma_2}}, \\
		\te(\hat{\beta}_t) - \te(\hat{\beta}_{s,t})
		&= \sigma^2 \cdot \bigbrace{\frac p {n_2 - p} -  \bigtr{\bigbrace{(n_1 + n_2)a_1\Sigma_2^{-1/2}\Sigma_1\Sigma_2^{-1/2} + (n_1 + n_2)a_2\id}^{-1}}}.
	\end{align*}
Because
\begin{align*}
	\te(\hat{\beta}_{s,t}) \le \te(\hat{\beta}_t)
	\Leftarrow~ & (n_2 - p)\Sigma_2 \preceq (n_1 + n_2) a_1 \Sigma_1 + (n_1 + n_2)a_2 \Sigma_2 \\
	\Leftrightarrow~ & \zeroMatrix \preceq (n_1 + n_2) a_1 \Sigma_1 + (n_1 - (n_1 + n_2)\cdot a_1) \Sigma_2,
\end{align*}
which is true since $a_1 \le n_1 / (n_1 + n_2)$ by equation \eqref{eq_a2}.
The proof for $\te(\hat{\beta}_{s,t}) \le \te(\hat{\beta}_t)$ follows by multiplying $\Sigma_2^{-1/2}$ on both sides of the inequalities above.

\medskip
