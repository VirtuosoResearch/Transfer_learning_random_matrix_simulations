\section{Supplementary Materials for Section \ref{sec_insight}}

From \cite{WZR20}, we know that we need to explicitly restrict the capacity $r$ of $B$ so that there is transfer between the two tasks.
for the rest of the section, we shall consider the case when $r=1$ we are considering the case of two tasks.
Here, equation \eqref{eq_mtl} simplifies to the following
\begin{align}\label{eq_mtl_2task}
	f(B; w_1, w_2) = \bignorm{X_1 B w_1 - Y_1}^2 + \bignorm{X_2 B w_2 - Y_2}^2,
\end{align}
where $B\in\real^p$ and $w_1, w_2$ are both real numbers.
To solve the above, suppose that $w_1, w_2$ are fixed, by local optimality, we solve $B$ as
\begin{align*}
	& \hat{B}(w_1, w_2) \\
	=& (w_1^2 X_1^{\top}X_1 + w_2^2 X_2^{\top}X_2)^{-1} (w_1 X_1^{\top}Y_1 + w_2 X_2^{\top}Y_2) \\
	=& \frac{1}{w_2} ((\frac{w_1}{w_2})^2 X_1^{\top}X_1 + X_2^{\top}X_2)^{-1} (\frac{w_1}{w_2} X_1^{\top}Y_1 + X_2^{\top}Y_2) \\
	=& \frac{1}{w_2}\bigbrace{\beta_t + ((\frac{w_1}{w_2})^2 X_1^{\top}X_1 + X_2^{\top}X_2)^{-1}\bigbrace{X_1^{\top}X_1(\frac{w_1}{w_2}\beta_s - w^2\beta_t) + (\frac{w_1}{w_2} X_1^{\top}\varepsilon_1 + X_2^{\top}\varepsilon_2)}}.
\end{align*}
As a remark, when $w_1 = w_2 = 1$, we recover the linear regression estimator.
The advantage of using $f(B; w_1, w_2)$ is that if $\theta_1$ is a scaling of $\theta_2$, then this case can be solved optimally using equation \eqref{eq_mtl_2task} \cite{KD12}.

\textbf{Defining the multi-task learning estimator.}
Using a validation set that is sub-sampled from the original training dataset, we get a validation loss as follows
\begin{align}
		\val(\hat{B}; w_1, w_2)
	= & n_1 \cdot \bignorm{\Sigma_1^{1/2}((\frac{w_1}{w_2})^2 X_1^{\top}X_1 + X_2^{\top}X_2)^{-1}X_2^{\top}X_2(\beta_s - \frac{w_1}{w_2}\beta_t)}^2 \nonumber \\
		&+ n_1 \sigma^2 \cdot \bigtr{((\frac{w_1}{w_2})^2 X_1^{\top}X_1 + X_2^{\top}X_2)^{-1}\Sigma_1} \nonumber \\
		&+ n_2 \cdot w^2\bignorm{\Sigma_2^{1/2}((\frac{w_1}{w_2})^2 X_1^{\top}X_1 + X_2^{\top}X_2)^{-1}X_1^{\top}X_1(\beta_s - \frac{w_1}{w_2}\beta_t)}^2 \nonumber \\
		&+ n_2 \cdot \sigma^2 \cdot \bigtr{((\frac{w_1}{w_2})^2 X_1^{\top}X_1 + X_2^{\top}X_2)^{-1}\Sigma_2} \label{eq_val_mtl}
\end{align}
Let $\hat{w_1}/\hat{w_2}$ be the global minimizer of $\val(\hat{B}; w_1, w_2)$.
We will define the multi-task learning estimator for the target task as
	\[ \hat{\beta}_t^{\MTL} = \hat{w}_{2}\hat{B}(\hat{w}_1, \hat{w}_2). \]

The intuition for deriving $\hat{\beta}_t^{\MTL}$ is akin to performing multi-task training in practice.
Let $\hat{v} = \hat{w_1} / \hat{w_2}$ for the simplicity of notation.
The test loss of using $\hat{\beta}_t^{\MTL}$ for the target task is
\begin{align}
	\te(\hat{\beta}_t^{\MTL}) =&~ \hat{v}^2 \bignorm{\Sigma_2^{1/2}(\hat{v}^2 X_1^{\top}X_1 + X_2^{\top}X_2)^{-1} X_1^{\top}X_1 (\beta_s - \hat{v} \beta_t)}^2 \nonumber \\
			&~ + \sigma^2 \cdot \bigtr{(\hat{v}^2 X_1^{\top}X_1 + X_2^{\top}X_2)^{-1} \Sigma_2}. \label{eq_te_mtl_2task}
\end{align}

Our goal is to study under model and covariate shifts, whether multi-task learning helps learn the target task better than single-task learning.
The baseline where we solve the target task with its own data is
\begin{align*}
	te(\hat{\beta}_t^{\STL}) = \sigma^2 \cdot \bigtr{(X_1^{\top}X_1 + X_2^{\top}X_2)^{-1}}, \text{ where } \hat{\beta}_t^{\STL} = (X_2^{\top}X_2)^{-1} X_2^{\top}Y_2.
\end{align*}
%Clearly, whether $\hat{\beta}_t^{\MTL}$ outperforms $\hat{\beta}_t^{\STL}$ depends on the covariate matrices, the difference of the task models, and the number of per-task data points.

We state several helper lemmas to get a bound on the variance of $\hat{\beta}_t^{\STL}$ and $\hat{\beta}_t^{\MTL}$.
The first lemma, which is a folklore result in random matrix theory, helps determine the asymptotic limit of $\te(\hat{\beta}_t^{\STL})$, as $p$ goes to infinity.
\begin{lemma}\label{lem_minv}[\todo{ref?}]
	Let $X\in\real^{n\times p}$ be a random matrix that contains i.i.d. row vectors with mean $0$ and covariance $\Sigma\in\real^{p\times p}$.
	In the setting when $n = c p$ we have that as $p$ goes to infinity,
	\[ \bigtr{(X^{\top}X)^{-1}\Sigma} = \frac{1}{c - 1}. \]
\end{lemma}

The second lemma, which deals the inverse of the sum of two random matrices, can be viewed as a special case of Theorem \ref{thm_model_shift}.
\begin{lemma}\label{lem_cov_shift}
	Let $X_i\in\real^{n_i\times p}$ be a random matrix that contains i.i.d. row vectors with mean $0$ and variance $\Sigma_i\in\real^{p\times p}$, for $i = 1, 2$.
	Denote by $M = \Sigma_1^{1/2}\Sigma_2^{-1/2}$ and let $\lambda_1, \lambda_2, \dots, \lambda_p$ be the singular values of $M^{\top}M$ in decreasing order.
	When $n_1 = c_1 p$ and $n_2 = c_2 p$, we have that with high probability over the randomness of $X_1$ and $X_2$, the following equation holds
	\begin{align}%\label{lem_cov_shift_eq}
		\bigtr{(X_1^{\top}X_1 + X_2^{\top}X_2)^{-1}\Sigma_2} = \frac{1}{c_1+c_2}\cdot \bigtr{ \frac{1}{p}\cdot(a_1 M^\top M + a_2)^{-1}} +\bigo{ n^{-1/2+\epsilon}},
	\end{align}
	for any constant $\epsilon>0$, where $a_1, a_2$ are solutions to the following deterministic equations:
	\begin{align}
		a_1 + a_2 = 1- \frac{1}{c_1 + c_2},~ a_1 + \frac{1}{p}\cdot\sum_{i=1}^p \frac{a_1}{(c_1 + c_2)(a_1 + a_2/ \lambda_i^2)} = \frac{c_1}{c_1 + c_2}. \label{eq_a12}
	\end{align}
\end{lemma}
We will give the proof of Lemma \ref{lem_cov_shift} in Section \ref{sec_maintools}.

Finally, we can get a bound tighter than Theorem \ref{thm_model_shift} as follows.
\todo{restate this}
\begin{lemma}\label{prop_model_shift_tight}
		In the setting of Theorem \ref{thm_model_shift}, assume that $\Sigma_1 =\id$,
		$\beta_t$ is i.i.d. with mean $0$ and variance $\kappa^2$ and $\beta_s - \beta_t$ is i.i.d. with mean $0$ and variance $d^2$.
		We set $\Delta_{\beta} = \bigbrace{(1 - \hat{w})^2 \kappa^2 + d^2)} \bigtr{Z}$
		and we have
		\begin{align*}
			\te(\hat{\beta}_t^{\MTL}) \le \te(\hat{\beta}_t^{\STL}) \text{ when: } & \Delta_{\vari} \ge \bigbrace{1 + \sqrt{\frac{p}{n_1}}}^4 \Delta_{\beta}, \\
			\te(\hat{\beta}_t^{\MTL}) \ge \te(\hat{\beta}_t^{\STL}) \text{ when: } & \Delta_{\vari} \le \bigbrace{1 - \sqrt{\frac{p}{n_1}}}^4 \Delta_{\beta}.
		\end{align*}
\end{lemma}

\subsection{Proof of Proposition \ref{prop_dist_transition}}

The proof will consist of two main steps.
\begin{itemize}
	\item First, we show that $\hat{v}$ is close to $1$.
	\item Second, we plug $\hat{v}$ back into $\te(\hat{\beta}_t^{\MTL})$ to show the result.
\end{itemize}

\begin{lemma}\label{lem_hat_v}
	Recall that $\hat{v} = \hat{w_1} / \hat{w_2}$.
	In the setting of Proposition \ref{prop_dist_transition}, we have that
	\[ \abs{\hat{v} - 1} \lesssim \frac 1 p. \]
\end{lemma}



Based on Lemma \ref{lem_hat_v}, we can prove Proposition \ref{prop_dist_transition}.
\begin{proof}[Proof of Proposition \ref{prop_dist_transition}]
	Since $\Sigma_1 = \Sigma_2$, we know that $M = \Sigma_1^{1/2}\Sigma_2^{-1/2} = \hat{v} \id$.
	Using Lemma \ref{lem_minv} and \ref{lem_cov_shift}, we can track the reduction of variance from $\hat{\beta}_t^{\MTL}$ to $\hat{\beta}_t^{\STL}$ as whp
	\begin{align*}
		\Delta_{\vari} &\define \sigma^2 \bigbrace{\frac{1}{c_2 - 1} - \frac{1}{c_1 + c_2}\cdot\frac{1}{a_1\hat{v}^2 + a_2} + \bigo{p^{-1/2+\varepsilon}}} \\
		&= \sigma^2 \bigbrace{\frac{c_1}{(c_2-1)(c_1 + c_2 -1)} + \bigo{p^{-1/2 + \varepsilon}}},
	\end{align*}
	where we use equation \eqref{eq_a12} and Lemma \ref{lem_hat_v}.
	Next we consider the model shift bias
	\begin{align*}
		\Delta_{\beta} &\define \hat{v}^2 \bignorm{(\hat{v}^2 X_1^{\top}X_1 + X_2^{\top}X_2)^{-1} X_1^{\top}X_1 (\beta_s - \hat{v}\beta_t)}^2 \\
		&= d^2 \cdot \bignormFro{(\hat{v}^2 X_1^{\top}X_1 + X_2^{\top}X_2)^{-1}X_1^{\top}X_1}^2 + \bigo{p^{-1/2 + \varepsilon}}
	\end{align*}
	Using Lemma \ref{prop_model_shift_tight}, we get an upper and lower bound on $\Delta_{\beta}$ as
	\[ \bigbrace{1 - \sqrt{\frac{1}{c_1}}}^4 \le \Delta_{\beta} / \bigbrace{p\cdot d^2 \cdot \frac{c_1^2}{(c_1+c_2)^2} \cdot \frac{1 + a_3 + a_4}{(a_1 + a_2)^2} + \bigo{\frac 1 p} } \le \bigbrace{1 + \sqrt{\frac{1}{c_1}}}^4. \]
	By solving equations \eqref{eq_a12}, \eqref{eq_a3} and \eqref{eq_a4}, we get
	\begin{align*}
		& a_1 = \frac{c_1(c_1 + c_2 - 1)}{(c_1 + c_2)^2} + O(p^{-1/2 + \varepsilon}),
		a_2 = \frac{c_2(c_1 + c_2 - 1)}{(c_1 + c_2)^2} + O(p^{-1/2 + \varepsilon}), \\
		& a_3 = \frac{c_2}{(c_1 + c_2)(c_1 + c_2 - 1)} + O(p^{-1/2 + \varepsilon}),
		a_4 = \frac{c_1}{(c_1 + c_2)(c_1 + c_2 - 1)} + O(p^{-1/2 + \varepsilon}).
	\end{align*}
	Hence we obtain that
	\begin{align*}
		\frac{1 + a_3 + a_4}{(a_1 + a_2)^2}
		= \frac{(c_1 + c_2)^3}{(c_1 + c_2 - 1)^3} + \bigo{p^{-1/2+\varepsilon}}.
	\end{align*}
	To sum up, we have shown that when
	\[ \frac{p\cdot d^2}{\sigma^2} < \frac{(c_1 + c_2 - 1)^2}{c_1(c_1 + c_2) (c_2 - 1)} \cdot \bigbrace{1 + \sqrt{\frac 1 {c_1}}}^{-4}, \]
	we have that $\Delta_{\vari} > \Delta_{\beta}$, which implies that $\te(\hat{\beta}_t^{\MTL})$ is lower than $\te(\hat{\beta}_t^{\STL})$.
	When
	\[ \frac{p\cdot d^2}{\sigma^2} > \frac{(c_1 + c_2 - 1)^2}{c_1 (c_1 + c_2) (c_2 - 1)} \cdot \bigbrace{1 - \sqrt{\frac 1 {c_1}}}^{-4}, \]
	we have that $\Delta_{\vari} < \Delta_{\beta}$, which implies that $\te(\hat{\beta}_t^{\MTL})$ is higher than $\te(\hat{\beta}_t^{\STL})$.
	The result follows by taking $c_1 = c_2 = c$ for the above equation.
\end{proof}



