\section{Proofs for the Extended Results}

\subsection{Many Tasks with the Same Covariate}\label{app_proof_many_tasks}



%As a remark, since the spectral norm of $U_r$ is less than $1$, we have that $\norm{U_r(i)} < 1$ for all $1 \le i \le t$. Compared to Theorem \ref{thm_main_informal}, we can get a simple expression for the two functions $\Delta_{vari}$ and $\Delta_{\bias}$. The proof of Theorem \ref{thm_many_tasks} can be found in Appendix \ref{app_proof_many_tasks}.



%In this section we consider the setting with $k$ many that have the same covariates.
%Since every task has the same number of data points as well as the same covariance, the only differences between different tasks are their models $\set{\beta_i}_{i=1}^k$.
%For this setting, we derive solutions for the multi-task training and the transfer learning setting that match our insights qualitatively from Section \ref{sec_denoise}.

\begin{proof}[Proof of Theorem \ref{thm_many_tasks}]
In this setting, we need to study the following loss function:
\begin{align}
	f(B; W_1, \dots, W_t) = \sum_{i=1}^t \bignorm{X B W_i - Y_i}^2. \label{eq_mtl_same_cov}
\end{align}
%In order to prove Theorem \ref{thm_many_tasks}, we will derive a closed form solution for equation \eqref{eq_mtl_same_cov}. \todo{check!}
For any fixed $W_1, W_2, \dots, W_t \in \R^r$, we can derive a closed form solution for $B$ as
	\begin{align*}
		\hat{B}(W_1, \dots, W_t) &= (X^{\top}X)^{-1} X^{\top} \bigbrace{\sum_{i=1}^t Y_i W_i^{\top}} (\cal W  \cal W^{\top})^{-1} \\
		&= (B^\star \cal W ^{\top}) (\cal W \cal W ^{\top})^{-1} + (X^{\top}X)^{-1}X^{\top} \bigbrace{\sum_{i=1}^t \varepsilon_i W_i^{\top}} (\cal W \cal W^{\top})^{-1},
	\end{align*}
	where we denote $\cal W \in\real^{r\times t}$ as $\cal W=[W_1, W_2, \dots, W_t]$.
	%Now we switch $\hat{B}$ back into equation \eqref{eq_mtl_same_cov} to
Then as in \eqref{approxvalid}, we pick $N$ independent samples of the training set for each task with $N\ge n^{1-\e_0}$, and use concentration to get the validation loss as
\be\label{eq_multival}\wt f(\hat B; \cal W)=  N\left[\val(\cal W) + t  \sigma^2 \right]\cdot \left( 1+\OO(p^{-(1-\e_0)/2+\e})\right).\ee
Here $\val(\cal W)$ is defined as
%it remains to consider minimizing the validation loss
	$$\val( \cal W):=\exarg{\varepsilon_j, \forall 1\le j\le t}{ \sum_{i=1}^t \bignorm{\Sigma^{1/2}( \hat B W_i - \beta_i)}^2} =  \delta_{\bias}(\cal W) + \delta_{\vari}(\cal W),$$
where the model shift bias term $\delta_{\bias}(\cal W) $ is given by
	\begin{align*}
		\delta_{\bias}(\cal W) :=\sum_{i=1}^t  \bignorm{\Sigma^{1/2}\bigbrace{(B^\star \cal W^\top) (\cal W\cal W^{\top})^{-1} W_i - \beta_i}}^2,
	\end{align*}
	and the variance term $\delta_{\vari}(\cal W)$ can be calculated as
	\begin{align*}
		\delta_{\vari}(\cal W):= \sigma^2 \cdot \bigtr{\Sigma (X^{\top}X)^{-1}}.
	\end{align*}
It suffices to minimize $\delta_{\bias}(\cal W)$ over $\cal W$, since both $t N \sigma^2$ and $\delta_{\vari}(\cal W)$ do not depend on the weights.

We denote $Q := \cal W^{\top} (\cal W\cal W^{\top})^{-1} \cal W \in\real^{k\times k}$, whose $(i,j)$-th entry is equal to $W_i^{\top} (\cal W\cal W^{\top})^{-1} W_j$.
	%Let $B^{\star} = [\beta_1, \beta_2, \dots, \beta_k] \in\real^{p \times k}$ denote the true model parameters.
	Now we can write $\delta_{\bias}(\cal W)$ succinctly as
	\begin{align*}
		\delta_{\bias}(\cal W)= \bignormFro{\Sigma^{1/2}B^{\star}  \bigbrace{Q -\id}}^2 .
	\end{align*}
	From this equation we can solve the minimizer optimally as $Q_0=U_{r}U_r^{\top}$. On the other hand, let $\hat {\cal W}$ be the minimizer of $\wt f$, and denote $\hat Q:= \hat{\cal W}^{\top} (\hat{\cal W}\hat{\cal W}^{\top})^{-1} \hat{\cal W} $. We claim that $\hat Q$ satisfies
	\be\label{Q-Q}\|Q_0^{-1}\hat Q - \id\| = \oo(1) \quad \text{whp}.\ee
	In fact, if \eqref{Q-Q} does not hold, then using the condition $\lambda_{\min}((B^{\star})^\top\Sigma B^{\star})\gtrsim \sigma^2$ and that $\delta_{\vari}(\cal W)=\OO(\sigma^2)$ by \eqref{eq_isometric}, we obtain that
	$$   \val(\hat {\cal W}) + t  \sigma^2 > (\val( {\cal W}_0) + t \sigma^2 )\cdot (1+\oo(1)) \ \Rightarrow \ \wt f(\hat B; \hat{\cal W})>\wt f(\hat B; {\cal W}_0),$$
	that is, $\hat {\cal W}$ is not a minimizer. This leads to a contradiction.

	In sum, we have solved that $\hat{\beta}_i^{\MTL}=B^{\star}\left( U_r U_r(i) +\oo(1)\right)$. Inserting it into the definition of the test loss, we get that
	\begin{align*}
		\te(\hat{\beta}_t^{\MTL}) &= \bignorm{\Sigma^{1/2} \left((B^\star \hat{\cal W}^\top) (\hat{\cal W}\hat{\cal W}^{\top})^{-1} \hat W_t - \beta_2 \right) }^2
		+ \sigma^2  \hat W_t^{\top} (\hat{\cal W}\hat{\cal W}^{\top})^{-1} \hat W_t \cdot \bigtr{\Sigma (X^{\top}X)^{-1}} \\
		&= \bignorm{\Sigma^{1/2} \bigbrace{B^{\star} U_r U_r(t)-\beta_2}}^2 + \oo\left(\|B^\star\|^2\right) + \sigma^2\norm{U_r(t)}^2 \bigtr{\Sigma (X^{\top}X)^{-1}} \cdot (1+\oo(1))\\
		&= \bignorm{\Sigma^{1/2} \bigbrace{B^{\star} U_r U_r(t)-\beta_2}}^2 + \frac{\sigma^2}{\rho-1}\norm{U_r(t)}^2 + \oo \left( \|B^\star\|^2 + \sigma^2\right),
	\end{align*}
	with high probability, where we used Lemma \ref{lem_minv} in the last step. Similar, by Lemma \ref{lem_minv} we have
	$$\te(\hat{\beta}_t^{\MTL})=\frac{\sigma^2}{\rho-1} \cdot \left( 1+\oo(1)\right).$$
Combining the above two estimates, we conclude the proof.
\end{proof}
%From the above we can obtain three conceptual insights that are consistent with Section \ref{sec_denoise} and \ref{sec_insight}.
%\begin{itemize}
%	\item The de-noising effect of multi-task learning.
%	\item Multi-task training vs single-task training can be either positive or negative.
%	\item Transfer learning is better than the other two. And the improvement over multi-task training increases as the model distances become larger.
%\end{itemize}

\section{Proof of ???}

The projection of $\beta_t$ onto the subspace spanned by $(\beta_1,\cdots, \beta_k)$ is 
$$\hat \beta_t = \sum_{i=1}^k\frac{\beta_t^\top \beta_i}{\|\beta_i\|} \beta_i .$$
On the other hand, the single task estimator $\hat\beta_i$ is given by 
$$\hat \beta_i = \beta_i + (X_i^\top X_i)^{-1}X_i \e_i.$$
Then the projection of $\beta_t$ onto the subspace spanned by $(\hat\beta_1,\cdots, \hat\beta_k)$ is 
$$ \hat \beta_t'= \sum_{i=1}^k\frac{\beta_t^\top \hat \beta_i}{\|\hat \beta_i\|} \hat \beta_i .$$
We now calculate the estimation error $ \exarg{\varepsilon_i, \forall 1\le i\le k} {\|\hat \beta_t - \hat\beta_t'\|^2}.$ Without loss of generality, we assume that $\|\beta_i\|^2 \sim 1$ for all $1\le i \le k$ and $\sigma^2\le n^{-c}$ for some constant $c>0$. Then we calculate that
\begin{align*}
\|\hat\beta_i\|^2 &= \|\beta_i\|^2 + 2\beta_t^\top\cdot (X_i^\top X_i)^{-1}X_i \e_i + \|(X_i^\top X_i)^{-1}X_i \e_i \|^2 .
\end{align*}
We can use concentration of random vector with independent entries, Lemma \ref{largedeviation}, to get that for any $\e>0$,
\be\label{largebetae}
\beta_t^\top (X_i^\top X_i)^{-1}X_i \e_i  \le p^{\e/2}\cdot \sigma \left[\beta_t^\top (X_i^\top X_i)^{-1}\beta_t \right]^{1/2}\le \sigma p^{-1/2+\e},  
\ee
with high probability, and 
\be\label{largebetae2}
\begin{split}
 \|(X_i^\top X_i)^{-1}X_i \e_i \|^2 - \sigma^2 \tr\left( (X_i^\top X_i)^{-1} \right)   \le p^{\e/2}\cdot \sigma^2 \left\{\tr\left[X_i^\top (X_i^\top X_i)^{-2}X_i\right]^2\right\}^{1/2} \\
 = p^{\e/2}\cdot \sigma^2 \left[ \tr (X_i^\top X_i)^{-2} \right]^{1/2} \le \sigma^2 p^{-1/2+\e},  
 \end{split}
\ee
with high probability, where we used Theorem \ref{LEM_SMALL} to get that
$$\beta_t^\top (X_i^\top X_i)^{-1}\beta_t  \le p^{-1+\e/2} \quad \text{and}\quad \tr \left[ (X_i^\top X_i)^{-2} \right]  \le p^{-1+\e/2} \quad \text{whp}.$$
Thus we get that with high probability,
\begin{align}
\|\hat\beta_i\|^2&= \|\beta_i\|^2 + \sigma^2\cdot \tr \left( (X_i^\top X_i)^{-1}\right) +\OO(\sigma p^{-1/2+\e}) \nonumber\\
&=  \|\beta_i\|^2 + \frac{\sigma^2}{c_i-1}\cdot \frac1p\tr \left(\Sigma_i^{-1}\right) +\OO(\sigma p^{-1/2+\e}) ,\label{betanormi}
\end{align}
where in the second step we used Lemma \ref{lem_minv}.

For simplicity we abbreviate $v_i:=(X_i^\top X_i)^{-1}X_i \e_i$. Then using \eqref{largebetae} and \eqref{betanormi} we can calculate that whp,
\begin{align*}
&\|\hat \beta_t - \hat\beta_t'\|^2= \left\|\sum_{i=1}^k\frac{\beta_t^\top \beta_i}{\|\beta_i\|} \beta_i - \sum_{i=1}^k\frac{\beta_t^\top (\beta_i + v_i)}{ \|\beta_i +v_i\| }(\beta_i + v_i) \right\|^2 \\
&=  \left\|\sum_{i=1}^k\left(\frac{\beta_t^\top \beta_i}{\|\beta_i\|}-\frac{\beta_t^\top \beta_i}{\|\beta_i+v_i\|} \right)\beta_i - \sum_{i=1}^k\frac{\beta_t^\top \beta_i }{ \|\beta_i+v_i\| } v_i - \sum_{i=1}^k\frac{\beta_t^\top v_i}{ \|\beta_i +v_i\| } (\beta_i+v_i) \right\|^2 \\
%& = \left\|\sum_{i=1}^k\frac{\beta_t^\top \beta_i}{\|\beta_i\|}\left(1-\frac{\|\beta_i\|}{(\|\beta_i\|^2 + 2\beta_i^\top v_i + \|v_i\|^2)^{1/2}} \right)\beta_i - \sum_{i=1}^k\frac{\beta_t^\top \beta_i }{ \|\beta_i\|+\OO(\sigma^2 + \sigma p^{-1/2+\e}) } v_i\right\|^2 + \OO\left(\sigma^2 p^{-1+2\e}\right) \\
%&= \left\|\sum_{i=1}^k\frac{\beta_t^\top \beta_i}{\|\beta_i\|}\frac{2\beta_i^\top v_i + \|v_i\|^2}{2\|\beta_i\|^2}\beta_i \right\|^2+\left\| \sum_{i=1}^k\frac{\beta_t^\top \beta_i }{ \|\beta_i\|} v_i\right\|^2 + \OO\left(\sigma^4+\sigma^2 p^{-1+2\e}\right)
&= \left\| \sum_{i=1}^k\frac{\beta_t^\top \beta_i }{ \|\beta_i\|} v_i\right\|^2 + \OO\left(\sigma^4+\sigma^2 p^{-1+2\e}\right).
\end{align*}
Thus taking expectation, we get that 
\begin{align*}
\exarg{\varepsilon_i, \forall 1\le i\le k} {\|\hat \beta_t - \hat\beta_t'\|^2}&=   \sum_{i=1}^k\frac{\left|\beta_t^\top \beta_i \right|^2}{ \|\beta_i\|^2 } \cdot \sigma^2 \tr \left( (X_i^\top X_i)^{-1}\right) + \OO\left(\sigma^4+\sigma^2 p^{-1+2\e}\right) \\
 &= \sum_{i=1}^k\frac{\left|\beta_t^\top \beta_i \right|^2}{ \|\beta_i\|^2 } \cdot \frac{\sigma^2}{c_i-1}\cdot \frac1p\tr \left(\Sigma_i^{-1}\right) + \OO\left(\sigma^4+\sigma^2 p^{-1/2+2\e}\right) .
\end{align*}
Note the first term is of order $\sigma^2$, and hence the second stem is an error term.




