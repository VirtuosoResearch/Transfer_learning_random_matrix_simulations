\section{Proof of Theorem \ref{thm_model_shift}}

The proof of Theorem \ref{thm_model_shift} involves two parts.

\paragraph{Part I: Bounding the bias from model shift.}
We relate the first term in equation \eqref{eq_te_model_shift} to $\Delta_{\beta}$.
\begin{proposition}\label{prop_model_shift}
	In the setting of Theorem \ref{thm_model_shift},
	denote by $K = (\hat{w}^2X_1^{\top}X_1 + X_2^{\top}X_1)^{-1}$, and
	\begin{align*}
		\delta_1 &= \hat{w}^2 \bignorm{\Sigma_2^{1/2} K X_1^{\top}X_1(\beta_s - \hat{w}\beta_t)}^2, \\
		\delta_2 &= n_1^2\cdot \hat{w}^2 \bignorm{\Sigma_2^{1/2}K\Sigma_1(\beta_s - \hat{w}\beta_t)}, \\
		\delta_3 &= n_1^2\cdot \hat{w}^2 \bignorm{\Sigma_1^{1/2} K \Sigma_2 K \Sigma_1^{1/2}} \cdot \bignorm{\Sigma_1^{1/2} (\beta_s - \hat{w}\beta_t)}^2.
	\end{align*}
	We have that
	\begin{align*}
		-2n_1^2\bigbrace{{2\sqrt{\frac{p}{n_1}}} + {\frac{p}{n_1}}} \delta_3
		\le  \delta_1 - \delta_2
		\le n_1^2\bigbrace{2\sqrt{\frac{p}{n_1}} + \frac{p}{n_1}}\bigbrace{2 + 2\sqrt{\frac{p}{n_1}} + \frac{p}{n_1}}\delta_3.
	\end{align*}
	For the special case when $\Sigma_1 = \id$ and $\beta_s - \beta_t$ is i.i.d. with mean $0$ and variance $d^2$, we further have
	\begin{align*}
		\bigbrace{1 - \sqrt{\frac{p}{n_1}}}^4 \Delta_{\beta}
		\le \bignorm{\Sigma_2^{1/2} (X_1^{\top}X_1 + X_2^{\top}X_2)^{-1}X_1^{\top}X_1(\beta_s - \beta_t)}^2.
	\end{align*}
\end{proposition}

\begin{proof}
	The proof follows by applying equation \eqref{eq_isometric}.
	Recall that $X_1^{\top}X_1 = \Sigma_1^{1/2}Z_1^{\top}Z_1\Sigma_1^{1/2}$.
	Denote by $\cE = Z_1^{\top}Z_1 - {n_1}\id$.
	Let
%	Let $\alpha = \bignorm{\Sigma_2^{1/2} K \Sigma_1 (\beta_s - \hat{w}\beta_t)}^2$.
	We have
	\begin{align}
%		& \bignorm{\Sigma_2^{1/2}(X_1^{\top}X_1 + X_2^{\top}X_2)^{-1}X_1^{\top}X_1(\beta_s - \hat{w}\beta_t)}^2 \nonumber \\
		\delta_1 = \delta_2 + {2\hat{w}^2n_1}(\beta_s - \hat{w}\beta_t)^{\top}\Sigma_1^{1/2}\cE\Sigma_1^{1/2}K \Sigma_2 K \Sigma_1 (\beta_s - \hat{w}\beta_t)
		+ \hat{w}^2\bignorm{\Sigma_2^{1/2} K \Sigma_1^{1/2}\cE \Sigma_1^{1/2}(\beta_s - \hat{w}\beta_t)}^2 \label{eq_lem_model_shift_1}
%		\le& n_1\bigbrace{{n_1^2}{} + \frac{2n_1}p(p + 2\sqrt{{n_1}p}) + (p + 2\sqrt{{n_1}p})^2} \alpha = n_1^2\bigbrace{1 + \sqrt{\frac{p}{n_1}}}^4 \alpha. \nonumber
	\end{align}
	Here we use the following on the second term in equation \eqref{eq_lem_model_shift_1}
	\begin{align*}
		& \bigabs{(\beta_s - \hat{w}\beta_t)^{\top} \Sigma_1^{1/2} \cE \Sigma_1^{1/2} K \Sigma_2 K \Sigma_1 (\beta_s - \hat{w}\beta_t)} \\
		= & \bigabs{\bigtr{\cE \Sigma_1^{1/2}K\Sigma_2 K \Sigma_1(\beta_s - \hat{w}\beta_t)(\beta_s - \hat{w}\beta_t)^{\top} \Sigma_1^{1/2}}} \\
		\le & \norm{\cE} \cdot \bignormNuclear{\Sigma_1^{1/2} K \Sigma_2 K \Sigma_1 (\beta_s - \hat{w}\beta_t) (\beta_s - \hat{w}\beta_t)^{\top} \Sigma_1^{1/2}} \\
		\le & n_1 \bigbrace{2\sqrt{\frac{p}{n_1}} + \frac{p}{n_1}} \cdot \bignormNuclear{\Sigma_1^{1/2} K \Sigma_2 K \Sigma_1 (\beta_s - \hat{w}\beta_t)(\beta_s - \hat{w}\beta_t)^{\top} \Sigma_1^{1/2}} \tag{by equation \eqref{eq_isometric}} \\
		\le   & n_1 \bigbrace{2\sqrt{\frac{p}{n_1}} + \frac{p}{n_1}} \bignorm{\Sigma_1^{1/2}K \Sigma_2 K \Sigma_1^{1/2}} \cdot \bignorm{\Sigma_1^{1/2}(\beta_s - \hat{w}\beta_t)}^2 \tag{since the matrix inside is rank 1}
	\end{align*}
	The third term in equation \eqref{eq_lem_model_shift_1} can be bounded with
	\begin{align*}
		\bignorm{\Sigma_2^{1/2}K\Sigma_1^{1/2}\cE\Sigma_1^{1/2}(\beta_s - \hat{w}\beta_t)}^2
		\le n_1^2 \bigbrace{2\sqrt{\frac{p}{n_1}} + \frac{p}{n_1}}^2 \bignorm{\Sigma_1^{1/2}K\Sigma_{2}K\Sigma_1^{1/2}} \cdot \bignorm{\Sigma_1^{1/2}(\beta_s - \hat{w}\beta_t)}^2.
	\end{align*}
	Combined together we have shown the right direction for $\delta_1 - \delta_2$.
	For the left direction, we simply note that the third term in equation \eqref{eq_lem_model_shift_1} is positive.
	And the second term is bigger than $-2n_1^2(2\sqrt{\frac{p}{n_1}} + \frac{p}{n_1}) \alpha$ using equation \eqref{eq_isometric}.
\end{proof}


%	\begin{align}
%		\left(\frac{n_2}{n_1+n_2}\frac1{a_1^2}- \frac1{n_1+n_2}\sum_{i=1}^p \frac{1}{ (a_1 + \lambda_i^2a_2)^2  }\right) a_3 -  \left(\frac1{n_1+n_2}\sum_{i=1}^p \frac{  \lambda_i^2 }{ (  a_1 + \lambda_i^2a_2)^2  }\right)a_4 &=  \frac1{n_1+n_2}\sum_{i=1}^p \frac{1 }{ (  a_1 + \lambda_i^2a_2)^2  } , \label{eq_a3} \\
%		\left( \frac{n_1}{n_1+n_2}\frac1{a_2^2} -  \frac1n\sum_{i=1}^p \frac{\lambda_i^4   }{  (a_1 + \lambda_i^2a_2)^2  }\right)a_4 -\left( \frac1 {n_1+n_2}\sum_{i=1}^p \frac{\lambda_i^2  }{  (a_1 + \lambda_i^2a_2)^2  }\right)a_3 &=   \frac1 {n_1+n_2}\sum_{i=1}^p \frac{\lambda_i^2 }{  (a_1 + \lambda_i^2a_2)^2  }. \label{eq_a4}
%	\end{align}
	%We have that
	%\begin{align*}
	%	&~ {\bignorm{\Sigma_2^{1/2} (X_1^{\top}X_1 + X_2^{\top}X_2)^{-1} X_1^{\top}X_1 %(\beta_s - \beta_t)}} \\
	%	&\le ~ \left[(\beta_s - \beta_t)^{\top}\Sigma_1^{1/2}M\frac{(1 + a_3)\id + a_4 M^{\top}M}{(a_1 + a_2 M^{\top}M)} M^{\top}\Sigma_1^{1/2} (\beta_s - \beta_t)\right]^{1/2} \\
	%	&+ \left\|M\frac{(1 + a_3)\id + a_4 M^{\top}M}{(a_1 + a_2 M^{\top}M)} M^{\top}\right\|_{op}^{1/2} \|\Sigma_1^{1/2} (\beta_s - \beta_t)\|_2 \left( 2\sqrt{\frac{p} {n_1}} + \frac{p}{n_1}\right),
	%\end{align*}
	%and
	%\begin{align*}
	%	&~ {\bignorm{\Sigma_2^{1/2} (X_1^{\top}X_1 + X_2^{\top}X_2)^{-1} X_1^{\top}X_1 (\beta_s - \beta_t)}} \\
	%	&\ge ~ \left[(\beta_s - \beta_t)^{\top}\Sigma_1^{1/2}M\frac{(1 + a_3)\id + a_4 M^{\top}M}{(a_1 + a_2 M^{\top}M)} M^{\top}\Sigma_1^{1/2} (\beta_s - \beta_t)\right]^{1/2} \\
	%	&- \left\|M\frac{(1 + a_3)\id + a_4 M^{\top}M}{(a_1 + a_2 M^{\top}M)} M^{\top}\right\|_{op}^{1/2} \|\Sigma_1^{1/2} (\beta_s - \beta_t)\|_2 \left( 2\sqrt{\frac{p} {n_1}} + \frac{p}{n_1}\right).
	%\end{align*}
	%Here in the error term $p$ can be replaced with the rank of $\Sigma_1$. Without any extra information, we can only use the fact the rank of $\Sigma_1$ is at most $p$.

\paragraph{Part II: The limit of $\bignorm{\Sigma_2^{1/2} (\hat{w}^2X_1^{\top}X_1 + X_2^{\top}X_2)^{-1}\Sigma_1 (\beta_s - \hat{w}\beta_t)}^2$ using random matrix theory.}
We consider the same setting as in previous subsection:
$$ X_1^{\top}X_1:=\Sigma_1^{1/2}  Z_1^T Z_1 \Sigma_1^{1/2} ,\quad X_2^{\top}X_2= \Sigma_2^{1/2}  Z_2^T Z_2 \Sigma_2^{1/2},$$
where $z_{ij}$, $1 \leq i \leq n_1+n_2\equiv n$, $1 \leq j \leq p$, are real independent random variables satisfying \eqref{eq_12moment}. For now, we assume that the random variables $z_{ij}$ are i.i.d. Gaussian, but we know that universality holds for generally distributed entries. Assume that $p/n_1$ is a small number such that $Z_1^TZ_1$ is roughly an isometry, that is, under \eqref{eq_12moment},
%\todo{
%\begin{align}\left\| Z_1^T Z_1 -  \frac{n_1}{n} I \right\| \le 2\sqrt{\frac{p}{n}} + {\frac{p}{n}} .
%\end{align}}
{\color{blue}
If we assume the variances of the entries of $Z_1$ are $1$, then we have
\begin{align}
- {n_1} \left(2\sqrt{\frac{p}{n_1}} - {\frac{p}{n_1}}\right)  \le {Z_1^T Z_1 -  n_1 \id}  \le {n_1} \left(2\sqrt{\frac{p}{n_1}} + {\frac{p}{n_1}}\right) . \label{eq_isometric}
\end{align}
}

\begin{lemma}\label{lem_cov_derivative}
	In the setting of Theorem \ref{thm_model_shift}, we have with high probability $1-{\rm o}(1)$,
\begin{equation}\label{lem_cov_derv_eq}
\begin{split}
&\wh w^{2}(n_1+n_2)^2\bignorm{\Sigma_2^{1/2} (\hat{w}X_1^{\top}X_1 + X_2^{\top}X_2)^{-1}\Sigma_1 (\beta_s - {w}\beta_t)}^2 \\
&= (\beta_s - {w}\beta_t)^{\top} \Sigma_1^{1/2} {M} \frac{(1 + a_3)\id + a_4 {M}^{\top}{M}}{(a_2 + a_1 {M}^{\top}{M})^2} {M}^{\top} \Sigma_1^{1/2} (\beta_s - \hat{w}\beta_t) +\OO(n^{-1/2+\e}),
\end{split}
\end{equation}
	for any constant $\epsilon>0$. %, where $a_{3,4}$ are found using equations in  \eqref{m35reduced}
\end{lemma}
We will give the proof of this lemma in Section \ref{sec_maintools}.

{\cor add some arguments with $\e$-net.}
%Then it is equivalent to study
%$$Q:=  ( X_1^T X_1 )^{-1}   D X_2^T X_2 D ,$$
%which has the same nonzero eigenvalues of
%$$\mathcal Q:=  X_2 D ( X_1^T X_1 )^{-1} D X_2^T,$$
%i.e. $\mathcal Q$ has the same nonzero eigenvalues of $Q$, but has $(n_2-p)$ more zero eigenvalues.

%----------The following is the results on the eigenvalues, not the singular values-------------
%We are interested in the eigenvalues of
%$$(X_1^{\top}X_1)^{-1} (X_2^{\top}X_2),$$
%which is called a generalized Fisher matrix. As in the previous setting, we write out their covariance explicitly and consider
%$$  (\Sigma_1^{1/2}  Z_1^T Z_1 \Sigma_1^{1/2})^{-1}   \Sigma_2^{1/2}  Z_2^T Z_2 \Sigma_2^{1/2} ,$$
%where $\Sigma_{1,2}$ are $p\times p$ deterministic covariance matrices, and $X_1=(x_{ij})_{1\le i \le n_1, 1\le j \le p}$ and $X_2=(x_{ij})_{n_1+1\le i \le n_1+n_2, 1\le j \le p}$ are $n_1\times p$ and $n_2 \times p$ random matrices, respectively, where the entries $x_{ij}$, $1 \leq i \leq n_1+n_2\equiv n$, $1 \leq j \leq p$, are real independent random variables satisfying \eqref{eq_12moment}.
%We can study it using the linearization matrix ({\color{red}for my own purpose right now})
%$$ H = \begin{pmatrix} - z I & X_2D & 0 \\ DX_2^T & 0 & X_1^T \\ 0 & X_1 & I \end{pmatrix}, \quad G(z):=H(z)^{-1}.$$
%Then the $(1,1)$-th block is equal to $\cal G(z):=(\mathcal Q-z)^{-1}$. We denote
%\begin{equation}\label{m1m}m_1(z):=\frac1n\tr \cal G(z), \quad m(z)= \frac1p \left[ nm_1(z) + \frac{n_2-p}{z}\right].\ee
%We can show that it satisfies the following self-consistent equations together with another $m_3(z)$:
%\begin{align}\label{m13}
%\frac{n_2}{n}\frac1{m_1} = - z +\frac1n\sum_{i=1}^p \frac{d_i^2}{ m_3 + d_i^2m_1  } ,\quad \frac{n_1}{n}\frac1{m_3} = 1 +\frac1n\sum_{i=1}^p \frac{1}{   m_3 + d_i^2m_1  } .
%\end{align}
%For these two equations, we can obtain one single equation for $m_1(z)$:
%\begin{align}\label{m1}
%m_3= 1-\frac pn + zm_1, \quad \frac{n_2}{n}\frac1{m_1} = - z +\frac1n\sum_{i=1}^p \frac{d_i^2}{ 1-\frac pn + (z + d_i^2)m_1  }  .
%\end{align}
%One can solve the above equation for $m_1$ with positive imaginary parts, and then calculate $m(z)$ using \eqref{m1m}.
%
%With $m(z)$, we can define
%$$\rho_c(z):=\frac1\pi \lim_{\eta\downarrow 0} m(z).$$
%It will be a compact supported probability density which gives the eigenvalue distribution of $Q$. Moreover, we have
%$$\frac{1}{p}\tr \frac1{(\Sigma_1^{1/2}  X_1^T X_1 \Sigma_1^{1/2})^{-1}   \Sigma_2^{1/2}  X_2^T X_2 \Sigma_2^{1/2} +1}\left(\frac1{(\Sigma_1^{1/2}  X_1^T X_1 \Sigma_1^{1/2})^{-1}   \Sigma_2^{1/2}  X_2^T X_2 \Sigma_2^{1/2} +1}\right)^T \approx \int \frac{\rho_c(x)}{(x+1)^2}{\rm d}x.$$
%We know that
%$$ m(z)=\int \frac{\rho_c(x)}{ x-z}{\rm d}x.$$
%Hence we have
%$$ m'(-1)=\int \frac{\rho_c(x)}{(x+1)^2}{\rm d}x.$$
%
%Moreover, the right edge $\lambda_+$ of $\rho_c$ gives the location of the largest eigenvalue, while the left edge $\lambda_-$  of $\rho_c$ gives the location of the smallest eigenvalue. They will provide the upper and lower bounds on the operator norm:
%$$\frac{1}{1+\lambda_+}\le \frac1{(\Sigma_1^{1/2}  X_1^T X_1 \Sigma_1^{1/2})^{-1}   \Sigma_2^{1/2}  X_2^T X_2 \Sigma_2^{1/2} +1}\le \frac1{1+\lambda_-}.$$
%The edges of the spectrum can be determined by solving the following equations of $(x,m_1)$:
%$$\frac{n_2}{n}\frac1{m_1} = - x +\frac1n\sum_{i=1}^p \frac{d_i^2}{ 1-\frac pn + (x + d_i^2)m_1  } ,\quad  \frac{n_2}{n}\frac1{m_1^2} = \frac1n\sum_{i=1}^p \frac{d_i^2 (x+d_i^2)}{\left[ 1-\frac pn + (x + d_i^2)m_1 \right]^2 } .$$
%The solutions for $x$ give the locations for the edges of the spectrum.
%--------------------------------------





%\section{The Case of Two Tasks}\label{sec_defspike}
%We denote task 1 as the source, i.e. $\beta_1 = \beta_s$.
%\subsection{Covariate Shift}
%\subsection{Model Shift}
