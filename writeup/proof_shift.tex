\section{Proofs of Theorem \todo{bla}}

The proofs of Theorem \ref{thm_model_shift} and Theorem \ref{thm_model_shift_neg} involve two parts.

\paragraph{Part I: Bounding the bias from model shift.}
We relate the first term in equation \eqref{eq_te_model_shift} to $\Delta_{\beta}$.
\begin{proposition}\label{prop_model_shift}
	In the setting of Theorem \ref{thm_model_shift}, we have that
	\begin{align*}
		\bigbrace{1 - {4\sqrt{\frac{p}{n_1}}} - {2\frac{p}{n_1}}} \Delta_{\beta}
		\le \bignorm{\Sigma_2^{1/2} (X_1^{\top}X_1 + X_2^{\top}X_2)^{-1}X_1^{\top}X_1(\beta_s - \beta_t)}^2
		\le \bigbrace{1 + \sqrt{\frac{p}{n_1}}}^4 \Delta_{\beta}.
	\end{align*}
	For the special case when $\Sigma_1 = \id$ and $\beta_s - \beta_t$ is i.i.d. with mean $0$ and variance $d^2$, we further have
	\begin{align*}
		\bigbrace{1 - \sqrt{\frac{p}{n_1}}}^4 \Delta_{\beta}
		\le \bignorm{\Sigma_2^{1/2} (X_1^{\top}X_1 + X_2^{\top}X_2)^{-1}X_1^{\top}X_1(\beta_s - \beta_t)}^2.
	\end{align*}
\end{proposition}

\begin{proof}
	The proof follows by applying equation \eqref{eq_isometric}.
	Recall that $X_1^{\top}X_1 = \Sigma_1^{1/2}Z_1^{\top}Z_1\Sigma_1^{1/2}$.
	Denote by $\cE = Z_1^{\top}Z_1 - {n_1}\id$.
	Let $K = (X_1^{\top}X_1 + X_2^{\top}X_1)^{-1}$.
	Let $\alpha = \bignorm{\Sigma_2^{1/2} K \Sigma_1 (\beta_s - \beta_t)}^2$.
	We have
	\begin{align}
		& \bignorm{\Sigma_2^{1/2}(X_1^{\top}X_1 + X_2^{\top}X_2)^{-1}X_1^{\top}X_1(\beta_s - \beta_t)}^2 \nonumber \\
		= &{n_1^2}{}\alpha + {2n_1}(\beta_s - \beta_t)^{\top}\Sigma_1^{1/2}\cE\Sigma_1^{1/2}K \Sigma_2 K \Sigma_1 (\beta_s - \beta_t)
		+ \bignorm{\Sigma_2^{1/2} K \Sigma_1^{1/2}\cE \Sigma_1^{1/2}(\beta_s - \beta_t)}^2 \label{eq_lem_model_shift_1}\\
		\le& n_1\bigbrace{{n_1^2}{} + \frac{2n_1}p(p + 2\sqrt{{n_1}p}) + (p + 2\sqrt{{n_1}p})^2} \alpha = n_1^2\bigbrace{1 + \sqrt{\frac{p}{n_1}}}^4 \alpha. \nonumber
	\end{align}
	Here we use the following on the second term in equation \eqref{eq_lem_model_shift_1}
	\begin{align*}
		& \bigabs{(\beta_s - \beta_t)^{\top} \Sigma_1^{1/2} \cE \Sigma_1^{1/2} K \Sigma_2 K \Sigma_1 (\beta_s - \beta_t)} \\
		= & \bigabs{\bigtr{\cE \Sigma_1^{1/2}K\Sigma_2 K \Sigma_1(\beta_s - \beta_t)(\beta_s - \beta_t)^{\top} \Sigma_1^{1/2}}} \\
		\le & \norm{\cE} \cdot \bignormNuclear{\Sigma_1^{1/2} K \Sigma_2 K \Sigma_1 (\beta_s - \beta_t) (\beta_s - \beta_t)^{\top} \Sigma_1^{1/2}} \\
		\le & n_1 \bigbrace{2\sqrt{\frac{p}{n_1}} + \frac{p}{n_1}} \cdot \bignormNuclear{\Sigma_1^{1/2} K \Sigma_2 K \Sigma_1 (\beta_s - \beta_t)(\beta_s - \beta_t)^{\top} \Sigma_1^{1/2}} \tag{by equation \eqref{eq_isometric}} \\
		=   & n_1 \bigbrace{2\sqrt{\frac{p}{n_1}} + \frac{p}{n_1}} \alpha \tag{since the matrix inside is rank 1}
	\end{align*}
	The third term in equation \eqref{eq_lem_model_shift_1} can be bounded similarly.

	For the other direction, we simply note that the third term in equation \eqref{eq_lem_model_shift_1} is positive.
	And the second term is bigger than $-2n_1^2(2\sqrt{\frac{p}{n_1}} + \frac{p}{n_1}) \alpha$ using equation \eqref{eq_isometric}.
\end{proof}


%	\begin{align}
%		\left(\frac{n_2}{n_1+n_2}\frac1{a_1^2}- \frac1{n_1+n_2}\sum_{i=1}^p \frac{1}{ (a_1 + \lambda_i^2a_2)^2  }\right) a_3 -  \left(\frac1{n_1+n_2}\sum_{i=1}^p \frac{  \lambda_i^2 }{ (  a_1 + \lambda_i^2a_2)^2  }\right)a_4 &=  \frac1{n_1+n_2}\sum_{i=1}^p \frac{1 }{ (  a_1 + \lambda_i^2a_2)^2  } , \label{eq_a3} \\
%		\left( \frac{n_1}{n_1+n_2}\frac1{a_2^2} -  \frac1n\sum_{i=1}^p \frac{\lambda_i^4   }{  (a_1 + \lambda_i^2a_2)^2  }\right)a_4 -\left( \frac1 {n_1+n_2}\sum_{i=1}^p \frac{\lambda_i^2  }{  (a_1 + \lambda_i^2a_2)^2  }\right)a_3 &=   \frac1 {n_1+n_2}\sum_{i=1}^p \frac{\lambda_i^2 }{  (a_1 + \lambda_i^2a_2)^2  }. \label{eq_a4}
%	\end{align}
	%We have that
	%\begin{align*}
	%	&~ {\bignorm{\Sigma_2^{1/2} (X_1^{\top}X_1 + X_2^{\top}X_2)^{-1} X_1^{\top}X_1 %(\beta_s - \beta_t)}} \\
	%	&\le ~ \left[(\beta_s - \beta_t)^{\top}\Sigma_1^{1/2}M\frac{(1 + a_3)\id + a_4 M^{\top}M}{(a_1 + a_2 M^{\top}M)} M^{\top}\Sigma_1^{1/2} (\beta_s - \beta_t)\right]^{1/2} \\
	%	&+ \left\|M\frac{(1 + a_3)\id + a_4 M^{\top}M}{(a_1 + a_2 M^{\top}M)} M^{\top}\right\|_{op}^{1/2} \|\Sigma_1^{1/2} (\beta_s - \beta_t)\|_2 \left( 2\sqrt{\frac{p} {n_1}} + \frac{p}{n_1}\right),
	%\end{align*}
	%and
	%\begin{align*}
	%	&~ {\bignorm{\Sigma_2^{1/2} (X_1^{\top}X_1 + X_2^{\top}X_2)^{-1} X_1^{\top}X_1 (\beta_s - \beta_t)}} \\
	%	&\ge ~ \left[(\beta_s - \beta_t)^{\top}\Sigma_1^{1/2}M\frac{(1 + a_3)\id + a_4 M^{\top}M}{(a_1 + a_2 M^{\top}M)} M^{\top}\Sigma_1^{1/2} (\beta_s - \beta_t)\right]^{1/2} \\
	%	&- \left\|M\frac{(1 + a_3)\id + a_4 M^{\top}M}{(a_1 + a_2 M^{\top}M)} M^{\top}\right\|_{op}^{1/2} \|\Sigma_1^{1/2} (\beta_s - \beta_t)\|_2 \left( 2\sqrt{\frac{p} {n_1}} + \frac{p}{n_1}\right).
	%\end{align*}
	%Here in the error term $p$ can be replaced with the rank of $\Sigma_1$. Without any extra information, we can only use the fact the rank of $\Sigma_1$ is at most $p$.

\paragraph{Part II: The limit of $\bignorm{\Sigma_2^{1/2} (X_1^{\top}X_1 + X_2^{\top}X_2)^{-1}\Sigma_1 (\beta_s - \beta_t)}^2$ using random matrix theory.}
We consider the same setting as in previous subsection:
$$ X_1^{\top}X_1:=\Sigma_1^{1/2}  Z_1^T Z_1 \Sigma_1^{1/2} ,\quad X_2^{\top}X_2= \Sigma_2^{1/2}  Z_2^T Z_2 \Sigma_2^{1/2},$$
where $z_{ij}$, $1 \leq i \leq n_1+n_2\equiv n$, $1 \leq j \leq p$, are real independent random variables satisfying \eqref{eq_12moment}. For now, we assume that the random variables $z_{ij}$ are i.i.d. Gaussian, but we know that universality holds for generally distributed entries. Assume that $p/n_1$ is a small number such that $Z_1^TZ_1$ is roughly an isometry, that is, under \eqref{eq_12moment},
%\todo{
%\begin{align}\left\| Z_1^T Z_1 -  \frac{n_1}{n} I \right\| \le 2\sqrt{\frac{p}{n}} + {\frac{p}{n}} .
%\end{align}}
{\color{blue}
If we assume the variances of the entries of $Z_1$ are $1$, then we have
\begin{align}
- {n_1} \left(2\sqrt{\frac{p}{n_1}} - {\frac{p}{n_1}}\right)  \le {Z_1^T Z_1 -  n_1 \id}  \le {n_1} \left(2\sqrt{\frac{p}{n_1}} + {\frac{p}{n_1}}\right) . \label{eq_isometric}
\end{align}
}

%Then we have
%$$  \left\|X_1^{\top}X_1 (\beta_s - \beta_t)  - \frac{n_1}{n}(\beta_s - \beta_t)\right\|_2 \le C \sqrt{\frac{p}{n}} \left\| (\beta_s - \beta_t)\right\|_2. $$
\todo{(revise the following proof)} It remain to study the following expression
\begin{align*}
\frac{1}{X_1^{\top}X_1 + X_2^{\top}X_2}  \Sigma_2 \frac{1}{X_1^{\top}X_1 + X_2^{\top}X_2}  = \Sigma_2^{-1/2}\left(\frac{1}{A^T Z_1^{\top}Z_1 A + Z_2^{\top}Z_2} \right)^2 \Sigma_2^{-1/2} \\
\stackrel{d}{=} \Sigma_2^{-1/2}V \left(\frac{1}{\Lambda Z_1^{\top}Z_1 \Lambda + Z_2^{\top}Z_2} \right)^2 V^T \Sigma_2^{-1/2},
\end{align*}
where
\be  \label{eigen2}
A:=\Sigma_1^{1/2}\Sigma_2^{-1/2} = U\Lambda V^T ,\quad \Lambda=\text{diag}(\lambda_1, \cdots, \lambda_p).
\ee
Using
$$\left(\frac{1}{\Lambda Z_1^{\top}Z_1 \Lambda + Z_2^{\top}Z_2} \right)^2 =\left. \frac{\dd }{\dd z}\right|_{z=0}\frac{1}{\Lambda Z_1^{\top}Z_1 \Lambda + Z_2^{\top}Z_2 - z} ,$$
we  need to study the resolvent of
$$G(z) = \left( \Lambda Z_1^{\top}Z_1 \Lambda + Z_2^{\top}Z_2 - z \right)^{-1}.$$
Its local law can be studied as in previous subsection (be careful we need to switch the roles of $Z_1$ and $Z_2$). More precisely,  we have that
$$ G(z) \approx \diag\left( \frac{1}{-z\left( 1+ m_3(z) + \lambda_i^2 m_4(z)\right)}\right)_{1\le i \le p}= \frac{1}{-z\left( 1+ m_3(z) + \Lambda^2 m_4(z)\right)} .$$
Here $m_{3,4}(z)$ satisfy the following self-consistent equations
%$$\frac{1}{G_{ii}} \approx -z \left( 1+m_3 + d_i^2m_4 \right), \quad \frac{1}{G_{\mu\mu}} = -z(1+m_1), \ \ \mu\in \cal I_1,\quad \frac{1}{G_{\nu\nu}} = -z(1+m_2), \ \ \nu\in \cal I_2,$$
%$$m_1= \frac1n\sum_{i}G_{ii}, \quad m_2= \frac1n\sum_{i}d_i^2 G_{ii}, \quad m_3 = \frac1n\sum_{\mu\in \cal I_1} G_{\mu\mu},\quad m_4 = \frac1n\sum_{\mu\in \cal I_2} G_{\mu\mu}.$$
\begin{align}\label{m34shift}
\frac{n_2}{n}\frac1{m_3} = - z +\frac1n\sum_{i=1}^p \frac1{  1+m_3 + \lambda_i^2m_4  } ,\quad \frac{n_1}{n}\frac1{m_4} = - z +\frac1n\sum_{i=1}^p \frac{\lambda_i^2 }{  1+m_3 + \lambda_i^2m_4  } .
\end{align}
Then we calculate the derivatives of $u_3:=zm_3$ and $u_4:=zm_4$ with respect to $z$:
\begin{align}\label{dotm34}
\frac{n_2}{n}\frac1{u_3^2}u'_3 =  \frac1n\sum_{i=1}^p \frac{1+u'_3 + \lambda_i^2 u'_4}{ ( z+u_3 + \lambda_i^2u_4)^2  } ,\quad \frac{n_1}{n}\frac1{u_4^2}u'_4 =   \frac1n\sum_{i=1}^p \frac{\lambda_i^2\left(1+u'_3 + \lambda_i^2 u'_4\right) }{  (z+u_3 + \lambda_i^2u_4)^2  } .
\end{align}
We can solve the above equations to get $u'_3$ and $u'_4$. Then we have
$$ G^2(z) \approx   \frac{1 +  u_3'(z) +\Lambda^2 u'_4(z)}{ \left( z+ u_3(z) + \Lambda^2  u_4(z)\right)^2}  $$
{\cob in certain sense}.

\cob $a_3\to a_2$, $a_4\to a_1$ \nc

We now simplify the expressions for $z\to 0$ case. When $z\to 0$, we shall have
$$u_3(z)= -  {a_3} + \OO(z), \quad u_4(z)= -  a_4 + \OO(z), \quad a_3,a_4 >0.$$
For $z\to0$, the equations in \eqref{m34shift} are reduced to
\begin{align}\label{m35shift}
\frac{n_2}{n}\frac{1}{\todo{a_2}} = 1 +\frac1n\sum_{i=1}^p \frac{1}{\todo{a_2} + \lambda_i^2\todo{a_1}  } ,\quad \frac{n_1}{n}\frac1{\todo{a_1}} = 1 +\frac1n\sum_{i=1}^p \frac{\lambda_i^2 }{  \todo{a_2} + \lambda_i^2 \todo{a_1} }.
\end{align}
It is easy to see that these equations are equivalent to
\begin{align} a_1 + a_2 = 1- \gamma_n, \quad a_1 +\frac1n\sum_{i=1}^p \frac{a_1}{a_1 + a_2/\lambda_i^2}=\frac{n_1}{n}  .\end{align}
The equations in \eqref{dotm34} reduce to
\be \label{dotm34red}
\begin{split}
\left(\frac{n_2}{n}\frac1{\todo{a_2}^2}- \frac1n\sum_{i=1}^p \frac{1 }{ (  \todo{a_2} + \lambda_i^2\todo{a_1})^2  }\right) a_3 -  \left(\frac1n\sum_{i=1}^p \frac{  \lambda_i^2 }{ (  \todo{a_2} + \lambda_i^2\todo{a_1})^2  }\right)a_4 =  \frac1n\sum_{i=1}^p \frac{1 }{ (  \todo{a_2} + \lambda_i^2\todo{a_1})^2  } ,\\
 %\frac{n_1}{n}\frac1{a_4^2}b_4 =   \frac1n\sum_{i=1}^p \frac{\lambda_i^2\left(1+b_3 + \lambda_i^2 b_4\right) }{  (a_3 + \lambda_i^2a_4)^2  } , \\
 \left( \frac{n_1}{n}\frac1{a_1^2} -  \frac1n\sum_{i=1}^p \frac{\lambda_i^4   }{  (a_2 + \lambda_i^2a_1)^2  }\right)a_4 -\left( \frac1n\sum_{i=1}^p \frac{\lambda_i^2  }{  (a_2 + \lambda_i^2a_1)^2  }\right)a_3 =   \frac1n\sum_{i=1}^p \frac{\lambda_i^2 }{  (a_2 + \lambda_i^2a_1)^2  } .
\end{split}
\ee
where we denote $b_3:=u_3'(0)$ and $b_4:=u_4'(0)$. Thus we have
$$\left(\frac{1}{\Lambda Z_1^{\top}Z_1 \Lambda + Z_2^{\top}Z_2} \right)^2  = G^2(0) \approx   \frac{ 1 + b_3 + \Lambda^2  b_4}{\left( a_3 + \Lambda^2  a_4\right)^2} .$$
This gives that
\be\label{derivative}
\begin{split}
& \frac{1}{X_1^{\top}X_1 + X_2^{\top}X_2}  \Sigma_2 \frac{1}{X_1^{\top}X_1 + X_2^{\top}X_2} =\Sigma_2^{-1/2}V \left(\frac{1}{\Lambda Z_1^{\top}Z_1 \Lambda + Z_2^{\top}Z_2} \right)^2 V^T \Sigma_2^{-1/2}\\
& \approx  \Sigma_2^{-1/2}V \frac{ 1 + b_3 + \Lambda^2  b_4}{\left( a_3 + \Lambda^2  a_4\right)^2}V^T \Sigma_2^{-1/2}= \Sigma_2^{-1/2}  \frac{ 1 + b_3 +  b_4 A^\top A }{\left( a_3 +   a_4 A^\top A\right)^2}\Sigma_2^{-1/2} .
\end{split}
\ee
Hence we have
\begin{align*}
& (\beta_s - \beta_t)^{\top}X_1^{\top}X_1 \frac{1}{X_1^{\top}X_1 + X_2^{\top}X_2}  \Sigma_2 \frac{1}{X_1^{\top}X_1 + X_2^{\top}X_2} X_1^{\top}X_1 (\beta_s - \beta_t) \\
 & \approx (\beta_s - \beta_t)^{\top}\Sigma_1 \Sigma_2^{-1/2}  \frac{ 1 + a_3 +  a_4 A^\top A }{\left( a_2 +   a_1 A^\top A\right)^2}\Sigma_2^{-1/2} \Sigma_1(\beta_s - \beta_t).
\end{align*}



%Then it is equivalent to study
%$$Q:=  ( X_1^T X_1 )^{-1}   D X_2^T X_2 D ,$$
%which has the same nonzero eigenvalues of
%$$\mathcal Q:=  X_2 D ( X_1^T X_1 )^{-1} D X_2^T,$$
%i.e. $\mathcal Q$ has the same nonzero eigenvalues of $Q$, but has $(n_2-p)$ more zero eigenvalues.

%----------The following is the results on the eigenvalues, not the singular values-------------
%We are interested in the eigenvalues of
%$$(X_1^{\top}X_1)^{-1} (X_2^{\top}X_2),$$
%which is called a generalized Fisher matrix. As in the previous setting, we write out their covariance explicitly and consider
%$$  (\Sigma_1^{1/2}  Z_1^T Z_1 \Sigma_1^{1/2})^{-1}   \Sigma_2^{1/2}  Z_2^T Z_2 \Sigma_2^{1/2} ,$$
%where $\Sigma_{1,2}$ are $p\times p$ deterministic covariance matrices, and $X_1=(x_{ij})_{1\le i \le n_1, 1\le j \le p}$ and $X_2=(x_{ij})_{n_1+1\le i \le n_1+n_2, 1\le j \le p}$ are $n_1\times p$ and $n_2 \times p$ random matrices, respectively, where the entries $x_{ij}$, $1 \leq i \leq n_1+n_2\equiv n$, $1 \leq j \leq p$, are real independent random variables satisfying \eqref{eq_12moment}.
%We can study it using the linearization matrix ({\color{red}for my own purpose right now})
%$$ H = \begin{pmatrix} - z I & X_2D & 0 \\ DX_2^T & 0 & X_1^T \\ 0 & X_1 & I \end{pmatrix}, \quad G(z):=H(z)^{-1}.$$
%Then the $(1,1)$-th block is equal to $\cal G(z):=(\mathcal Q-z)^{-1}$. We denote
%\begin{equation}\label{m1m}m_1(z):=\frac1n\tr \cal G(z), \quad m(z)= \frac1p \left[ nm_1(z) + \frac{n_2-p}{z}\right].\ee
%We can show that it satisfies the following self-consistent equations together with another $m_3(z)$:
%\begin{align}\label{m13}
%\frac{n_2}{n}\frac1{m_1} = - z +\frac1n\sum_{i=1}^p \frac{d_i^2}{ m_3 + d_i^2m_1  } ,\quad \frac{n_1}{n}\frac1{m_3} = 1 +\frac1n\sum_{i=1}^p \frac{1}{   m_3 + d_i^2m_1  } .
%\end{align}
%For these two equations, we can obtain one single equation for $m_1(z)$:
%\begin{align}\label{m1}
%m_3= 1-\frac pn + zm_1, \quad \frac{n_2}{n}\frac1{m_1} = - z +\frac1n\sum_{i=1}^p \frac{d_i^2}{ 1-\frac pn + (z + d_i^2)m_1  }  .
%\end{align}
%One can solve the above equation for $m_1$ with positive imaginary parts, and then calculate $m(z)$ using \eqref{m1m}.
%
%With $m(z)$, we can define
%$$\rho_c(z):=\frac1\pi \lim_{\eta\downarrow 0} m(z).$$
%It will be a compact supported probability density which gives the eigenvalue distribution of $Q$. Moreover, we have
%$$\frac{1}{p}\tr \frac1{(\Sigma_1^{1/2}  X_1^T X_1 \Sigma_1^{1/2})^{-1}   \Sigma_2^{1/2}  X_2^T X_2 \Sigma_2^{1/2} +1}\left(\frac1{(\Sigma_1^{1/2}  X_1^T X_1 \Sigma_1^{1/2})^{-1}   \Sigma_2^{1/2}  X_2^T X_2 \Sigma_2^{1/2} +1}\right)^T \approx \int \frac{\rho_c(x)}{(x+1)^2}{\rm d}x.$$
%We know that
%$$ m(z)=\int \frac{\rho_c(x)}{ x-z}{\rm d}x.$$
%Hence we have
%$$ m'(-1)=\int \frac{\rho_c(x)}{(x+1)^2}{\rm d}x.$$
%
%Moreover, the right edge $\lambda_+$ of $\rho_c$ gives the location of the largest eigenvalue, while the left edge $\lambda_-$  of $\rho_c$ gives the location of the smallest eigenvalue. They will provide the upper and lower bounds on the operator norm:
%$$\frac{1}{1+\lambda_+}\le \frac1{(\Sigma_1^{1/2}  X_1^T X_1 \Sigma_1^{1/2})^{-1}   \Sigma_2^{1/2}  X_2^T X_2 \Sigma_2^{1/2} +1}\le \frac1{1+\lambda_-}.$$
%The edges of the spectrum can be determined by solving the following equations of $(x,m_1)$:
%$$\frac{n_2}{n}\frac1{m_1} = - x +\frac1n\sum_{i=1}^p \frac{d_i^2}{ 1-\frac pn + (x + d_i^2)m_1  } ,\quad  \frac{n_2}{n}\frac1{m_1^2} = \frac1n\sum_{i=1}^p \frac{d_i^2 (x+d_i^2)}{\left[ 1-\frac pn + (x + d_i^2)m_1 \right]^2 } .$$
%The solutions for $x$ give the locations for the edges of the spectrum.
%--------------------------------------





%\section{The Case of Two Tasks}\label{sec_defspike}
%We denote task 1 as the source, i.e. $\beta_1 = \beta_s$.
%\subsection{Covariate Shift}
%\subsection{Model Shift}
