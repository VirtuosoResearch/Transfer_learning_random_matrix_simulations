
\begin{abstract}
When does multi-task learning outperform single-task learning?
In this work, we address this question by studying the test performance of predicting a particular task given multiple tasks using a commonly used hard parameter sharing architecture.
In the setting of high-dimensional linear regression, we provide a sharp analysis of the bias-variance tradeoff for multi-task learning estimators.
A key technical tool that we develop is the trace of $(X_1^{\top}X_1 + X_2^{\top}X_2)^{-1}$ for two random matrices $X_1$ and $X_2$ in the setting of two tasks.
	Based on the theory, we provide more precise interpretations of many empirical phenomena in multi-task learning.
	For example, we theoretically analyze the benefit of multi-task learning for reducing the amount of labeled data needed to achieve comparable performance to single-task learning. %, which has been a key empirical finding in recent work.
	Finally, we show practical implications of our theory for detecting and mitigating negative effects in image and text classification tasks.
\end{abstract}

\section{Introduction}

%Multi-task learning is an inductive learning mechanism to improve generalization performance using related task data.
%Many state-of-the-art results in computer vision and natural language processing are obtained using multi-task learning.
Multi-task learning represents a powerful paradigm to solve complex prediction tasks in computer vision \cite{chexnet17,ZSSGM18}, natural language processing \cite{GLUE,superglue} and numerous other areas \cite{ZY17}.
%In multi-task learning, having related task data is fundamental to its performance.
%Multi-task learning is particularly powerful when there is limited labeled data for a task to be solved, meanwhile more labeled data from different but related tasks is available.
By combining multiple information sources, it is possible to share all the information in the same model.
%For example, many applications in , and many other areas have been achieved by learning from multiple tasks together.
The performance of multi-task learning depends on the relationship of the information sources \cite{C97}.
When the information sources are heterogeneous, negative transfer -- where multi-task learning performs worse than single-task learning -- has often been observed \cite{AP16,BS17}.
While numerous studies have sought to alleviate negative transfer \cite{ZY17}, a rigorous understanding of the contributing causes of negative transfer has remained elusive in the literature \cite{R17}.
%This phenomenon, known as \textit{negative transfer}, is fundamental to the understanding of multi-task learning.
In this work, we develop technical tools to compare multi-task learning to single-task learning for the setting of high-dimensional linear regression. % for learning from multiple linear regression tasks.
We use the tools to explain negative transfer and provide implications for positive transfer.
Based on the theory, we show practical implications for detecting and mitigating negative transfer.
%We consider a setting where the target task has limited labeled data and show
%On the other hand, unless the structures across task data are well-understood, applying multi-task learning on several different datasets often result in suboptimal models (or negative transfer in more technical terms).

Identifying negative transfer requires developing tight generalization bounds for both multi-task learning and single-task learning.
In classical Rademacher or VC based theory of multi-task learning \cite{B00,AZ05,M06}, the generalization bounds are usually presented so that the error reduces as the data sizes of all tasks increase.
For instance, the data sizes of all tasks are assumed to be equal \cite{B00}.
On the other hand, uneven data sizes or dominating tasks have been empirically observed as a cause of negative transfer \cite{YKGLHF20}.
More recent work has shown the benefit of learning multi-task representations for certain half-spaces \cite{MPR16} and sparse regression \cite{LPTV09,LPVT11}.
To rigorously study negative transfer, the technical challenge is to develop generalization bounds that scale tightly with properties of the data.


In this work, we consider a setting where multiple labeled linear regression tasks are available and focus on predicting a particular task whose amount of labeled data is limited.
Following Hastie et al. \cite{HMRT19} and Bartlett et al. \cite{BLLT20}, we assume that for every task $1\le i\le t$, its features are random vectors $x = \Sigma_i^{1/2}z$, where $z\in\real^p$ consists of i.i.d. entries with mean zero and unit variance, and $\Sigma_i\in\real^{p\times p}$ is a positive semidefinite matrix.
Let $n_i$ denote the data size and $X_i\in\real^{n_i\times p}$ denote the features of task $i$, for every $1\le i\le t$.
The label of task $i$ is given by $Y_i = X_i\beta_i + \varepsilon_i$, where $\beta_i\in\real^p$ denotes the ground truth parameters for task $i$ and $\varepsilon_i$ denotes i.i.d. random noise with mean zero and variance $\sigma^2$.
Without loss of generality, let the $t$-th task denote the target task.
We assume that the data size of task $t$ is a small constant $\rho_t > 1$ times $p$ to capture the need for more labeled data.
%We shall assume that each task data follows a linear model, i.e. $y_i = X_i \beta_i + \varepsilon_i$, $1\le i\le k$.
%Here $\beta_i\in\real^p$ is the model parameter for the $i$-th task.
%Each row of $X_i\in\real^{n_i\times p}$ is assumed to be drawn i.i.d. from a fixed
%distribution with covariance matrix $\Sigma_i$.

We combine all the labeled data using a hard parameter sharing architecture that contains a shared body $B\in\real^{p\times r}$ for all tasks and a separate prediction head $\set{W_i \in \real^{r}}_{i=1}^t$ for each task \cite{R17,MTDNN19,WZR20}.
This corresponds to minimizing the following objective.
\begin{align}
	\label{eq_mtl}
	f(B; W_1, \dots, W_t) = \sum_{i=1}^t \norm{X_i B W_i - Y_i}^2.
\end{align}
For a target task $t$,
let $\hat{\beta}_t^{\MTL} = B W_t$ denote the optimal multi-task estimator by minimizing equation \eqref{eq_mtl}.
Let $\hat{\beta}_t^{\STL}$ denote the standard linear regression estimator using task $t$ alone.
We say there is negative transfer if the test error of $\hat{\beta}_t^{\MTL}$  is larger than that of $\hat{\beta}_t^{\STL}$, or positive transfer otherwise.

\textbf{Main results.}
We revisit the bias-variance tradeoff of the multi-task estimator.
Interestingly, we observe that the variance of $\hat{\beta}_t^{\MTL}$ is always smaller than that of $\hat{\beta}_t^{\STL}$, hence resulting in a positive effect of variance reduction.
The bias of $\hat{\beta}_t^{\MTL}$, which we term as \textit{model shift bias}, results in a negative effect caused by the difference between $\beta_t$ and the rest $\set{\beta_i}_{i=1}^{t-1}$.
Hence, the tradeoff between the variance reduction effect and model shift bias determines whether the transfer is positive.

To quantify the tradeoff more precisely, we develop a new technical result that provides a tight bound on the trace of $(X_1^{\top}X_1 + X_2^{\top}X_2)^{-1}$, for the setting of two tasks with general covariance matrices.
Our result extends a well-known result on the trace of $(X_1^{\top}X_1)^{-1}$ for a single random matrix \cite{S07}.
Furthermore, we provide a nearly optimal error bound, which may be of independent interest.

Using the result, we provide a sharp analysis of the bias-variance tradeoff of $\hat{\beta}_t^{\MTL}$ for the setting of two tasks.
For more than two tasks, we show a similar result if all tasks have the same features.

Finally, we apply our technical tools to the related setting of transfer learning.
We study the transfer function used by Taskonomy \cite{ZSSGM18}, which pools learnt representations from source tasks into a shared body similar to $B$ in equation \eqref{eq_mtl}.
We find that the model shift bias can be captured by the orthogonal projection of $\beta_t$ to the subspace spanned by $\set{\beta_i}_{i=1}^{t-1}$.
These results are presented more precisely in Section \ref{sec_main}.

The tradeoff that we develop in Section \ref{sec_main} depends on the properties of each task such as its data size and covariance matrix.
In Section \ref{sec_insight}, we use the technical results to provide guidance on when positive transfer is likely to occur and identify causes of negative transfer.
\squishlist
		\item First, we provide a more precise interpretation to a folklore understanding in multi-task learning, which posits that negative transfer is caused by having dissimilar tasks.
		For an example of two tasks, we show a sharp transition from positive to negative transfer determined by the ratio of $\norm{\beta_1 - \beta_2}^2$ and a certain function of data sizes (cf. Proposition \ref{prop_dist_transition}).
		We further show that positive transfer is more likely to occur when transferring from a less noisy source task to a more noisy target task.
		In Section \ref{sec_validate}, we validate the observation on text and image classification tasks.
%	In , we provide the trade-off between $\norm{\beta_1 - \beta_2}^2$ and a certain function $\Phi(\rho_1, \rho_2)$ to determine the type of transfer.
		\item Second, we find that depending on how large $\norm{\beta_1 - \beta_2}^2$ is, adding more labeled data from the source task does not always reduce the test error of $\hat{\beta}_t^{\MTL}$ (cf. Proposition \ref{prop_data_size}).
	We connect the observation to explain a key finding of Taskonomy \cite{ZSSGM18}.
	We define the \textit{data efficiency ratio} as the smallest $\alpha\in(0, 1)$ such that if we only use an $\alpha$ fraction of labeled data, then the test error of $\hat{\beta}_t^{\MTL}$ matches that of the $\hat{\beta}_t^{\STL}$ on the entire set.
	In Proposition \ref{prop_data_efficiency}, we show that the data efficiency ratio of an illustrative example is at most $\alert{\frac {1}{2\rho_t}}$.
	In Section \ref{sec_validate}, we validate that performing multi-task learning can reduce the need for labeled data on 6 sentiment analysis tasks. %, we observe that just by using \alert{40\%} of the labeled data, the overall accuracy of multi-task learning matches that of learning every task in isolation.
		\item Finally, we find that covariate shift, i.e. having different covariance matrices, is another cause for suboptimal performance for $\hat{\beta}_t^{\MTL}$.
		We show that as $n_1 / n_2$ becomes large, the best performing source task has have the same covariance matrix as the target task (cf. Proposition \ref{prop_covariate}).
%		On the other hand, when $n_1 / n_2$ is small, there are counter examples where having the same covariance matrix is not necessarily the optimal choice.
\squishend

%	A crucial technical tool that we develop for showing Theorem \ref{thm_main_informal} is the asymptotic limit of the trace of $(X_1^{\top}X_1 + X_2^{\top}X_2)^{-1}$ (for the setting of two tasks), which may be of independent interest.

We show practical implications of our theory on text and image classification tasks.
	First, we provide a metric to determine positive versus negative transfer by comparing the test accuracies of single-task models.
	Second, \todo{}
	Finally, we show that as the data size between the source and target task becomes more imbalanced, aligning the covariances of the tasks becomes more beneficial.
% as the data size becomes more imbalanced between the source and target task, having mis-aligned task covariance matrices can result in suboptimal performance.
	%validate on text and image classification tasks that comparing single-task accuracies can help determine whether multi-task learning performs better than single-task learning.


%	On the other hand, if the number of source task datapoints is comparable to the target task, aligning task covariances may hurt performance.


