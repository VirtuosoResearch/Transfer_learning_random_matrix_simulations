\subsection{The Transfer Learning Setting}\label{app_proof_sec4}

\begin{theorem}\label{prop_taskonomy}
%For $i=1,\cdots, t$, let $Y_i = X_i\beta_i + \varepsilon_i$ be independent data models, where $X_i$, $\beta_i$ and $\varepsilon_i$ are also independent of each other. Suppose $X_i=Z_i\Sigma_i^{1/2}\in \R^{n_i\times p}$ satisfy Assumption \ref{assm_secA1} with $\rho_i:=n_i/p>1$ being fixed constants, and $\e_i\in \R^{n_i}$ are  random vectors with i.i.d. entries with mean zero, variance $\sigma^2$ and all moments as in \eqref{assmAhigh}.  
Consider $t$ data models $Y_i = X\beta_i + \varepsilon_i$, $i=1,2,\cdots, t$, that satisfy Assumption \ref{assm_secA2}.
Moreover, assume that $\max_{1\le i \le t}\|\beta_i\|=\OO(1)$, $\sigma^2=\OO(1)$, and 
\be\label{BTBassm}
\left\|\left[(B^\star_{t-1})^\top B^\star_{t-1}\right]^{-1}\right\|=\OO(1),\quad B_{t-1}^{\star}: = [{\beta}_1,{\beta}_2,\dots,{\beta}_{t-1}].
\ee
Then for any constant $\e>0$, we have 
\be\label{eq_taskonomy}
\te(B W_t) =\|\e_\beta + \sigma^2 \e_{\vari}^{(1)}\|^2+ \sigma^2 \|\e_{\vari}^{(2)}\|^2 + \OO(p^{-1/2+\e})
\ee
with high probability, where 
\begin{align*}
\e_\beta &:= \Sigma_t^{1/2}\left\{1 - B^\star_{t-1}\left[(B^\star_{t-1})^\top B^\star_{t-1}\right]^{-1} (B^\star_{t-1})^\top\right\} \beta_t,\\
\e_{\vari}^{(1)}&:=  \Sigma_t^{1/2}  B^\star_{t-1}\left[(B^\star_{t-1})^\top B^\star_{t-1}\right]^{-1}\cal M_1 \left[(B^\star_{t-1})^\top B^\star_{t-1} + \sigma^2\cal M_1\right]^{-1} (B^\star_{t-1})^\top \beta_t,\\
\e_{\vari}^{(2)}&:= \cal M_2^{1/2}\left[(B^\star_{t-1})^\top B^\star_{t-1} + \sigma^2\cal M_1\right]^{-1} (B^\star_{t-1})^\top \beta_t.
\end{align*}
Here $\cal M_1$ and $\cal M_2$ are $(t-1)\times (t-1)$ diagonal matrices with 
$$(\cal M_1)_{ii}=\frac{1}{\rho_i-1}\cdot \frac1p\tr \left(\Sigma_i^{-1}\right),\quad \text{and}\quad (\cal M_2)_{ii}=\frac{1}{\rho_i-1}\cdot \frac1p\tr \left(\Sigma_i^{-1}\Sigma_t\right).$$
\end{theorem}

\begin{proof}[Proof of Theorem \ref{prop_taskonomy}]
 For each task $i$, $1\le i \le t-1$, we can find that  
$$\hat \beta_i = \beta_i + (X_i^\top X_i)^{-1}X_i^\top \e_i.$$
We first calculate $B^\top B$, where recall that $B= [\hat{\beta}_1,\hat{\beta}_2,\dots,\hat{\beta}_{t-1}]$.

For the diagonal entries of $B^\top B$ we can calculate that
\begin{align*}
\|\hat\beta_i\|^2 &= \|\beta_i\|^2 + 2\beta_i^\top (X_i^\top X_i)^{-1}X_i^\top \e_i + \|(X_i^\top X_i)^{-1}X_i^\top \e_i \|^2 .
\end{align*}
Using concentration of random vector with i.i.d. centered entries, Lemma \ref{largedeviation}, we get that for any deterministic vector $\beta$ with $\|\beta\|=\OO(1)$,
\be\label{largebetae}
\beta^\top (X_i^\top X_i)^{-1}X_i^\top \e_i  \le p^{\e/2}\cdot \sigma \left[\beta^\top (X_i^\top X_i)^{-1}\beta \right]^{1/2}\le \sigma p^{-1/2+\e},  
\ee
with high probability, where we used \eqref{eq_isometric} to bound 
\be\label{opXiXi}\| (X_i^\top X_i)^{-1}\|\lesssim n_i^{-1} \quad \text{whp}. \ee
Similarly, using Lemma \ref{largedeviation} we get that with high probability,
\be\label{largebetae2}
\begin{split}
 \left|\|(X_i^\top X_i)^{-1}X_i^\top \e_i \|^2 - \sigma^2 \tr\left[ (X_i^\top X_i)^{-1} \right]\right|   \le p^{\e/2}\cdot \sigma^2 \tr^{1/2}\left\{\left[X_i (X_i^\top X_i)^{-2}X_i^\top\right]^2\right\} \\
 = p^{\e/2}\cdot \sigma^2 \tr^{1/2} \left[  (X_i^\top X_i)^{-2} \right]\le \sigma^2 p^{-1/2+\e},
 \end{split}
\ee 
and for $i\ne j$,
\be \label{largebetae3}
\begin{split}
\left| \e_j^\top X_j (X_j^\top X_j)^{-1}(X_i^\top X_i)^{-1}X_i^\top \e_i \right| &\le   p^{\e/2}\cdot \sigma^2 \tr^{1/2} \left[(X_i^\top X_i)^{-1}(X_j^\top X_j)^{-1}\right]\\
&\le \sigma^2 p^{-1/2+\e}.  
 \end{split}
\ee
%where in the we used Theorem \ref{LEM_SMALL} to get that
%$$\beta_t^\top (X_i^\top X_i)^{-1}\beta_t  \le p^{-1+\e/2} \quad \text{and}\quad \tr \left[ (X_i^\top X_i)^{-2} \right]  \le p^{-1+\e/2} \quad \text{whp}.$$
With \eqref{largebetae}-\eqref{largebetae3}, we get that with high probability,
\begin{align}\label{betanormij}
\hat\beta_i^\top \hat \beta_j &= \beta_i^\top \beta_j + \OO(\sigma p^{-1/2+\e})  , \quad i\ne j,
\end{align}
and
\begin{align}
\|\hat\beta_i\|^2&= \|\beta_i\|^2 + \sigma^2 \tr \left[ (X_i^\top X_i)^{-1}\right] +\OO(\sigma p^{-1/2+\e}) \nonumber\\
&=  \|\beta_i\|^2 + \frac{\sigma^2}{\rho_i-1}\cdot \frac1p\tr \left(\Sigma_i^{-1}\right) +\OO(\sigma p^{-1/2+\e}) ,\label{betanormi}
\end{align}
where we used Lemma \ref{lem_minv} in the second step. %where in the second step we used Lemma \ref{lem_minv}. Similarly, we have

With \eqref{betanormij} and \eqref{betanormi}, we obtain that with high probability,
\be\label{BTB}
B^\top B= (B^\star_{t-1})^\top B^\star_{t-1} + \sigma^2\cal M_1 +\OO(\sigma p^{-1/2+\e}),
\ee
where $\OO(\sigma p^{-1/2+\e})$ means a $(t-1)\times (t-1)$ matrix, say $\cal E$, satisfying $\|\cal E\|\le C\sigma p^{-1/2+\e}$. Notice that by \eqref{BTBassm}, we have that 
$$\|(B^\top B)^{-1}\| = \OO( 1) \quad \text{whp}.$$
Moreover, using \eqref{largebetae} we get 
\be\label{BTBeta}
B^\top \beta =  (B^\star_{t-1})^\top \beta + \OO(\sigma p^{-1/2+\e}),
\ee
for any deterministic vector $\beta$ with $\|\beta\|=\OO(1)$.

Now we are ready to calculate $\te(B W_t) $ using the above estimates. For the $t$-th target model, by optimizing over $W_t$ we get
$$\hat W_t= (B^\top X_t^\top X_t B)^{-1} B^\top X_t^\top Y_k =(B^\top X_t^\top X_t B)^{-1} B^\top X_t^\top (X_t \beta_t + \e_t) .$$
We then calculate that 
\be
\begin{split}
\exarg{\varepsilon_t} {\left\|\Sigma_t^{1/2}(B\hat W_t -\beta_t)\right\|^2} &= \left\| \Sigma_t^{1/2}\left[\id - B(B^\top X_t^\top X_t B)^{-1} B^\top X_t^\top X_t\right] \beta_t \right\|^2 \\
&+\sigma^2\cdot \tr \left[\Sigma_t (B^\top X_t^\top X_t B)^{-1}\right]. \label{EBWk}
%\\&= \sigma^2\cdot \tr (B^\top X_t^\top X_t B)^{-1} + \left\|\beta_t\right\|^2 - 2 \beta_t^\top B(B^\top X_t^\top X_t B)^{-1} B^\top X_t^\top X_t \beta_t \nonumber\\
%&+ \beta_t^\top X_t^\top X_t B(B^\top X_t^\top X_t B)^{-1}B^\top B (B^\top X_t^\top X_t B)^{-1}B^\top X_t^\top X_t \beta_t.\label{EBWk}
\end{split}
\ee
Using the restricted isometry property for $X_t$ on the subspace spanned by the column vectors of $B$, we get that with high probability, 
$$ B^\top X_t^\top X_t B = n_t  \left( B^\top B + \OO(p^{-1/2+\e})\right),\quad B^\top X_t^\top X_t \beta_t = n_t  \left( B^\top \beta_t  + \OO(p^{-1/2+\e})\right).$$
Together with \eqref{opXiXi}, \eqref{BTB} and \eqref{BTBeta}, we can simplify the right-hand side of \eqref{EBWk} as
\begin{align}
&\exarg{\varepsilon_t} {\left\|\Sigma_t^{1/2} (B\hat W_t -\beta_t)\right\|^2} =\left\|\Sigma_t^{1/2} \beta_t - \Sigma_t^{1/2} B(B^\top B)^{-1} B^\top \beta_t \right\|^2 + \OO(p^{-1/2+\e}) \nonumber\\
&=\left\|\Sigma_t^{1/2} \beta_t - \Sigma_t^{1/2} B\left[ (B^\star_{t-1})^\top B^\star_{t-1} + \sigma^2\cal M_1 \right]^{-1}(B_{t-1}^{\star})^\top \beta_t \right\|^2 + \OO(p^{-1/2+\e}). \label{EBWk2}
\end{align}
Then as in \eqref{BTB}, we can show by concentration and Lemma \ref{lem_minv} that 
\be\label{BTB2}
B^\top \Sigma_t B=  (B_{t-1}^{\star})^\top \Sigma_t B_{t-1}^{\star}+\sigma^2 {\cal M}_2 +\OO(\sigma p^{-1/2+\e}).
\ee
Combining \eqref{BTB}, \eqref{BTBeta} and \eqref{BTB2}, we can further take the expectation and simplify \eqref{EBWk2} as
\begin{align*}
&\exarg{\varepsilon_i, \forall 1\le i\le t} {\left\|\Sigma_t^{1/2} (B\hat W_t -\beta_t)\right\|^2}\\
&= \left\|\Sigma_t^{1/2} \beta_t\right\|^2 - 2\beta_t^\top \Sigma_t B_{t-1}^{\star} \wt\beta_t + \wt\beta_t^\top \left((B_{t-1}^{\star})^\top \Sigma_t B_{t-1}^{\star}+\sigma^2 {\cal M}_2 \right) \wt\beta_t + \OO(p^{-1/2+\e}).
%& =\|\e_\beta + \sigma^2 \e_{\vari}^{(1)}\|^2+ \sigma^2 \|\e_{\vari}^{(2)}\|^2 + \OO(p^{-1/2+\e}),
\end{align*}
where we abbreviated 
$$\wt\beta_t:=\left[(B_{t-1}^{\star})^\top B_{t-1}^{\star} +\sigma^2\cal M_1\right]^{-1}  (B_{t-1}^{\star})^\top \beta_t .$$
Finally, rearranging terms and performing some basic calculations we can conclude the proof.
\end{proof}

