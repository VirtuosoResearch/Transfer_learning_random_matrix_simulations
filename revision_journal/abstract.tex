\iffalse
\begin{frontmatter}

\title{Precise Asymptotics of Transferring from Different Regression in High Dimensions}

\runtitle{Precise Asymptotics of Transferring from Different Regression}

\begin{aug}
	\author[A]{\fnms{Fan} \snm{Yang}\ead[label=e1]{}},
	\author[B]{\fnms{Hongyang R.} \snm{Zhang}\ead[label=e2]{}},
	\author[C]{\fnms{Sen} \snm{Wu}\ead[label=e3]{}},\\
	\author[A]{\fnms{Weijie J.} \snm{Su}\ead[label=e4]{}},
	\and
	\author[C]{\fnms{Christopher} \snm{R\'e}\ead[label=e5]{}}
	\address{${ }^{1}$University of Pennsylvania, ${ }^{2}$Northeastern University, ${ }^{3}$Stanford University
}
\end{aug}
\fi

\begin{abstract}
	Hard parameter sharing for multi-task learning is widely used in empirical research despite the fact that its generalization properties have not been well established in many cases. This paper studies its generalization properties in a fundamental setting: How does hard parameter sharing work given multiple linear regression tasks? We develop new techniques and establish a number of new results in the high-dimensional setting, where the sample size and feature dimension increase at a fixed ratio. First, we show a sharp bias-variance decomposition of hard parameter sharing, given multiple tasks with the same features. Second, we characterize the asymptotic bias-variance limit for two tasks, even when they have arbitrarily different sample size ratios and covariate shifts. We also demonstrate that these limiting estimates for the empirical loss are incredibly accurate in moderate dimensions. Finally, we explain an intriguing phenomenon where increasing one task's sample size helps another task initially by reducing variance but hurts eventually due to increasing bias. This suggests progressively adding data for optimizing hard parameter sharing, and we validate its efficiency in text classification tasks.
\end{abstract}

%\begin{keyword}[class=MSC2020]
%\kwd[Primary ]{62J05}
%\kwd{60B20}
%\kwd[; secondary ]{62E20, 62H10}
%\end{keyword}

%\begin{keyword}
%\kwd{Transfer learning, random matrix theory, covariate shift, sample covariance matrices.}
%\end{keyword}


%\end{frontmatter}
%%%%%%%%%%%%%%%%%%%%%%%%%%%%%%%%%%%%%%%%%%%%%%
%% Please use \tableofcontents for articles %%
%% with 50 pages and more                   %%
%%%%%%%%%%%%%%%%%%%%%%%%%%%%%%%%%%%%%%%%%%%%%%
%\tableofcontents


