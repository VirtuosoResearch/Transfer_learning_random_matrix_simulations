\section{Bias-variance Limits: Different Sample Sizes and Covariate Shifts}\label{sec_diff}

The previous section assumes that all tasks have the same sample size and feature vectors.
This section discusses how having different sample sizes and different covariance matrices impact hard parameter sharing.
%The different covariates case differs from the same covariates case in two aspects.
%First, different tasks may have different sample sizes. In extreme scenarios, one task may have much less labeled data compared to another task.
The setting where covariates differ across tasks is often known as ``covariate shift''.

Unlike the previous section, we can no longer characterize the global minimum of $f(A, B)$.
This is because $f(A, B)$ is in general non-convex.
Instead, our result implies sharp bias-variance tradeoffs for any \emph{local minimizer} of $f(A, B)$.
We focus on the two-task case to better understand the impact of having different sample sizes and covariates.
Let $n_1, n_2$ denote task one  and two's sample size, respectively.
Suppose
\begin{align*}
	X^{(1)} = Z^{(1)}(\Sigma^{(1)})^{1/2} \in \real^{n_1 \times p} \text{ and }
	X^{(2)} = Z^{(2)}(\Sigma^{(2)})^{1/2} \in \real^{n_2 \times p},
\end{align*}
where the entries of $Z^{(1)}$ and $ Z^{(2)}$ are drawn independently from a one dimensional distribution with zero mean, unit variance, and constant $\varphi$-th moment for a fixed $\varphi > 4$. $\Sigma^{(1)}\in \R^{p\times p}$ and $\Sigma^{(2)}\in \R^{p\times p}$ denote the population covariance matrices of task 1 and task 2, respectively.

\paragraph{Bias-variance equations.}
Our key result characterizes the asymptotic limit of the inverse of the sum of two arbitrarily different sample covariance matrices.
Without loss of generality, we consider task two's prediction loss, and the same result applies to task one.
We consider the case of $r = 1 < t = 2$, since when $r > 1$, the global minimum of $f(A, B)$ reduces to single-task learning (cf. Proposition 1 of \citet{WZR20}).
When $r = 1$, $B$ is a vector, and $A_1, A_2$ are both scalars.
To motivate our study, we consider a special case where $A_1=A_2=1$.
Hence the HPS estimator is equal to $B$.
%Hence we can write down a closed form equation for any local minimizer of $f(A, B)$.
By solving $B$ in equation \eqref{eq_mtl}, we obtain the estimator for task two as follows:
\begin{align}
	\hat{\beta}_2^{\MTL} = {\hat{\Sigma}}^{-1} ({X^{(1)}}^{\top} Y^{(1)} + {X^{(2)}}^{\top} Y^{(2)}), \text{ where }
	\hat{\Sigma} = {X^{(1)}}^{\top} X^{(1)} + {X^{(2)}}^{\top} X^{(2)}. \label{def hatsig}
\end{align}
The matrix $\hat{\Sigma}$ adds up both tasks' sample covariance matrices, and the expectation of $\hat{\Sigma}$ is equal to a mixture of their population covariance matrices, with mixing proportions determined by their sample sizes.

To derive the bias and variance equation, we consider the expected loss conditional on the covariates as follows (the empirical loss is close to this expectation as will be shown in equation \eqref{claim_largedev2}):
 %similar to Claim \ref{claim_pred_err}
\begin{align}
	 \exarg{\cE}{L(\hat{\beta}_2^{\MTL}) \mid X^{(1)}, X^{(2)}}
	=& \bignorm{{\Sigma^{(2)}}^{1/2} \hat{\Sigma}^{-1} {X^{(1)}}^{\top} X^{(1)} (\beta^{(1)} - \beta^{(2)})}^2 \label{eq_bias_2task} \\
	& + \sigma^2 \bigtr{\Sigma^{(2)}\hat{\Sigma}^{-1}}. \label{eq_variance_2task}
\end{align}
Equations \eqref{eq_bias_2task} and \eqref{eq_variance_2task} correspond to the bias and variance of HPS for two tasks, respectively.
Our main result in this section characterizes the asymptotic bias-variance limits in the high-dimensional setting.
Intuitively, the spectrum of $\hat{\Sigma}^{-1}$ (and hence its trace) not only depends on both tasks' sample sizes but also depends on the ``alignment'' between $\Sigma^{(1)}$ and $\Sigma^{(2)}$.
However, capturing this intuition quantitatively turns out to be technically challenging.
%The main technical challenge of our result deals with the ``covariate shift'' between tasks one and two.
We introduce a critical quantity $M \define (\Sigma^{(1)})^{1/2}(\Sigma^{(2)})^{-1/2}$, and as we show below, the trace of $\hat{\Sigma}^{-1}$ has an intricate dependence on the spectrum of $M$.
%Let $U\Lambda V^\top$ denote the SVD of $M$ and
Let $\lambda_1, \lambda_2, \dots, \lambda_p$ denote $M$'s singular values in descending order.
Our main result is stated as follows.

\begin{theorem}\label{thm_main_RMT}
	Let $c_{\varphi}$ be any fixed value within $(0, \frac{\varphi - 4}{2\varphi})$.
	Assume that: a) the sample sizes $n_1$ and $n_2$ both satisfy Assumption \ref{assume_rm};
	b) $M$'s singular values are all greater than $\tau$ and less than $1/\tau$;
	c) task one's sample size is greater than $\tau p$ and task two's sample size is greater than $(1 + \tau) p$.
	With high probability over the randomness of $X^{(1)}$ and $X^{(2)}$, we have the following limits:

\noindent(i) The variance equation \eqref{eq_variance_2task} $\tr[\Sigma^{(2)} \hat{\Sigma}^{-1}]$ (leaving out $\sigma^2$) satisfies the following estimate:
			\begin{align}\label{lem_cov_shift_eq}
				\bigabs{\bigtr{\Sigma^{(2)} \bigbrace{ {\hat{\Sigma}^{-1}} - \frac{(a_1 \Sigma^{(1)} + a_2 \Sigma^{(2)})^{-1}}{n_1+n_2} }}}
				\le p^{-c_{\varphi}},
			\end{align}
			where $a_1$ and $a_2$ are the solutions of the following self-consistent equations
			\begin{align}
				a_1 + a_2 = 1- \frac{p}{n_1 + n_2},
				a_1 + \frac1{n_1 + n_2}\cdot \bigbrace{\sum_{i=1}^p \frac{\lambda_i^2 a_1}{\lambda_i^2 a_1 + a_2}} = \frac{n_1}{n_1 + n_2}. \label{eq_a12extra}
			\end{align}

\noindent(ii) The bias equation \eqref{eq_bias_2task} satisfies the following limit with high probability: Let $S$ be an arbitrary subset of the unit sphere in dimension $p$ whose size is polynomial in $p$, for any unit vector $w\in S$,
			\begin{align}\label{lem_cov_derv_eq}
				\bigabs{w^{\top} \Sigma^{(1)} \bigbrace{\hat{\Sigma}^{-1} \Sigma^{(2)} \hat{\Sigma}^{-1} - \frac{1}{(n_1+n_2)^2}{\Sigma^{(2)}}^{-1/2} V {\frac{a_3 \Lambda^2 + (a_4 + 1)\id}{(a_1 \Lambda^2 + a_2\id)^2}} V^{\top} {\Sigma^{(2)}}^{-1/2}} \Sigma^{(1)} w} \le  \frac{p^{-c_{\varphi}}}{(n_1+n_2)^2},
			\end{align}
				where $a_{3}$ and $a_4$ are the solutions of the following self-consistent equations % with $b_k = \frac1{p}\sum_{i=1}^p \frac{\lambda_i^{2k}} {(\lambda_i^2 a_1 + a_2)^2}$, for $k = 0, 1, 2$:
			\begin{align}\label{eq_a34extra}
				a_3 + a_4 = \frac{1}{n_1 + n_2}\sum_{i=1}^p \frac{1}{\lambda_i^2 a_1 + a_2}, \ \
				a_3 + \frac{1}{n_1 + n_2} \sum_{i=1}^p \frac{\lambda_i^2 (a_2 a_3-a_1 a_4 )}{(\lambda_i^2 a_1 + a_2)^2} = \frac{1}{n_1 + n_2} \sum_{i=1}^p \frac{\lambda_i^2 a_1}{(\lambda_i^2 a_1 + a_2)^{2}}.
%				\left(\frac{\rho_1}{a_1^{2}} -  b_2  \right)\cdot  a_3 -  b_1 \cdot  a_4 = b_1,\quad \left(\frac{\rho_2}{a_2^{2}}-  b_0\right)\cdot  a_4 - b_1 \cdot  a_3
%				= b_0.
			\end{align}
\end{theorem}
%Due to space limit, we defer the bias limit result to Appendix \eqref{appendix RMT}.
Our result extends Fact \ref{fact_tr} to the inverse of the sum of two sample covariance matrices.
To see this, when $n_1$ is zero, we solve equation \eqref{eq_a12extra} to obtain that $a_1 = 0$ and $a_2 = (n_2-p) / n_2$, and apply them to equation \eqref{lem_cov_shift_eq}.
For general $A_1,A_2$ that are not equal to one, we can still apply our result by rescaling $X^{(1)}$ and $M$ with $A_1 / A_2$.
We defer a proof sketch of Theorem \ref{thm_main_RMT} until the end of the section.
%This amounts to replacing $M$ with $\frac{A_1}{A_2}M$ in Theorem \ref{thm_main_RMT}.

\paragraph{How does hard parameter sharing scale with sample sizes and covariate shift $M$?}
One can see that the variance limit depends intricately on both tasks' samples sizes and covariate shift.
Next, we illustrate how varying them impact the prediction loss.
\begin{example}[Sample size ratio]\label{ex_sample_ratio}
	We first consider the impact of varying sample sizes.
	Consider the random-effect model from Section \ref{sec_same}, with both tasks having an isotropic population covariance matrix.

	Applying Theorem \ref{thm_main_RMT} to the above setting, we get that
	\begin{align*}
		\frac{1}{n_1 + n_2} \tr[\Sigma^{(2)} (a_1\Sigma^{(1)} + a_2\Sigma^{(2)})^{-1}]
		= \frac{1}{n_1 + n_2} \bigtr{((a_1 + a_2)\id_p)^{-1}}
		= \frac{p}{n_1 + n_2 - p},
	\end{align*}
	because $a_1 + a_2 = 1 - \frac{p}{n_1 + n_2}$ by equation \eqref{eq_a12extra}.
	%Similarly, for the bias limit, we solve the self-consistent equations \eqref{eq_a34extra} to get $a_3$ and $a_4$ after we have obtained $a_1, a_2$.
	Similarly, we can calculate the bias limit.
	Combined together, we obtain the following corollary of Theorem \ref{thm_main_RMT}.
\end{example}

\begin{corollary}\label{cor_MTL_loss}
	In the setting of Example \ref{ex_sample_ratio}, assume that
	%a) the sample sizes $n_1$ and $n_2$ are greater than $(1 + \tau) p$, b) $\Sigma_1=\Sigma_2=\id_p$, and c) %there exists a small constant $c_0>0$ such that
	(i) both tasks sample sizes are at least $3p$;
	(ii) noise variance is smaller than the shared signal variance: $\sigma^2 \lesssim  \kappa^2$;
%	\be\label{choiceofpara0}
%	p^{-1/2+c_0}\sigma^2 + p^{c_0}d^2\le \kappa^2\le p^{1-c_0} (\sigma^2 +d^2)  .  	\ee
	%\be\label{choiceofpara0}
%	(ii) the task-specific variance of $\beta_i$ is much smaller than the signal strength {\color{red}$d^2 = \oo( {\kappa^2})$}; \HZ{what does $\ll$ mean exactly?}
%	(iii) the sample sizes $n_1$ and $n_2$ are greater than $(1 + \tau) p$.
	(iii) task-specific variance is much smaller than the shared signal variance: $d^2 \le p^{-\e}{\kappa^2}$ for a small constant $c>0$.
	Let $\varepsilon = (1 + \sqrt{p/n_1})^ 4 - 1$, which decreases as $n_1$ increases.
	Let $\hat{A},\hat{B}$ be the global minimizer of $f(A, B)$.
	With high probability over the randomness of the input,
	the prediction loss of $\hat{\beta}_2^{\MTL} = \hat{B} \hat{A}_2$ for task two satisfies that
	\begin{align}
	%-\left[1- \left( 1-\frac{1}{\sqrt{\rho_1}}\right)^4\right] pd^2\cdot \frac{\rho_1^2 (\rho_1+\rho_2)}{(\rho_1 + \rho_2 - 1)^3} +\OO(p^{-c}\sigma^2)  \le
	   \left|L(\hat{\beta}_2^{\MTL}) - \frac{2d^2 n_1^2 (n_1 + n_2)}{(n_1 + n_2 - p)^3} -\frac{\sigma^2 p}{n_1 + n_2 - p}  \right|
	\le \varepsilon \cdot \frac{2d^2 n_1^2 (n_1 + n_2)}{(n_1 + n_2 - p)^3} +  \OO(p^{-c/2}).\label{cor_MTL_error}
	%\left[\left( 1+\frac{1}{\sqrt{\rho_1}}\right)^4-1\right] d^2\cdot \frac{\rho_1^2 (\rho_1+\rho_2)}{(\rho_1 + \rho_2 - 1)^3} \\
	%& +C \left[(p^{-c_\varphi}+p^{-c_\infty/2})(\sigma^2 +d^2)+p^{-c_\infty}\kappa^2 + %\frac{d^4+\sigma^2 d^2}{\kappa^2}\right],\nonumber
	 \end{align}
%	 with high probability for any fixed $c\in(0, \min(\frac{1}{4}, \delta,\frac{\varphi-4}{2\varphi}))$.
%	 {\color{red}[FY: the error also contains $p^{-1/2+2c}\kappa^2 +  p^{-1/4+c} (\sigma^2 +d^2) $, both of which cannot be omitted, because (i) there is no assumption on the upper bound of $\kappa^2$, and (ii) we do not necessarily have $c_\varphi<1/4$. We can decide how to present the result concisely (for instance we can impose an upper bound on $\kappa^2$ and that $c_\varphi<1/4$), but it needs to be correct.]}
	 \end{corollary}

In the above inequality, the $d^2$ scaling term is the bias limit, and the $\sigma^2$ scaling term is the variance limit.
This result allows for a more concrete interpretation since the dependence on datasets' properties is explicit.
The proof of Corollary \ref{cor_MTL_loss} can be found in Appendix \ref{app_iso_cov}.
As a remark, %in equation \eqref{cor_MTL_error}, the predication loss $L(\hat{\beta}_2^{\MTL}) $ was obtained using the global minimizer $(\hat A,\hat B)$. B
by combining the bias and variance limits, we can also obtain a bias-variance tradeoff for any local minimizer of $f(A, B)$.
The proof is similar to Corollary \ref{cor_MTL_loss}, so we omit the details.

Next, we use the bias-variance limits to study how varying sample sizes impacts HPS.
For example, imagine if we want to decide whether to collect more of task one's data or not, how does increasing $n_1$ affect the prediction loss?
We assume that $n_2$ is fixed for simplicity.
%Now we illustrate an interesting phenomenon that adding task one's samples helps task two initially, but may hurt eventually.
The variance limit in equation \eqref{cor_MTL_error} obviously decreases with $n_1$.
It turns out that the bias term always increases with $n_1$, which can be verified by showing that the bias limit's derivative is always nonnegative.
%As a function of the sample ratio, the limiting estimate always decreases first from $\frac{\sigma^2 p}{n_2 - p}$ with $n_1$ being zero, and then increases to $d^2$ when $n_1$ goes to infinity.
%We describe a sketch of the proof.
By comparing the derivative of the bias and variance limits with respect to $n_1$ (details omitted), we obtain the following dichotomy.
\begin{enumerate}
	\item When $\frac{d^2}{\sigma^2} < \frac{p}{4n_2 - 6p}$, the prediction loss decreases monotonically as $n_1$ increases.
	Intuitively, this regime of $d^2$ always helps task two.
	\item When $\frac{d^2}{\sigma^2} > \frac{p}{4n_2 - 6p}$, the prediction loss always decreases first from $\frac{\sigma^2 p}{n_2 - p}$ (when $n_1 = 0$), and then increases to $d^2$ (when $n_1 \rightarrow \infty$).
	To see this, near the point where $n_1$ is zero, one can verify (from the derivatives) that bias increases less while variance decreases more, and there is \textit{exactly} one critical point where the derivative is zero, which corresponds to the \textit{optimal sample size ratio}.
\end{enumerate}

\begin{example}[Covariate shift]\label{ex_covshift}
%So far we have considered the isotropic model where $\Sigma_1 = \Sigma_2$.
%This setting is relevant for settings where different tasks share the same input features such as multi-class image classification.
%In general, the covariance matrices of the two tasks may be different such as in text classification.
Our second example focuses on how varying covariate shifts impacts the \textit{variance} limit in equation \eqref{lem_cov_shift_eq}. For large enough $p$,
%in the left hand side of equation \eqref{lem_cov_shift_eq}:
\begin{align*}
	\bigtr{\Sigma^{(2)} \hat{\Sigma}^{-1}} &\rightarrow \frac{1}{n_1 + n_2} \bigtr{\Sigma^{(2)} (a_1 \Sigma^{(1)} + a_2 \Sigma^{(2)})^{-1}}
	= \frac{1}{n_1 + n_2} \bigtr{(a_1 M^{\top} M + a_2 \id)^{-1}}.
\end{align*}
%As we are going to show later, covariate shift is accurately captured by the spectrum of $\Sigma^{1/2}\Sigma^{-1/2}$.
Hence the variance limit depends on the spectrum of $M$. % To be clear, for this example we assume that the bias is $0$.
To illustrate the result, suppose that half of $M$'s singular values are equal to $\lambda > 1$ and the other half are equal to $\lambda^{-1}$.
In particular, when $\lambda = 1$, there is no covariate shift.
As $\lambda$ increases, the severity of covariate shift increases.
We observe the following dichotomy.
\begin{enumerate}
	\item If $n_1 \ge n_2$, then the variance limit is smallest when there is no covariate shift.
	\item If $n_1 < n_2$, then the variance limit is largest when there is no covariate shift.
\end{enumerate}
\end{example}
\noindent We explain why the dichotomy happens. The variance limit in this example is equal to
$\frac{p}{2(n_1 + n_2)} f(\lambda)$, where
\[ f(\lambda) = {(\lambda^{-2} a_1 + a_2)^{-1} + (\lambda^2 a_1 + a_2)^{-1}}. \]
Using the fact that $a_1 + a_2 = 1 - \frac{p}{n_1 + n_2}$, we can verify
%\begin{align*}
%	f(\lambda) - f(1) &= \left(\lambda^2 a_1 + \frac{n_1 + n_2 - p}{n_1 + n_2} - a_1\right)^{-1} \\
%	&+ \left(\lambda^{-2} a_1 + \frac{n_1 + n_2 - p}{n_1 + n_2} - a_1\right)^{-1} \\
%	&- \frac{2(n_1 + n_2)}{n_1 + n_2 - p} \\
%	&= \left(2a_1 - \frac{n_1 + n_2-p} {n_1 + n_2 }\right)  g(\lambda, a_1), %\cdot (\lambda^2-1)^2
%\end{align*}
\begin{align*}
	f(\lambda) - f(1) &= \left(2a_1 - \frac{n_1 + n_2-p} {n_1 + n_2 }\right)  g(\lambda, a_1), %\cdot (\lambda^2-1)^2
\end{align*}
where $g(\lambda, a_1) \ge 0$.
%and can be derived from algebraic calculations (details omitted).
We claim that $a_1 \ge \frac{n_1 + n_2-p}{2(n_1 + n_2 )}$ if and only if $n_1 \ge n_2$, which explains the dichonomy. In fact, if $a_1>a_2$, then equation \eqref{eq_a12extra} gives that $a_1> \frac{n_1 + n_2-p}{2 (n_1 + n_2)}$, and equation \eqref{eq_a12extra} gives that
\begin{align*}
 \frac{n_1}{n_1 + n_2} &> a_1 + \frac{p}{2(n_1+n_2)} \left(\frac{\lambda^2}{\lambda^2+1}+\frac{\lambda^{-2}}{\lambda^{-2}+1}\right) >\frac{1}{2}.
\end{align*}
%\begin{align*}
% \frac{n_1}{n_1 + n_2} &=a_1 + \frac1{n_1 + n_2}\cdot \bigbrace{\sum_{i=1}^p \frac{\lambda_i^2 a_1}{\lambda_i^2 a_1 + a_2}} \\
%	&> a_1 + \frac{p}{2(n_1+n_2)} \left(\frac{\lambda^2}{\lambda^2+1}+\frac{\lambda^{-2}}{\lambda^{-2}+1}\right) =\frac{1}{2}.
%\end{align*}
This implies $n_1>n_2$. The other direction follows from similar arguments. % Similarly, if $a_1<a_2$, equations  \eqref{eq_a12extra000} and  \eqref{eq_a12extra} give that $a_1 < \frac{n_1 + n_2-p}{2 (n_1 + n_2)}$ and $n_1<n_2$.
%Thus, we conclude that $f(\lambda) \ge f(1)$ if and only if $n_1 \ge n_2$.

\iffalse
%FY: Below was the word explanation. I still think the above simple proof is clearer.
In fact, due to the fact that $M=M^{-1}$, $a_1$ and $a_2$ play symmetric roles in equations \eqref{eq_a12extra000} and \eqref{eq_a12extra}.
Hence, when $n_1 \ge n_2$, we have that $a_1 \ge a_2$, hence $a_1 \ge \frac{1}{2}(1 - \frac{p}{n_1 + n_2 - p}) = \frac{n_1 + n_2}{2 (n_1 + n_2 - p)}$.
The other case when $n_1 < n_2$ is similar.
A formal proof follows easily from the self-consistent equations \eqref{lem_cov_shift_eq} and we omit the details.
Thus, we conclude that if $n_1 \ge n_2$, then $f(\lambda) > f(1)$.
If $n_1< n_2$, then $f(\lambda)< f(1)$.
\fi
 
 


\paragraph{Proof overview of Theorem \ref{thm_main_RMT}.}
For the rest of this section, we present an overview of the proof of Theorem \ref{thm_main_RMT}.
The central quantity of interest is the inverse of the sum of two sample covariance matrices.
We note that the variance equation $\tr[\Sigma^{(2)} \hat{\Sigma}^{-1}]$ is equal to $(n_1 + n_2)^{-1} \bigtr{W^{-1}}$, where $W$ is
\begin{align}\label{eigen2extra}
	\frac{\Lambda U^\top (Z^{(1)})^\top Z^{(1)} U\Lambda  + V^\top (Z^{(2)})^\top Z^{(2)}V}{n_1 + n_2}.
\end{align}
Here $U\Lambda V^\top$ is defined as the SVD of $M$.
This formulation is helpful because we know that $(Z^{(1)})^{\top} Z^{(1)}$ and $(Z^{(2)})^{\top} Z^{(2)}$ are both sample covariance matrices with isotropic population covariance, and $U, V$ are both orthonormal matrices.
For example, if $Z^{(1)},Z^{(2)}$ are both Gaussian random matrices, by rotational invariance, $Z^{(1)} U, Z^{(2)}V$ are still Gaussian random matrices.

Our proof uses the Stieltjes transform or the resolvent method in random matrix theory.
We briefly describe the key ideas and refer the interested readers to classical texts such as  \citet{bai2009spectral,tao2012topics,erdos2017dynamical}.
For any probability measure $\mu$ supported on $[0,\infty)$, the Stieltjes transform of $\mu$ is a complex function defined as
$$m_\mu(z):= \int_0^\infty \frac{\dd\mu(x)}{x-z}, \text{ for any complex } z\in \C\setminus \set{0}.$$
Thus, the Stieltjes transform method reduces the study of a probability measure $\mu$ to the study of a complex function $m_\mu(z)$.
%To study the trace of $\hat{\Sigma}^{-1}$, we consider the Stieltjes transform of its empirical spectral distribution.





%Recall that  $U\Lambda V^\top$ is the singular value decomposition of $M$, %$\Lambda$ consists of the singular values of $M$, $V$ is an orthonormal matrix,
%It is not hard to verify that $(n_1 + n_2)\hat{\Sigma}^{-1}= (\Sigma^{(2)})^{-1/2} V W^{-1}V^\top (\Sigma^{(2)})^{-1/2}$ and .
Let $\mu=p^{-1}\sum_{i} \delta_{\sigma_i}$ denote the empirical spectral distribution of $W$, where the $\sigma_i$'s are the eigenvalues of $W$ and $\delta_{\sigma_i}$ is the point mass measure at $\sigma_i$. Then it is easy to see that the Stieltjes transform of $\mu$ is equal to
 \[ m_{\mu}(z) \define \frac{1}{p}\sum_{i=1}^p \frac{1}{\sigma_i - z}= p^{-1}\tr\left[(W-z\id)^{-1}\right]. \]
The above matrix $(W - z\id)^{-1}$ is known as $W$'s resolvent or Green's function.
%When $p$ goes to infinity, it is well-known that $m_{\mu}(z)$ converges to a fixed distribution governed by a set of self-consistent equations.
%These self-consistent equations give the asymptotic limit of the trace of $\hat{\Sigma}^{-1}$.
%The above approach applies when $\Sigma^{(2)}$ is isotropic in Theorem \ref{thm_main_RMT}.
%Since our goal is to show the limit of $\tr\left[ (Y - z\id)^{-1}V^{\top}{\Sigma^{(2)}}^{-1}V \right]$ as shown in equation \eqref{eigen2extra}, we study the  $(Y-z\id)^{-1}$.
%Compared to the Stieltjes transform, the resolvent also applies to random matrices.
We prove the convergence of $W$'s resolvent using the so-called ``local law'' with a sharp convergence rate \cite{isotropic,erdos2017dynamical,Anisotropic}.
%Recent developments in the random matrix literature have shown the convergence of the resolvent matrix using the so-called ``local laws'' or ``deterministic equivalents'' (cf. \cite{Hachem2007deterministic,DS18}).

%For our purpose, we use a convenient linearization trick in linear algebra, that is, the SVD of a rectangular matrix $A$ is equivalent to the study of the eigendecomposition of the symmetric block matrix
%$$H(A):=\begin{pmatrix}0 & A \\ A^\top & 0\end{pmatrix},$$
%which is linear in $A$. This trick has been used in many random matrix literatures, such as \cite{Anisotropic, AEK_Gram, XYY_circular,DY20201}.
We say that $(W-z\id)^{-1}$ converges to a deterministic $p\times p$ matrix limit $R(z)$ if for any sequence of deterministic unit vectors $v\in \R^p$,
$$v^\top \left[(W-z\id)^{-1}-R(z)\right]v\to 0\ \ \ \text{when $p$ goes to infinity.
}$$
To study $W$'s resolvent, we observe that $W$ is equal to $\AF\AF^{\top}$ for a $p$ by $n_1 + n_2$ matrix
	\be\label{defn AF} \AF := (n_1+ n_2)^{-1/2} [\Lambda U^\top (Z^{(1)})^\top,V^\top (Z^{(2)})^\top]. \ee
%}\HZ{what does this trick mean? use less technical words},
%This idea dates back at least to Girko, see e.g., the works \cite{girko1975random,girko1985spectral} and references therein.
Consider the following symmetric block matrix whose dimension is $p + n_1 + n_2$
%which is a linear function of $(Z^{(1)})$ and $(Z^{(2)})$:
%\begin{definition}[Linearizing block matrix]\label{def_linearHG}%definiton of the Green function
%We define the $(n+N)\times (n+N)$ block matrix
 \begin{equation}\label{linearize_block}
    H \define \left( {\begin{array}{*{20}c}
   0 & \AF  \\
   \AF^{\top} & 0
%   {Z^{(2)}V} & 0 & 0
   \end{array}} \right).
 \end{equation}
For this block matrix, we define its resolvent as
$$G(z) \define \left[H - \begin{pmatrix}z\id_{p\times p}&0\\ 0 & \id_{(n_1+n_2)\times (n_1+n_2)} \end{pmatrix}\right]^{-1},$$
%as the resolvent of $H$,
for any complex value $z\in \mathbb C$.
Using Schur complement formula for the inverse of a block matrix, it is not hard to verify that
	\begin{equation} \label{green2}
	  G(z) =  \left( {\begin{array}{*{20}c}
			(W- z\id)^{-1} & (W - z\id)^{-1} \AF  \\
      \AF^\top (W - z\id)^{-1} & z(\AF^\top \AF - z\id)^{-1}
		\end{array}} \right).%\quad \cal G_R:=(W^\top W - z)^{-1} ,
  \end{equation}



\paragraph{Variance asymptotic limit.}
In Theorem \ref{main_cor}, we will show that for $z$ in a small neighborhood around $0$, when $p$ goes to infinity, $G(z)$ converges to the following limit
\be \label{defn_piw}
	\Gi(z) \define \begin{pmatrix} (a_{1}(z)\Lambda^2  +  (a_{2}(z)- z)\id_{p\times p})^{-1} & 0 & 0 \\ 0 & - \frac{n_1+n_2}{n_1} a_{1}(z)\id_{n_1\times n_1} & 0 \\ 0 & 0 & -\frac{n_1+n_2}{n_2}a_{2}(z)\id_{n_2\times n_2}  \end{pmatrix},\ee
where $a_1(z)$ and $a_2(z)$ are the unique solutions to the following self-consistent equations
\be\label{selfomega_a}
\begin{split}
	&a_1(z) + a_2(z) = 1 - \frac{1}{n_1 + n_2} \bigbrace{\sum_{i=1}^p \frac{\lambda_i^2 a_1(z) + a_2(z)}{\lambda_i^2 a_1(z) + a_2(z) - z}}, \\ %\label{selfomega_a000} \\
	&a_1(z) + \frac{1}{n_1 + n_2}\bigbrace{\sum_{i=1}^p \frac{\lambda_i^2 a_1(z)}{\lambda_i^2 a_1(z) + a_2(z) - z}} = \frac{n_1}{n_1 + n_2}.
% \frac{\rho_1}{a_{1}(z)} = \frac{1}{p}\sum_{i=1}^p \frac{\lambda_i^2}{ - z+\lambda_i^2 a_{1}(z) +a_{2} (z) } + (\rho_1+\rho_2),\  \frac{\rho_2}{a_{2}(z)} = \frac{1}{p}\sum_{i=1}^p \frac{1 }{  -z+\lambda_i^2 a_{1}(z) +  a_{2}(z)  }+ (\rho_1+\rho_2) .
\end{split}
\ee
The existence and uniqueness of solutions to the above system are shown in Lemma \ref{lem_mbehaviorw}.
%First, we define the deterministic limits of $(m_1(z), m_{2}(z))$ by $\left(-\frac{\rho_1+\rho_2}{\rho_1}a_{1}(z),-\frac{\rho_1+\rho_2}{\rho_2}a_{2}(z)\right)$, where
%satisfying that $\im a_{1}(z)< 0$ and $\im a_{2}(z)<0$ for $z\in \C_+$ with $\im z$.
%\be\label{ratios}
% \gamma_n :=\frac{p}{n}=\frac{1}{\rho_1+\rho_2},\quad r_1 :=\frac{n_1}{n}=\frac{\rho_1}{\rho_1+\rho_2},\quad r_2 :=\frac{n_2}{n}=\frac{\rho_2}{\rho_1+\rho_2}.
%\ee
Given this result, we now show that when $z = 0$, the matrix limit $\Gi(0)$ implies the variance limit shown in equation \eqref{lem_cov_shift_eq}.
First, we have that $a_1 = a_1(0)$ and $a_2 = a_2(0)$ since the equations in \eqref{selfomega_a} reduce to equation \eqref{eq_a12extra} when $z=0$.
Second, since $W^{-1}$ is the upper-left block matrix of $G(0)$, we have that $W^{-1}$ converges to $ (a_1\Lambda^2 + a_2\id)^{-1} $.
Using the fact that $\tr[\Sigma^{(2)} \hat{\Sigma}^{-1}] = (n_1 + n_2)^{-1}\bigtr{W^{-1}} $, we get that when $p$ goes to infinity, % the trace of $$ converges to
\begin{align*}
  \bigtr{\Sigma^{(2)} \hat{\Sigma}} \rightarrow \frac{1}{n_1+n_2}\bigtr{(a_1 \Lambda^2 + a_2\id)^{-1}} &= \frac1{n_1+n_2}\bigtr{(a_1 M^{\top}M + a_2 \id)^{-1}} \\
  &=\frac{1}{n_1+n_2} \bigtr{\Sigma^{(2)} (a_1 \Sigma^{(1)} + a_2 \Sigma^{(2)})^{-1}},
  \end{align*}
%\noindent{\bf Variance asymptotics.} Using definition \eqref{mainG}, we can write equation \eqref{eigen2extra} as
%\be\label{rewrite X as R} [(X^{(1)})^\top X^{(1)}+(X^{(2)})^\top X^{(2)}]^{-1}=n^{-1}\Sigma_2^{-1/2}V\cal G(0)V^\top\Sigma_2^{-1/2}.\ee
%When $z=0$, it is easy to check that , which means that we actually have $a_1(0)=a_1$ and $a_2(0)=a_2$. Hence the matrix limit of $\cal G(0)$ is given by $(a_{1}\Lambda^2 + a_{2}\id_p)^{-1}$. Then inserting this limit into equation \eqref{rewrite X as R}, we can write the left-hand side of equation \eqref{lem_cov_shift_eq} as
%\begin{align}
%&\bigtr{\left( (X^{(1)})^{\top}X^{(1)} + (X^{(2)})^{\top}X^{(2)}\right)^{-1} \Sigma}\approx n^{-1}\bigtr{\Sigma_2^{-1/2}V\cal (a_{1}\Lambda^2 + a_{2}\id_p)^{-1}V^\top\Sigma_2^{-1/2}\Sigma}  \nonumber\\
%&=n^{-1}\bigtr{\Sigma_2^{-1/2}\cal (a_{1}\Sigma_2^{-1/2}\Sigma_1\Sigma_2^{-1/2} + a_{2}\id_p)^{-1}\Sigma_2^{-1/2}\Sigma}  = n^{-1}\bigtr{\cal (a_{1} \Sigma_1  + a_{2}\Sigma_2)^{-1}\Sigma}  ,\label{Gi00}
%\end{align}
where we note that $M^\top M = (\Sigma^{(2)})^{-1/2} \Sigma^{(1)} (\Sigma^{(2)})^{-1/2}$ and its SVD is equal to $V^{\top}\Lambda^2 V$.
%For the asymptotic limit, its concentration error is shown in Appendix \ref{appendix RMT}.



\paragraph{Bias asymptotic limit.}
 For the bias limit in equation \eqref{lem_cov_derv_eq}, we show that it is governed by the derivative of $(W - z\id)^2$ with respect to $z$ at $z = 0$.
First, we can express the empirical bias term in equation \eqref{lem_cov_derv_eq} as %$W$
\begin{align}\label{calculate G'}
	(n_1 + n_2)^2 \hat{\Sigma}^{-1}\Sigma^{(2)}\hat{\Sigma}^{-1} = {\Sigma^{(2)}}^{-1/2} V W^{-2} V^{\top} {\Sigma^{(2)}}^{-1/2}.
\end{align}
Let $\cal G(z):=(W-z\id )^{-1}$ denote the resolvent of $W$.
Our key observation is that $\frac{\dd{\cal G(z)}}{\dd z} =  \cal G^2(z)$.
Hence, provided that the limit of $(W - z\id)^{-1}$ is $(a_1(z) \Lambda^2 + (a_2(z) - z) \id)^{-1}$ near $z = 0$, the limit of $\frac{\dd{\cal G(0)}}{\dd z}$ satisfies
\begin{align}\label{cal G'0}
	\frac{\dd \cal G(0)}{\dd z} \to \frac{-\frac{\dd a_1(0)}{\dd z}\Lambda^2 - (\frac{\dd a_2(0)}{\dd z} - 1)\id}{(a_{1}(0)\Lambda^2 + a_{2}(0)\id_p)^2}.
\end{align}
To find the derivatives of $a_1(z)$ and $a_2(z)$, we take the derivatives on both sides of the system of equations \eqref{selfomega_a}.
%\begin{align*}
%	\frac{\dd a_1(z)}{\dd z} + \frac{\dd a_2(z)}{\dd z} = -\frac{1}{n_1 + n_2} \sum_{i=1}^p \frac{1}{\lambda_i^2 a_1 + a_2},
%	\frac{\dd a_1(z)}{\dd z} + \frac{1}{n_1 + n_2}\sum_{i=1}^p \frac{\lambda_i^2 (a_1'(z) a_2 - a_2'(z) a_1)}{(\lambda_i^2 a_1 + a_2)^2} = -\frac{1}{n_1 + n_2} \sum_{i=1}^p \frac{\lambda_i^2 a_1}{(\lambda_i^2 a_1 + a_2)^2}
%\end{align*}
Let $a_3 = - \frac{\dd a_1(0)}{\dd z}$ and $a_4 = - \frac{\dd a_2(0)}{\dd z}$.
%then taking implicit differentiation of equation \eqref{selfomega_a}
One can verify that $a_3$ and $a_4$ satisfy the self-consistent equations in \eqref{eq_a34extra} (details omitted).
Applying equation \eqref{cal G'0} to equation \eqref{calculate G'}, we obtain the bias limit.
%\begin{align}
%& n^2\bignorm{\Sigma_2^{1/2} \bigbrace{ (X^{(1)})^{\top}(X^{(1)}) + (X^{(2)})^{\top}(X^{(2)}) }^{-1} \Sigma_1^{1/2} w}^2 \nonumber\\
%&\approx  w^\top \Sigma_1^{1/2}\Sigma_2^{-1/2}V\frac{a_3\Lambda^2 +(1+ a_4)\id_p}{(a_{1}\Lambda^2 + a_{2}\id)^2}V^\top \Sigma_2^{-1/2}\Sigma_1^{1/2}w= w^{\top} \Pi_\bias w, \label{calculatePibias}
%\end{align}
%where in the last step we used $M = \Sigma_1^{1/2}\Sigma_2^{-1/2}$ and $V \Lambda^2 V^\top=M^\top M$. This concludes equation \eqref{lem_cov_derv_eq}.

As a remark, in order for $\frac{\dd \cal G(z)}{\dd z}$ to stay close to its limit at $z = 0$, we not only need to find the limit of $\cal G(0)$, but also the limit of $\cal G(z)$ within a small neighborhood of $0$.
This is why we consider $W$'s resolvent for a general $z$ (as opposed to the Stieljes transform of its empirical spectral distribution discussed earlier).

\paragraph{Schur complement and self-consistent equations.}
First, we consider the special case where $Z^{(1)}$ and $Z^{(2)}$ are both multivariate Gaussian random matrices.
By rotational invariance, we have that $Z^{(1)} U$ and $Z^{(2)} V$ are still multivariate Gaussian random matrices.
Next, we use the Schur complement formula to deal with the resolvent $G(z)$.
We show that $G(z)$'s diagonal entries satisfy a set of self-consistent equations in the limit, leading to equations in \eqref{selfomega_a}.
On the other hand, $G(z)$'s off-diagonal entries are approximately zero using standard concentration bounds.
Finally, we extend our result to general random matrices under the finite $\varphi$-th moment condition.
We prove an anisotropic local law using recent developments in random matrix theory \cite{erdos2017dynamical,Anisotropic}.
The proof of Theorem \ref{thm_main_RMT} is shown in Appendix \ref{appendix RMT}. %Appendix \ref{appendix RMT}.

